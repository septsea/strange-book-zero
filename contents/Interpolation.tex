\subsection*{Interpolation}
\addcontentsline{toc}{subsection}{Interpolation}
\markright{Interpolation}

本节讨论多项式插值 \term{polynomial interpolation} 问题\period

``插值'' 听上去可能比较陌生\period 不过, 您在初中一定见过这样的问题:

\begin{example}
    已知一次函数的图像经过点 $(-1,2)$ 与 $(1,3)$, 求其解析式\period
\end{example}

\begin{example}
    已知二次函数的图像经过点 $(-1,-1)$, $(1,1)$ 与 $(2,5)$, 求其解析式\period
\end{example}

在初中, 我们是用 ``待定系数法'' \term{the method of undetermined coefficients} 求解的\period 它的基本思想是 ``求什么, 设什么''\period 设此一次函数的解析式为
\begin{align*}
    y = ax + b, \quad a \neq 0 \period
\end{align*}
代入已知条件, 得到二元一次方程组
\begin{align*}
    \begin{cases}
        2 = -a + b, \\
        3 = a + b \period
    \end{cases}
\end{align*}
由此可解出
\begin{align*}
    a = \frac12, \quad b = \frac52 \period
\end{align*}
所以此一次函数的解析式为
\begin{align*}
    y = \frac12 x + \frac52 \period
\end{align*}

完全类似地, 设此二次函数的解析式为
\begin{align*}
    y = ax^2 + bx + c, \quad a \neq 0 \period
\end{align*}
代入已知条件, 得到三元一次方程组
\begin{align*}
    \begin{cases}
        -1 = a - b + c, \\
        1 = a + b + c,  \\
        5 = 4a + 2b + c \period
    \end{cases}
\end{align*}
由此可解出
\begin{align*}
    a = 1, \quad b = 1, \quad c = -1 \period
\end{align*}
所以此二次函数的解析式为
\begin{align*}
    y = x^2 + x - 1 \period
\end{align*}

在初中, 一般用左 $y$ 右 $x$ 的等式表示函数 (的解析式)\period 这种表示法强调因变元 \term{dependent variable} $y$ 与自变元 \term{independent variable} $x$ 的关系\period 不过, 既然我们有 $f(x)$ 这样的记号, 那么因变元就不必写出了\period 并且, 我们在前节提到, 我们不再区分多项式与多项式函数\period 所以, 为方便, 我们用另一种方式叙述这二个问题:

\begin{example}
    求次为 $1$ 的多项式 $f(x)$, 使 $f(-1)=2$, $f(1)=3$\period
\end{example}

\begin{example}
    求次为 $2$ 的多项式 $f(x)$, 使 $f(=1)=-1$, $f(1)=1$, $f(2)=5$\period
\end{example}

设 $x_0$, $x_1$, $\cdots$, $x_{n}$ 是 $\FF$ 的 $n+1$ 个互不相同的元\period 设 $y_0$, $y_1$, $\cdots$, $y_{n} \in \FF$\period
