\subsection*{\Interpolation}
\addcontentsline{toc}{subsection}{\Interpolation}
\markright{\Interpolation}

本节讨论多项式插值问题.

``插值'' 听上去可能比较陌生. 不过, 读者在初中一定见过这样的问题:

\begin{example}
    已知一次函数的图像经过点 $(-1,2)$ 与 $(1,3)$, 求其解析式.
\end{example}

\begin{example}
    已知二次函数的图像经过点 $(-1,-1)$, $(1,1)$ 与 $(2,5)$, 求其解析式.
\end{example}

在初中, 我们是用 ``待定系数法'' \term{the method of undetermined coefficients} 求解的. 它的基本思想是 ``求什么, 设什么''. 设此一次函数的解析式为
\begin{align*}
    y = ax + b, \quad a \neq 0.
\end{align*}
代入已知条件, 得到二元一次方程组
\begin{align*}
    \begin{cases}
        2 = -a + b, \\
        3 = a + b.
    \end{cases}
\end{align*}
由此可解出
\begin{align*}
    a = \frac12, \quad b = \frac52.
\end{align*}
所以此一次函数的解析式为
\begin{align*}
    y = \frac12 x + \frac52.
\end{align*}

完全类似地, 设此二次函数的解析式为
\begin{align*}
    y = ax^2 + bx + c, \quad a \neq 0.
\end{align*}
代入已知条件, 得到三元一次方程组
\begin{align*}
    \begin{cases}
        -1 = a - b + c, \\
        1 = a + b + c,  \\
        5 = 4a + 2b + c.
    \end{cases}
\end{align*}
由此可解出
\begin{align*}
    a = 1, \quad b = 1, \quad c = -1.
\end{align*}
所以此二次函数的解析式为
\begin{align*}
    y = x^2 + x - 1.
\end{align*}

在初中, 一般用左 $y$ 右 $x$ 的等式表示函数 (的解析式). 这种表示法强调因变元 \term{dependent variable} $y$ 与自变元 \term{independent variable} $x$ 的关系. 不过, 既然我们有 $f(x)$ 这样的记号, 那么因变元就不必写出了. 并且, 我们在前节提到, 我们不再区分多项式与多项式函数. 所以, 为方便, 我们用另一种方式叙述这二个问题:

\begin{example}
    求次为 $1$ 的多项式 $f(x)$, 使 $f(-1)=2$, $f(1)=3$.
\end{example}

\begin{example}
    求次为 $2$ 的多项式 $f(x)$, 使 $f(-1)=-1$, $f(1)=1$, $f(2)=5$.
\end{example}

设 $x_0$, $x_1$, $\cdots$, $x_{n}$ 是 $\FF$ 的 $n+1$ 个互不相同的元. 这 $n+1$ 个不同的元称为 $n+1$ 个节点 \term{node}. 设 $y_0$, $y_1$, $\cdots$, $y_{n} \in \FF$. 通俗地说, 多项式插值 \term{polynomial interpolation} 的任务是: 找一个多项式 $f(x) \in \FF[x]$ 使
\begin{align*}
    f(x_i) = y_i \quad (i = 0,1,\cdots,n),
\end{align*}
且适合 ``附加条件''.

这里, ``附加条件'' 是有必要的: 如果太松, 可能找出的 $f(x)$ 不止一个; 如果太紧, 则可能找不到 $f(x)$.

\begin{example}
    找一个多项式 $f(x)$ 使 $f(-1)=-1$, $f(0)=0$, $f(1)=1$.

    如果不作任何别的约束, 那么 $n$ 是奇数时, $f(x) = x^n$ 适合这些条件. 不仅如此, 下面的多项式也适合条件:
    \begin{align*}
        \frac{1}{6}x + \frac{1}{3}x^3 + \frac{1}{2}x^5, \quad -x + 2x^7, \quad \frac{x + x^3 + \cdots + x^{2k-1}}{k}.
    \end{align*}

    在初中, 我们知道, 若平面直角坐标系的三点 $A$, $B$, $C$ 不在同一直线上, 且任意二点的连线既不与 $y$ 轴平行也不与 $y$ 轴重合, 则存在 (唯一的) 二次函数 $y = ax^2 + bx + c$ ($a \neq 0$) 使其图像过此三点. 假如 ``附加条件'' 是 ``$f(x)$ 是次为 $2$ 的多项式'' 呢? 设
    \begin{align*}
        f(x) = ax^2 + bx + c, \quad a \neq 0.
    \end{align*}
    代入已知条件, 得到三元一次方程组
    \begin{align*}
        \begin{cases}
            -1 = a - b + c, \\
            0 = c,          \\
            1 = a + b + c.
        \end{cases}
    \end{align*}
    由此可解出
    \begin{align*}
        a = 0, \quad b = 1, \quad c = 0.
    \end{align*}
    这与假定 $a \neq 0$ 不符. 所以, 这个条件太紧了.
\end{example}

有没有什么 ``松紧得当的'' ``附加条件'' 呢? 回想一下这个命题:

\begin{proposition}
    设 $a_0$, $b_0$, $a_1$, $b_1$, $\cdots$, $a_n$, $b_n$ 是 $\FF$ 的元. 设 $n$ 是非负整数. 设
    \begin{align*}
        f(x) = \sum_{i = 0}^n a_i x^i, \quad g(x) = \sum_{i = 0}^n b_i x^i.
    \end{align*}
    若 $t_0$, $t_1$, $\cdots$, $t_n$ 是 $n+1$ 个互不相同的 $\FF$ 的元, 且
    \begin{align*}
        f(t_0) = g(t_0), \quad f(t_1) = g(t_1), \quad \cdots, \quad f(t_n) = g(t_n),
    \end{align*}
    则 $f(x)$ 必等于 $g(x)$. 通俗地说, 若次不高于 $n$ (且系数为 $\FF$ 的元) 的二个多项式若在多于 $n$ 处取一样的值, 则这二个多项式相等.
\end{proposition}

由此, 我们可以试着作出这样的 ``附加条件'': 多项式的次低于节点数. 至少, 这个条件不是太松: 因为上面的命题说, 这样的多项式若存在, 必唯一.

这个 ``附加条件'' 一定能让我们求出这个多项式吗? 不好说.

\begin{example}
    如果把 $\FF$ 跟 $\FF[x]$ 改为 $\ZZ$ 跟 $\ZZ[x]$, 那么就没有 $1$ 次多项式 $f(x)$ 使 $f(-1)=2$, $f(1)=3$. 为啥? 看二元一次方程组
    \begin{align*}
        \begin{cases}
            2 = -a + b, \\
            3 = a + b.
        \end{cases}
    \end{align*}
    二式相加, 可得 $5 = 2b$. 可是, 如果 $b$ 是整数, 那么 $2b$ 是偶数. 偶数 $2b$ 不可能等于奇数 $5$ 呀!
\end{example}

具体地说, 设次低于节点数 $n+1$ 的多项式
\begin{align*}
    f(x) = \sum_{i=0}^n a_i x^i = a_0 + a_1 x + \cdots + a_n x^n \in \FF[x]
\end{align*}
适合
\begin{align*}
    f(x_i) = y_i \quad (i = 0,1,\cdots,n),
\end{align*}
则可得到下面的方程组:
\begin{align*}
    \begin{cases}
        y_0 = 1 a_0 + x_0 a_1 + \cdots + x_0^n a_n, \\
        y_1 = 1 a_0 + x_1 a_1 + \cdots + x_1^n a_n, \\
        \cdots \cdots \cdots \cdots \cdots \cdots \cdots \cdots
        \cdots \cdots \cdots \cdots,                \\
        y_n = 1 a_0 + x_n a_1 + \cdots + x_n^n a_n.
    \end{cases}
\end{align*}
这是一个有 $n+1$ 个 $n+1$ 元一次方程的方程组, 且未知元是 $a_0$, $a_1$, $\cdots$, $a_n$. 假如我们能解出这个方程组, 且这个方程组的解 ``不超出 $\FF$ 的范围'' (我们说, 上面的二元一次方程组超出了 $\ZZ$ 的范围, 但没有超出 $\FF$ 的范围), 那么就能说明 ``多项式的次低于节点数'' 这个 ``附加条件'' 是 ``松紧得当的''.

可惜, 我们在初中并没有研究一般的多元一次方程组. 我们在学习二 (三) 元一次方程组的时候, 主要学习怎么用代入消元法与加减消元法解方程组, 并没有过多地讨论方程组什么时候有解与解的结构这样的问题.

我们换一个角度看问题. 首先, 我们有如下命题:

\begin{proposition}
    设 $t_0$, $t_1$, $\cdots$, $t_{s-1} \in \FF$ 互不相同. 则 $t_0$, $t_1$, $\cdots$, $t_{s-1}$ ($1 \leq s \leq n$) 是 $n$ 次多项式 $f(x)$ 的根的一个必要与充分条件是: 存在 $n-s$ 次多项式 $q(x) \in \FF[x]$ 使
    \begin{align*}
        f(x) = (x - t_0)(x - t_1) \cdots (x-t_{s-1}) q(x).
    \end{align*}
\end{proposition}

\begin{pf}
    先看充分性. 既然 $f(x)$ 能写为这种形式, 将 $x$ 换为 $t_i$ ($i = 0$, $1$, $\cdots$, $s-1$), 则有 $f(t_i) = 0$.

    再看必要性. 因为 $t_0$ 是 $f(x)$ 的根, 故存在 $n-1$ 次多项式 $q_1 (x) \in \FF[x]$ 使
    \begin{align*}
        f(x) = (x - t_0) q_1 (x).
    \end{align*}
    设 $t_j$ 是 $t_1$, $t_2$, $\cdots$, $t_{s-1}$ 的一个. 则 $t_j \neq t_0$. 因为 $t_j$ 也是 $f(x)$ 的根, 故
    \begin{align*}
        (t_j - t_0) q_1 (t_j) = f(t_j) = 0 = (t_j - t_0) 0.
    \end{align*}
    根据消去律, $q_1 (t_j) = 0$. 这样, $t_1$, $\cdots$, $t_{s-1}$ 这 $s-1$ 个 $\FF$ 中元是 $q_1 (x)$ 的根. 所以, 对 $q_1 (x)$ 来说, 存在 $n-1-1 = n-2$ 次多项式 $q_2 (x) \in \FF[x]$ 使
    \begin{align*}
        q_1 (x) = (x - t_1) q_2 (x) \implies f(x) = (x - t_0) (x - t_1) q_2 (x),
    \end{align*}
    且 $t_2$, $\cdots$, $t_{s-1}$ 这 $s-2$ 个 $\FF$ 中元是 $q_2 (x)$ 的根. 再将这个过程进行 $s-2$ 次, 可得到 $n-s$ 次多项式 $q_s (x) \in \FF[x]$ 使
    \begin{align*}
        f(x) = (x - t_0) (x - t_1) \cdots (x - t_{s-1}) q_s (x).
    \end{align*}
    取 $q(x) = q_s (x)$ 即可.
\end{pf}

\begin{example}
    我们考虑非常特殊的情形. 如果 $y_0$, $y_1$, $\cdots$, $y_n$ 中恰有一个是 $1$, 而剩下的全是 $0$, 那这样的多项式应该长什么样呢?

    以 $y_0 = 1$, $y_1 = y_2 = \cdots = y_n = 0$ 为例. 这样, 多项式 $f(x)$ 有根 $x_1$, $x_2$, $\cdots$, $x_n$. 根据上个命题, 存在多项式 $q(x)$ 使
    \begin{align*}
        f(x) = q(x) (x - x_1) (x - x_2) \cdots (x - x_n).
    \end{align*}
    因为 $f(x)$ 的次低于 $n+1$, 而 $(x - x_1) (x - x_2) \cdots (x - x_n)$ 的次为 $n$, 故 $q(x)$ 一定是非零的数 $c$, 即
    \begin{align*}
        f(x) = c (x - x_1) (x - x_2) \cdots (x - x_n).
    \end{align*}
    因为 $f(x_0) = y_0 = 1$, 故
    \begin{align*}
        1 = c (x_0 - x_1) (x_0 - x_2) \cdots (x_0 - x_n),
    \end{align*}
    也就是
    \begin{align*}
        c = \frac{1}{(x_0 - x_1) (x_0 - x_2) \cdots (x_0 - x_n)}.
    \end{align*}
    故
    \begin{align*}
        f(x) = \frac{(x - x_1) (x - x_2) \cdots (x - x_n)}{(x_0 - x_1) (x_0 - x_2) \cdots (x_0 - x_n)}.
    \end{align*}

    类似地, 适合条件 $y_1 = 1$, $y_0 = y_2 = y_3 = \cdots = y_n = 0$ 的多项式是
    \begin{align*}
        \frac{(x - x_0)(x - x_2)(x - x_3) \cdots (x - x_n)}{(x_1 - x_0)(x_1 - x_2)(x_1 - x_3) \cdots (x_1 - x_n)}.
    \end{align*}
    可以将这个多项式简单地写为
    \begin{align*}
        \prod_{\begin{smallmatrix}0 \leq \ell \leq n \\\ell \neq 1\end{smallmatrix}} \frac{x - x_\ell}{x_1 - x_\ell}.
    \end{align*}
    上面的 $f(x)$ 也可以写为
    \begin{align*}
        \prod_{\begin{smallmatrix}0 \leq \ell \leq n \\\ell \neq 0\end{smallmatrix}} \frac{x - x_\ell}{x_0 - x_\ell}.
    \end{align*}
\end{example}

回到一般的设定 (也就是说, $y_0$, $y_1$, $\cdots$, $y_n$ 是 $\FF$ 的任意元). 作 $n+1$ 个多项式
\begin{align*}
    L_{i} (x) = \prod_{\begin{smallmatrix}0 \leq \ell \leq n \\\ell \neq i\end{smallmatrix}} \frac{x - x_\ell}{x_i - x_\ell} \quad (i = 0,1,\cdots,n).
\end{align*}
不难看出, 任取 $i,j = 0,1,\cdots,n$,
\begin{align*}
    L_{i} (x_j) = \begin{cases}
        1, \quad i = j; \\
        0, \quad i \neq j.
    \end{cases}
\end{align*}
所以
\begin{align*}
    f(x) = \sum_{i = 0}^{n} y_i L_{i} (x) = y_0 L_0 (x) + y_1 L_1 (x) + \cdots + y_n L_n (x)
\end{align*}
适合条件
\begin{align*}
    f(x_i) = y_i \quad (i = 0,1,\cdots,n),
\end{align*}
且
\begin{align*}
    \deg f(x) \leq n < n + 1.
\end{align*}

综合上面的事实, 我们已经证明了

\begin{proposition}
    设 $x_0$, $x_1$, $\cdots$, $x_{n}$ 是 $\FF$ 的 $n+1$ 个互不相同的元. 设 $y_0$, $y_1$, $\cdots$, $y_{n} \in \FF$. 存在唯一的多项式
    \begin{align*}
        f(x) = \sum_{i = 0}^{n} y_i \prod_{\begin{smallmatrix}0 \leq \ell \leq n \\\ell \neq i\end{smallmatrix}} \frac{x - x_\ell}{x_i - x_\ell}
    \end{align*}
    适合条件
    \begin{align*}
        f(x_i) = y_i \quad (i = 0,1,\cdots,n),
    \end{align*}
    且
    \begin{align*}
        \deg f(x) < n + 1.
    \end{align*}
    这个公式的一个名字是 ``Lagrange 插值公式'' \term{Lagrange's interpolation formula}.
\end{proposition}

\begin{remark}
    我们在前面接触的线性无关的多项式组 (几乎都) 是次不等的多项式. Lagrange 插值公式告诉我们, $L_0 (x)$, $L_1 (x)$, $\cdots$, $L_n (x)$ 适合:

    (i) $L_0 (x)$, $L_1 (x)$, $\cdots$, $L_n (x)$ 是线性无关的;

    (ii) 任意次不高于 $n$ 的多项式都可唯一地写为 $L_0 (x)$, $L_1 (x)$, $\cdots$, $L_n (x)$ 的线性组合;

    (iii) $L_0 (x)$, $L_1 (x)$, $\cdots$, $L_n (x)$ 全为 $n$ 次多项式.
\end{remark}

\begin{remark}
    由上面的公式, 可以看出, $f(x)$ 的 $n$ 次系数是
    \begin{align*}
        \sum_{i = 0}^{n} y_i \prod_{\begin{smallmatrix}0 \leq \ell \leq n \\\ell \neq i\end{smallmatrix}} \frac{1}{x_i - x_\ell}.
    \end{align*}
    看上去有点复杂. 我们想个办法简单地写出 $\prod$ 符号代表的内容. 作 $n+1$ 次多项式
    \begin{align*}
        N_{n+1} (x) = (x - x_0) (x - x_1) \cdots (x - x_n).
    \end{align*}
    从 $0$, $1$, $\cdots$, $n$ 里任取一个整数 $i$. 那么
    \begin{align*}
        N_{n+1} (x) = (x - x_i) \prod_{\begin{smallmatrix}0 \leq \ell \leq n \\\ell \neq i\end{smallmatrix}} (x - x_\ell).
    \end{align*}
    二边求微商, 有
    \begin{align*}
        N_{n+1} (x) = \prod_{\begin{smallmatrix}0 \leq \ell \leq n \\\ell \neq i\end{smallmatrix}} (x - x_\ell) + (x - x_i) \left( \prod_{\begin{smallmatrix}0 \leq \ell \leq n \\\ell \neq i\end{smallmatrix}} (x - x_\ell) \right)^{\prime}.
    \end{align*}
    用 $x_i$ 代替 $x$, 有
    \begin{align*}
        N_{n+1}^{\prime} (x) = \prod_{\begin{smallmatrix}0 \leq \ell \leq n \\\ell \neq i\end{smallmatrix}} (x_i - x_\ell) + 0,
    \end{align*}
    即
    \begin{align*}
        \prod_{\begin{smallmatrix}0 \leq \ell \leq n \\\ell \neq i\end{smallmatrix}} \frac{1}{x_i - x_\ell} = \frac{1}{N_{n+1}^{\prime} (x_i)}.
    \end{align*}
    这样, $f(x)$ 的 $n$ 次系数可简单地写为
    \begin{align*}
        \sum_{i = 0}^{n} \frac{y_i}{N_{n+1}^{\prime} (x_i)}.
    \end{align*}
\end{remark}

\begin{example}
    取 $n = 2$. 取
    \begin{align*}
         & x_0 = -1, \quad x_1 = 1, \quad x_2 = 2, \\
         & y_0 = -1, \quad y_1 = 1, \quad y_2 = 5.
    \end{align*}
    计算 $L_0 (x)$, $L_1 (x)$, $L_2 (x)$:
    \begin{align*}
         & L_0 (x) = \prod_{\begin{smallmatrix}0 \leq \ell \leq 2 \\\ell \neq 0\end{smallmatrix}} \frac{x - x_\ell}{x_0 - x_\ell} = \frac{(x - 1)(x - 2)}{(-1 - 1)(-1 - 2)} = \frac16 x^2 - \frac12 x + \frac13, \\
         & L_1 (x) = \prod_{\begin{smallmatrix}0 \leq \ell \leq 2 \\\ell \neq 1\end{smallmatrix}} \frac{x - x_\ell}{x_1 - x_\ell} = \frac{(x + 1)(x - 2)}{(1 + 1)(1 - 2)} = -\frac12 x^2 + \frac12 x + 1,        \\
         & L_2 (x) = \prod_{\begin{smallmatrix}0 \leq \ell \leq 2 \\\ell \neq 2\end{smallmatrix}} \frac{x - x_\ell}{x_2 - x_\ell} = \frac{(x + 1)(x - 1)}{(2 + 1)(2 - 1)} = \frac13 x^2 - \frac13.
    \end{align*}
    所以, 适合条件
    \begin{align*}
         & f(-1) = -1, \quad f(1) = 1, \quad f(2) = 5, \\
         & \deg f(x) < n + 1 = 3
    \end{align*}
    的多项式 $f(x)$ 就是
    \begin{align*}
             & (-1)L_0 (x) + 1L_1 (x) + 5L_2 (x)  \\
        = {} & -L_0 (x) + L_1 (x) + 5L_2 (x)      \\
        = {} & -\frac16 x^2 + \frac12 x - \frac13
        - \frac12 x^2 + \frac12 x + 1
        + \frac53 x^2 - \frac53                   \\
        = {} & x^2 + x - 1.
    \end{align*}
    这跟前面用三元一次方程组算出的答案完全一致.
\end{example}

\begin{example}
    取 $n = 3$. 在上例的基础上, 追加
    \begin{align*}
        x_3 = -2, \quad y_3 = -11.
    \end{align*}
    我们的目标是: 找多项式 $f(x)$ 适合条件
    \begin{align*}
         & f(-1) = -1, \quad f(1) = 1, \quad f(2) = 5, \quad f(-2) = -11, \\
         & \deg f(x) < n + 1 = 4.
    \end{align*}
    在原理上, 并没有什么复杂的地方. 求出 $L_0 (x)$, $L_1 (x)$, $L_2 (x)$, $L_3 (x)$ 后, 答案就出来了:
    \begin{align*}
        f(x)
        = {} & - \frac{(x-1)(x-2)(x+2)}{(-1-1)(-1-2)(-1+2)} + \frac{(x+1)(x-2)(x+2)}{(1+1)(1-2)(1+2)}                   \\
             & + 5 \cdot \frac{(x+1)(x-1)(x+2)}{(2+1)(2-1)(2+2)} - 11 \cdot \frac{(x+1)(x-1)(x-2)}{(-2+1)(-2-1)(-2-2)}.
    \end{align*}
    不过, 实践告诉我们, 拆开 $4$ 个 $3$ 次多项式后再相加可不是什么轻松的事儿——至少比前一个例复杂一些. 而且, 加一个节点后, $L_0 (x)$, $L_1 (x)$, $L_2 (x)$ (跟之前相比) 都要多乘一个一次多项式. 有无稍微容易一些的算法呢?
\end{example}

\begin{definition}
    设 $x_0$, $x_1$, $\cdots$, $x_{n}$ 是 $\FF$ 的 $n+1$ 个互不相同的元. 设 $y_0$, $y_1$, $\cdots$, $y_{n} \in \FF$. 定义
    \begin{align*}
        [x_i, x_j] = \frac{y_i - y_j}{x_i - x_j} \quad (i \neq j).
    \end{align*}
    这称为 $1$ 级差商 \term{first-order divided difference}. 类似地, 当 $i$, $j$, $k$ 互不相同时, $2$ 级差商是
    \begin{align*}
        [x_i, x_j, x_k] = \frac{[x_i, x_j] - [x_j, x_k]}{x_i - x_k}.
    \end{align*}
    一般地, 当 $i_0$, $i_1$, $\cdots$, $i_{\ell - 1}$ 互不相同时, $\ell - 1$ 级差商定义为
    \begin{align*}
        [x_{i_0}, x_{i_1}, \cdots, x_{i_{\ell - 1}}] = \frac{[x_{i_0}, x_{i_1}, \cdots, x_{i_{\ell - 2}}] - [x_{i_1}, x_{i_2}, \cdots, x_{i_{\ell - 1}}]}{x_0 - x_{\ell - 1}}.
    \end{align*}

    ``差商'' 可指代任意级差商. 高级差商可任意指代 $\ell$ 级差商, 此处 $\ell > 1$.
\end{definition}

\begin{example}
    取 $n = 2$. 取
    \begin{align*}
         & x_0 = -1, \quad x_1 = 1, \quad x_2 = 2, \\
         & y_0 = -1, \quad y_1 = 1, \quad y_2 = 5.
    \end{align*}
    我们随意地计算三个 $1$ 级差商:
    \begin{align*}
         & [x_0, x_1] = \frac{y_0 - y_1}{x_0 - x_1} = 1, \\
         & [x_0, x_2] = \frac{y_0 - y_2}{x_0 - x_2} = 2, \\
         & [x_1, x_2] = \frac{y_1 - y_2}{x_1 - x_2} = 4.
    \end{align*}
    由此可知
    \begin{align*}
         & [x_0, x_1, x_2] = \frac{[x_0, x_1] - [x_1, x_2]}{x_0 - x_2} = \frac{1 - 4}{-1 - 2} = 1.
    \end{align*}
    根据 $1$ 级差商的定义,
    \begin{align*}
        [x_j, x_i] = \frac{y_j - y_i}{x_j - x_i} = \frac{y_i - y_j}{x_i - x_j} = [x_i, x_j],
    \end{align*}
    故
    \begin{align*}
        [x_2, x_1] = [x_1, x_2] = 4.
    \end{align*}
    所以
    \begin{align*}
         & [x_0, x_2, x_1] = \frac{[x_0, x_2] - [x_2, x_1]}{x_0 - x_1} = \frac{2 - 4}{-1 - 1} = 1.
    \end{align*}
    同样的道理,
    \begin{align*}
        [x_1, x_0] = [x_0, x_1] = 1.
    \end{align*}
    所以
    \begin{align*}
         & [x_1, x_0, x_2] = \frac{[x_1, x_0] - [x_0, x_2]}{x_1 - x_2} = \frac{1 - 2}{1 - 2} = 1.
    \end{align*}
    我们发现, 在这些特殊的 $x_i$ 与 $y_j$ ($i,j = 0,1,2$) 下
    \begin{align*}
        [x_0, x_1, x_2] = [x_0, x_2, x_1] = [x_1, x_0, x_2].
    \end{align*}
    类似地, 读者还可以计算 $[x_1, x_2, x_0]$, $[x_2, x_0, x_1]$, $[x_2, x_1, x_0]$, 它们跟上面三个 $2$ 级差商有着同样的值. 换句话说, 我们猜想, $2$ 级差商 $[x_i, x_j, x_k]$ 的三个文字 $x_i$, $x_j$, $x_k$ 的次序可以任意交换, 且值不变 (当然, $y_i$, $y_j$, $y_k$ 的次序也要交换).
\end{example}

幸运的事儿是, 我们没猜错:

\begin{proposition}
    设 $m$ 是高于 $1$ 的整数. $m - 1$ 级差商 $[x_0, x_1, \cdots, x_{m-1}]$ 可表示为
    \begin{align*}
        [x_0, x_1, \cdots, x_{m-1}] = \sum_{k = 0}^{m-1} \frac{y_k}{N_m^{\prime} (x_k)},
    \end{align*}
    这里
    \begin{align*}
        N_m (x) = (x - x_0)(x - x_1) \cdots (x - x_{m-1}) = \prod_{k=0}^{m-1} (x - x_k).
    \end{align*}
    由此立得: 随意交换 $x_0$, $x_1$, $\cdots$, $x_{m-1}$ 的次序, 若 $y_0$, $y_1$, $\cdots$, $y_{m-1}$ 的次序也跟着改变, 得到的新 $m - 1$ 级差商的值不变.
\end{proposition}

\begin{pf}
    回想一下, $\ell$ 级差商 ($\ell > 1$) 是用 $\ell - 1$ 级差商定义的. 所以, 我们用数学归纳法证明这个结论.

    当 $m = 2$ 时,
    \begin{align*}
        N_2 (x) = (x - x_0) (x - x_1) = x^2 - (x_0 + x_1) x + x_0 x_1,
    \end{align*}
    故
    \begin{align*}
        N_2^{\prime} (x) = 2x - (x_0 + x_1).
    \end{align*}
    从而
    \begin{align*}
        N_2^{\prime} (x_0) = x_0 - x_1, \quad N_2^{\prime} (x_1) = x_1 - x_0.
    \end{align*}
    根据定义,
    \begin{align*}
        [x_0, x_1]
        = {} & \frac{y_0 - y_1}{x_0 - x_1}                                     \\
        = {} & \frac{y_0}{x_0 - x_1} - \frac{y_1}{x_0 - x_1}                   \\
        = {} & \frac{y_0}{x_0 - x_1} + \frac{y_1}{x_1 - x_0}                   \\
        = {} & \frac{y_0}{N_2^{\prime} (x_0)} + \frac{y_1}{N_2^{\prime} (x_1)} \\
        = {} & \sum_{k = 0}^{2-1} \frac{y_k}{N_2^{\prime} (x_k)}.
    \end{align*}
    所以, 结论对 $m = 2$ 成立.

    假设结论对 $m = \ell \geq 2$ 成立. 我们要由此推出: 结论对 $m = \ell + 1$ 也成立. $x_0$, $x_1$, $\cdots$, $x_{\ell}$ 这 $\ell + 1$ 个元的 $\ell$ 级差商, 按定义, 是
    \begin{align*}
        [x_{0}, x_{1}, \cdots, x_{\ell}] = \frac{[x_{0}, x_{1}, \cdots, x_{\ell - 1}] - [x_{1}, x_{2}, \cdots, x_{\ell}]}{x_0 - x_{\ell}}.
    \end{align*}
    这里, $[x_{0}, x_{1}, \cdots, x_{\ell - 1}]$ 与 $[x_{1}, x_{2}, \cdots, x_{\ell}]$ 都是 $\ell - 1$ 级差商. 按归纳假设,
    \begin{align*}
         & [x_{0}, x_{1}, \cdots, x_{\ell - 1}] = \sum_{k = 0}^{\ell-1} \frac{y_k}{P^{\prime} (x_k)}, \\
         & [x_{1}, x_{2}, \cdots, x_{\ell}] = \sum_{k = 1}^{\ell} \frac{y_k}{Q^{\prime} (x_k)},
    \end{align*}
    其中
    \begin{align*}
         & P(x) = (x - x_0) (x - x_1) \cdots (x - x_{\ell - 1}), \\
         & Q(x) = (x - x_1) (x - x_2) \cdots (x - x_{\ell}).
    \end{align*}
    作
    \begin{align*}
        N_{\ell + 1} (x) = (x - x_0) (x - x_1) \cdots (x - x_{\ell - 1}) (x - x_{\ell}),
    \end{align*}
    我们观察 $N_{\ell + 1} (x)$ 与 $P(x)$ (或 $Q(x)$) 的关系. 显然,
    \begin{align*}
        N_{\ell + 1} (x) = P(x) (x - x_{\ell}).
    \end{align*}
    二边求微商, 有
    \begin{align*}
        N_{\ell + 1}^{\prime} (x) = P^{\prime} (x) (x - x_{\ell}) + P(x).
    \end{align*}
    用 $x_u$ ($u \neq \ell$) 代替 $x$, 有
    \begin{align*}
        N_{\ell + 1}^{\prime} (x_u) = {} & P^{\prime} (x_u) (x_u - x_{\ell}) + P(x_u) = P^{\prime} (x_u) (x_u - x_{\ell})            \\
                                         & \implies \frac{1}{P^{\prime} (x_u)} = \frac{x_u - x_{\ell}}{N_{\ell + 1}^{\prime} (x_u)}.
    \end{align*}
    同理, 若 $v \neq 0$, 则
    \begin{align*}
        \frac{1}{Q^{\prime} (x_v)} = \frac{x_v - x_0}{N_{\ell + 1}^{\prime} (x_v)}.
    \end{align*}
    所以
    \begin{align*}
             & [x_{0}, x_{1}, \cdots, x_{\ell - 1}] - [x_{1}, x_{2},
        \cdots, x_{\ell}]                                            \\
        = {} & \sum_{k = 0}^{\ell-1} \frac{y_k}{P^{\prime} (x_k)}
        - \sum_{k = 1}^{\ell} \frac{y_k}{Q^{\prime} (x_k)}           \\
        = {} & \sum_{k = 0}^{\ell-1} \frac{y_k (x_k - x_{\ell})}
        {N_{\ell + 1}^{\prime} (x_k)}
        + \sum_{k = 1}^{\ell} \frac{-y_k (x_k - x_0)}
        {N_{\ell + 1}^{\prime} (x_k)}                                \\
        = {} & \sum_{k = 0}^{\ell} \frac{y_k (x_k - x_{\ell})}
        {N_{\ell + 1}^{\prime} (x_k)}
        + \sum_{k = 0}^{\ell} \frac{y_k (x_0 - x_k)}
        {N_{\ell + 1}^{\prime} (x_k)}                                \\
        = {} & \sum_{k = 0}^{\ell} \frac{y_k (x_k - x_{\ell})
            + y_k (x_0 - x_k)}
        {N_{\ell + 1}^{\prime} (x_k)}                                \\
        = {} & \sum_{k = 0}^{\ell} \frac{y_k (x_0 - x_{\ell})}
        {N_{\ell + 1}^{\prime} (x_k)}                                \\
        = {} & (x_0 - x_{\ell}) \sum_{k = 0}^{\ell} \frac{y_k}
        {N_{\ell + 1}^{\prime} (x_k)}.
    \end{align*}
    这样
    \begin{align*}
        [x_{0}, x_{1}, \cdots, x_{\ell}]
        = {} & \frac{[x_{0}, x_{1}, \cdots, x_{\ell - 1}]
        - [x_{1}, x_{2}, \cdots, x_{\ell}]}{x_0 - x_{\ell}}          \\
        = {} & \frac{1}{x_0 - x_{\ell}} \cdot (x_0 - x_{\ell})
        \sum_{k = 0}^{\ell} \frac{y_k} {N_{\ell + 1}^{\prime} (x_k)} \\
        = {} & \sum_{k = 0}^{(\ell + 1) - 1} \frac{y_k}
        {N_{\ell + 1}^{\prime} (x_k)}. \qedhere
    \end{align*}
\end{pf}

\begin{remark}
    前面, 我们知道, 用 Lagrange 插值公式算出的次不高于 $n$ 的多项式的 $n$ 次系数是
    \begin{align*}
        \sum_{i = 0}^{n} \frac{y_i}{N_{n+1}^{\prime} (x_i)},
    \end{align*}
    其中
    \begin{align*}
        N_{n+1} (x) = (x - x_0) (x - x_1) \cdots (x - x_n).
    \end{align*}
    用差商的语言, 有: $f(x)$ 的 $n$ 次系数可用 $n$ 级差商
    \begin{align*}
        [x_0, x_1, \cdots, x_n]
    \end{align*}
    表示.
\end{remark}

现在, 我们来看看差商在多项式插值里的用处. 设 $x_0$, $x_1$, $\cdots$, $x_{n}$ 是 $\FF$ 的 $n+1$ 个互不相同的元. 设 $y_0$, $y_1$, $\cdots$, $y_{n} \in \FF$. 作 $n+1$ 个多项式:
\begin{align*}
     & N_0 (x) = 1,                                          \\
     & N_1 (x) = x - x_0,                                    \\
     & N_2 (x) = (x - x_0) (x - x_1),                        \\
     & \cdots \cdots \cdots \cdots \cdots \cdots \cdots
    \cdots \cdots \cdots \cdots \cdots \cdots \cdots,        \\
     & N_{n} (x) = (x - x_0) (x - x_1) \cdots (x - x_{n-1}).
\end{align*}

因为 $N_0 (x)$, $N_1 (x)$, $\cdots$, $N_n (x)$ 的次分别是 $0$, $1$, $\cdots$, $n$, 所以:

(i) $N_0 (x)$, $N_1 (x)$, $\cdots$, $N_n (x)$ 是线性无关的;

(ii) 任意次不高于 $n$ 的多项式都可唯一地写为 $N_0 (x)$, $N_1 (x)$, $\cdots$, $N_n (x)$ 的线性组合.

由前面的 Lagrange 插值公式可知, 存在一个次不高于 $n$ 的多项式 $f(x)$ 使
\begin{align*}
    f(x_i) = y_i \quad (i = 0,1,\cdots,n).
\end{align*}
对这个 $f(x)$ 而言, 存在 (唯一的) $c_0$, $c_1$, $\cdots$, $c_n \in \FF$ 使
\begin{align*}
    f(x) = \sum_{i = 0}^{n} c_i N_{n} (x).
\end{align*}
我们的任务就是找出 $c_0$, $c_1$, $\cdots$, $c_n$. 先从 $c_n$ 看起. 显然, 左侧的 $n$ 次系数是 $[x_0, x_1, \cdots, x_n]$, 而右侧的 $n$ 次系数是 $c_n$, 故
\begin{align*}
    c_n = [x_0, x_1, \cdots, x_n].
\end{align*}
找出 $c_n$, 还有 $n$ 个系数要找呢! 接下来的系数该怎么找呢?

\begin{proposition}
    设 $x_0$, $x_1$, $\cdots$, $x_{n}$ 是 $\FF$ 的 $n+1$ 个互不相同的元 ($n \geq 1$). 设 $y_0$, $y_1$, $\cdots$, $y_{n} \in \FF$. 作 $n+1$ 个多项式:
    \begin{align*}
         & N_0 (x) = 1,                                          \\
         & N_1 (x) = x - x_0,                                    \\
         & N_2 (x) = (x - x_0) (x - x_1),                        \\
         & \cdots \cdots \cdots \cdots \cdots \cdots \cdots
        \cdots \cdots \cdots \cdots \cdots \cdots \cdots,        \\
         & N_{n} (x) = (x - x_0) (x - x_1) \cdots (x - x_{n-1}).
    \end{align*}
    由 Lagrange 插值公式可知, 存在一个次不高于 $n$ 的多项式 $f(x)$ 使
    \begin{align*}
        f(x_i) = y_i \quad (i = 0,1,\cdots,n).
    \end{align*}
    对这个 $f(x)$ 而言, 存在 (唯一的) $c_0$, $c_1$, $\cdots$, $c_n \in \FF$ 使
    \begin{align*}
        f(x) = \sum_{i = 0}^{n} c_i N_{n} (x).
    \end{align*}
    这些系数有着简单的形式:
    \begin{align*}
         & c_0 = y_0,                                              \\
         & c_i = [x_0, x_1, \cdots, x_i] \quad (i = 1,2,\cdots,n).
    \end{align*}
\end{proposition}

\begin{pf}
    用数学归纳法. 当 $n=1$ 时,
    \begin{align*}
        f(x)
        = {} & y_0 \frac{x - x_1}{x_0 - x_1}
        + y_1 \frac{x - x_0}{x_1 - x_0}                      \\
        = {} & y_0 \frac{(x - x_0) + (x_0 - x_1)}{x_0 - x_1}
        - y_1 \frac{x - x_0}{x_0 - x_1}                      \\
        = {} & y_0 + y_0 \frac{x - x_0}{x_0 - x_1}
        - y_1 \frac{x - x_0}{x_0 - x_1}                      \\
        = {} & y_0 + \frac{y_0 - y_1}{x_0 - x_1} (x - x_0)   \\
        = {} & y_0 N_0 (x) + [x_0, x_1] N_1 (x).
    \end{align*}
    这样, 结论对 $n=1$ 成立.

    设结论对 $n = \ell \geq 1$ 成立. 我们看 $n = \ell + 1$ 的情形.

    由 Lagrange 插值公式可知, 存在一个次不高于 $\ell + 1$ 的多项式 $f(x)$ 使
    \begin{align*}
        f(x_i) = y_i \quad (i = 0,1,\cdots,\ell + 1).
    \end{align*}
    对这个 $f(x)$ 而言, 存在 (唯一的) $c_0$, $c_1$, $\cdots$, $c_{\ell}, c_{\ell + 1} \in \FF$ 使
    \begin{align*}
        f(x) = \sum_{i = 0}^{\ell} c_i N_{i} (x) + c_{\ell + 1} N_{\ell + 1} (x).
    \end{align*}
    左侧的 $\ell + 1$ 次系数是 $[x_0, x_1, \cdots, x_\ell, x_{\ell + 1}]$, 右侧的 $\ell + 1$ 次系数是 $c_{\ell + 1}$, 故
    \begin{align*}
        c_{\ell + 1} = [x_0, x_1, \cdots, x_\ell, x_{\ell + 1}].
    \end{align*}
    作
    \begin{align*}
        g(x) = f(x) - [x_0, x_1, \cdots, x_\ell, x_{\ell + 1}] N_{\ell + 1} (x).
    \end{align*}
    则
    \begin{align*}
        g(x) = \sum_{i = 0}^{\ell} c_i N_{i} (x),
    \end{align*}
    且 $i \neq \ell + 1$ 时,
    \begin{align*}
        g(x_i) = f(x_i) - [x_0, x_1, \cdots, x_\ell, x_{\ell + 1}] 0 = y_i.
    \end{align*}
    这个 $g(x)$ 的次不会高于 $\ell$. 并且, $i = 0$, $1$, $\cdots$, $\ell$ 时, $g(x_i) = y_i$.

    由 Lagrange 插值公式, 存在一个次不高于 $\ell$ 的多项式 $h(x)$ 使
    \begin{align*}
        h(x_i) = y_i \quad (i = 0,1,\cdots,\ell).
    \end{align*}
    对这个 $h(x)$ 而言, 存在 (唯一的) $d_0$, $d_1$, $\cdots$, $d_{\ell} \in \FF$ 使
    \begin{align*}
        h(x) = \sum_{i = 0}^{\ell} d_i N_{i} (x).
    \end{align*}
    根据归纳假设,
    \begin{align*}
         & d_0 = y_0,                                                 \\
         & d_i = [x_0, x_1, \cdots, x_i] \quad (i = 1,2,\cdots,\ell).
    \end{align*}
    由插值的唯一性, $g(x) = h(x)$. 所以
    \begin{align*}
         & c_0 = d_0 = y_0,                                                 \\
         & c_i = d_i = [x_0, x_1, \cdots, x_i] \quad (i = 1,2,\cdots,\ell).
    \end{align*}

    所以, $n = \ell + 1$ 时, 结论是正确的.
\end{pf}

为方便, 记 $[x_i] = y_i$, 称其为 $x_i$ 的 $0$ 级差商. 我们证明了

\begin{proposition}
    设 $x_0$, $x_1$, $\cdots$, $x_{n}$ 是 $\FF$ 的 $n+1$ 个互不相同的元. 设 $y_0$, $y_1$, $\cdots$, $y_{n} \in \FF$. 存在唯一的多项式
    \begin{align*}
        f(x)
        = {} & \sum_{i = 0}^{n} [x_0, x_1, \cdots, x_i]
        \prod_{j = 0}^{i - 1} (x - x_j)                                                \\
        = {} & [x_0] + [x_0, x_1] (x - x_0) + \cdots + [x_0, x_1, \cdots, x_n] (x-x_0) \\
             & \qquad \qquad \cdot (x-x_1)\cdots (x-x_{n-1})
    \end{align*}
    适合条件
    \begin{align*}
        f(x_i) = y_i \quad (i = 0,1,\cdots,n),
    \end{align*}
    且
    \begin{align*}
        \deg f(x) < n + 1.
    \end{align*}
    这个公式的一个名字是 ``Newton 插值公式'' \term{Newton's interpolation formula}.
\end{proposition}

我们举三个具体的例帮读者消化这个 Newton 插值公式.

\begin{example}
    求次不高于 $1$ 的多项式 $f(x)$, 使 $f(-1)=2$, $f(1)=3$.

    这里, $n = 1$, 且
    \begin{align*}
         & x_0 = -1, \quad x_1 = 1, \\
         & y_0 = 2, \quad y_1 = 3.
    \end{align*}
    不难算出
    \begin{align*}
         & [x_0] = y_0 = 2,                                    \\
         & [x_0, x_1] = \frac{y_0 - y_1}{x_0 - x_1} = \frac12.
    \end{align*}
    所以
    \begin{align*}
        f(x) = 2 + \frac12 (x - (-1)) = \frac12 x + \frac52.
    \end{align*}
\end{example}

\begin{example}
    求次不高于 $2$ 的多项式 $f(x)$, 使 $f(-1)=-1$, $f(1)=1$, $f(2)=5$.

    这里, $n = 2$, 且
    \begin{align*}
         & x_0 = -1, \quad x_1 = 1, \quad x_2 = 2, \\
         & y_0 = -1, \quad y_1 = 1, \quad y_2 = 5.
    \end{align*}
    不难算出
    \begin{align*}
         & [x_0] = y_0 = -1,                                \\
         & [x_0, x_1] = \frac{y_0 - y_1}{x_0 - x_1}
        = 1,                                                \\
         & [x_1, x_2] = \frac{y_1 - y_2}{x_1 - x_2}
        = 4,                                                \\
         & [x_0, x_1, x_2] = \frac{[x_0, x_1] - [x_1, x_2]}
        {x_0 - x_2} = 1.
    \end{align*}
    所以
    \begin{align*}
        f(x) = -1 + (x + 1) + (x + 1)(x - 1) = x^2 + x - 1.
    \end{align*}
    前面, 我们用 Lagrange 插值公式, 得到了一样的结果, 不过计算过程稍繁.

    实操时, 往往用名为 ``差商表'' 的表进行计算. 当 $n = 2$ 时, 它长这样:
    \begin{align*}
        \arraycolsep=0.25cm
        \begin{array}{c|ccc}
            x_2 & [x_2] & \          & \               \\
            x_1 & [x_1] & [x_1, x_2] & \               \\
            x_0 & [x_0] & [x_0, x_1] & [x_0, x_1, x_2] \\
        \end{array}
    \end{align*}
    在这个问题里, 差商表如下:
    \begin{align*}
        \arraycolsep=0.25cm
        \begin{array}{c|ccc}
            2  & 5  & \  & \ \\
            1  & 1  & 4  & \ \\
            -1 & -1 & 1  & 1 \\
        \end{array}
    \end{align*}
\end{example}

\begin{example}
    求次不高于 $3$ 的多项式 $f(x)$, 使 $f(-1)=-1$, $f(1)=1$, $f(2)=5$, $f(-2) = -11$.

    这里, $n = 3$, 且
    \begin{align*}
         & x_0 = -1, \quad x_1 = 1, \quad x_2 = 2, \quad x_3 = -2   \\
         & y_0 = -1, \quad y_1 = 1, \quad y_2 = 5, \quad y_3 = -11.
    \end{align*}

    画出 $n = 3$ 时的差商表:
    \begin{align*}
        \arraycolsep=0.25cm
        \begin{array}{c|cccc}
            \textcolor{red}{x_3} & \textcolor{red}{[x_3]} & \                           & \                                & \                                     \\
            x_2                  & [x_2]                  & \textcolor{red}{[x_2, x_3]} & \                                & \                                     \\
            x_1                  & [x_1]                  & [x_1, x_2]                  & \textcolor{red}{[x_1, x_2, x_3]} & \                                     \\
            x_0                  & [x_0]                  & [x_0, x_1]                  & [x_0, x_1, x_2]                  & \textcolor{red}{[x_0, x_1, x_2, x_3]} \\
        \end{array}
    \end{align*}
    我们已经在上个例里算出了 $[x_0, x_1]$, $[x_1, x_2]$, $[x_0, x_1, x_2]$:
    \begin{align*}
        \arraycolsep=0.25cm
        \begin{array}{c|cccc}
            \textcolor{red}{x_3} & \textcolor{red}{[x_3]} & \                           & \                                & \                                     \\
            2                    & 1                      & \textcolor{red}{[x_2, x_3]} & \                                & \                                     \\
            1                    & 1                      & 4                           & \textcolor{red}{[x_1, x_2, x_3]} & \                                     \\
            -1                   & -1                     & 1                           & 1                                & \textcolor{red}{[x_0, x_1, x_2, x_3]} \\
        \end{array}
    \end{align*}
    我们的目标是算出 $[x_0, x_1, x_2, x_3]$. 所以, 我们要算出 $[x_1, x_2, x_3]$; 所以, 我们要算出 $[x_2, x_3]$; 所以, 我们要算出 $[x_3]$. 不过, $[x_3]$ 是已知的, 它就是 $y_3$, 也就是 $-11$.

    列出算式:
    \begin{align*}
         & x_3 = -2,                                                                       \\
         & [x_3] = y_3 = -11,                                                              \\
         & [x_2, x_3] = \frac{y_2 - y_3}{x_2 - x_3} = 4,                                   \\
         & [x_1, x_2, x_3] = \frac{[x_1, x_2] - [x_2, x_3]}{x_1 - x_3} = 0,                \\
         & [x_0, x_1, x_2, x_3] = \frac{[x_0, x_1, x_2] - [x_1, x_2, x_3]}{x_0 - x_3} = 1.
    \end{align*}
    此时, 差商表如下:
    \begin{align*}
        \arraycolsep=0.25cm
        \begin{array}{c|cccc}
            \textcolor{red}{-2} & \textcolor{red}{-11} & \                  & \                  & \                  \\
            2                   & 1                    & \textcolor{red}{4} & \                  & \                  \\
            1                   & 1                    & 4                  & \textcolor{red}{0} & \                  \\
            -1                  & -1                   & 1                  & 1                  & \textcolor{red}{1} \\
        \end{array}
    \end{align*}
    所以
    \begin{align*}
        f(x)
        = {} & -1 + (x + 1) + (x + 1)(x - 1) + (x + 1)(x - 1)(x - 2) \\
        = {} & (x^2 + x - 1) + (x^3 - 2x^2 - x + 2)                  \\
        = {} & x^3 - x^2 + 1.
    \end{align*}

    用 Lagrange 插值公式, 有
    \begin{align*}
        f(x)
        = {} & - \frac{(x-1)(x-2)(x+2)}{(-1-1)(-1-2)(-1+2)} + \frac{(x+1)(x-2)(x+2)}{(1+1)(1-2)(1+2)}                   \\
             & + 5 \cdot \frac{(x+1)(x-1)(x+2)}{(2+1)(2-1)(2+2)} - 11 \cdot \frac{(x+1)(x-1)(x-2)}{(-2+1)(-2-1)(-2-2)}.
    \end{align*}
    有兴趣的读者可展开上式, 以验证我们的计算是否正确. 由此可见, Newton 插值公式在实操上优于 Lagrange 插值公式.
\end{example}

我们以带余除法与插值的关系结束本节.

\begin{proposition}
    设 $x_0$, $x_1$, $\cdots$, $x_{n}$ 是 $\FF$ 的 $n+1$ 个互不相同的元. 设
    \begin{align*}
        d(x) = (x - x_0) (x - x_1) \cdots (x - x_n) \in \FF[x]
    \end{align*}
    是 $n+1$ 次多项式. 由带余除法知, 任取 $f(x) \in \FF[x]$, 存在唯一的 $q(x)$, $r(x) \in \FF[x]$ 使
    \begin{align*}
        f(x) = q(x) d(x) + r(x), \quad \deg r(x) < n + 1.
    \end{align*}
    余式 $r(x)$ 可具体地写出:
    \begin{align*}
        r(x) = \sum_{i = 0}^{n} f(x_i) \prod_{\begin{smallmatrix}0 \leq \ell \leq n \\\ell \neq i\end{smallmatrix}} \frac{x - x_\ell}{x_i - x_\ell}
    \end{align*}
    或
    \begin{align*}
        r(x) = \sum_{i = 0}^{n} [x_0, x_1, \cdots, x_i] \prod_{j = 0}^{i - 1} (x - x_j),
    \end{align*}
    其中差商的 $y_i$ 取 $f(x_i)$, $i = 0$, $1$, $\cdots$, $n$.
\end{proposition}

\begin{pf}
    由带余除法知, 任取 $f(x) \in \FF[x]$, 存在唯一的 $q(x)$, $r(x) \in \FF[x]$ 使
    \begin{align*}
        f(x) = q(x) d(x) + r(x), \quad \deg r(x) < n + 1.
    \end{align*}
    用 $x_i$ 代替 $x$, 有
    \begin{align*}
        f(x_i) = q(x_i) d(x_i) + r(x_i) = r(x_i).
    \end{align*}
    因为 $\deg r(x) < n + 1$, 故由插值公式即得待证命题.
\end{pf}
