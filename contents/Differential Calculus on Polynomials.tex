\subsection*{Differential Calculus on Polynomials}
\addcontentsline{toc}{subsection}{Differential Calculus on Polynomials}
\markright{Differential Calculus on Polynomials}

本节讨论多项式的微分学 \term{differential calculus}\period 这也是本文的最终节\period

How time flies! 一开始, 我们在 Prerequisites 给读者介绍预备知识\period 然后, 我们给读者介绍了系数为整环的元的多项式\period 当初, 多项式还是有点抽象的\period 我们利用带余除法推出了几个很重要的命题, 并指出: 当多项式的系数为 $\FF$ 的元时, 多项式与中学学的多项式 (函数) 没有根本上的区别\period 我们在 Interpolation 节开始介绍多项式的应用\period 后面的 Summation Formulae 节告诉读者一种方便的求和法\period 在上节, 我们捡起很久未出场的导数, 并把它重讲了一遍\period 我们利用高级导数推广了二项展开, 得到了 Taylor 公式, 并用它重证多项式的积的导数规则\period

现在, 我们要用 Taylor 公式给读者讲述多项式的微分学\period 如果您知道一点微积分 \term{calculus}, 您将不会对本节感到特别陌生; 如果您没有学过微积分, 无妨将本节作为 ``入微作''\period
