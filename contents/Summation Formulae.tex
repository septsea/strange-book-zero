\subsection*{\SummationFormulae}
\addcontentsline{toc}{subsection}{\SummationFormulae}
\markright{\SummationFormulae}

本节讨论求和公式 \term{summation formula} 问题: 设 $f(x) \in \FF[x]$, 求
\begin{align*}
    S(n) = \sum_{\ell = 0}^{n - 1} f(\ell) = f(0) + f(1) + \cdots + f(n - 1).
\end{align*}

\begin{example}
    相信大家应该听说过德意志算学家 Carl Friedrich Gauß. 1787 年, Gauß 还只是一个 10 岁的孩子. 据说, 当时他的算学教师给全班同学出了这样的算术题:
    \begin{align*}
        1 + 2 + 3 + \cdots + 100 = ?
    \end{align*}
    这里, 后一个数比前一个数多 $1$, 且共有 $100$ 个数. 教师刚写完问题, Gauß 就算出, 答案是 $5\,050$. 他的同学还在一个一个地加, 算了很久, 还没算对.

    Gauß 是怎么快速算出答案的呢? 设
    \begin{align*}
        S = 1 + 2 + 3 + \cdots + 100.
    \end{align*}
    因为加法适合交换律, 故
    \begin{align*}
        S = 100 + 99 + 98 + \cdots + 1.
    \end{align*}
    所以
    \begin{align*}
        2S
        = {} & (1 + 100) + (2 + 99) + (3 + 98) + \cdots + (100 + 1) \\
        = {} & \underbrace{101 + 101 + 101 + \cdots + 101}
        _{\text{a hundred $101$'s}}                                 \\
        = {} & 100 \cdot 101                                        \\
        = {} & 10\,100.
    \end{align*}
    由此可得
    \begin{align*}
        S = \frac{10\,100}{2} = 5\,050.
    \end{align*}

    如果记 $f(x) = x + 1$, 则
    \begin{align*}
        S
        = {} & 1 + 2 + 3 + \cdots + 100                 \\
        = {} & f(0) + f(1) + f(2) + \cdots + f(100 - 1) \\
        = {} & \sum_{\ell = 0}^{100 - 1} f(\ell).
    \end{align*}

    考虑更一般的情形. 设 $f(x) = a + bx$. 记
    \begin{align*}
        S(n) = f(0) + f(1) + \cdots + f(n-1).
    \end{align*}
    类似地, 把右侧倒着写:
    \begin{align*}
        S(n) = f(n-1) + f(n-1) + \cdots + f(0).
    \end{align*}
    因为
    \begin{align*}
        f(k) + f(n-1-k) = a + bk + a + b(n-1-k) = 2a + b(n - 1),
    \end{align*}
    故
    \begin{align*}
             & 2S(n)                                                        \\
        = {} & (f(0) + f(n-1)) + (f(1) + f(n-2)) + \cdots + (f(n-1) + f(0)) \\
        = {} & n(2a + b(n - 1)),
    \end{align*}
    即
    \begin{align*}
        S(n) = \frac{n(2a + b(n - 1))}{2} = \left(a - \frac{b}{2}\right) n + \frac{b}{2} n^2.
    \end{align*}
    我们还可以看出: $S(n)$ 是多项式, 且
    \begin{align*}
        \deg S(n) = \deg f(n) + 1.
    \end{align*}
\end{example}

上面讨论了当 $f(x)$ 的次不高于 $1$ 时如何求 $S(n)$. 那么, 当 $f(x)$ 的次高于 $1$ 时, 怎么找 $S(n)$? 它还是多项式吗?

在求和前, 我们看 $S(n)$ 适合什么性质. $S(n)$ 是 $f(0)$, $f(1)$, $\cdots$, $f(n-1)$ 这 $n$ 个数的和. 因为 $0$ 个数的和是 $0$, 故 $S(0) = 0$. 同时, 不难看出, $S(n+1)$ 比 $S(n)$ 多出 $f(n)$, 也即
\begin{align*}
    S(n+1) - S(n) = f(n).
\end{align*}
反过来, 设 $\NN$ 到 $\FF$ 的函数 $W(n)$ 适合 $W(0) = 0$ 与 $W(n+1) - W(n) = f(n)$, 则
\begin{align*}
    \sum_{\ell = 0}^{n - 1} f(\ell)
    = {} & \sum_{\ell = 0}^{n - 1} (W(\ell+1) - W(\ell))                       \\
    = {} & \sum_{\ell = 0}^{n - 1} W(\ell+1) - \sum_{\ell = 0}^{n - 1} W(\ell) \\
    = {} & \sum_{\ell = 1}^{n} W(\ell) - \sum_{\ell = 0}^{n - 1} W(\ell)       \\
    = {} & W(n) - W(0)                                                         \\
    = {} & W(n).
\end{align*}
这样, 任给 $f(x) \in \FF[x]$, 若我们能找到适合条件 $S(0) = 0$ 与 $S(x+1) - S(x) = f(x)$ 的多项式, 则
\begin{align*}
    \sum_{\ell = 0}^{n - 1} f(\ell) = S(n).
\end{align*}

\begin{proposition}
    设 $f(x) \in \FF[x]$ 是 $m$ 次多项式. 存在唯一的 $m+1$ 次多项式 $F(x) \in \FF[x]$ 适合条件:

    (i) $F(0) = 0$;

    (ii) $F(x + 1) - F(x) = f(x)$.
\end{proposition}

\begin{pf}
    先看存在性. 若 $f(x) = 0$, 则 $F(x) = 0$ 显然适合 (i) (ii), 且
    \begin{align*}
        \deg F(x) = -\infty = -\infty + 1 = \deg f(x) + 1.
    \end{align*}
    设 $m \geq 0$. 根据广义二项系数的性质, 存在 $m+1$ 个 $\FF$ 中元 $c_0$, $\cdots$, $c_{m}$ 使
    \begin{align*}
        f(x) = \sum_{\ell = 0}^{m} c_\ell \binom{x}{\ell}, \quad c_{m} \neq 0.
    \end{align*}
    (读者可思考: 若 $c_{m} = 0$, $f(x)$ 还能是 $m$ 次多项式吗?) 作多项式
    \begin{align*}
        F(x) = \sum_{\ell = 0}^{m} c_\ell \binom{x}{\ell+1} \in \FF[x].
    \end{align*}
    显然 $\deg F(x) = m + 1$. 验证 (i):
    \begin{align*}
        F(0) = \sum_{\ell = 0}^{m} c_\ell \binom{0}{\ell+1} = \sum_{\ell = 0}^{m} 0 = 0.
    \end{align*}
    验证 (ii):
    \begin{align*}
        F(x + 1) - F(x)
        = {} & \sum_{\ell = 0}^{m} c_\ell \binom{x+1}{\ell+1} - \sum_{\ell = 0}^{m} c_\ell \binom{x}{\ell+1} \\
        = {} & \sum_{\ell = 0}^{m} c_\ell \left( \binom{x+1}{\ell+1} - \binom{x}{\ell+1} \right)             \\
        = {} & \sum_{\ell = 0}^{m} c_\ell \binom{x}{\ell}                                                    \\
        = {} & f(x).
    \end{align*}

    再看唯一性. 设 $G(x) \in \FF[x]$ 是 $m+1$ 次多项式, 并适合条件 $G(0) = 0$ 与 $G(x + 1) - G(x) = f(x)$. 作
    \begin{align*}
        H(x) = F(x) - G(x).
    \end{align*}
    则 $H(0) = 0$, $H(x + 1) - H(x) = 0$. 所以, $r$ 为非负整数时, $H(r) = 0$. 从而 $H(x)$ 一定是零多项式, 即 $F(x) = G(x)$.
\end{pf}

\begin{example}
    记 $f(x) = x^2$. 我们求
    \begin{align*}
        S(n) = f(0) + f(1) + \cdots + f(n-1) = \sum_{\ell = 0}^{n - 1} f(\ell).
    \end{align*}
    由上个命题可知, 存在唯一的次为 $3$ 的多项式 $F(x)$ 使 $F(0) = 0$, $F(x + 1) - F(x) = f(x)$, 且 $S(n) = F(n)$.

    可以用插值的思想求 $F(x)$. 取 $x_0$, $x_1$, $x_2$, $x_3$ 为 $0$, $1$, $-1$, $2$. 不难算出:
    \begin{align*}
         & y_{0} = F(0) = 0,                  \\
         & y_{1} = F(1) = F(0) + f(0) = 0,    \\
         & y_{2} = F(-1) = F(0) - f(-1) = -1, \\
         & y_{3} = F(2) = F(1) + f(1) = 1.
    \end{align*}
    注意到 $y_0 = y_1 = 0$, 故可以考虑 Lagrange 插值 (只要算 $L_2 (x)$ 与 $L_3 (x)$):
    \begin{align*}
         & L_2 (x) = \frac{(x - 0)(x - 1)(x - 2)}{(-1 - 0)(-1 - 1)(-1 - 2)} = -\frac{x(x-1)(x-2)}{6}, \\
         & L_3 (x) = \frac{(x - 0)(x - 1)(x + 1)}{(2 - 0)(2 - 1)(2 + 1)} = \frac{x(x-1)(x+1)}{6},     \\
         & F(x) = y_2 L_2 (x) + y_3 L_3 (x) = \frac{x(x-1)(2x-1)}{6}.
    \end{align*}
    当然, 也可利用 Newton 插值. 作出差商表:
    \begin{align*}
        \arraycolsep=0.25cm
        \begin{array}{c|cccc}
            2  & 1  & \       & \        & \       \\
            -1 & -1 & \frac23 & \        & \       \\
            1  & 0  & \frac12 & \frac16  & \       \\
            0  & 0  & 0       & -\frac12 & \frac13 \\
        \end{array}
    \end{align*}
    故
    \begin{align*}
        F(x)
        = {} & [0] + [0,1] (x - 0) + [0,1,-1] (x - 0)(x - 1)    \\
             & \qquad \qquad + [0,1,-1,2] (x - 0)(x - 1)(x + 1) \\
        = {} & -\frac{1}{2} x(x-1) + \frac{1}{3} x(x-1)(x+1)    \\
        = {} & \frac{x(x-1)(2x-1)}{6}.
    \end{align*}

    综上, 我们有
    \begin{align*}
        \sum_{\ell = 0}^{n - 1} \ell^2 = 0^2 + 1^2 + \cdots + (n-1)^2 = \frac{n(n-1)(2n-1)}{6}.
    \end{align*}
\end{example}

其实, 我们可以在此处结束本节. 设 $f(x)$ 是 $n$ 次多项式. 上面的命题告诉我们, 存在唯一的 $n+1$ 次多项式 $F(x)$ 使 $F(0) = 0$, $F(x + 1) - F(x) = f(x)$, 且 $S(n) = F(n)$. 利用这些条件, 可以确定 $F(x)$ 在 $n+2$ 个整数点处的值, 从而可用插值公式求出 $F(x)$. 不过, 为了使实操容易一些, 我们还得多研究一点.

由上个命题的证明过程, 有

\begin{proposition}
    若
    \begin{align*}
        f(x) = c_0 \binom{x}{0} + c_1 \binom{x}{1} + \cdots + c_m \binom{x}{m},
    \end{align*}
    则
    \begin{align*}
        S(n) = \sum_{\ell = 0}^{n - 1} f(\ell) = c_0 \binom{n}{1} + c_1 \binom{n}{2} + \cdots + c_m \binom{n}{m + 1}.
    \end{align*}
\end{proposition}

由此可见, 若我们能把 $f(x)$ 写为广义二项系数的线性组合, 则寻找 $S(n)$ 的过程将十分简单. 接下来, 我们讨论怎么方便地把多项式写为广义二项系数的线性组合.

\begin{definition}
    设 $f(x) \in \FF[x]$. 定义 $f(x)$ 的差分 \term{difference} 为
    \begin{align*}
        \Delta f(x) = f(x+1) - f(x) \in \FF[x].
    \end{align*}

    设 $t \in \FF$. 我们把
    \begin{align*}
        f(t+1) - f(t) \in \FF
    \end{align*}
    也写为 $\Delta f(t)$.
\end{definition}

\begin{example}
    取 $f(x) = x^2 + x - 1$. 则
    \begin{align*}
        f(x + 1) = (x + 1)^2 + (x + 1) - 1 = x^2 + 3x + 1,
    \end{align*}
    故
    \begin{align*}
        \Delta f(x) = 2x + 2.
    \end{align*}
    所以
    \begin{align*}
        \Delta f(332) = 2 \cdot 332 + 2 = 666.
    \end{align*}
\end{example}

\begin{proposition}
    设 $k$ 是整数. 则
    \begin{align*}
        \Delta \binom{x}{k} = \binom{x}{k - 1}.
    \end{align*}
\end{proposition}

\begin{pf}
    也许, 这就是所谓的 ``新瓶装旧酒'' 吧! 不过, 为了方便, 我们还是单独列出来.
\end{pf}

回忆一下, 微商适合如下二条性质:

(i) $(cf(x))^{\prime} = c f^{\prime} (x)$;

(ii) $(f(x) \pm g(x))^{\prime} = f^{\prime} (x) \pm g^{\prime} (x)$.

差分也有类似的性质.

\begin{proposition}
    设 $f(x)$, $g(x) \in \FF[x]$, $c \in \FF$. 则

    (i) $\Delta (cf(x)) = c \Delta f(x)$;

    (ii) $\Delta (f(x) \pm g(x)) = \Delta f(x) \pm \Delta g(x)$.

    由 (i) (ii) 与算学归纳法可知: 当 $c_0$, $c_1$, $\cdots$, $c_{k-1} \in \FF$, 且 $f_0 (x)$, $f_1 (x)$, $\cdots$, $f_{k-1} (x) \in \FF[x]$ 时,
    \begin{align*}
        \Delta \left( \sum_{\ell = 0}^{k-1} c_\ell f_\ell (x) \right)
        = \sum_{\ell = 0}^{k-1} c_\ell \Delta f_\ell (x).
    \end{align*}
\end{proposition}

\begin{pf}
    老样子, 我们证明 (i) (ii), 将剩下的推论留给读者作练习. 设

    (i) 设 $p(x) = cf(x)$. 则
    \begin{align*}
        \Delta (cf(x))
        = {} & \Delta p(x)      \\
        = {} & p(x+1) - p(x)    \\
        = {} & cf(x+1) - cf(x)  \\
        = {} & c(f(x+1) - f(x)) \\
        = {} & c \Delta f(x).
    \end{align*}

    (ii) 设 $q(x) = f(x) \pm g(x)$. 则
    \begin{align*}
        \Delta (f(x) \pm g(x))
        = {} & \Delta q(x)                           \\
        = {} & q(x+1) - q(x)                         \\
        = {} & (f(x+1) \pm g(x+1)) - (f(x) \pm g(x)) \\
        = {} & (f(x+1) - f(x)) \pm (g(x+1) - g(x))   \\
        = {} & \Delta f(x) \pm \Delta g(x). \qedhere
    \end{align*}
\end{pf}

\begin{definition}
    设 $f(x) \in \FF[x]$. 记
    \begin{align*}
        \Delta^0 f(x) = f(x) \in \FF[x],
    \end{align*}
    并称其为 $f(x)$ 的 $0$ 级差分 \term{zeroth-order difference}. $1$ 级差分就是差分:
    \begin{align*}
        \Delta^1 f(x) = \Delta f(x) = \Delta (\Delta^0 f(x)) \in \FF[x].
    \end{align*}
    $1$ 级差分的差分是 $2$ 级差分:
    \begin{align*}
        \Delta^2 f(x) = \Delta (\Delta^1 f(x)) \in \FF[x].
    \end{align*}
    $2$ 级差分的差分是 $3$ 级差分:
    \begin{align*}
        \Delta^3 f(x) = \Delta (\Delta^2 f(x)) \in \FF[x].
    \end{align*}
    一般地, $e$ 级差分就是 $e - 1$ 级差分的差分:
    \begin{align*}
        \Delta^e f(x) = \Delta (\Delta^{e-1} f(x)) \in \FF[x].
    \end{align*}
    高级差分可指代任意 $e$ 级差分, 此处 $e > 1$.

    设 $t \in \FF$. 既然 $\Delta^e f(x)$ 是某个多项式
    \begin{align*}
        v_0 + v_1 x + \cdots + v_s x^s \in \FF[x],
    \end{align*}
    我们将
    \begin{align*}
        v_0 + v_1 t + \cdots + v_s t^s \in \FF
    \end{align*}
    简单地写为 $\Delta^e f(t)$.
\end{definition}

\begin{example}
    设
    \begin{align*}
        f(x) = 2x^3 + 3x^2 + 5x + 7.
    \end{align*}
    根据定义, $f(x)$ 的 $0$ 级差分就是自己:
    \begin{align*}
        \Delta^0 f(x) = 2x^3 + 3x^2 + 5x + 7.
    \end{align*}
    因为
    \begin{align*}
        (1+x)^3
        = {} & (1+x)^2 (1+x)         \\
        = {} & (1+2x+x^2) (1+x)      \\
        = {} & 1+2x+x^2 + x+2x^2+x^3 \\
        = {} & 1+3x+3x^2+x^3,
    \end{align*}
    故
    \begin{align*}
        f(x+1)
        = {} & 2(x+1)^3 + 3(x+1)^2 + 5(x+1) + 7 \\
        = {} & 2x^3 + 9x^2 + 17x + 17.
    \end{align*}
    从而 $f(x)$ 的 $1$ 级差分是
    \begin{align*}
        \Delta^1 f(x) = \Delta f(x) = f(x+1) - f(x) = 6x^2 + 12x + 10.
    \end{align*}
    因为
    \begin{align*}
        \Delta^1 f(x+1) = 6(x+1)^2 + 12(x+1) + 10 = 6x^2 + 24x + 28,
    \end{align*}
    故 $f(x)$ 的 $2$ 级差分是
    \begin{align*}
        \Delta^2 f(x) = \Delta (\Delta^1 f(x)) = \Delta^1 f(x+1) - \Delta^1 f(x) = 12x + 18.
    \end{align*}
    因为
    \begin{align*}
        \Delta^2 f(x+1) = 12(x+1) + 18 = 12x + 30,
    \end{align*}
    故 $f(x)$ 的 $3$ 级差分是
    \begin{align*}
        \Delta^3 f(x) = \Delta (\Delta^2 f(x)) = \Delta^2 f(x+1) - \Delta^2 f(x) = 12.
    \end{align*}
    因为
    \begin{align*}
        \Delta^3 f(x+1) = 12,
    \end{align*}
    故 $f(x)$ 的 $4$ 级差分是
    \begin{align*}
        \Delta^4 f(x) = \Delta (\Delta^3 f(x)) = \Delta^3 f(x+1) - \Delta^3 f(x) = 0.
    \end{align*}
    读者不难验证: 对任意超出 $3$ 的整数 $e$, 必有
    \begin{align*}
        \Delta^e f(x) = 0.
    \end{align*}

    由上面的计算, 可知
    \begin{align*}
         & \Delta^0 f(1) = 2 \cdot 1^3 + 3 \cdot 1^2 + 5 \cdot 1 + 7 = 17, \\
         & \Delta^1 f(1) = 6 \cdot 1^2 + 12 \cdot 1^2 + 10 = 28,           \\
         & \Delta^2 f(1) = 12 \cdot 1 + 18 = 30,                           \\
         & \Delta^3 f(1) = 12,                                             \\
         & \Delta^e f(1) = 0 \quad (e > 3).
    \end{align*}
\end{example}

高级差分适合如下性质:

\begin{proposition}
    设 $e$ 是非负整数. 当 $c_0$, $c_1$, $\cdots$, $c_{k-1} \in \FF$, 且 $f_0 (x)$, $f_1 (x)$, $\cdots$, $f_{k-1} (x) \in \FF[x]$ 时,
    \begin{align*}
        \Delta^e \left( \sum_{\ell = 0}^{k-1} c_\ell f_\ell (x) \right)
        = \sum_{\ell = 0}^{k-1} c_\ell \Delta^e f_\ell (x).
    \end{align*}
\end{proposition}

\begin{pf}
    用算学归纳法. 我们把具体过程留给读者当练习.
\end{pf}

\begin{proposition}
    设 $e$ 是非负整数. 设 $k$ 是整数. 则
    \begin{align*}
        \Delta^e \binom{x}{k} = \binom{x}{k - e}.
    \end{align*}
\end{proposition}

\begin{pf}
    用算学归纳法. 我们把具体过程留给读者当练习.
\end{pf}

\begin{proposition}
    设 $e$ 是非负整数. 设 $f(x) \in \FF[x]$. 则
    \begin{align*}
        \Delta^e f(x) = \sum_{k = 0}^{e} (-1)^{e-k} \binom{e}{k} f(x+k).
    \end{align*}
\end{proposition}

\begin{pf}
    当 $e = 0$ 时, 左侧是 $f(x)$, 右侧是
    \begin{align*}
        (-1)^0 \binom{0}{0} f(x+0) = f(x).
    \end{align*}
    当 $e = 1$ 时, 左侧是 $f(x+1) - f(x)$, 右侧是
    \begin{align*}
        (-1)^1 \binom{1}{0} f(x+0) + (-1)^0 \binom{1}{1} f(x+1) = -f(x) + f(x+1).
    \end{align*}
    所以, 命题对 $e=0$ 或 $e=1$ 成立.

    设命题对 $e = \ell \geq 1$ 成立, 即
    \begin{align*}
        \Delta^{\ell} f(x) = \sum_{k = 0}^{\ell} (-1)^{\ell-k} \binom{\ell}{k} f(x+k).
    \end{align*}
    则 $e = \ell + 1$ 时,
    \begin{align*}
             & \Delta^{\ell + 1} f(x)                                                     \\
        = {} & \Delta (\Delta^{\ell} f(x))                                                \\
        = {} & \Delta^{\ell} f(x+1) - \Delta^{\ell} f(x)                                  \\
        = {} & \sum_{k = 0}^{\ell} (-1)^{\ell-k} \binom{\ell}{k} f(x+1+k)
        - \sum_{k = 0}^{\ell} (-1)^{\ell-k} \binom{\ell}{k} f(x+k)                        \\
        = {} & \sum_{k = 0}^{\ell} (-1)^{(\ell+1)-(k+1)} \binom{\ell}{k+1-1} f(x+k+1)     \\
             & \qquad \qquad + \sum_{k = 0}^{\ell} (-1)^{\ell+1-k} \binom{\ell}{k} f(x+k) \\
        = {} & \sum_{k = 1}^{\ell+1} (-1)^{\ell+1-k} \binom{\ell}{k-1} f(x+k)
        + \sum_{k = 0}^{\ell} (-1)^{\ell+1-k} \binom{\ell}{k} f(x+k)                      \\
        = {} & \sum_{k = 0}^{\ell+1} (-1)^{\ell+1-k} \binom{\ell}{k-1} f(x+k)
        + \sum_{k = 0}^{\ell+1} (-1)^{\ell+1-k} \binom{\ell}{k} f(x+k)                    \\
        = {} & \sum_{k = 0}^{\ell+1} \left( (-1)^{\ell+1-k} \binom{\ell}{k-1} f(x+k)
        + (-1)^{\ell+1-k} \binom{\ell}{k} f(x+k) \right)                                  \\
        = {} & \sum_{k = 0}^{\ell+1} (-1)^{\ell+1-k} \left( \binom{\ell}{k-1}
        + \binom{\ell}{k} \right) f(x+k)                                                  \\
        = {} & \sum_{k = 0}^{\ell+1} (-1)^{\ell+1-k} \binom{\ell+1}{k} f(x+k). \qedhere
    \end{align*}
\end{pf}

我们再补充一个跟广义二项系数有关的性质:

\begin{proposition}
    设 $k$ 是整数. 则
    \begin{align*}
        \binom{0}{k} = \begin{cases}
            1, & \quad k = 0;    \\
            0, & \quad k \neq 0.
        \end{cases}
    \end{align*}
\end{proposition}

\begin{pf}
    显然.
\end{pf}

设 $f(x) \in \FF[x]$ 是次不高于 $m$ 的多项式. 我们知道, $f(x)$ 一定可以写为广义二项系数的线性组合:
\begin{align*}
    f(x) = \sum_{k = 0}^{m} c_{k} \binom{x}{k}.
\end{align*}
对左右二侧求 $e$ 级差分 ($e \leq m$), 有
\begin{align*}
    \Delta^e f(x) = \sum_{k = 0}^{m} c_{k} \binom{x}{k - e}.
\end{align*}
用 $0$ 替换 $x$, 有
\begin{align*}
    \Delta^e f(0) = \sum_{k = 0}^{m} c_{k} \binom{0}{k - e} = c_{e}.
\end{align*}
所以
\begin{align*}
    f(x)
    = {} & \sum_{k = 0}^{m} \Delta^k f(0) \binom{x}{k}                                                   \\
    = {} & \Delta^0 f(0) \binom{x}{0} + \Delta^1 f(0) \binom{x}{1} + \cdots + \Delta^m f(0) \binom{x}{m} \\
    = {} & f(0) + \Delta f(0) \binom{x}{1} + \cdots + \Delta^m f(0) \binom{x}{m}.
\end{align*}

我们已经证明了

\begin{proposition}
    设 $f(x) \in \FF[x]$ 是次不高于 $m$ 的多项式. 则
    \begin{align*}
        f(x)
        = {} & \sum_{k = 0}^{m} \Delta^k f(0) \binom{x}{k}                                                   \\
        = {} & \Delta^0 f(0) \binom{x}{0} + \Delta^1 f(0) \binom{x}{1} + \cdots + \Delta^m f(0) \binom{x}{m} \\
        = {} & f(0) + \Delta f(0) \binom{x}{1} + \cdots + \Delta^m f(0) \binom{x}{m},
    \end{align*}
    所以
    \begin{align*}
        S(n) = \sum_{\ell = 0}^{n - 1} f(\ell) = f(0) \binom{n}{1} + \Delta f(0) \binom{n}{2} + \cdots + \Delta^{m} f(0) \binom{n}{m + 1}.
    \end{align*}
\end{proposition}

注意到
\begin{align*}
    \Delta^k f(0) = \sum_{u = 0}^{k} (-1)^{k-u} \binom{k}{u} f(u),
\end{align*}
故计算 $\Delta^k f(0)$ 需要用到 $f(0)$, $f(1)$, $\cdots$, $f(k)$ 这 $k+1$ 个数. 也就是说, 计算 $\Delta^0 f(0)$, $\Delta^1 f(0)$, $\cdots$, $\Delta^m f(0)$ 需要用到 $f(0)$, $f(1)$, $\cdots$, $f(m)$ 这 $m+1$ 个数.

下面我们举几个具体的例, 帮助读者消化这种求和方法.

\begin{example}
    设 $f(x) = x^2 + x - 1$. 求
    \begin{align*}
        S(n) = \sum_{\ell = 0}^{n - 1} f(\ell) = f(0) + f(1) + \cdots + f(n - 1).
    \end{align*}

    这里, $m = 2$. 所以, 我们计算 $f(0)$, $f(1)$, $f(2)$:
    \begin{align*}
        f(0) = -1, \quad f(1) = 1, \quad f(2) = 5.
    \end{align*}
    由此, 不难算出:
    \begin{align*}
         & \Delta^0 f(0) = f(0) = -1,                         \\
         & \Delta^1 f(0) = f(1) - f(0) = 2,                   \\
         & \Delta^1 f(1) = f(2) - f(1) = 4,                   \\
         & \Delta^2 f(0) = \Delta^1 f(1) - \Delta^1 f(0) = 2.
    \end{align*}
    所以
    \begin{align*}
        f(x)
        = {} & f(0) + \Delta f(0) \binom{x}{1} + \Delta^2 f(0) \binom{x}{2} \\
        = {} & -1 + 2 \binom{x}{1} + 2 \binom{x}{2}.
    \end{align*}
    从而
    \begin{align*}
        S(n)
        = {} & \sum_{\ell = 0}^{n - 1} f(\ell)                   \\
        = {} & -1 \binom{n}{1} + 2 \binom{n}{2} + 2 \binom{n}{3} \\
        = {} & -n + n(n-1) + \frac{n(n-1)(n-2)}{3}               \\
        = {} & \frac{n(n+2)(n-2)}{3}.
    \end{align*}

    实操时, 往往用名为 ``差分表'' 的表进行计算. 当 $m = 2$ 时, 它长这样:
    \begin{align*}
        \arraycolsep=0.25cm
        \begin{array}{ccc}
            \Delta^{0} f(2) & \               & \               \\
            \Delta^{0} f(1) & \Delta^{1} f(1) & \               \\
            \Delta^{0} f(0) & \Delta^{1} f(0) & \Delta^{2} f(0) \\
        \end{array}
    \end{align*}
    在这个问题里, 差分表如下:
    \begin{align*}
        \arraycolsep=0.25cm
        \begin{array}{ccc}
            5  & \  & \ \\
            1  & 4  & \ \\
            -1 & 2  & 2 \\
        \end{array}
    \end{align*}
\end{example}

\begin{example}
    求前 $n$ 个非负整数的立方和
    \begin{align*}
        S(n) = 0^3 + 1^3 + \cdots + (n-1)^3 = \sum_{\ell = 0}^{n - 1} \ell^3.
    \end{align*}
    取 $f(x) = x^3$. 这里, $m = 3$. 画出 $m = 3$ 时的差分表:
    \begin{align*}
        \arraycolsep=0.25cm
        \begin{array}{cccc}
            \Delta^{0} f(3) & \               & \               & \               \\
            \Delta^{0} f(2) & \Delta^{1} f(2) & \               & \               \\
            \Delta^{0} f(1) & \Delta^{1} f(1) & \Delta^{2} f(1) & \               \\
            \Delta^{0} f(0) & \Delta^{1} f(0) & \Delta^{2} f(0) & \Delta^{3} f(0) \\
        \end{array}
    \end{align*}
    $\Delta^{0} f(t)$ 就是 $f(t)$:
    \begin{align*}
        f(0) = 0, \quad f(1) = 1, \quad f(2) = 8, \quad f(3) = 27.
    \end{align*}
    写在表上, 就是
    \begin{align*}
        \arraycolsep=0.25cm
        \begin{array}{cccc}
            27 & \               & \               & \               \\
            8  & \Delta^{1} f(2) & \               & \               \\
            1  & \Delta^{1} f(1) & \Delta^{2} f(1) & \               \\
            0  & \Delta^{1} f(0) & \Delta^{2} f(0) & \Delta^{3} f(0) \\
        \end{array}
    \end{align*}
    由此可确定 $1$ 级差分:
    \begin{align*}
         & \Delta^{1} f(2) = f(3) - f(2) = 19, \\
         & \Delta^{1} f(1) = f(2) - f(1) = 7,  \\
         & \Delta^{1} f(0) = f(1) - f(0) = 1.
    \end{align*}
    写在表上, 就是
    \begin{align*}
        \arraycolsep=0.25cm
        \begin{array}{cccc}
            27 & \  & \               & \               \\
            8  & 19 & \               & \               \\
            1  & 7  & \Delta^{2} f(1) & \               \\
            0  & 1  & \Delta^{2} f(0) & \Delta^{3} f(0) \\
        \end{array}
    \end{align*}
    类似地, 可确定 $2$ 级差分:
    \begin{align*}
         & \Delta^{2} f(1) = \Delta^{1} f(2) - \Delta^{1} f(1) = 12, \\
         & \Delta^{2} f(0) = \Delta^{1} f(1) - \Delta^{1} f(0) = 6.
    \end{align*}
    写在表上, 就是
    \begin{align*}
        \arraycolsep=0.25cm
        \begin{array}{cccc}
            27 & \  & \  & \               \\
            8  & 19 & \  & \               \\
            1  & 7  & 12 & \               \\
            0  & 1  & 6  & \Delta^{3} f(0) \\
        \end{array}
    \end{align*}
    最后, 可确定 $3$ 级差分:
    \begin{align*}
         & \Delta^{3} f(0) = \Delta^{2} f(1) - \Delta^{2} f(0) = 6.
    \end{align*}
    写在表上, 就是
    \begin{align*}
        \arraycolsep=0.25cm
        \begin{array}{cccc}
            27 & \  & \  & \ \\
            8  & 19 & \  & \ \\
            1  & 7  & 12 & \ \\
            0  & 1  & 6  & 6 \\
        \end{array}
    \end{align*}
    所以
    \begin{align*}
        f(x)
        = {} & f(0) + \Delta f(0) \binom{x}{1} + \Delta^2 f(0) \binom{x}{2} + \Delta^3 f(0) \binom{x}{3} \\
        = {} & \binom{x}{1} + 6 \binom{x}{2} + 6 \binom{x}{3}.
    \end{align*}
    从而
    \begin{align*}
        S(n)
        = {} & \sum_{\ell = 0}^{n - 1} f(\ell)                             \\
        = {} & \binom{n}{2} + 6 \binom{n}{3} + 6 \binom{n}{4}              \\
        = {} & \frac{n(n-1)}{2} + n(n-1)(n-2) + \frac{n(n-1)(n-2)(n-3)}{4} \\
        = {} & \frac{n(n-1)}{4} (2 + 4(n-2) + (n-2)(n-3))                  \\
        = {} & \frac{n(n-1)}{4} n(n-1)                                     \\
        = {} & \left( \frac{n(n-1)}{2} \right)^2.
    \end{align*}
\end{example}

\begin{remark}
    回忆一下, 前 $n$ 个非负整数的和
    \begin{align*}
        0 + 1 + \cdots + (n - 1) = \frac{n(n-1)}{2}.
    \end{align*}
    上面的例告诉我们,
    \begin{align*}
        0^3 + 1^3 + \cdots + (n - 1)^3 = (0 + 1 + \cdots + (n - 1))^2.
    \end{align*}
    所以, 前 $n$ 个非负整数的立方和等于前 $n$ 个非负整数的和的平方.
\end{remark}

\begin{example}
    求前 $n$ 个非负整数的 $4$ 次幂和
    \begin{align*}
        S(n) = 0^4 + 1^4 + \cdots + (n-1)^4 = \sum_{\ell = 0}^{n - 1} \ell^4.
    \end{align*}
    取 $f(x) = x^4$. 这里, $m = 4$. 画出 $m = 4$ 时的差分表:
    \begin{align*}
        \arraycolsep=0.25cm
        \begin{array}{ccccc}
            \Delta^{0} f(4) & \               & \               & \               & \               \\
            \Delta^{0} f(3) & \Delta^{1} f(3) & \               & \               & \               \\
            \Delta^{0} f(2) & \Delta^{1} f(2) & \Delta^{2} f(2) & \               & \               \\
            \Delta^{0} f(1) & \Delta^{1} f(1) & \Delta^{2} f(1) & \Delta^{3} f(1) & \               \\
            \Delta^{0} f(0) & \Delta^{1} f(0) & \Delta^{2} f(0) & \Delta^{3} f(0) & \Delta^{4} f(0) \\
        \end{array}
    \end{align*}
    我们直接填差分表:
    \begin{align*}
         & {
                \arraycolsep=0.25cm
                \begin{array}{ccccc}
                    256 & \               & \               & \               & \               \\
                    81  & \Delta^{1} f(3) & \               & \               & \               \\
                    16  & \Delta^{1} f(2) & \Delta^{2} f(2) & \               & \               \\
                    1   & \Delta^{1} f(1) & \Delta^{2} f(1) & \Delta^{3} f(1) & \               \\
                    0   & \Delta^{1} f(0) & \Delta^{2} f(0) & \Delta^{3} f(0) & \Delta^{4} f(0) \\
                \end{array}
        }    \\
         & {
                \arraycolsep=0.25cm
                \begin{array}{ccccc}
                    256 & \   & \               & \               & \               \\
                    81  & 175 & \               & \               & \               \\
                    16  & 65  & \Delta^{2} f(2) & \               & \               \\
                    1   & 15  & \Delta^{2} f(1) & \Delta^{3} f(1) & \               \\
                    0   & 1   & \Delta^{2} f(0) & \Delta^{3} f(0) & \Delta^{4} f(0) \\
                \end{array}
        }    \\
         & {
                \arraycolsep=0.25cm
                \begin{array}{ccccc}
                    256 & \   & \   & \               & \               \\
                    81  & 175 & \   & \               & \               \\
                    16  & 65  & 110 & \               & \               \\
                    1   & 15  & 50  & \Delta^{3} f(1) & \               \\
                    0   & 1   & 14  & \Delta^{3} f(0) & \Delta^{4} f(0) \\
                \end{array}
        }    \\
         & {
                \arraycolsep=0.25cm
                \begin{array}{ccccc}
                    256 & \   & \   & \  & \               \\
                    81  & 175 & \   & \  & \               \\
                    16  & 65  & 110 & \  & \               \\
                    1   & 15  & 50  & 60 & \               \\
                    0   & 1   & 14  & 36 & \Delta^{4} f(0) \\
                \end{array}
        }    \\
         & {
                \arraycolsep=0.25cm
                \begin{array}{ccccc}
                    256 & \   & \   & \  & \  \\
                    81  & 175 & \   & \  & \  \\
                    16  & 65  & 110 & \  & \  \\
                    1   & 15  & 50  & 60 & \  \\
                    0   & 1   & 14  & 36 & 24 \\
                \end{array}
            }
    \end{align*}
    所以
    \begin{align*}
        f(x)
        = {} & f(0) + \Delta f(0) \binom{x}{1} + \Delta^2 f(0) \binom{x}{2} + \Delta^3 f(0) \binom{x}{3} + \Delta^4 f(0) \binom{x}{4} \\
        = {} & \binom{x}{1} + 14 \binom{x}{2} + 36 \binom{x}{3} + 24 \binom{x}{4}.
    \end{align*}
    从而
    \begin{align*}
        S(n)
        = {} & \sum_{\ell = 0}^{n - 1} f(\ell)                                         \\
        = {} & \binom{n}{2} + 14 \binom{n}{3} + 36 \binom{n}{4} + 24 \binom{n}{5}      \\
        = {} & \frac{n(n-1)}{2} + \frac{7n(n-1)(n-2)}{3} + \frac{3n(n-1)(n-2)(n-3)}{2} \\
             & \qquad \qquad + \frac{n(n-1)(n-2)(n-3)}{5}                              \\
        = {} & \frac{n(n-1)}{30} (15 + 70(n-2) + 45(n-3)(n-2)                          \\
             & \qquad \qquad + 6(n-4)(n-3)(n-2))                                       \\
        = {} & \frac{n(n-1)}{30} (6n^3 - 9n^2 + n + 1)                                 \\
        = {} & \frac{n(n-1)}{120} (24n^3 - 36n^2 + 4n + 4)                             \\
        = {} & \frac{n(n-1)}{120} (3(2n)^3 - 9(2n)^2 + 2(2n) + 4)                      \\
        = {} & \frac{n(n-1)}{120} (3(2n)^3 - 3 - 9(2n)^2 + 9 + 2(2n) - 2)              \\
        = {} & \frac{n(n-1)}{120} (3((2n)^3 - 1) - 9((2n)^2 - 1) + 2((2n) - 1))        \\
        = {} & \frac{n(n-1)}{120} (2n-1) (3((2n)^2 + 2n + 1) - 9(2n + 1) + 2)          \\
        = {} & \frac{n(n-1)(2n-1)}{120} (12n^2 - 12n - 4)                              \\
        = {} & \frac{n(n-1)(2n-1)}{30} (3n^2 - 3n - 1)                                 \\
        = {} & \frac{n(n-1)(2n-1)(3n^2 - 3n - 1)}{30}.
    \end{align*}
\end{example}
