\subsection*{Summation Formulae}
\addcontentsline{toc}{subsection}{Summation Formulae}
\markright{Summation Formulae}

本节讨论求和公式 \term{summation formula} 问题: 设 $f(x) \in \FF[x]$, 求
\begin{align*}
    S(n) = \sum_{\ell = 0}^{n - 1} f(\ell) = f(0) + f(1) + \cdots + f(n - 1) \period
\end{align*}

\begin{example}
    相信大家应该听说过德意志数学家 Gauss\period 1787 年, Gauss 还只是一个 10 岁的孩子\period 据说, 当时他的数学教师给全班同学出了这样的算术题:
    \begin{align*}
        1 + 2 + 3 + \cdots + 100 = ?
    \end{align*}
    这里, 后一个数比前一个数多 $1$, 且共有 $100$ 个数\period 教师刚写完问题, Gauss 就算出, 答案是 $5\,050$\period 他的同学还在一个一个地加, 算了很久, 还没算对\period

    Gauss 是怎么快速算出答案的呢? 设
    \begin{align*}
        S = 1 + 2 + 3 + \cdots + 100 \period
    \end{align*}
    因为加法适合交换律, 故
    \begin{align*}
        S = 100 + 99 + 98 + \cdots + 1 \period
    \end{align*}
    所以
    \begin{align*}
        2S
        = {} & (1 + 100) + (2 + 99) + (3 + 98) + \cdots + (100 + 1) \\
        = {} & \underbrace{101 + 101 + 101 + \cdots + 101}
        _{\text{a hundred $101$'s}}                                 \\
        = {} & 100 \cdot 101                                        \\
        = {} & 10\,100 \period
    \end{align*}
    由此可得
    \begin{align*}
        S = \frac{10\,100}{2} = 5\,050 \period
    \end{align*}

    如果记 $f(x) = x + 1$, 则
    \begin{align*}
        S
        = {} & 1 + 2 + 3 + \cdots + 100                  \\
        = {} & f(0) + f(1) + f(2) + \cdots + f(100 - 1)  \\
        = {} & \sum_{\ell = 0}^{100 - 1} f(\ell) \period
    \end{align*}

    考虑更一般的情形\period 设 $f(x) = a + bx$\period 记
    \begin{align*}
        S(n) = f(0) + f(1) + \cdots + f(n-1) \period
    \end{align*}
    类似地, 把右侧倒着写:
    \begin{align*}
        S(n) = f(n-1) + f(n-1) + \cdots + f(0) \period
    \end{align*}
    因为
    \begin{align*}
        f(k) + f(n-1-k) = a + bk + a + b(n-1-k) = 2a + b(n - 1),
    \end{align*}
    故
    \begin{align*}
             & 2S(n)                                                        \\
        = {} & (f(0) + f(n-1)) + (f(1) + f(n-2)) + \cdots + (f(n-1) + f(0)) \\
        = {} & n(2a + b(n - 1)),
    \end{align*}
    即
    \begin{align*}
        S(n) = \frac{n(2a + b(n - 1))}{2} = \left(a - \frac{b}{2}\right) n + \frac{b}{2} n^2 \period
    \end{align*}
    我们还可以看出: $S(n)$ 是多项式, 且
    \begin{align*}
        \deg S(n) = \deg f(n) + 1 \period
    \end{align*}
\end{example}

上面讨论了当 $f(x)$ 的次不高于 $1$ 时如何求 $S(n)$\period 那么, 当 $f(x)$ 的次高于 $1$ 时, 怎么找 $S(n)$? 它还是多项式吗?

在求和前, 我们看 $S(n)$ 适合什么性质\period $S(n)$ 是 $f(0)$, $f(1)$, $\cdots$, $f(n-1)$ 这 $n$ 个数的和\period 因为 $0$ 个数的和是 $0$, 故 $S(0) = 0$\period 同时, 不难看出, $S(n+1)$ 比 $S(n)$ 多出 $f(n)$, 也即
\begin{align*}
    S(n+1) - S(n) = f(n) \period
\end{align*}
反过来, 设 $\NN$ 到 $\FF$ 的函数 $W(n)$ 适合 $W(0) = 0$ 与 $W(n+1) - W(n) = f(n)$, 则
\begin{align*}
    \sum_{\ell = 0}^{n - 1} f(\ell)
    = {} & \sum_{\ell = 0}^{n - 1} (W(\ell+1) - W(\ell))                       \\
    = {} & \sum_{\ell = 0}^{n - 1} W(\ell+1) - \sum_{\ell = 0}^{n - 1} W(\ell) \\
    = {} & \sum_{\ell = 1}^{n} W(\ell) - \sum_{\ell = 0}^{n - 1} W(\ell)       \\
    = {} & W(n) - W(0)                                                         \\
    = {} & W(n) \period
\end{align*}
这样, 任给 $f(x) \in \FF[x]$, 若我们能找到适合条件 $S(0) = 0$ 与 $S(x+1) - S(x) = f(x)$ 的多项式, 则
\begin{align*}
    \sum_{\ell = 0}^{n - 1} f(\ell) = S(n) \period
\end{align*}

\begin{proposition}
    设 $f(x) \in \FF[x]$ 是 $m$ 次多项式\period 存在唯一的 $m+1$ 次多项式 $F(x) \in \FF[x]$ 适合条件:

    (i) $F(0) = 0$;

    (ii) $F(x + 1) - F(x) = f(x)$\period
\end{proposition}

\begin{pf}
    先看存在性\period 若 $f(x) = 0$, 则 $F(x) = 0$ 显然适合 (i) (ii), 且
    \begin{align*}
        \deg F(x) = -\infty = -\infty + 1 = \deg f(x) + 1 \period
    \end{align*}
    设 $m \geq 0$\period 根据广义二项系数的性质, 存在 $m+1$ 个 $\FF$ 中元 $c_0$, $\cdots$, $c_{m}$ 使
    \begin{align*}
        f(x) = \sum_{\ell = 0}^{m} c_\ell \binom{x}{\ell}, \quad c_{m} \neq 0 \period
    \end{align*}
    (读者可思考: 若 $c_{m} = 0$, $f(x)$ 还能是 $m$ 次多项式吗?) 作多项式
    \begin{align*}
        F(x) = \sum_{\ell = 0}^{m} c_\ell \binom{x}{\ell+1} \in \FF[x] \period
    \end{align*}
    显然 $\deg F(x) = m + 1$\period 验证 (i):
    \begin{align*}
        F(0) = \sum_{\ell = 0}^{m} c_\ell \binom{0}{\ell+1} = \sum_{\ell = 0}^{m} 0 = 0 \period
    \end{align*}
    验证 (ii):
    \begin{align*}
        F(x + 1) - F(x)
        = {} & \sum_{\ell = 0}^{m} c_\ell \binom{x+1}{\ell+1} - \sum_{\ell = 0}^{m} c_\ell \binom{x}{\ell+1} \\
        = {} & \sum_{\ell = 0}^{m} c_\ell \left( \binom{x+1}{\ell+1} - \binom{x}{\ell+1} \right)             \\
        = {} & \sum_{\ell = 0}^{m} c_\ell \binom{x}{\ell}                                                    \\
        = {} & f(x) \period
    \end{align*}

    再看唯一性\period 设 $G(x) \in \FF[x]$ 是 $m+1$ 次多项式, 并适合条件 $G(0) = 0$ 与 $G(x + 1) - G(x) = f(x)$\period 作
    \begin{align*}
        H(x) = F(x) - G(x) \period
    \end{align*}
    则 $H(0) = 0$, $H(x + 1) - H(x) = 0$\period 所以, $r$ 为非负整数时, $H(r) = 0$\period 从而 $H(x)$ 一定是零多项式, 即 $F(x) = G(x)$\period
\end{pf}

\begin{example}
    记 $f(x) = x^2$\period 我们求
    \begin{align*}
        S(n) = f(0) + f(1) + \cdots + f(n-1) = \sum_{\ell = 0}^{n - 1} f(\ell) \period
    \end{align*}
    由上个命题可知, 存在唯一的次为 $3$ 的多项式 $F(x)$ 使 $F(0) = 0$, $F(x + 1) - F(x) = f(x)$, 且 $S(n) = F(n)$\period

    可以用插值的思想求 $F(x)$\period 取 $x_0$, $x_1$, $x_2$, $x_3$ 为 $0$, $1$, $-1$, $2$\period 不难算出:
    \begin{align*}
         & y_{0} = F(0) = 0,                      \\
         & y_{1} = F(1) = F(0) + f(0) = 0,        \\
         & y_{2} = F(-1) = F(0) - f(-1) = -1,     \\
         & y_{3} = F(2) = F(1) + f(1) = 1 \period
    \end{align*}
    注意到 $y_0 = y_1 = 0$, 故可以考虑 Lagrange 插值 (只要算 $L_2 (x)$ 与 $L_3 (x)$):
    \begin{align*}
         & L_2 (x) = \frac{(x - 0)(x - 1)(x - 2)}{(-1 - 0)(-1 - 1)(-1 - 2)} = -\frac{x(x-1)(x-2)}{6}, \\
         & L_3 (x) = \frac{(x - 0)(x - 1)(x + 1)}{(2 - 0)(2 - 1)(2 + 1)} = \frac{x(x-1)(x+1)}{6},     \\
         & F(x) = y_2 L_2 (x) + y_3 L_3 (x) = \frac{x(x-1)(2x-1)}{6} \period
    \end{align*}
    当然, 也可利用 Newton 插值\period 作出差商表:
    \begin{align*}
        \arraycolsep=0.25cm
        \begin{array}{c|cccc}
            2  & 1  & \       & \        & \       \\
            -1 & -1 & \frac23 & \        & \       \\
            1  & 0  & \frac12 & \frac16  & \       \\
            0  & 0  & 0       & -\frac12 & \frac13 \\
        \end{array}
    \end{align*}
    故
    \begin{align*}
        F(x)
        = {} & [0] + [0,1] (x - 0) + [0,1,-1] (x - 0)(x - 1)    \\
             & \qquad \qquad + [0,1,-1,2] (x - 0)(x - 1)(x + 1) \\
        = {} & -\frac{1}{2} x(x-1) + \frac{1}{3} x(x-1)(x+1)    \\
        = {} & \frac{x(x-1)(2x-1)}{6} \period
    \end{align*}

    综上, 我们有
    \begin{align*}
        \sum_{\ell = 0}^{n - 1} \ell^2 = 0^2 + 1^2 + \cdots + (n-1)^2 = \frac{n(n-1)(2n-1)}{6} \period
    \end{align*}
\end{example}

由上个命题的证明过程, 有

\begin{proposition}
    若
    \begin{align*}
        f(x) = c_0 \binom{x}{0} + c_1 \binom{x}{1} + \cdots + c_m \binom{x}{m},
    \end{align*}
    则
    \begin{align*}
        S(n) = \sum_{\ell = 0}^{n - 1} f(\ell) = c_0 \binom{n}{1} + c_1 \binom{n}{2} + \cdots + c_m \binom{n}{m + 1} \period
    \end{align*}
\end{proposition}

由此可见, 若我们能把 $f(x)$ 写为广义二项系数的线性组合, 则寻找 $S(n)$ 的过程将十分简单\period 接下来, 我们讨论怎么方便地把多项式写为广义二项系数的线性组合\period

\begin{definition}
    设 $f(x) \in \FF[x]$\period 定义 $f(x)$ 的差分 \term{difference} 为
    \begin{align*}
        \Delta f(x) = f(x+1) - f(x) \in \FF[x]\period
    \end{align*}
    设 $t \in \FF$\period 我们把
    \begin{align*}
        f(t+1) - f(t) \in \FF
    \end{align*}
    也写为 $\Delta f(t)$\period
\end{definition}

\begin{example}
    取 $f(x) = x^2 + x - 1$\period 则
    \begin{align*}
        f(x + 1) = (x + 1)^2 + (x + 1) - 1 = x^2 + 3x + 1,
    \end{align*}
    故
    \begin{align*}
        \Delta f(x) = 2x + 2 \period
    \end{align*}
\end{example}

\begin{proposition}
    设 $k$ 是整数\period 则
    \begin{align*}
        \Delta \binom{x}{k} = \binom{x}{k - 1} \period
    \end{align*}
\end{proposition}

\begin{pf}
    也许, 这就是所谓的 ``新瓶装旧酒'' 吧! 不过, 为了方便, 我们还是单独列出来\period
\end{pf}
