\subsection*{\RootsOfPolynomials}
\addcontentsline{toc}{subsection}{\RootsOfPolynomials}
\markright{\RootsOfPolynomials}

我们回顾一下熟悉的整式函数.

\begin{definition}
    设 $a_0, a_1, \cdots, a_n \in D$. 称
    \begin{align*}
         & D \to D, \tag*{$f \colon$}               \\
         & t \mapsto a_0 + a_1 t + \cdots + a_n t^n
    \end{align*}
    为 $D$ 的整式函数 \term{polynomial function}. 我们也说, 这个 $f$ 是由 $D$ 上 $x$ 的整式
    \begin{align*}
        f(x) = a_0 + a_1 x + \cdots + a_n x^n
    \end{align*}
    诱导的整式函数 \term{the polynomial function induced by $f$}. 不难看出, 若二个整式相等, 则其诱导的整式函数也相等.
\end{definition}

\begin{definition}
    设 $f$ 与 $g$ 是 $D$ 的二个整式函数. 二者的和 $f+g$ 定义为
    \begin{align*}
         & D \to D, \tag*{$f+g \colon$} \\
         & t \mapsto f(t) + g(t).
    \end{align*}
    二者的积 $fg$ 定义为
    \begin{align*}
         & D \to D, \tag*{$fg \colon$} \\
         & t \mapsto f(t) g(t).
    \end{align*}
\end{definition}

设 $f$, $g$ 是 $D$ 的二个整式函数:
\begin{align*}
     & D \to D, \tag*{$f \colon$}                \\
     & t \mapsto a_0 + a_1 t + \cdots + a_n t^n, \\
     & D \to D, \tag*{$g \colon$}                \\
     & t \mapsto b_0 + b_1 t + \cdots + b_n t^n.
\end{align*}
利用 $D$ 的运算律, 可以得到
\begin{align*}
     & D \to D, \tag*{$f+g \colon$}                                      \\
     & t \mapsto (a_0 + b_0) + (a_1 + b_1) t + \cdots + (a_n + b_n) t^n, \\
     & D \to D, \tag*{$fg \colon$}                                       \\
     & t \mapsto c_0 + c_1 t + \cdots + c_{2n} t^{2n},
\end{align*}
其中
\begin{align*}
    c_k = a_0 b_k + a_1 b_{k-1} + \cdots + a_k b_0.
\end{align*}
由此可得下面的命题:

\begin{proposition}
    设 $f(x)$, $g(x) \in D[x]$, $f$, $g$ 分别是 $f(x)$, $g(x)$ 诱导的整式函数. 那么 $f+g$ 是 $f(x)+g(x)$ 诱导的整式函数, 且 $fg$ 是 $f(x)g(x)$ 诱导的整式函数.

    通俗地说, 若整式 $f_0 (x)$, $f_1 (x)$, $\cdots$, $f_{n-1} (x)$ 之间有一个由加法与乘法计算得到的关系, 那么将 $x$ 换为 $D$ 的元 $t$, 这样的关系仍成立.
\end{proposition}

\begin{example}
    考虑 $\FF$ 与 $\FF[x]$. 前面, 利用带余除法, 得到关系
    \begin{align*}
        8x^6 + 1 = (4x^3 + 12x - 8) \cdot 2(x-1)^2 (x+2) + (72x^2 - 96x + 33).
    \end{align*}
    这里 $x$ 只是一个文字, 不是数! 但是, 上面的命题告诉我们, 可以把 $x$ 看成一个数. 比如, 由上面的式可以立即看出, $8t^6 + 1$ 与 $72t^2 - 96t + 33$ 在 $t = 1$ 或 $t = -2$ 时值是一样的.

    可是, 对于这样的式, 我们不能将 $x$ 改写为 $\FF$ 的元 $t$:
    \begin{align*}
        \deg 3x^2 < \deg 2x^3.
    \end{align*}
    可以看到, 若 $t=0$, 则 $3t^2 = 2t^3 = 0$, 而 $0$ 的次是 $-\infty$; 若 $t \neq 0$, 则 $3t^2$ 与 $2t^3$ 都是非零数, 次都是 $0$.
\end{example}

\begin{remark}
    我们已经知道, 整式确定整式函数. 自然地, 有这样的问题: 整式函数能否确定整式? 一般情况下, 这个问题的答案是 no.

    考虑 $4$ 元集 $V$. 作 $V$ 上 $x$ 的二个整式:
    \begin{align*}
        f(x) = x^4 - x, \quad g(x) = 0.
    \end{align*}
    显然, 这是二个不相等的整式. 但是, 任取 $t \in V$, 都有
    \begin{align*}
        t^4 - t = 0.
    \end{align*}
    因此, $f(x)$ 与 $g(x)$ 诱导的整式函数是同一函数!

    不过, 在某些场合下, 整式函数可以确定整式. 之后我们还会提到这一点.
\end{remark}

\begin{remark}
    设 $f(x) = a_0 + a_1 x + \cdots + a_n x^n \in D[x]$. 设 $t$ 是 $D$ 的元. 以后, 我们直接写
    \begin{align*}
        f(t) = a_0 + a_1 t + \cdots + a_n t^n.
    \end{align*}
    并称 $f(t)$ 是整式 $f(x)$ 在点 \term{point} $t$ 的值. 至少, 一方通行 \term{one-way traffic} 是没问题的.

    顺便一提, $f(x)$ 的微商也是整式:
    \begin{align*}
        f^{\prime} (x) = a_1 + 2a_2 x + \cdots + na_n x^{n-1}.
    \end{align*}
    我们把
    \begin{align*}
        a_1 + 2a_2 t + \cdots + na_n t^{n-1} \in D
    \end{align*}
    简单地写为 $f^{\prime} (t)$.
\end{remark}

了解了整式与整式函数的关系后, 下面的这个命题就不会太凸兀了.

\begin{proposition}
    设 $f(x) \in D[x]$ 是 $n$ 次整式 ($n \geq 1$), $a \in D$. 则存在 $n-1$ 次整式 $q(x)$ ($\in D[x]$) 使
    \begin{align*}
        f(x) = q(x) (x-a) + f(a).
    \end{align*}
    根据带余除法, 这样的 $q(x)$ 一定是唯一的.
\end{proposition}

\begin{pf}
    因为 $x-a$ 的首项系数 $1$ 是单位, 故存在 $D[x]$ 的二元 $q(x)$, $r(x)$ 使
    \begin{align*}
        f(x) = q(x) (x-a) + r(x), \quad \deg r(x) < \deg (x-a) = 1.
    \end{align*}
    所以, $r(x) = c$, $c \in D$. 用 $D$ 的元 $a$ 替换 $x$, 有
    \begin{align*}
        f(a) = q(a) (a-a) + c = c.
    \end{align*}
    所以
    \begin{align*}
        f(x) = q(x) (x-a) + f(a).
    \end{align*}
    再看这个 $q(x)$ 的次. 因为 $f(x)$ 的次不低于 $x-a$ 的次, 故
    \begin{align*}
         & \deg q(x) = \deg f(x) - \deg (x-a) = n-1. \qedhere
    \end{align*}
\end{pf}

\begin{remark}
    如果用 $D$ 的元 $b$ 替换 $x$, 则
    \begin{align*}
        f(b) = (b-a)q(b) + f(a),
    \end{align*}
    也就是说, 存在 $r \in D$ 使
    \begin{align*}
        f(b) - f(a) = (b-a)r.
    \end{align*}
    所以, 若 $f(x) \in D[x]$ 是 $n$ 次整式 ($n \geq 1$), $a,b \in D$, 则存在 $r \in D$ 使 $f(b) - f(a) = (b-a)r$. 当 $f(x)$ 的次低于 $1$ 时, 这个命题也对 (取 $r=0$).

    举个简单的例. 我们说, 不存在系数为整数的整式 $f(x)$ 使 $f(1) = f(-1) + 1$. 假如说这样的 $f$ 存在, 那么应存在整数 $r$ 使
    \begin{align*}
        1 = f(1) - f(-1) = (1 - (-1))r = 2r,
    \end{align*}
    而 $1$ 不是偶数, 矛盾.
\end{remark}

现在, 我们讨论整式的根的基本性质.

\begin{definition}
    设 $f(x)$ 是 $D$ 上 $x$ 的整式. 若有 $a \in D$ 使 $f(a) = 0$, 则说 $a$ 是 (整式) $f(x)$ 的根 \term{root}.
\end{definition}

\begin{example}
    设 $D \subset \CC$, 且 $\ZZ \subset D$. 看 $D$ 上 $x$ 的整式
    \begin{align*}
        f(x) = (2x-1)(x+1)(x^2 - 3)(x^2 + 1)(x^2 + 4).
    \end{align*}
    如果 $D = \ZZ$, 则 $f(x)$ 有一个在 $D$ 里的根: $-1$. 如果 $D = \QQ$, 则 $f(x)$ 有二个在 $D$ 里的根: $-1$, $\frac12$. 如果 $D = \RR$, 则 $f(x)$ 有四个在 $D$ 里的根: $-1$, $\frac12$, $\pm \sqrt{3}$. 如果 $D = \CC$, 则 $f(x)$ 有八个在 $D$ 里的根: $-1$, $\frac12$, $\pm \sqrt{3}$, $\pm \ii$, $\pm 2\ii$.
\end{example}

\begin{example}
    再来一个例. 看 $D$ 上 $x$ 的整式
    \begin{align*}
        f(x) = x^2 + x - 1.
    \end{align*}
    若 $D = \RR$, 则 $f(x)$ 的二个根是 $\frac{-1 \pm \sqrt{5}}{2}$. 若 $D = V$, 则 $f(x)$ 的二个根是 $\tau$, $\tau^2$. 当然, 若 $D \subset \QQ$, 则 $f(x)$ 无 ($D$ 的) 根.
\end{example}

\begin{remark}
    设 $a,b \in D$, 且 $a \neq 0$.

    若 $f(x) = a$, 则 $f(x)$ 无根. 换句话说, 零次整式至多有零个根.

    再设 $f(x) = ax + b$ 是 $1$ 次整式. 若存在 $c \in D$ 使 $b = ac$, 则 $f(x)$ 有一个根 $-c$. 并且, $f(x)$ 也不会有另一个根 (若 $at_1 + b = at_2 + b$, 则 $at_1 = at_2$, 故 $t_1 = t_2$). 若这样的 $c$ 不存在, 则 $f(x)$ 无根 (反设 $f(x)$ 有根 $d$, 则由 $ad + b = 0$ 知 $b = a(-d)$, 矛盾). 换句话说, $1$ 次整式至多有一个根.

    结合上面的二个例, 我们猜想: $n$ 次整式 ($n \in \NN$) 至多有 $n$ 个 (不同的) 根. 幸运的事儿是, 这个猜想是正确的.
\end{remark}

\begin{proposition}
    设 $f(x) \in D[x]$ 是 $n$ 次整式 ($n \geq 1$). $a$ 是 $f(x)$ 的根的一个必要与充分条件是: 存在 $n-1$ 次整式 $q(x)$ ($\in D[x]$) 使
    \begin{align*}
        f(x) = q(x) (x-a).
    \end{align*}
    根据带余除法, 这样的 $q(x)$ 一定是唯一的.
\end{proposition}

\begin{pf}
    先看充分性. 若这样的 $q(x)$ 存在, 则
    \begin{align*}
        f(a) = q(a) (a-a) = 0.
    \end{align*}
    再看必要性. 设 $f(a)=0$. 根据上面的命题, 存在 $n-1$ 次整式 $q(x) \in D[x]$ 使
    \begin{align*}
         & f(x) = q(a) (x-a) + f(a) = q(a) (x-a). \qedhere
    \end{align*}
\end{pf}

\begin{proposition}
    设 $f(x) \in D[x]$ 是 $n$ 次整式 ($n \in \NN$). 则 $f(x)$ 至多有 $n$ 个不同的根.
\end{proposition}

\begin{pf}
    $n = 0$ 或 $n = 1$ 时, 我们已经知道这是对的. 用算学归纳法. 假设 $\ell$ 次整式至多有 $\ell$ 个不同的根. 看 $\ell + 1$ 次整式 $f(x)$. 如果它没有根, 当然至多有 $\ell + 1$ 个不同的根. 如果它有一个根 $a$, 则存在 $\ell$ 次整式 $q(x)$ 使
    \begin{align*}
        f(x) = q(x) (x-a).
    \end{align*}
    根据归纳假设, $q(x)$ 至多有 $\ell$ 个不同的根. 而且, 若 $b \neq a$, 且 $b$ 不是 $q(x)$ 的根, 利用消去律可知 $f(b) \neq 0$. 这样, $f(x)$ 至多有 $\ell + 1$ 个不同的根.
\end{pf}

由此可推出一个很有用的事实:

\begin{proposition}
    设 $a_0$, $a_1$, $\cdots$, $a_n$ 是 $D$ 的元. 设 $n$ 是非负整数. 设
    \begin{align*}
        f(x) = a_0 + a_1 x + \cdots + a_n x^n.
    \end{align*}
    若 $t_0$, $t_1$, $\cdots$, $t_n$ 是 $n+1$ 个互不相同的 $D$ 的元, 且
    \begin{align*}
        f(t_0) = f(t_1) = \cdots = f(t_n) = 0,
    \end{align*}
    则 $f(x)$ 必为零整式. 通俗地说, 次不高于 $n$ (且系数为整环的元) 的整式不可能有 $n$ 个以上的互不相同的根, 除非这个整式是零.
\end{proposition}

\begin{pf}
    反证法. 设 $f(x)$ 不是零整式. 设 $f(x)$ 的次为 $m$, 则 $0 \leq m \leq n$. 根据上个命题, $f(x)$ 至多有 $m$ 个不同的根, 这与题设矛盾! 故 $f(x) = 0$.
\end{pf}

\begin{remark}
    再看前面提到的 $4$ 元集 $V$. 可以看出, 因为 $V$ 的元 ``不够多'', 所以出现了取零值的非零整式.
\end{remark}

此事实的一个推论是:

\begin{proposition}
    设 $a_0$, $b_0$, $a_1$, $b_1$, $\cdots$, $a_n$, $b_n$ 是 $D$ 的元. 设 $n$ 是非负整数. 设
    \begin{align*}
         & f(x) = a_0 + a_1 x + \cdots + a_n x^n, \\
         & g(x) = b_0 + b_1 x + \cdots + b_n x^n.
    \end{align*}
    若 $t_0$, $t_1$, $\cdots$, $t_n$ 是 $n+1$ 个互不相同的 $D$ 的元, 且
    \begin{align*}
        f(t_0) = g(t_0), \quad f(t_1) = g(t_1), \quad \cdots, \quad f(t_n) = g(t_n),
    \end{align*}
    则 $f(x)$ 必等于 $g(x)$. 通俗地说, 若次不高于 $n$ (且系数为整环的元) 的二个整式若在多于 $n$ 处取一样的值, 则这二个整式相等.
\end{proposition}

\begin{pf}
    考虑 $h(x) = f(x) - g(x)$. 则 $\deg h(x) \leq n$. $h(x)$ 有 $n+1$ 个不同的根. 根据上个命题, $h(x)$ 是零整式. 这样, $f(x) = g(x)$.
\end{pf}

在中学, 我们学过解 $1$ 元 $2$ 次方程 $at^2 + bt + c = 0$ ($a$, $b$, $c$ 为实数, 且 $a \neq 0$) 的一种方法: 直接套用公式
\begin{align*}
    t = \frac{-b \pm \sqrt{\Delta}}{2a},
\end{align*}
其中
\begin{align*}
    \Delta = b^2 - 4ac
\end{align*}
是判别式: 当 $\Delta > 0$ 时, 方程有二个不等的实数解; 当 $\Delta = 0$ 时, 方程有二个相等的实数解; 当 $\Delta < 0$ 时, 方程无实数解.

当 $\Delta = 0$ 时, $c = \frac{b^2}{4a}$, 则
\begin{align*}
    at^2 + bt + c = a \left( t^2 + 2\frac{b}{2a}t + \left(\frac{b}{2a}\right)^2 \right) = a \left( t + \frac{b}{2a} \right)^2.
\end{align*}
记
$$
    f(x) = a \left( x + \frac{b}{2a} \right)^2 \in \RR[x].
$$
根据根的定义, $-\frac{b}{2a} \in \RR$ 是 $f(x)$ 的根. 我们发现, 这个根 ``出现了'' $2$ 次, 是重复的. 我们给这样的根一个特殊点的称呼.

\begin{definition}
    设 $a \in D$ 是整式 $f(x) \in D[x]$ 的根. 那么, 存在唯一的整式 $q(x) \in D[x]$ 使
    \begin{align*}
        f(x) = (x - a) q(x).
    \end{align*}
    若 $q(a) = 0$, 则说 $a$ 是 $f(x)$ 的一个重根 \term{multiple root}. 若 $q(a) \neq 0$, 则说 $a$ 是 $f(x)$ 的一个单根 \term{simple root}.
\end{definition}

\begin{example}
    看 $\ZZ$ 上 $x$ 的整式
    \begin{align*}
        f(x) = (x^2 - 3)(x^2 + 2)(x - 1)^2 (x + 2).
    \end{align*}
    显然, $f(x)$ 的根是 $1$ 与 $-2$. 因为
    \begin{align*}
        f(x) = (x + 2) \underbrace{(x^2 - 3)(x^2 + 2)(x - 1)^2}_{q_1 (x)},
    \end{align*}
    且 $q_1 (x) \neq 0$, 故 $-2$ 是 $f(x)$ 的单根. 类似地, 由于
    \begin{align*}
        f(x) = (x - 1) \underbrace{(x^2 - 3)(x^2 + 2)(x - 1)(x + 2)}_{q_2 (x)},
    \end{align*}
    且 $q_2 (x) = 0$, 故 $1$ 是 $f(x)$ 的重根.
\end{example}

\begin{proposition}
    设 $a \in D$ 是整式 $f(x) \in D[x]$ 的根. 则:

    (i) 若 $a$ 是 $f(x)$ 的重根, 则 $a$ 是 $f^{\prime} (x)$ 的根;

    (ii) 若 $a$ 是 $f(x)$ 的单根, 则 $a$ 不是 $f^{\prime} (x)$ 的根.

    所以, $f(x)$ 有重根的一个必要与充分条件是: $f(x)$ 与 $f^{\prime} (x)$ 有公共根.
\end{proposition}

\begin{pf}
    因为 $a$ 是 $f(x)$ 的根, 故存在唯一的 $q(x)$ 使
    \begin{align*}
        f(x) = (x - a) q(x).
    \end{align*}
    从而
    \begin{align*}
        f^{\prime} (x) = (x - a)^{\prime} q(x) + (x - a) q^{\prime} (x) = q(x) + (x - a) q^{\prime} (x).
    \end{align*}
    这样
    \begin{align*}
        f^{\prime} (a) = q(a) + (a - a) q^{\prime} (a) = q(a).
    \end{align*}

    (i) 若 $a$ 是 $f(x)$ 的重根, 则 $q(a) = 0$, 故 $f^{\prime} (a) = 0$.

    (ii) 若 $a$ 是 $f(x)$ 的单根, 则 $q(a) \neq 0$, 故 $f^{\prime} (a) \neq 0$.
\end{pf}

\begin{example}
    我们看
    \begin{align*}
        f(x) = ax^2 + bx + c \in \RR[x], \quad a \neq 0.
    \end{align*}
    它的微商 $f^{\prime} (x) = 2ax + b$ 恰有一个根 $t_0 = -\frac{b}{2a}$. 由上个命题, $f(x)$ 有重根相当于 $f(t_0) = 0$, 即
    \begin{align*}
        0 = f(t_0) = a \cdot \frac{b^2}{4a^2} - \frac{b^2}{2a} + c = \frac{4ac - b^2}{4a} = -\frac{\Delta}{4a}.
    \end{align*}
\end{example}
