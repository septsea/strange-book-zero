\subsection*{Roots of Polynomials}
\addcontentsline{toc}{subsection}{Roots of Polynomials}
\markright{Roots of Polynomials}

我们回顾一下熟悉的多项式函数\period

\begin{definition}
    设 $a_0, a_1, \cdots, a_n \in D$\period 称
    \begin{align*}
         & D \to D, \tag*{$f \colon$}               \\
         & t \mapsto a_0 + a_1 t + \cdots + a_n t^n
    \end{align*}
    为 $D$ 的多项式函数 \term{polynomial function}\period 我们也说, 这个 $f$ 是由 $D$ 上 $x$ 的多项式
    \begin{align*}
        f(x) = a_0 + a_1 x + \cdots + a_n x^n
    \end{align*}
    诱导的多项式函数 \term{the polynomial function induced by $f$}\period 不难看出, 若二个多项式相等, 则其诱导的多项式函数也相等\period
\end{definition}

\begin{definition}
    设 $f$ 与 $g$ 是 $D$ 的二个多项式函数\period 二者的和 $f+g$ 定义为
    \begin{align*}
         & D \to D, \tag*{$f+g \colon$}  \\
         & t \mapsto f(t) + g(t) \period
    \end{align*}
    二者的积 $fg$ 定义为
    \begin{align*}
         & D \to D, \tag*{$fg \colon$} \\
         & t \mapsto f(t) g(t) \period
    \end{align*}
\end{definition}

设 $f$, $g$ 是 $D$ 的二个多项式函数:
\begin{align*}
     & D \to D, \tag*{$f \colon$}                       \\
     & t \mapsto a_0 + a_1 t + \cdots + a_n t^n,        \\
     & D \to D, \tag*{$g \colon$}                       \\
     & t \mapsto b_0 + b_1 t + \cdots + b_n t^n \period
\end{align*}
利用 $D$ 的运算律, 可以得到
\begin{align*}
     & D \to D, \tag*{$f+g \colon$}                                      \\
     & t \mapsto (a_0 + b_0) + (a_1 + b_1) t + \cdots + (a_n + b_n) t^n, \\
     & D \to D, \tag*{$fg \colon$}                                       \\
     & t \mapsto c_0 + c_1 t + \cdots + c_{2n} t^{2n},
\end{align*}
其中
\begin{align*}
    c_k = a_0 b_k + a_1 b_{k-1} + \cdots + a_k b_0 \period
\end{align*}
由此可得下面的命题:

\begin{proposition}
    设 $f(x)$, $g(x) \in D[x]$, $f$, $g$ 分别是 $f(x)$, $g(x)$ 诱导的多项式函数\period 那么 $f+g$ 是 $f(x)+g(x)$ 诱导的多项式函数, 且 $fg$ 是 $f(x)g(x)$ 诱导的多项式函数\period

    通俗地说, 若多项式 $f_0 (x)$, $f_1 (x)$, $\cdots$, $f_{n-1} (x)$ 之间有一个由加法与乘法计算得到的关系, 那么将 $x$ 换为 $D$ 的元 $t$, 这样的关系仍成立\period
\end{proposition}

\begin{example}
    考虑 $\FF$ 与 $\FF[x]$\period 前面, 利用带余除法, 得到关系
    \begin{align*}
        8x^6 + 1 = (4x^3 + 12x - 8) \cdot 2(x-1)^2 (x+2) + (72x^2 - 96x + 33) \period
    \end{align*}
    这里 $x$ 只是一个文字, 不是数! 但是, 上面的命题告诉我们, 可以把 $x$ 看成一个数\period 比如, 由上面的式可以立即看出, $8t^6 + 1$ 与 $72t^2 - 96t + 33$ 在 $t = 1$ 或 $t = -2$ 时值是一样的\period

    可是, 对于这样的式, 我们不能将 $x$ 改写为 $\FF$ 的元 $t$:
    \begin{align*}
        \deg 3x^2 < \deg 2x^3 \period
    \end{align*}
    可以看到, 若 $t=0$, 则 $3t^2 = 2t^3 = 0$, 而 $0$ 的次是 $-\infty$; 若 $t \neq 0$, 则 $3t^2$ 与 $2t^3$ 都是非零数, 次都是 $0$\period
\end{example}

\begin{remark}
    我们已经知道, 多项式确定多项式函数\period 自然地, 有这样的问题: 多项式函数能否确定多项式? 一般情况下, 这个问题的答案是 no\period

    考虑 $4$ 元集 $V$\period 作 $V$ 上 $x$ 的二个多项式:
    \begin{align*}
        f(x) = x^4 - x, \quad g(x) = 0 \period
    \end{align*}
    显然, 这是二个不相等的多项式\period 但是, 任取 $t \in V$, 都有
    \begin{align*}
        t^4 - t = 0 \period
    \end{align*}
    因此, $f(x)$ 与 $g(x)$ 诱导的多项式函数是同一函数!

    不过, 在某些场合下, 多项式函数可以确定多项式\period 之后我们还会提到这一点\period
\end{remark}

\begin{remark}
    设 $f(x) = a_0 + a_1 x + \cdots + a_n x^n \in D[x]$\period 设 $t$ 是 $D$ 的元\period 以后, 我们直接写
    \begin{align*}
        f(t) = a_0 + a_1 t + \cdots + a_n t^n \period
    \end{align*}
    并称 $f(t)$ 是多项式 $f(x)$ 在点 \term{point} $t$ 的值\period 至少, 一方通行 \term{one-way traffic} 是没问题的\period

    顺便一提, $f(x)$ 的导数也是多项式:
    \begin{align*}
        f^{\prime} (x) = a_1 + 2a_2 x + \cdots + na_n x^{n-1} \period
    \end{align*}
    我们把
    \begin{align*}
        a_1 + 2a_2 t + \cdots + na_n t^{n-1} \in D
    \end{align*}
    简单地写为 $f^{\prime} (t)$\period
\end{remark}

了解了多项式与多项式函数的关系后, 下面的这个命题就不会太凸兀了\period

\begin{proposition}
    设 $f(x) \in D[x]$ 是 $n$ 次多项式 ($n \geq 1$), $a \in D$\period 则存在 $n-1$ 次多项式 $q(x)$ ($\in D[x]$) 使
    \begin{align*}
        f(x) = q(x) (x-a) + f(a) \period
    \end{align*}
    根据带余除法, 这样的 $q(x)$ 一定是唯一的\period
\end{proposition}

\begin{pf}
    因为 $x-a$ 的首项系数 $1$ 是单位, 故存在 $D[x]$ 的二元 $q(x)$, $r(x)$ 使
    \begin{align*}
        f(x) = q(x) (x-a) + r(x), \quad \deg r(x) < \deg (x-a) = 1 \period
    \end{align*}
    所以, $r(x) = c$, $c \in D$\period 用 $D$ 的元 $a$ 替换 $x$, 有
    \begin{align*}
        f(a) = q(a) (a-a) + c = c \period
    \end{align*}
    所以
    \begin{align*}
        f(x) = q(x) (x-a) + f(a) \period
    \end{align*}
    再看这个 $q(x)$ 的次\period 因为 $f(x)$ 的次不低于 $x-a$ 的次, 故
    \begin{align*}
         & \deg q(x) = \deg f(x) - \deg (x-a) = n-1 \period \qedhere
    \end{align*}
\end{pf}

\begin{remark}
    如果用 $D$ 的元 $b$ 替换 $x$, 则
    \begin{align*}
        f(b) = (b-a)q(b) + f(a),
    \end{align*}
    也就是说, 存在 $r \in D$ 使
    \begin{align*}
        f(b) - f(a) = (b-a)r \period
    \end{align*}
    所以, 若 $f(x) \in D[x]$ 是 $n$ 次多项式 ($n \geq 1$), $a,b \in D$, 则存在 $r \in D$ 使 $f(b) - f(a) = (b-a)r$\period 当 $f(x)$ 的次低于 $1$ 时, 这个命题也对 (取 $r=0$)\period

    举个简单的例\period 我们说, 不存在系数为整数的多项式 $f(x)$ 使 $f(1) = f(-1) + 1$\period 假如说这样的 $f$ 存在, 那么应存在整数 $r$ 使
    \begin{align*}
        1 = f(1) - f(-1) = (1 - (-1))r = 2r,
    \end{align*}
    而 $1$ 不是偶数, 矛盾\period
\end{remark}

现在, 我们讨论多项式的根的基本性质\period

\begin{definition}
    设 $f(x)$ 是 $D$ 上 $x$ 的多项式\period 若有 $a \in D$ 使 $f(a) = 0$, 则说 $a$ 是 (多项式) $f(x)$ 的根 \term{root}\period
\end{definition}

\begin{example}
    设 $D \subset \CC$, 且 $\ZZ \subset D$\period 看 $D$ 上 $x$ 的多项式
    \begin{align*}
        f(x) = (2x-1)(x+1)(x^2 - 3)(x^2 + 1)(x^2 + 4) \period
    \end{align*}
    如果 $D = \ZZ$, 则 $f(x)$ 有一个在 $D$ 里的根: $-1$\period 如果 $D = \QQ$, 则 $f(x)$ 有二个在 $D$ 里的根: $-1$, $\frac12$\period 如果 $D = \RR$, 则 $f(x)$ 有四个在 $D$ 里的根: $-1$, $\frac12$, $\pm \sqrt{3}$\period 如果 $D = \CC$, 则 $f(x)$ 有八个在 $D$ 里的根: $-1$, $\frac12$, $\pm \sqrt{3}$, $\pm \ii$, $\pm 2\ii$\period
\end{example}

\begin{example}
    再来一个例\period 看 $D$ 上 $x$ 的多项式
    \begin{align*}
        f(x) = x^2 + x - 1 \period
    \end{align*}
    若 $D = \RR$, 则 $f(x)$ 的二个根是 $\frac{-1 \pm \sqrt{5}}{2}$\period 若 $D = V$, 则 $f(x)$ 的二个根是 $\tau$, $\tau^2$\period 当然, 若 $D \subset \QQ$, 则 $f(x)$ 无 ($D$ 的) 根\period
\end{example}

\begin{remark}
    设 $a,b \in D$, 且 $a \neq 0$\period

    若 $f(x) = a$, 则 $f(x)$ 无根\period 换句话说, 零次多项式至多有零个根\period

    再设 $f(x) = ax + b$ 是一次多项式\period 若存在 $c \in D$ 使 $b = ac$, 则 $f(x)$ 有一个根 $-c$\period 并且, $f(x)$ 也不会有另一个根 (若 $at_1 + b = at_2 + b$, 则 $at_1 = at_2$, 故 $t_1 = t_2$)\period 若这样的 $c$ 不存在, 则 $f(x)$ 无根 (反设 $f(x)$ 有根 $d$, 则由 $ad + b = 0$ 知 $b = a(-d)$, 矛盾)\period 换句话说, 一次多项式至多有一个根\period

    结合上面的二个例, 我们猜想: $n$ 次多项式 ($n \in \NN$) 至多有 $n$ 个 (不同的) 根\period 幸运的事儿是, 这个猜想是正确的\period
\end{remark}

\begin{proposition}
    设 $f(x) \in D[x]$ 是 $n$ 次多项式 ($n \geq 1$)\period $a$ 是 $f(x)$ 的根的一个必要与充分条件是: 存在 $n-1$ 次多项式 $q(x)$ ($\in D[x]$) 使
    \begin{align*}
        f(x) = q(x) (x-a) \period
    \end{align*}
    根据带余除法, 这样的 $q(x)$ 一定是唯一的\period
\end{proposition}

\begin{pf}
    先看充分性\period 若这样的 $q(x)$ 存在, 则
    \begin{align*}
        f(a) = q(a) (a-a) = 0 \period
    \end{align*}
    再看必要性\period 设 $f(a)=0$\period 根据上面的命题, 存在 $n-1$ 次多项式 $q(x) \in D[x]$ 使
    \begin{align*}
         & f(x) = q(a) (x-a) + f(a) = q(a) (x-a) \period \qedhere
    \end{align*}
\end{pf}

\begin{proposition}
    设 $f(x) \in D[x]$ 是 $n$ 次多项式 ($n \in \NN$)\period 则 $f(x)$ 至多有 $n$ 个不同的根\period
\end{proposition}

\begin{pf}
    $n = 0$ 或 $n = 1$ 时, 我们已经知道这是对的\period 用数学归纳法\period 假设 $\ell$ 次多项式至多有 $\ell$ 个不同的根\period 看 $\ell + 1$ 次多项式 $f(x)$\period 如果它没有根, 当然至多有 $\ell + 1$ 个不同的根\period 如果它有一个根 $a$, 则存在 $\ell$ 次多项式 $q(x)$ 使
    \begin{align*}
        f(x) = q(x) (x-a) \period
    \end{align*}
    根据归纳假设, $q(x)$ 至多有 $\ell$ 个不同的根\period 而且, 若 $b \neq a$, 且 $b$ 不是 $q(x)$ 的根, 利用消去律可知 $f(b) \neq 0$\period 这样, $f(x)$ 至多有 $\ell + 1$ 个不同的根\period
\end{pf}

由此可推出一个很有用的事实:

\begin{proposition}
    设 $a_0$, $a_1$, $\cdots$, $a_n$ 是 $D$ 的元\period 设 $n$ 是非负整数\period 设
    \begin{align*}
        f(x) = a_0 + a_1 x + \cdots + a_n x^n \period
    \end{align*}
    若 $t_0$, $t_1$, $\cdots$, $t_n$ 是 $n+1$ 个互不相同的 $D$ 的元, 且
    \begin{align*}
        f(t_0) = f(t_1) = \cdots = f(t_n) = 0,
    \end{align*}
    则 $f(x)$ 必为零多项式\period 通俗地说, 次不高于 $n$ (且系数为整环的元) 的多项式不可能有 $n$ 个以上的互不相同的根, 除非这个多项式是零\period
\end{proposition}

\begin{pf}
    反证法\period 设 $f(x)$ 不是零多项式\period 设 $f(x)$ 的次为 $m$, 则 $0 \leq m \leq n$\period 根据上个命题, $f(x)$ 至多有 $m$ 个不同的根, 这与题设矛盾! 故 $f(x) = 0$\period
\end{pf}

\begin{remark}
    再看前面提到的 $4$ 元集 $V$\period 可以看出, 因为 $V$ 的元 ``不够多'', 所以出现了取零值的非零多项式\period
\end{remark}

此事实的一个推论是:

\begin{proposition}
    设 $a_0$, $b_0$, $a_1$, $b_1$, $\cdots$, $a_n$, $b_n$ 是 $D$ 的元\period 设 $n$ 是非负整数\period 设
    \begin{align*}
         & f(x) = a_0 + a_1 x + \cdots + a_n x^n,        \\
         & g(x) = b_0 + b_1 x + \cdots + b_n x^n \period
    \end{align*}
    若 $t_0$, $t_1$, $\cdots$, $t_n$ 是 $n+1$ 个互不相同的 $D$ 的元, 且
    \begin{align*}
        f(t_0) = g(t_0), \quad f(t_1) = g(t_1), \quad \cdots, \quad f(t_n) = g(t_n),
    \end{align*}
    则 $f(x)$ 必等于 $g(x)$\period 通俗地说, 若次不高于 $n$ (且系数为整环的元) 的二个多项式若在多于 $n$ 处取一样的值, 则这二个多项式相等\period
\end{proposition}

\begin{pf}
    考虑 $h(x) = f(x) - g(x)$\period 则 $\deg h(x) \leq n$\period $h(x)$ 有 $n+1$ 个不同的根\period 根据上个命题, $h(x)$ 是零多项式\period 这样, $f(x) = g(x)$\period
\end{pf}

在中学, 我们学过解一元二次方程 $at^2 + bt + c = 0$ ($a$, $b$, $c$ 为实数, 且 $a \neq 0$) 的一种方法: 直接套用公式
\begin{align*}
    t = \frac{-b \pm \sqrt{\Delta}}{2a},
\end{align*}
其中
\begin{align*}
    \Delta = b^2 - 4ac
\end{align*}
是判别式: 当 $\Delta > 0$ 时, 方程有二个不等的实数解; 当 $\Delta = 0$ 时, 方程有二个相等的实数解; 当 $\Delta < 0$ 时, 方程无实数解\period

当 $\Delta = 0$ 时, $c = \frac{b^2}{4a}$, 则
\begin{align*}
    at^2 + bt + c = a \left( t^2 + 2\frac{b}{2a}t + \left(\frac{b}{2a}\right)^2 \right) = a \left( t + \frac{b}{2a} \right)^2 \period
\end{align*}
记
$$
    f(x) = a \left( x + \frac{b}{2a} \right)^2 \in \RR[x] \period
$$
根据根的定义, $-\frac{b}{2a} \in \RR$ 是 $f(x)$ 的根\period 我们发现, 这个根 ``出现了'' $2$ 次, 是重复的\period 我们给这样的根一个特殊点的称呼\period

\begin{definition}
    设 $a \in D$ 是多项式 $f(x) \in D[x]$ 的根\period 那么, 存在唯一的多项式 $q(x) \in D[x]$ 使
    \begin{align*}
        f(x) = (x - a) q(x) \period
    \end{align*}
    若 $q(a) = 0$, 则说 $a$ 是 $f(x)$ 的一个重根 \term{multiple root}\period 若 $q(a) \neq 0$, 则说 $a$ 是 $f(x)$ 的一个单根 \term{simple root}\period
\end{definition}

\begin{example}
    看 $Z$ 上 $x$ 的多项式
    \begin{align*}
        f(x) = (x^2 - 3)(x^2 + 2)(x - 1)^2 (x + 2) \period
    \end{align*}
    显然, $f(x)$ 的根是 $1$ 与 $-2$\period 因为
    \begin{align*}
        f(x) = (x + 2) \underbrace{(x^2 - 3)(x^2 + 2)(x - 1)^2}_{q_1 (x)},
    \end{align*}
    且 $q_1 (x) \neq 0$, 故 $-2$ 是 $f(x)$ 的单根\period 类似地, 由于
    \begin{align*}
        f(x) = (x - 1) \underbrace{(x^2 - 3)(x^2 + 2)(x - 1)(x + 2)}_{q_2 (x)},
    \end{align*}
    且 $q_2 (x) = 0$, 故 $1$ 是 $f(x)$ 的重根\period
\end{example}

\begin{proposition}
    设 $a \in D$ 是多项式 $f(x) \in D[x]$ 的根\period 则:

    (i) 若 $a$ 是 $f(x)$ 的重根, 则 $a$ 是 $f^{\prime} (x)$ 的根;

    (ii) 若 $a$ 是 $f(x)$ 的单根, 则 $a$ 不是 $f^{\prime} (x)$ 的根\period

    所以, $f(x)$ 有重根的一个必要与充分条件是: $f(x)$ 与 $f^{\prime} (x)$ 有公共根\period
\end{proposition}

\begin{pf}
    因为 $a$ 是 $f(x)$ 的根, 故存在唯一的 $q(x)$ 使
    \begin{align*}
        f(x) = (x - a) q(x) \period
    \end{align*}
    从而
    \begin{align*}
        f^{\prime} (x) = (x - a)^{\prime} q(x) + (x - a) q^{\prime} (x) = q(x) + (x - a) q^{\prime} (x) \period
    \end{align*}
    这样
    \begin{align*}
        f^{\prime} (a) = q(a) + (a - a) q^{\prime} (a) = q(a) \period
    \end{align*}

    (i) 若 $a$ 是 $f(x)$ 的重根, 则 $q(a) = 0$, 故 $f^{\prime} (a) = 0$\period

    (ii) 若 $a$ 是 $f(x)$ 的单根, 则 $q(a) \neq 0$, 故 $f^{\prime} (a) \neq 0$\period
\end{pf}

\begin{example}
    我们看
    \begin{align*}
        f(x) = ax^2 + bx + c \in \RR[x], \quad a \neq 0 \period
    \end{align*}
    它的导数 $f^{\prime} (x) = 2ax + b$ 恰有一个根 $t_0 = -\frac{b}{2a}$\period 由上个命题, $f(x)$ 有重根相当于 $f(t_0) = 0$, 即
    \begin{align*}
        0 = f(t_0) = a \cdot \frac{b^2}{4a^2} - \frac{b^2}{2a} + c = \frac{4ac - b^2}{4a} = -\frac{\Delta}{4a} \period
    \end{align*}
\end{example}
