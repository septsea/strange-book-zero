\subsection*{\PolynomialsOverZAndOverQ}
\addcontentsline{toc}{subsection}{\PolynomialsOverZAndOverQ}
\markright{\PolynomialsOverZAndOverQ}

在 ``\SomePropertiesOfIntegers '' 与 ``\SomePropertiesOfPolynomials '' 里, 我们系统地介绍了整数与 (系数为 $\FF$ 的元的) 多项式的一些性质. 它们有一个共同点: 都可以作带余除法. 因为带余除法, 我们证明了最大公因子的存在性与 Bézout 等式; 因为最大公因子与 Bézout 等式, 我们考察了互素, 进而考虑了不可约的整数 (多项式).

读者可能注意到, 在 ``\SomePropertiesOfPolynomials '' 里, 我们没有讨论整系数多项式. 为什么没讨论呢? 读者可以想一想, 整系数多项式是否还有带余除法.

\begin{example}
    以 $f = x^2 + 1$, $g = 2x$ 为例. 设存在整系数多项式 $q$, $r$ 使
    \begin{align*}
        f = gq + r, \quad \deg r < \deg g = 1.
    \end{align*}
    由此可设 $r = c$, $c$ 是某个待确定的整数. 设
    \begin{align*}
        q = a_0 + a_1 x + \cdots + a_n x^n,
    \end{align*}
    且 $a_0$, $a_1$, $\cdots$, $a_n$ 都是整数. 所以
    \begin{align*}
        x^2 + 1 = c + 2a_0 x + 2a_1 x^2 + \cdots + 2a_n x^{n+1}.
    \end{align*}
    由此可知 $n+1 = 2$, 且
    \begin{align*}
        1 = c, \quad 0 = 2a_0, \quad 1 = 2a_1.
    \end{align*}
    问题来了: 哪个整数乘 $2$ 等于 $1$? 所以这样的 $q$ 不存在.

    当然, 如果读者视 $f$, $g$, $q$, $r$ 为有理系数多项式, 立即可得
    \begin{align*}
        q = \frac{1}{2}x, \quad r = 1.
    \end{align*}
\end{example}

在 ``\SomePropertiesOfPolynomials '' 里, 我们把 ``\SomePropertiesOfIntegers '' 的套路几乎原封不动地搬了过来. 不过, 由于整系数多项式不一定有带余除法, 故我们没法 ``偷懒地'' 讨论整系数多项式.

但情况不是特别糟. 首先, 整数是有理数, 故整系数多项式是有理系数多项式. 其次, 读者知道, 有理数是二个整数的比 (分母不为零). 取不为零的有理系数多项式
\begin{align*}
    f = \frac{p_0}{q_0} + \frac{p_1}{q_1} x + \cdots + \frac{p_n}{q_n} x^n,
\end{align*}
这里 $p_0$, $q_0$, $p_1$, $q_0$, $\cdots$, $p_n$, $q_n$ 都是整数, 且 $q_0$, $q_1$, $\cdots$, $q_n$ 都不是零. 作整数
\begin{align*}
     & Q = q_0 q_1 \cdots q_n,                       \\
     & Q_0 = q_1 q_2 \cdots q_n = \frac{Q}{q_0},     \\
     & Q_1 = q_0 q_2 \cdots q_n = \frac{Q}{q_1},     \\
     & \cdots \cdots \cdots \cdots,                  \\
     & Q_n = q_0 q_1 \cdots q_{n-1} = \frac{Q}{q_n},
\end{align*}
将 $f$ 改写为
\begin{align*}
    f = \frac{p_0 Q_0}{Q} + \frac{p_1 Q_1}{Q} x + \cdots + \frac{p_n Q_n}{Q} x^n.
\end{align*}
设 $d$ 是整数 (而不是多项式) $p_0 Q_0$, $p_1 Q_1$, $\cdots$, $p_n Q_n$ 的最大公因子. 这样, 存在整数 $m_0$, $m_1$, $\cdots$, $m_n$ 使
\begin{align*}
    p_0 Q_0 = d m_0, \quad p_1 Q_1 = d m_1, \quad \cdots, \quad p_n Q_n = d m_n.
\end{align*}
所以
\begin{align*}
    f = \frac{d}{Q} (m_0 + m_1 x + \cdots + m_n x^n).
\end{align*}
由最大公因子的性质, 知 $m_0$, $m_1$, $\cdots$, $m_n$ 互素. 最后, 设 $D$ 是 $d$ 与 $Q$ 的最大公因子, 且 $d = Dd^{\prime}$, $Q = DQ^{\prime}$. 所以
\begin{align*}
    f = \frac{d^{\prime}}{Q^{\prime}} (m_0 + m_1 x + \cdots + m_n x^n).
\end{align*}

上面的叙述看起来有些抽象, 实则很好理解.

\begin{example}
    取
    \begin{align*}
        f = 1 + \frac{2}{3} x + \frac{1}{6}x^2 + \frac{3}{5}x^3.
    \end{align*}
    这里
    \begin{align*}
        q_0 = 1, \quad q_1 = 3, \quad q_2 = 6, \quad q_3 = 5.
    \end{align*}
    所以
    \begin{align*}
        Q = 180, \quad Q_0 = 180, \quad Q_1 = 60, \quad Q_2 = 30, \quad Q_3 = 36.
    \end{align*}
    因为
    \begin{align*}
        p_0 = 1, \quad p_1 = 2, \quad p_2 = 1, \quad p_3 = 3,
    \end{align*}
    故
    \begin{align*}
        p_0 Q_0 = 180, \quad p_1 Q_1 = 120, \quad p_2 Q_2 = 30, \quad p_3 Q_3 = 108.
    \end{align*}
    所以 $f$ 可被改写为
    \begin{align*}
        f = \frac{180}{180} + \frac{120}{180} x + \frac{30}{180} x^2 + \frac{108}{180} x^3.
    \end{align*}
    读者可能一眼就认出来, 这如果不是通分, 那它什么都不是.

    不难算出 $6$ 是 $180$, $120$, $30$, $108$ 的最大公因子是 $6$. 所以
    \begin{align*}
        f = \frac{6}{180} (30 + 20x + 5x^2 + 18x^3).
    \end{align*}
    不难算出, $1$ 是 $30$, $20$, $5$, $18$ 的最大公因子. 最后, 因为 $6$ 是 $6$ 与 $180$ 的最大公因子, 故可进一步将 $f$ 改写为
    \begin{align*}
        f = \frac{1}{30} (30 + 20x + 5x^2 + 18x^3).
    \end{align*}
\end{example}

上面假定 $f \neq 0$; 现在考虑 $0$. 显然 $0 = 0 \cdot 1$, 其中 $1$ 是整系数多项式, 且其系数互素.

上面的文字说明: 有理系数多项式 $f$ 总可以写为一个有理数 $c_f$ 与一个整系数多项式 $f^{\ast}$ 的积, 且 $f^{\ast}$ 的系数互素.

因此, 我们可以借助有理系数多项式讨论整系数多项式.

\begin{definition}
    设
    \begin{align*}
        f = a_0 + a_1 x + \cdots + a_n x^n.
    \end{align*}
    若系数 $a_0$, $a_1$, $\cdots$, $a_n$ 都是整数, 且整数 $a_0$, $a_1$, $\cdots$, $a_n$ 互素, 则 $f$ 是本原的 \term{primitive}.
\end{definition}

\begin{proposition}
    设 $f$ 是有理系数多项式, 且 $f$ 不是零.

    (i) $f$ 一定可以写为有理数 $c_f$ 与本原的多项式 $f^{\ast}$ 的积, 即 $f = c_f f^{\ast}$;

    (ii) 若有理数 $r$ 与本原的多项式 $g$ 适合 $f = rg$, 必有 $r = \varepsilon c_f$, $g = \varepsilon^{-1} f^{\ast}$, 其中 $\varepsilon = \pm 1$.

    $c_f$ 称为 $f$ 的容量 \term{capacity}; $f^{\ast}$ 称为 $f$ 的本原的相伴 \term{primitive associate}.
\end{proposition}

\begin{pf}
    (i) 显然.

    (ii) 设 $c_f f^{\ast} = r g$, 其中 $c_f$, $r$ 是有理数, $f^{\ast}$, $g$ 是本原的多项式. 不难看出, $f$ 的次一定等于 $g$ 的次. 设
    \begin{align*}
        f^{\ast} = {} & s_0 + s_1 x + \cdots + s_n x^n, \\
        g = {}        & t_0 + t_1 x + \cdots + t_n x^n.
    \end{align*}
    设 $\frac{c_f}{r} = \frac{p}{q}$, $p$, $q$ 为整数, $q \geq 1$ 且 $p$ 与 $q$ 互素. 所以
    \begin{align*}
        p f^{\ast} = q \frac{c_f f^{\ast}}{r} = q \frac{r g}{r} = q g.
    \end{align*}
    所以
    \begin{align*}
        p s_i = q t_i, \quad i = 0,1,\cdots,n.
    \end{align*}
    因为 $p$ 与 $q$ 互素, 故任取 $g$ 的系数 $t_i$, $p$ 一定是 $t_i$ 的因子. 所以 $p$ 是 $t_0$, $t_1$, $\cdots$, $t_n$ 的公因子. 因为 $t_0$, $t_1$, $\cdots$, $t_n$ 互素, 故 $p$ 是 (整数的) 单位 $\varepsilon_1$. 既然 $p$ 与 $q$ 互素, 则 $q$ 也是 (整数的) 单位 $\varepsilon_2$. 所以
    \begin{align*}
        r = c_f \frac{q}{p} = (\varepsilon_1^{-1} \varepsilon_2) c_f.
    \end{align*}
    从而
    \begin{align*}
        g = f^{\ast} \frac{c_f}{f} = (\varepsilon_1 \varepsilon_2^{-1}) f^{\ast}.
    \end{align*}
    记 $\varepsilon = \varepsilon_1^{-1} \varepsilon_2$, 则 $\varepsilon^{-1} = \varepsilon_1 \varepsilon_2^{-1}$. 因为 $\varepsilon_1$, $\varepsilon_2$ 都是整数的单位, 故 $\varepsilon$ 也是整数的单位. 所以, $\varepsilon = \pm 1$.
\end{pf}

\begin{remark}
    我们可以这么叙述我们刚才证明的命题: 若忽略 (整数的) 单位的区别, 有理系数多项式可唯一地写为有理数与本原的多项式的积.
\end{remark}

\begin{proposition}
    设多项式 $f$, $g$, $h$ 的系数都是整数. 设 $f = gh$.

    (i) 若 $f$ 是本原的, 则 $g$ 与 $h$ 也是本原的;

    (ii) 若 $g$ 与 $h$ 是本原的, 则 $f$ 也是本原的.
\end{proposition}

\begin{pf}
    (i) 反证法. 因为乘法可交换, 故不失一般性, 设 $g$ 不是本原的. 这样, 存在整系数多项式 $\ell$ 与不是 (整数的) 单位的整数 $t$, 使 $g = t\ell$. 这样, $f = t \cdot (\ell h)$. 所以 $t$ 是 $f$ 的所有系数的公因子, 故 $t$ 是 (整数的) 单位的公因子, 即 $t$ 也是 (整数的) 单位. 矛盾!

    (ii) 任取不可约的整数 $p$. 我们证明: 存在 $f$ 的系数 $c$, 使 $p$ 不是 $c$ 的因子. 设
    \begin{align*}
         & g = g_m x^m + g_{m-1} x^{m-1} + \cdots + g_0, \\
         & h = h_n x^n + h_{n-1} x^{n-1} + \cdots + h_0
    \end{align*}
    是二个本原的多项式. 所以, 从次高的项往次低的项看, 一定存在二个整数 $s$, $t$ 使 $p$ 是 $g_m$, $g_{m-1}$, $\cdots$, $g_{s+1}$, $h_n$, $h_{n-1}$, $\cdots$, $h_{t+1}$ 的因子, 但 $p$ 不是 $g_s$ 的因子, 且 $p$ 不是 $h_t$ 的因子. 我们看 $f$ 的 $s+t$ 次系数:
    \begin{align*}
        f_{s+t}
        = {} & g_s h_t + g_{s+1} h_{t-1} + \cdots + g_{s+t} h_0  \\
             & \quad + g_{s-1} h_{t+1} + \cdots + g_{0} h_{s+t}.
    \end{align*}
    由此可见, $p$ 是上式右侧除 $g_s h_t$ 外的任意一项的因子. 这样, $p$ 不是 $f$ 的 $s+t$ 次系数 $f_{s+t}$ 的因子. 所以 $f$ 的全部系数一定互素.
\end{pf}

\begin{proposition}
    设多项式 $f$, $g$ 的系数都是整数.

    (i) 若 $g$ 是本原的, 且存在多项式 $h$ 使 $f = gh$, 则 $h$ 的系数也都是整数;

    (ii) 在 (i) 的基础上, 若还假定 $f$ 也是本原的, 则 $h$ 也是本原的.
\end{proposition}

\begin{pf}
    (i) $f$ 与 $g$ 当然可以视为有理系数多项式. 由带余除法知, $h$ 至少也是有理系数多项式. 将 $h$ 写为 $c_h h^{\ast}$, 其中 $c_h$ 是某有理数, $h$ 是本原的多项式. 所以
    \begin{align*}
        f = gh = g(c_h h^{\ast}) = c_h (g h^{\ast}).
    \end{align*}
    显然, $g h^{\ast}$ 是本原的. 当然, $f$ 也可写为
    \begin{align*}
        f = c_f f^{\ast},
    \end{align*}
    其中 $c_f$ 是整数 (因为 $f$ 的系数都是整数), 且 $f^{\ast}$ 是本原的多项式. 所以, 存在 (整数的) 单位 $\varepsilon$, 使
    \begin{align*}
        c_h = \varepsilon c_f, \quad g h^{\ast} = \varepsilon^{-1} f^{\ast}.
    \end{align*}
    从而
    \begin{align*}
        h = c_h h^{\ast} = \varepsilon c_f h^{\ast}
    \end{align*}
    的系数都是整数.

    (ii) 若 $f$ 也是本原的, 则由 (i) 的证明过程, 知 $h$ 是本原的.
\end{pf}

\begin{proposition}
    设多项式 $f$ 的系数都是整数. 设 $f$ 可写为二个有理系数多项式 $g$, $h$ 的积. 则 $f$ 可写为
    \begin{align*}
        f = c_f g^{\ast} h^{\ast}.
    \end{align*}
    上式应这么理解: 存在 $g$ 的某个本原的相伴 $g^{\ast}$, 存在 $h$ 的某个本原的相伴 $h^{\ast}$, 存在 $f$ 的某个容量 $c_f$, 使上式成立.
\end{proposition}

\begin{pf}
    设 $f = c_f f^{\ast}$, $g = c_g g^{\ast}$, $h = c_h h^{\ast}$, 其中 $f^{\ast}$, $g^{\ast}$, $h^{\ast}$ 都是本原的多项式, $c_g$ 与 $c_h$ 使有理数, 且 $c_f$ (由题设) 是整数. 因为 $f = gh$, 故
    \begin{align*}
        c_f f^{\ast} = (c_g c_h) (g^{\ast} h^{\ast}).
    \end{align*}
    $g^{\ast} h^{\ast}$ 是本原的. 所以, 存在 (整数的) 单位 $\varepsilon$, 使
    \begin{align*}
        c_g c_h = \varepsilon c_f, \quad g^{\ast} h^{\ast} = \varepsilon^{-1} f^{\ast}.
    \end{align*}
    所以
    \begin{align*}
         & f = c_g c_h g^{\ast} h^{\ast} = (\varepsilon c_f) g^{\ast} h^{\ast} = c_f^{\prime} g^{\ast} h^{\ast}. \qedhere
    \end{align*}
\end{pf}

\begin{remark}
    设多项式 $f$ 的系数都是整数. 上个命题表明: 若 $f$ 可写为二个有理系数多项式的积, 则 $f$ 可写为二个整系数多项式的积. 反过来, 因为整数是有理数, 故若 $f$ 可写为二个整系数多项式的积, $f$ 当然可写为二个有理系数多项式的积. 每个有理系数多项式都可写为有理数与本原的多项式的积. 所以, 我们可以借整数的性质研究有理系数多项式是否是可约的.
\end{remark}

\myLine

作者本想到此结束本文. 不过, 抱着认真、负责的态度, 作者再给几个重要的命题就结束本文吧.

先从几个简单的小命题开始吧. 这里, 为了方便, 称正的不可约的整数为素数.

\begin{proposition}
    设 $p$ 是素数. 若 $j$ 是低于 $p$ 的正整数, 则 $p$ 是 (广义) 二项系数 $\binom{p}{j}$ 的因子.
\end{proposition}

\begin{pf}
    易知
    \begin{align*}
        \binom{p}{j} = \frac{p \cdot (p-1) \cdots (p-(j-1))}{j!} = K,
    \end{align*}
    其中 $K$ 是整数. 所以
    \begin{align*}
        p \cdot (p-1) \cdots (p-(j-1)) = K \cdot j!.
    \end{align*}
    我们的目标是: 证明 $p$ 是 $K$ 的因子. 这里, $p$ 已经是 $K \cdot j!$ 的因子了. 如果我们能证明 $p$ 与 $j!$ 互素, 那么 $p$ 一定是 $K$ 的因子. 想法很美好, 是吧? 确实.

    继续分解这个目标. 假如我们能说明 $1$, $2$, $\cdots$, $j$ 都与 $p$ 互素, 那 $1! = 1$ 与 $p$ 互素, $2! = 1! \cdot 2$ 与 $p$ 也互素, $3! = 2! \cdot 3$ 与 $p$ 也互素……一直到 $j! = (j-1)! \cdot j$ 与 $p$ 也互素.

    好! 任取低于 $p$ 的正整数 $\ell$. 我们证明: $p$ 与 $\ell$ 互素. 反证法. 若 $p$ 与 $\ell$ 不互素, 则 $p$ 一定是 $\ell$ 的因子. 所以, 存在整数 $q$ 使 $\ell = pq$. 因为 $\ell \neq 0$, 故 $q \neq 0$, 即 $|q| \geq 1$. 所以
    \begin{align*}
        \ell = |\ell| = |p| |q| \geq |p| \cdot 1 = p.
    \end{align*}
    但是, 这与假定 $\ell < p$ 矛盾. 彳亍. 完了.
\end{pf}

前面, 我们讨论多项式的性质时, 为了简单, 我们把 $f(x)$, $g(x)$, $h(x)$, $\cdots$ 写为 $f$, $g$, $h$, $\cdots$. 现在, 因为我们需要多项式的复合, 我们需要写出被省略的 ``$(x)$''.

\begin{proposition}
    设
    \begin{align*}
        f(x) = a_0 + a_1 x + \cdots + a_n x^n
    \end{align*}
    是有理系数多项式, 且 $n \geq 1$, $a_n \neq 0$ (这表明, $f(x)$ 不是 $0$, 也不是多项式的单位, 且 $f(x)$ 的次为 $n$). 设 $\alpha$, $\beta$ 是有理数, 且 $\alpha \neq 0$. 设
    \begin{align*}
        g(x) = f(\alpha x + \beta) = a_0 + a_1 (\alpha x + \beta) + \cdots + a_n (\alpha x + \beta)^n.
    \end{align*}
    显然, $g(x)$ 也是有理系数多项式, 且次仍为 $n$ ($g(x)$ 的次不超过 $n$, 且其 $n$ 次系数 $a_n \alpha^n \neq 0$). 因为
    \begin{align*}
        x = \alpha \cdot \left( \frac{1}{\alpha} x + \frac{-\beta}{\alpha} \right) + \beta,
    \end{align*}
    故
    \begin{align*}
        f(x) = f\left( \alpha \cdot \left( \frac{1}{\alpha} x + \frac{-\beta}{\alpha} \right) + \beta \right) = g\left( \frac{1}{\alpha} x + \frac{-\beta}{\alpha} \right).
    \end{align*}
    这里, $\frac{1}{\alpha}$, $\frac{-\beta}{\alpha}$ 当然也是有理数, 且 $\frac{1}{\alpha} \neq 0$.

    (i) 若 $f(x)$ 是可约的\myFN{
        这里的 ``可约的'' 是指 $f(x)$ 作为有理系数多项式是可约的 (也就是说, 这是 ``\SomePropertiesOfPolynomials '' 里的 ``可约的''). 作者不打算讨论详细讨论整系数多项式的 ``可约的'' 的含义; 相反, 作者决定只在脚注里简单地提一提.

        设 $f$ 是整系数多项式. 若存在整系数多项式 $g$ 使 $fg = 1$, 则说 $f$ 是整系数多项式的单位. 由此, 读者可以证明: 整系数多项式的单位恰为 $1$, $-1$——这跟有理系数多项式的单位很不一样.

        设 $f$ 是整系数多项式, 且 $f$ 既不是 $0$, 也不是单位. $f$ 作为整系数多项式是可约的, 是指存在二个不是单位的 (整系数) 多项式 $f_1$, $f_2$, 使 $f = f_1 f_2$. 所以, 若把 $4x$ 视为整系数多项式, $4x$ 是可约的: $4 = 4 \cdot x$, 且 $4$ 与 $x$ 都不是 (整系数多项式的) 单位. 但若视 $4x$ 为有理系数多项式, 则 $4x$ 当然是不可约的.

        作者不希望这些小差异影响读者. 而且, 这不是什么很重要的点.
    }, 则 $g(x)$ 是可约的;

    (ii) 若 $g(x)$ 是可约的, 则 $f(x)$ 是可约的.

    简单点说, ``$f(x)$ 是可约的 (不可约的)'' 的一个必要与充分条件是: ``$f(\alpha x + \beta)$ ($\alpha$, $\beta$ 是有理数, 且 $\alpha \neq 0$) 是可约的 (不可约的)''.
\end{proposition}

\begin{pf}
    事实上, 我们只要证明 (i). (ii) 的证明就是把 (i) 的证明里的 $f$ 与 $g$ 互换, 且 $\alpha$, $\beta$ 分别换为 $\frac{1}{\alpha}$, $\frac{-\beta}{\alpha}$.

    设 $f(x)$ 是可约的. 所以, 存在二个不是单位的 (次高于 $0$ 的) 多项式 $f_1 (x)$, $f_2 (x)$ 使
    \begin{align*}
        f(x) = f_1 (x) f_2 (x).
    \end{align*}
    记
    \begin{align*}
        g_1 (x) = f_1 (\alpha x + \beta), \quad g_2 (x) = f_2 (\alpha x + \beta),
    \end{align*}
    则
    \begin{align*}
        g(x) = f(\alpha x + \beta) = f_1 (\alpha x + \beta) f_2 (\alpha x + \beta) = g_1 (x) g_2 (x).
    \end{align*}
    因为 $\deg g_1 (x) = \deg f_1 (x)$, $\deg g_2 (x) = \deg f_2 (x)$, 故 $g_1 (x)$, $g_2 (x)$ 都不是单位. 从而 $g(x)$ 也是可约的.
\end{pf}

\begin{remark}
    有一点值得读者注意.

    设 $f(x) = x + 4$. 显然, $f(x)$ 是不可约的.

    设 $g(x) = f(x^2) = x^2 + 4$. 我们证明: $g(x)$ 是不可约的.

    反证法. 假定存在二个有理系数多项式 $g_1 (x)$, $g_2 (x)$ 使
    \begin{align*}
        g(x) = g_1 (x) g_2 (x),
    \end{align*}
    且 $g_1 (x)$, $g_2 (x)$ 都不是单位. 根据前面的命题, 可进一步假定 $g_1 (x)$, $g_2 (x)$ 的系数都是整数 (这可以简化讨论). 因为 $\deg g_1 (x) + \deg g_2 (x) = 2$, 而 $\deg g_1 (x) > 0$, $\deg g_2 (x) > 0$, 故 $g_1 (x)$ 与 $g_2 (x)$ 的次都是 $1$. 所以, 设
    \begin{align*}
        g_1 (x) = ax + b, \quad g_2 (x) = cx + d,
    \end{align*}
    其中 $a$, $b$, $c$, $d$ 都是整数. 从而
    \begin{align*}
        x^2 + 4 = (ax + b)(cx + d) = (ac) x^2 + (ad + bc) x + (bd),
    \end{align*}
    也就是
    \begin{align*}
        ac = 1, \quad ad + bc = 0, \quad bd = 4.
    \end{align*}
    由 $ac = 1$ 知 $a = c = 1$ 或 $a = c = -1$. 所以
    \begin{align*}
         & b + d = \frac{ad + ba}{a} = \frac{ad + bc}{a} = 0, \\
         & bd = 4.
    \end{align*}
    消去 $d$, 有
    \begin{align*}
        b^2 = -4.
    \end{align*}
    看到这里, 读者可能笑了: 整数的平方不可能是 $-4$ 呀! 所以, $g(x)$ 一定是不可约的.

    设 $h(x) = g(x^2) = x^4 + 4$. 我们证明: $h(x)$ 是可约的.

    这里就没必要反证了. 作者直接点吧. 无非就是添平方嘛! 具体一点, 就是
    \begin{align*}
        x^4 + 4
        = {} & x^4 + 4x^2 + 4 - 4x^2          \\
        = {} & (x^2 + 2)^2 - (2x)^2           \\
        = {} & (x^2 + 2x + 2) (x^2 - 2x + 2).
    \end{align*}
    显然 $x^2 \pm 2x + 2$ 不是单位. 所以, $h(x)$ 是可约的.

    设 $\ell (x) = h(x^2) = x^8 + 4$. 显然,
    \begin{align*}
        x^8 + 4
        = {} & (x^2)^4 + 4                               \\
        = {} & ((x^2)^2 + 2x^2 + 2) ((x^2)^2 - 2x^2 + 2) \\
        = {} & (x^4 + 2x^2 + 2) (x^4 - 2x^2 + 2),
    \end{align*}
    且 $x^4 \pm 2x^2 + 2$ 不是单位, 故 $\ell (x)$ 是可约的.

    作者举这个例的目的是提醒读者: 上个命题的 $\alpha x + \beta$ 不能改为较高次的多项式; 否则, 命题不一定成立.
\end{remark}

\begin{definition}
    设
    \begin{align*}
        f(x) = a_0 + a_1 x + \cdots + a_n x^n
    \end{align*}
    是多项式, $a_n \neq 0$, 且 $a_0 \neq 0$. $f(x)$ 的反多项式 \term{reciprocal polynomial} 是
    \begin{align*}
        f^{\mathrm{r}} (x) = a_n + a_{n-1} x + \cdots + a_0 x^n.
    \end{align*}
    也就是说, $f^{\mathrm{r}} (x)$ 的 $j$ 次系数是 $a_{n-j}$ ($j = 0,1,\cdots,n$).

    请读者注意: 上面的 $f(x)$ 的 $0$ 次系数不是 $0$. 如果 $a_0 = 0$, 它的反多项式是未定义的.
\end{definition}

\begin{example}
    设
    \begin{align*}
        f(x) = 1 + 3x + 6x^2 + 10x^3 + 15x^4 + 21x^5.
    \end{align*}
    所以
    \begin{align*}
        f^{\mathrm{r}} (x) = 21 + 15x + 10x^2 + 6x^3 + 3x^4 + x^5.
    \end{align*}
\end{example}

\begin{example}
    设
    \begin{align*}
        g(x) = -6 - 5 (x - 1) + 2 (x - 1)^2 + (x - 1)^3.
    \end{align*}
    读者可能会觉得
    \begin{align*}
        g^{\mathrm{r}} (x) = 1 + 2 (x - 1) - 5 (x - 1)^2 - 6 (x - 1)^3.
    \end{align*}
    但这不对. 按照定义, 我们要先展开 $g(x)$:
    \begin{align*}
        g(x)
        = {} & -6 - 5(x - 1) + 2 (x^2 - 2x + 1) + (x^3 - 3x^2 + 3x - 1) \\
        = {} & -6 + (-5x + 5) + (2x^2 - 4x + 2) + (x^3 - 3x^2 + 3x - 1) \\
        = {} & -6x - x^2 + x^3.
    \end{align*}
    由此可见, $g(x)$ 的 $0$ 次系数为 $0$. 所以, $g^{\mathrm{r}} (x)$ 是未定义的.
\end{example}

\begin{proposition}
    设
    \begin{align*}
        f(x) = a_0 + a_1 x + \cdots + a_n x^n
    \end{align*}
    是多项式, $a_n \neq 0$, 且 $a_0 \neq 0$.

    (i) $f(x)$ 的反多项式 $f^{\mathrm{r}} (x)$ 的次仍为 $n$;

    (ii) $f^{\mathrm{r}}(x)$ 的反多项式 $(f^{\mathrm{r}})^{\mathrm{r}} (x)$ 是 $f(x)$;

    (iii) 若 $t$ 是非零数, 则
    \begin{align*}
        f^{\mathrm{r}} (t) = t^n f \left( \frac{1}{t} \right).
    \end{align*}
\end{proposition}

\begin{pf}
    (i) $f(x)$ 的反多项式是
    \begin{align*}
        f^{\mathrm{r}} (x) = a_n + a_{n-1} x + \cdots + a_0 x^n.
    \end{align*}
    因为 $a_0 \neq 0$, 故 $f^{\mathrm{r}} (x)$ 的次仍为 $n$.

    (ii) 设
    \begin{align*}
        b_j = a_{n - j}, \quad j = 0,1,\cdots,n.
    \end{align*}
    则 $f(x)$ 的反多项式可写为
    \begin{align*}
        f^{\mathrm{r}} (x) = b_0 + b_1 x + \cdots + b_n x^n.
    \end{align*}
    因为 $b_0 = a_n \neq 0$, 且 $b_n = a_0 \neq 0$, 故 $f^{\mathrm{r}}(x)$ 的反多项式是
    \begin{align*}
        (f^{\mathrm{r}})^{\mathrm{r}} (x)
        = {} & b_n + b_{n-1} x + \cdots + b_0 x^n \\
        = {} & a_0 + a_1 x + \cdots + a_n x^n     \\
        = {} & f(x).
    \end{align*}

    (iii) 设 $t$ 是非零数. 则
    \begin{align*}
        f \left( \frac{1}{t} \right)
        = {} & a_0 + a_1 \frac{1}{t} + a_2 \left( \frac{1}{t} \right)^2 + \cdots + a_n \left( \frac{1}{t} \right)^n \\
        = {} & a_0 + a_1 \frac{1}{t} + a_2 \frac{1}{t^2} + \cdots + a_n \frac{1}{t^n}                               \\
        = {} & a_0 \frac{t^n}{t^n} + a_1 \frac{t^{n-1}}{t^n} + a_2 \frac{t^{n-2}}{t^n} + \cdots + a_n \frac{1}{t^n} \\
        = {} & \frac{a_0 t^n + a_1 t^{n-1} + a_2 t^{n-2} + \cdots + a_n}{t^n}                                       \\
        = {} & \frac{a_n + a_{n-1} t + \cdots + a_0 t^n}{t^n}                                                       \\
        = {} & \frac{f^{\mathrm{r}} (t)}{t^n}.
    \end{align*}
    所以
    \begin{align*}
         & f^{\mathrm{r}} (t) = t^n f \left( \frac{1}{t} \right). \qedhere
    \end{align*}
\end{pf}

\begin{proposition}
    设
    \begin{align*}
         & f(x) = a_0 + a_1 x + \cdots + a_n x^n,    \\
         & f_1 (x) = p_0 + p_1 x + \cdots + p_u x^u, \\
         & f_2 (x) = q_0 + q_1 x + \cdots + q_v x^v
    \end{align*}
    是多项式, 其中 $p_u \neq 0$, $q_v \neq 0$, $a_n \neq 0$ 且 $a_0 \neq 0$. 设
    \begin{align*}
        f(x) = f_1 (x) f_2 (x).
    \end{align*}

    (i) $u + v = n$;

    (ii) $p_0 \neq 0$, $q_0 \neq 0$;

    (iii) $f^{\mathrm{r}} (x) = f_1^{\mathrm{r}} (x) f_2^{\mathrm{r}} (x)$.
\end{proposition}

\begin{pf}
    (i) 显然\myFN{因为 $f(x) = f_1 (x) f_2 (x)$, 故 $\deg f(x) = \deg f_1 (x) + \deg f_2 (x)$.}.

    (ii) 因为 $f(x) = f_1 (x) f_2 (x)$, 故 $p_0 q_0 = a_0$. 因为 $a_0 \neq 0$, 故 $p_0 \neq 0$, $q_0 \neq 0$.

    (iii) 因为 $a_0 \neq 0$, $p_0 \neq 0$, $q_0 \neq 0$, 故 $f(x)$, $f_1 (x)$, $f_2 (x)$ 都有反多项式:
    \begin{align*}
         & f^{\mathrm{r}} (x) = a_n + a_{n-1} x + \cdots + a_0 x^n,   \\
         & f_1^{\mathrm{r}} (x) = p_u + p_{u-1} x + \cdots + p_0 x^u, \\
         & f_2^{\mathrm{r}} (x) = q_v + q_{v-1} x + \cdots + q_0 x^v.
    \end{align*}
    记
    \begin{align*}
        E(x) = f^{\mathrm{r}} (x) - f_1^{\mathrm{r}} (x) f_2^{\mathrm{r}} (x).
    \end{align*}
    任取非零数 $t$. 则
    \begin{align*}
        E(t)
        = {} & f^{\mathrm{r}} (t) - f_1^{\mathrm{r}} (t) f_2^{\mathrm{r}} (t)                                           \\
        = {} & t^n f \left( \frac{1}{t} \right) - t^u f_1 \left( \frac{1}{t} \right) t^v f_2 \left( \frac{1}{t} \right) \\
        = {} & t^n f \left( \frac{1}{t} \right) - t^u t^v f_1 \left( \frac{1}{t} \right) f_2 \left( \frac{1}{t} \right) \\
        = {} & t^n f \left( \frac{1}{t} \right) - t^{u+v} f \left( \frac{1}{t} \right)                                  \\
        = {} & t^n f \left( \frac{1}{t} \right) - t^n f \left( \frac{1}{t} \right)                                      \\
        = {} & 0.
    \end{align*}
    这说明, 多项式 $E(x)$ 有无限多个根. 所以, $E(x)$ 一定是零多项式, 即
    \begin{align*}
         & f^{\mathrm{r}} (x) = f_1^{\mathrm{r}} (x) f_2^{\mathrm{r}} (x). \qedhere
    \end{align*}
\end{pf}

\begin{proposition}
    设
    \begin{align*}
         & f(x) = a_0 + a_1 x + \cdots + a_n x^n,    \\
         & f_1 (x) = p_0 + p_1 x + \cdots + p_u x^u, \\
         & f_2 (x) = q_0 + q_1 x + \cdots + q_v x^v
    \end{align*}
    是多项式, 其中 $p_u \neq 0$, $q_v \neq 0$, $a_n \neq 0$ 且 $p_0 \neq 0$, $q_0 \neq 0$, $a_0 \neq 0$. 设
    \begin{align*}
        f^{\mathrm{r}} (x) = f_1^{\mathrm{r}} (x) f_2^{\mathrm{r}} (x).
    \end{align*}
    则
    \begin{align*}
        f(x) = f_1 (x) f_2 (x).
    \end{align*}
\end{proposition}

\begin{pf}
    因为 ($0$ 次系数不为 $0$ 的) 多项式 $g$ 的反多项式的反多项式是 $g$, 故由上命题知本命题也对.
\end{pf}

跟前面的 ``$\alpha x + \beta$'' 类似, 我们有
\begin{proposition}
    设多项式 $f(x)$ 既不是 $0$, 也不是单位, 且 $0$ 次系数不为 $0$. ``$f(x)$ 是可约的 (不可约的)'' 的一个必要与充分条件是: ``反多项式 $f^{\mathrm{r}} (x)$ 是可约的 (不可约的)''.
\end{proposition}

\begin{pf}
    这是作者留给读者的练习题; 请读者尝试自行补充细节 (可以说, 这跟 ``$\alpha x + \beta$'' 几乎一致).
\end{pf}

抱歉, 读者朋友. 作者一不小心, 又写多了. ``喧宾夺主了, 属于是.'' 所以, 请读者消化一下.

\myLine

我们继续吧!

本来, 作者只想用下面的判别法结束本文, 但又觉得只是冷冰冰地丢下一个判别法不是很负责. 所以, 作者决定加几个例. 为尽可能地消除读者的疑惑, 作者再加了一点细节. 加着加着, 又写多了……

下述命题一般被称为 Eisenstein 判别法 \term{Eisenstein criterion}. 当然了, 此命题仅仅是 ``$f$ 是不可约的'' 的一个充分条件哟.

\begin{proposition}
    设多项式
    \begin{align*}
        f = a_0 + a_1 x + \cdots + a_n x^n
    \end{align*}
    的系数都是整数. 若存在不可约的整数 $p$ 适合如下三条件, 则 $f$ 是不可约的:

    (i) $p$ 不是 $a_n$ 的因子 (这说明 $a_n \neq 0$);

    (ii) $p$ 是 $a_{n-1}$, $a_{n-2}$, $\cdots$, $a_0$ 的因子.

    (iii) $p^2$ 不是 $a_0$ 的因子 (这说明 $a_0 \neq 0$).
\end{proposition}

\begin{pf}
    用反证法. 设二个 (系数都是整数的) 的多项式 $g$, $h$ 使 $f = gh$, 其中
    \begin{align*}
         & g = g_0 + g_1 x + \cdots + g_\ell x^\ell, \\
         & h = h_0 + h_1 x + \cdots + h_m x^m
    \end{align*}
    都不是单位, 且 $g_\ell \neq 0$, $h_m \neq 0$. 所以 $1 \leq \ell < n$, $1 \leq m < n$, 且 $\ell + m = n$.

    因为 $p$ 是 $a_0 = g_0 h_0$ 的因子, 故 $p$ 是 $g_0$ 的因子, 或 $p$ 是 $h_0$ 的因子. 因为 $p^2$ 不是 $a_0$ 的因子, 故 $p$ 不可能是 $g_0$ 与 $h_0$ 的公因子. 不失一般性, 设 $p$ 是 $g_0$ 的因子, 且 $p$ 不是 $h_0$ 的因子. 因为 $p$ 不是 $a_n = g_\ell h_m$ 的因子, 故 $p$ 不是 $g_\ell$ 的因子. 所以, 从次低的项往次高的项看, 必有整数 $k$ ($1 \leq k \leq \ell < n$) 使 $p$ 是 $g_0$, $g_1$, $\cdots$, $g_{k-1}$ 的因子, 但 $p$ 不是 $g_k$ 的因子. 所以
    \begin{align*}
        a_k = g_k h_0 + g_{k-1} h_1 + \cdots + g_0 h_k.
    \end{align*}
    由此可见, $p$ 是上式右侧除 $g_k h_0$ 外的任意一项的因子. 这样, $p$ 不是 $a_k$ 的因子. 这跟 $k < n$ 矛盾!
\end{pf}

我们用几个例帮助读者消化此判别法.

\begin{example}
    我们可以随手写出任意次的不可约的多项式: $x^n + p$, 这里 $p$ 是某个不可约的整数, 且 $n$ 是正整数.
\end{example}

\begin{example}
    设 $h = x^{m+1} + x^n - 2$, 其中 $m$ 是正整数. 所以, $h$ 的次不低于 $2$. 所以, $h$ 既不是 $0$, 也不是单位. 这样, $h$ 一定可以写为不可约的多项式的积. 我们试写 $h$ 为不可约的多项式的积.

    读者可能还记得
    \begin{align*}
        f^n - g^n = (f - g)(f^{n-1} + f^{n-2} g + \cdots + f^{n-i} g^{i-1} + \cdots + g^{n-1}).
    \end{align*}
    由此, 我们可以将 $h$ 改写为
    \begin{align*}
        h = (x^{m+1} - 1) + (x^m - 1).
    \end{align*}
    因为
    \begin{align*}
             & x^n - 1                                                                                \\
        = {} & x^n - 1^n                                                                              \\
        = {} & (x - 1)(x^{n-1} + x^{n-2} \cdot 1 + \cdots + x^{n-i} \cdot 1^{i-1} + \cdots + 1^{n-1}) \\
        = {} & (x - 1)(x^{n-1} + x^{n-2} + \cdots + 1),
    \end{align*}
    故
    \begin{align*}
        h
        = {} & (x - 1)(x^m + x^{m-1} + x^{m-2} + \cdots + 1)                     \\
             & \qquad + (x - 1)(x^{m-1} + x^{m-2} + \cdots + 1)                  \\
        = {} & (x - 1)\underbrace{(x^m + 2x^{m-1} + 2x^{m-2} + \cdots + 2)}_{q}.
    \end{align*}
    显然 $x-1$ 是不可约的. 我们再看看 $q$ 是不是可约的. 读者不难看出, 取 $p = 2$, 则 $p$ (与 $q$) 适合 Eisenstein 判别法的三条件, 故 $q$ 是不可约的. 所以, $h$ 可写为二个不可约的多项式的积: $(x - 1) \cdot q$.
\end{example}

\begin{example}
    设 $f = -7 + 9x + 3x^6$. 不难看出, 不存在不可约的整数 $p$ 适合 Eisenstein 判别法的三条件.

    因为 (不是单位的) 多项式与其反多项式 (若存在) 要么都是可约的, 要么都是不可约的, 所以我们可以试试\myFN{只是 ``试试''; 不一定管用哟.}反多项式. $f$ 的 $0$ 次系数不是 $0$, 故 $f$ 的反多项式存在, 且 $f^{\mathrm{r}} = 3 + 9x^5 - 7x^6$. 由此可见, 取 $p = 3$, 则 $p$ (与 $f^{\mathrm{r}}$) 适合 Eisenstein 判别法的三条件, 故 $f^{\mathrm{r}}$ 是不可约的. 从而 $f$ 也是不可约的.
\end{example}

一般地, 下面的 Eisenstein 判别法的变体成立:

\begin{proposition}
    设多项式
    \begin{align*}
        f = a_0 + a_1 x + \cdots + a_n x^n
    \end{align*}
    的系数都是整数. 若存在不可约的整数 $p$ 适合如下三条件, 则 $f$ 是不可约的:

    (i) $p$ 不是 $a_0$ 的因子 (这说明 $a_0 \neq 0$);

    (ii) $p$ 是 $a_1$, $a_2$, $\cdots$, $a_n$ 的因子.

    (iii) $p^2$ 不是 $a_n$ 的因子 (这说明 $a_n \neq 0$).
\end{proposition}

\begin{pf}
    考虑 $f$ 的反多项式 $f^{\mathrm{r}}$. 对 $f^{\mathrm{r}}$ 施行 Eisenstein 判别法, 可知 $f^{\mathrm{r}}$ 是不可约的. 故 $f$ 也是不可约的. (作者邀请感兴趣的读者补全细节.)
\end{pf}

我们用一个经典的例结束本文.

\begin{example}
    设 $q$ 是素数. 设
    \begin{align*}
        f(x) = 1 + x + \cdots + x^{q-2} + x^{q-1}.
    \end{align*}
    我们证明: $f(x)$ 是不可约的.

    显然, 不存在不可约的整数 $p$ 适合 Eisenstein 判别法的三条件. 反多项式也没有帮助: $f(x)$ 的反多项式刚好是 $f(x)$. 我们试试 ``$\alpha x + \beta$'' 吧.

    考虑
    \begin{align*}
        g(x) = f(x + 1) = (1 + x)^0 + (1 + x)^1 + \cdots + (1 + x)^{q - 2} + (1 + x)^{q - 1}.
    \end{align*}
    我们需要展开 $g(x)$. 我们知道, $(1 + x)^\ell$ 的 $j$ 次系数是 $\binom{\ell}{j}$. 所以, $g(x)$ 的 $j$ 次系数是
    \begin{align*}
        \binom{0}{j} + \binom{1}{j} + \cdots + \binom{q-2}{j} + \binom{q-1}{j} = \binom{q}{j + 1}.
    \end{align*}
    也就是说,
    \begin{align*}
        g(x) = \binom{q}{1} + \binom{q}{2} x + \cdots + \binom{q}{q - 1} x^{q-2} + x^{q-1}.
    \end{align*}
    取 $p = q$. 因为 $q$ 是素数 (正的不可约的整数), $p$ 当然是不可约的整数. 由此可见, $p$ (与 $g(x)$) 适合 Eisenstein 判别法的三条件, 故 $g(x)$ 是不可约的. 故 $f(x)$ 也是不可约的.
\end{example}

感谢读者的阅读! 再见.
