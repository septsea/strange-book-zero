\subsection*{\RationalRootsOfPolynomialsOverQ}
\addcontentsline{toc}{subsection}{\RationalRootsOfPolynomialsOverQ}
\markright{\RationalRootsOfPolynomialsOverQ}

本文讨论有理系数多项式的有理根.

首先, 作者还是给一个定义吧; 作者怕读者不知道什么叫 ``有理根''.

\begin{definition}
    设 $f(x)$ 是有理系数多项式. 若有理数 $a$ 适合 $f(a) = 0$, 则 $a$ 是 $f(x)$ 的有理根 \term{rational root}.
\end{definition}

\begin{example}
    设 $a$, $b$ 是有理数, 且 $a \neq 0$. $-\frac{b}{a}$ 是 $f(x) = ax + b$ 的有理根.
\end{example}

\begin{example}
    设 $f(x) = x^2 - 1$. $\pm 1$ 是 $f(x)$ 的有理根.
\end{example}

读者或许还能记起
\begin{proposition}
    设 $f(x)$ 是 $n$ 次多项式 ($n \geq 1$), $a$ 是数. 则存在 $n-1$ 次多项式 $q(x)$ 使
    \begin{align*}
        f(x) = q(x) (x-a) + f(a).
    \end{align*}
    根据带余除法, 这样的 $q(x)$ 一定是唯一的.
\end{proposition}

利用此命题, 我们可得
\begin{proposition}
    设 $f(x)$ 是 $n$ 次多项式 ($n \geq 1$). $a$ 是 $f(x)$ 的根的一个必要与充分条件是: 存在 $n-1$ 次多项式 $q(x)$ 使
    \begin{align*}
        f(x) = q(x) (x-a).
    \end{align*}
    根据带余除法, 这样的 $q(x)$ 一定是唯一的.
\end{proposition}

此命题告诉我们: 找到 $f$ 的有理根, 就相当于找到了 $f$ 的 $1$ 次因子. 在前面, 我们也知道, $1$ 次 (有理系数) 多项式总是不可约的. 本文就是要告诉读者如何寻找有理系数多项式的最简单的不可约的因子——$1$ 次因子.

前面, 我们也讨论过整系数多项式与有理系数多项式的关系. 作者帮读者整理一下知识要点:

(i) 整系数多项式当然是有理系数多项式. 虽然有理系数多项式不一定是整系数多项式, 可我们总能把非零的有理系数多项式写为一个有理数与整系数多项式的积. 我们可以进一步要求, 此整系数多项式的系数互素. 一般地, 系数全是整数, 且系数还互素的多项式是本原的多项式.

(ii) 二个本原的多项式的积也是本原的; 若本原的多项式是二个整系数多项式 $g$, $h$ 的积, 则 $g$ 与 $h$ 也是本原的.

(iii) 若 $f$ 是整系数多项式, 且本原的多项式 $g$ 是 $f$ 的因子, 则存在整系数多项式 $h$ 使 $f = gh$.

(iv) 若一个整系数多项式 $f$ 可写为二个有理系数多项式的积, 则 $f$ 也一定可写为二个整系数多项式的积. 反过来, 若整系数多项式 $f$ 可写为二个整系数多项式的积, 则 $f$ 已经是二个有理系数多项式的积了.

比方说, 我们想研究
\begin{align*}
    f = 1 + \frac{1}{2}x + \frac{1}{3}x^2 + \frac{1}{4}x^3.
\end{align*}
我们可以 ``通分'':
\begin{align*}
    f
    = {} & \frac{24}{24} + \frac{12}{24} x + \frac{8}{24} x^2 + \frac{6}{24} x^3 \\
    = {} & \frac{1}{24} (24 + 12x + 8x^2 + 6x^3)                                 \\
    = {} & \frac{2}{24} (12 + 6x + 4x^2 + 3x^3)                                  \\
    = {} & \frac{1}{12} \underbrace{(12 + 6x + 4x^2 + 3x^3)}_{f^{\ast}}.
\end{align*}
这样, 我们把 $f$ 化为了一个有理数与整系数多项式 $f^{\ast}$ 的积; 读者不难看出, $f^{\ast}$ 的系数互素, 故 $f^{\ast}$ 是本原的.

读者朋友也应该看到了, 虽然 ``\HEADING '' 主要讨论多项式, 可如果我们想对多项式作更深入的研究, 我们必须好好地研究多项式的系数. 或许, 有理系数多项式是读者经常遇见的对象——我们从初中就知道了. 有理数与整数的关系甚是密切——有理数是二个整数的比 (分母自然是非零的). 所以, 作者用很多文字, 带领读者好好地回忆小学就学过的 ``因数'' ``最大公因数'' ``素数''……当然, 为了让读者更好地关联知识 (使读者更好地理解同人文里整数的与多项式的知识), 作者用 ``因子'' 代替 ``因数''. 为了让读者了解素数的含义, 作者引入 ``不可约的整数'', 并指出: 正的不可约的整数即为素数. 读者了解足够多的整数知识后, 就可以具体地讨论有理系数多项式的因子分解了 (作者之后会正式地引入它). 作者再说一次: 本文就是要告诉读者如何用整数的性质找有理系数多项式的最简单的不可约的因子——$1$ 次因子. 准备好了吗? 我们开始了哟.

\begin{proposition}
    设整数 $u$, $v$ 互素, 且 $u \neq 0$. 这样, $g = ux - v$ 是本原的 $1$ 次多项式. 设 $f$ 是整系数多项式. 若 $g$ 是 $f$ 的因子, 则 $u$ 是 $f$ 的首项系数的因子, 且 $v$ 是 $f$ 的 $0$ 次系数的因子.
\end{proposition}

\begin{pf}
    设
    \begin{align*}
        f = a_n x^n + a_{n-1} x^{n-1} + \cdots + a_1 x + a_0
    \end{align*}
    是整系数多项式, 且 $a_n \neq 0$. 因为 $g = ux - v$ 是本原的, $g$ 是 $f$ 的因子, 故存在整系数多项式 $h$ 使 $f = gh$. 因为 $\deg f = \deg g + \deg h$, 故 $\deg h = n - 1$. 所以, 可设
    \begin{align*}
        h = b_{n-1} x^{n-1} + \cdots + b_1 x + b_0,
    \end{align*}
    且 $b_{n-1} \neq 0$. 从而
    \begin{align*}
        a_n = ub_{n-1}, \quad a_0 = -vb_0.
    \end{align*}
    故 $u$ 是 $a_n$ 的因子, 且 $v$ 是 $a_0$ 的因子.
\end{pf}

\begin{remark}
    事实上, 找出所有的本原的 $1$ 次因子, 也就找到了所有的 $1$ 次因子——也就差个单位. 此命题充分地运用了 (ii).
\end{remark}

既然 $ux - v$ 是 $f$ 的因子, 那么有多项式 $h$ 使
\begin{align*}
    f = (ux - v) h.
\end{align*}
适当地改写一下:
\begin{align*}
    f = \left( x - \frac{v}{u} \right) (uh)
\end{align*}
由此可见, $\frac{v}{u}$ 是 $f$ 的根. 所以, 用根的语言描述上个命题, 就是
\begin{proposition}
    设整数 $u$, $v$ 互素, 且 $u \neq 0$. 设 $f$ 是整系数多项式. 若 $\frac{v}{u}$ 是 $f$ 的根, 则 $u$ 是 $f$ 的首项系数的因子, 且 $v$ 是 $f$ 的 $0$ 次系数的因子.
\end{proposition}

\begin{pf}
    把上个命题用根的语言说一遍就彳亍了.
\end{pf}

\begin{remark}
    每个有理数都可约分. 所以, 此命题指出: 若有理数 $r$ 是 $f(x)$ 的根, 则 $r$ 的 ``最简形式'' 一定适合某些 ``因子条件''.
\end{remark}

\begin{remark}
    我们也可以如此证明此命题. 还是设
    \begin{align*}
        f = a_n x^n + a_{n-1} x^{n-1} + \cdots + a_1 x + a_0
    \end{align*}
    是整系数多项式. 既然 $\frac{v}{u}$ 是 $f$ 的根, 那么
    \begin{align*}
        0 = a_n \left( \frac{v}{u} \right)^n + a_{n-1} \left( \frac{v}{u} \right)^{n-1} + \cdots + a_1 \frac{v}{u} + a_0.
    \end{align*}
    等式二侧同乘 $u^n$, 有
    \begin{align*}
        0 = a_n v^n + a_{n-1} v^{n-1} u + \cdots + a_{i} v^{i} u^{n-i} + \cdots + a_1 v u^{n-1} + a_0 u^n. \tag*{(\myStar)}
    \end{align*}

    将 (\myStar) 改写为
    \begin{align*}
        -a_n v^n = u (a_{n-1} v^{n-1} + \cdots + a_{i} v^{i} u^{n-i-1} + \cdots + a_1 v u^{n-2} + a_0 u^{n-1}).
    \end{align*}
    上式右侧是 $u$ 的因子, 故上式左侧也是 $u$ 的因子. 因为 $u$ 与 $v$ 互素, 故 $u$ 与 $-v^n$ 互素. 所以 $u$ 是 $a_n$ 的因子.

    将 (\myStar) 改写为
    \begin{align*}
        -a_0 u^n = v (a_n v^{n-1} + a_{n-1} v^{n-2} u + \cdots + a_{i} v^{i-1} u^{n-i} + \cdots + a_1 u^{n-1}).
    \end{align*}
    上式右侧是 $v$ 的因子, 故上式左侧也是 $v$ 的因子. 因为 $u$ 与 $v$ 互素, 故 $-u^n$ 与 $v$ 互素. 所以 $v$ 是 $a_0$ 的因子.

    作者承认, 这个证明简单一些. 不过, 借助本原的多项式, 我们可以方便地获取更多情报. 读者马上就会看到这一点.
\end{remark}

\begin{remark}
    如果 $f$ 的 $0$ 次系数为 $0$, 则因每个整数都是 $0$ 的因子, 可知此命题在 ``说废话''. 它真地在说废话吗? 嘛, 它本身不会说话; 这取决于读者怎么想.

    设 $f = x^5 + x^3 - 2x^2$. 明显地, $x^2$ 是 $x^5$, $x^3$, $-2x^2$ 的公因子. 所以, $f = x^2 (x^3 + x^2 - 2)$. 如果读者还记得 $m$ 重因子的概念, 读者不难看出, $x$ 是 $f$ 的 $2$ 重因子. $g = x^3 + x^2 - 2$ 的 $0$ 次系数不是 $0$. 所以, 若 $ux - v$ 是 $g$ 的因子, 且 $u$ 与 $v$ 互素, 则 $u$ 是 $1$ 的因子, $v$ 是 $2$ 的因子. 这还是在 ``说废话'' 吗? $g$ 的 $1$ 次因子就不是 $f$ 的 $1$ 次因子了吗? 读者一定不要 ``死'' 学算学呀!

    一般地, 若 $x$ 是 $f(x)$ 的 $m$ 重因子 ($m \geq 1$), 我们总是可写 $f(x)$ 为 $x^m g(x)$ 的形式, 且 $x = x - 0$ 不是 $g(x)$ 的因子. 因为 $x - 0$ 除 $g$ 的余式是 $g(0)$, 故 $g(0) \neq 0$. 因为 $g(x)$ 的 $0$ 次系数是 $g(0)$, 故 $g(x)$ 的 $0$ 次系数不是 $0$.

    作者再具体一点吧. 设
    \begin{align*}
        f(x) = a_n x^n + a_{n-1} x^{n-1} + \cdots + a_{m+1} x^{m+1} + a_m x^m,
    \end{align*}
    且 $n \geq m \geq 1$, $a_n \neq 0$, $a_m \neq 0$. $f(x)$ 的每一项 $a_i x^i$ ($m \leq i \leq n$) 均可写为 $x^m \cdot a_i x^{i-m}$. 所以
    \begin{align*}
        f(x) = x^m \underbrace{(a_n x^{n-m} + a_{n-1} x^{n-m-1} + \cdots + a_{m+1} x + a_m)}_{g(x)}.
    \end{align*}
    不难看出, $g(x)$ 的 $0$ 次系数 $a_m \neq 0$.

    以后, 当读者要寻找 $f$ 的 $1$ 次因子时, 如果 $x$ 的幂是 $f$ 的因子, 就先将它与 $f$ ``分离''.
\end{remark}

设 $f$ 是 $n$ 次整系数多项式. 怎么方便地判断一次因式 $ux - v$ 是 $f$ 的因子呢? 可考虑综合除法. 一次因式 $ux - v$ 是 $f$ 的因子, 相当于 $x - \frac{v}{u}$ 是 $f$ 的因子. 作者在前面也讲过, 计算 $x - a$ 除 $f$ 的一个简便方法就是综合除法——$n$ 次加法与 $n$ 次乘法即可给出商与余式. 这样, 剩下的问题就是: 列出适合上个命题条件的全部 $\frac{v}{u}$.

设 $f$ 的首项系数是 $a_n$, $0$ 次系数是 $a_0$. 根据前面的评注, 我们可假定 $a_0 \neq 0$. 所以, $a_n$ 与 $a_0$ 的因子的数目都是有限多个的. 设 $a_n$ 的全部因子是 $\pm u_1$, $\pm u_2$, $\cdots$, $\pm u_s$; 设 $a_0$ 的全部因子是 $\pm v_1$, $\pm v_2$, $\cdots$, $\pm v_t$. 这样, $\frac{v}{u}$ 必形如\myFN{$\frac{+v}{+u}$ 与 $\frac{-v}{-u}$ 表示同一个数 $\frac{v}{u}$, 且 $\frac{-v}{+u}$ 与 $\frac{+v}{-u}$ 表示同一个数 $-\frac{v}{u}$. 因为现假定 $a_0 \neq 0$, 而首项系数 $a_n$ 也非零, 故 $\frac{v}{u}$ 与 $-\frac{v}{u}$ 是二个不同的数. 表面上, 将 $\pm v$ 放在分子, $\pm u$ 放在分母, 会产生四个数; 实则恰有二个.}
\begin{align*}
    \pm \frac{v_j}{u_i}, \qquad 1 \leq i \leq s, \quad 1 \leq j \leq t.
\end{align*}
当然, 还有一个条件: $v_j$ 与 $u_i$ 互素.

``One more thing.'' 若 $f$ 是有理系数多项式, 但 $f$ 不是整系数多项式, 我们总可将其变为本原的相伴.

有了上面的讨论, 作者举二个例帮助读者消化.

\begin{example}
    设 $f(x) = 2 x^4-3 x^3+3 x^2-13 x+6$. 我们试找 $f(x)$ 的有理根.

    $f(x)$ 的首项系数与 $0$ 次系数分别是 $2$ 与 $6$. $2$ 的全部因子是
    \begin{align*}
        \pm 1, \pm 2.
    \end{align*}
    $6$ 的全部因子是
    \begin{align*}
        \pm 1, \pm 2, \pm 3, \pm 6.
    \end{align*}
    这应该是好找的; 作者教过读者怎么找整数的全部因子, 对吧? 所以, 若 $\frac{v}{u}$ 是 $f(x)$ 的有理根, 则 $\frac{v}{u}$ 必形如
    \begin{align*}
         & {\pm \frac{1}{1}}, {\pm \frac{2}{1}}, {\pm \frac{3}{1}}, {\pm \frac{6}{1}}; \\
         & {\pm \frac{1}{2}}, {\pm \frac{3}{2}}.
    \end{align*}
    因为 $2$ 与 $2$ 不互素, $6$ 与 $2$ 不互素, 故不必考虑这些组合 (读者牢记: $u$ 与 $v$ 互素!). 适当简化上面的有理数:
    \begin{align*}
        \pm 1, \pm 2, \pm 3, \pm 6, \pm \frac{1}{2}, \pm \frac{3}{2}.
    \end{align*}
    这里有 $12$ 个有理数. 有点多, 但作者请读者忍一下.

    先试试 $\pm 1$ 吧. 再具体点, 先试试 $1$ 吧. 记 $a = 1$. 临时地, 记
    \begin{align*}
        f(x) = a_4 x^4 + a_3 x^3 + a_2 x^2 + a_1 x + a_0.
    \end{align*}
    我们用综合除法算算:
    \begin{align*}
         & b_3 = a_4 = 2,                       \\
         & b_2 = a b_3 + a_3 = 2 - 3 = -1,      \\
         & b_1 = a b_2 + a_2 = -1 + 3 = 2,      \\
         & b_0 = a b_1 + a_1 = 2 - 13 = -11,    \\
         & b_{-1} = a b_0 + a_0 = -11 + 6 = -5.
    \end{align*}
    很遗憾, $1$ 不是 $f(x)$ 的根. 再看看 $-1$? 记 $a = -1$. 上综合除法!
    \begin{align*}
         & b_3 = a_4 = 2,                      \\
         & b_2 = a b_3 + a_3 = -2 - 3 = -5,    \\
         & b_1 = a b_2 + a_2 = 5 + 3 = 8,      \\
         & b_0 = a b_1 + a_1 = -8 - 13 = -21,  \\
         & b_{-1} = a b_0 + a_0 = 21 + 6 = 27.
    \end{align*}
    很遗憾, $-1$ 不是 $f(x)$ 的根. 再看看 $2$? 记 $a = 2$. 上综合除法!
    \begin{align*}
         & b_3 = a_4 = 2,                     \\
         & b_2 = a b_3 + a_3 = 4 - 3 = 1,     \\
         & b_1 = a b_2 + a_2 = 2 + 3 = 5,     \\
         & b_0 = a b_1 + a_1 = 10 - 13 = -3,  \\
         & b_{-1} = a b_0 + a_0 = -6 + 6 = 0.
    \end{align*}
    好消息! $2$ 是 $f(x)$ 的根. 不仅如此, 我们还知
    \begin{align*}
        f(x) = (x - 2)\underbrace{(2x^3 + x^2 + 5x - 3)}_{g(x)}.
    \end{align*}

    任务完成了吗? 还没呢. 不过, 我们的任务变轻松了——毕竟, $g(x)$ 的次为 $3$; 少了一点, 是不是? 继续看. $g(x)$ 的首项系数与 $0$ 次系数分别是 $2$ 与 $-3$. $2$ 的因子——不必再找一次了. $3$ 的因子呢? 不就是 $\pm 1$ 与 $\pm 3$ 嘛! 所以, 若 $\frac{v}{u}$ 是 $g(x)$ 的有理根, 则 $\frac{v}{u}$ 必形如
    \begin{align*}
         & {\pm \frac{1}{1}}, {\pm \frac{3}{1}}; \\
         & {\pm \frac{1}{2}}, {\pm \frac{3}{2}}.
    \end{align*}
    读者可能会注意到, 这里只有 $8$ 个数. 一般地, 若我们找到的有理根不是 $\pm 1$, 则首项系数的绝对值或 $0$ 次系数的绝对值会变小, 从而 ``候选根'' 的数目也会变少; 这一点, 读者可从多项式的乘法看出. 我们知道 $\pm 1$ 不是 $f(x)$ 的根; 所以, $\pm 1$ 一定不会是 $g(x)$ 的根 (读者可用反证法使自己相信这一点). 去掉不可能的选择, 并简化上面的有理数:
    \begin{align*}
        \pm 3, \pm \frac{1}{2}, \pm \frac{3}{2}.
    \end{align*}
    先试试 $\pm 3$ 吧. 再具体点, 先试试 $3$ 吧. 记 $a = 3$. 临时地, 记
    \begin{align*}
        g(x) = a_3 x^3 + a_2 x^2 + a_1 x + a_0.
    \end{align*}
    我们用综合除法算算:
    \begin{align*}
         & b_2 = a_3 = 2,                      \\
         & b_1 = a b_2 + a_2 = 6 + 1 = 7,      \\
         & b_0 = a b_1 + a_1 = 21 + 5 = 26,    \\
         & b_{-1} = a b_0 + a_0 = 78 - 3 = 75.
    \end{align*}
    很遗憾, $3$ 不是 $g(x)$ 的根. 再看看 $-3$? 记 $a = -3$. 上综合除法!
    \begin{align*}
         & b_2 = a_3 = 2,                        \\
         & b_1 = a b_2 + a_2 = -6 + 1 = -5,      \\
         & b_0 = a b_1 + a_1 = 15 + 5 = 20,      \\
         & b_{-1} = a b_0 + a_0 = -60 - 3 = -63.
    \end{align*}
    很遗憾, $-3$ 不是 $g(x)$ 的根. 再看看 $\frac{1}{2}$? 记 $a = \frac{1}{2}$. 上综合除法!
    \begin{align*}
         & b_2 = a_3 = 2,                    \\
         & b_1 = a b_2 + a_2 = 1 + 1 = 2,    \\
         & b_0 = a b_1 + a_1 = 1 + 5 = 6,    \\
         & b_{-1} = a b_0 + a_0 = 3 - 3 = 0.
    \end{align*}
    好消息! $\frac{1}{2}$ 是 $g(x)$ 的根. 不仅如此, 我们还知
    \begin{align*}
        g(x) = \left(x - \frac{1}{2}\right) (2x^2 + 2x + 6) = (2x - 1) \underbrace{(x^2 + x + 3)}_{h(x)}.
    \end{align*}

    任务完成了吗? 还没呢. 不过, 我们的任务变轻松了——毕竟, $h(x)$ 的次为 $2$; 少了一点, 是不是? 继续看. $h(x)$ 的首项系数与 $0$ 次系数分别是 $1$ 与 $3$. $1$ 的因子是 $\pm 1$; $3$ 的因子——不必再找一次了. 所以, 若 $\frac{v}{u}$ 是 $h(x)$ 的有理根, 则 $\frac{v}{u}$ 必形如 $\pm 3$. 我们知道 $\pm 3$ 不是 $g(x)$ 的根; 所以, $\pm 3$ 一定不会是 $h(x)$ 的根. 换句话说, $h(x)$ 无有理根.

    综上, $f(x)$ 恰有二个有理根: $2$ 与 $\frac{1}{2}$.
\end{example}

\begin{example}
    令 $f(x) = 12 x^5-7 x^3+21 x^2+56$. 我们试找 $f(x)$ 的有理根.

    $f(x)$ 的首项系数与 $0$ 次系数分别是 $12$ 与 $56$. $12$ 的全部因子是
    \begin{align*}
        \pm 1, \pm 2, \pm 3, \pm 4, \pm 6, \pm 12.
    \end{align*}
    $56$ 的全部因子是
    \begin{align*}
        \pm 1, \pm 2, \pm 7, \pm 4, \pm 14, \pm 8, \pm 28, \pm 56.
    \end{align*}
    所以, 若 $\frac{v}{u}$ 是 $f(x)$ 的有理根, 则 $\frac{v}{u}$ 必形如
    \begin{align*}
         & {\pm \frac{1}{1}}, {\pm \frac{2}{1}}, {\pm \frac{7}{1}}, {\pm \frac{4}{1}}, {\pm \frac{14}{1}}, {\pm \frac{8}{1}}, {\pm \frac{28}{1}}, {\pm \frac{56}{1}}; \\
         & {\pm \frac{1}{2}}, {\pm \frac{7}{2}};                                                                                                                      \\
         & {\pm \frac{1}{3}}, {\pm \frac{2}{3}}, {\pm \frac{7}{3}}, {\pm \frac{4}{3}}, {\pm \frac{14}{3}}, {\pm \frac{8}{3}}, {\pm \frac{28}{3}}, {\pm \frac{56}{3}}; \\
         & {\pm \frac{1}{4}}, {\pm \frac{7}{4}};                                                                                                                      \\
         & {\pm \frac{1}{6}}, {\pm \frac{7}{6}};                                                                                                                      \\
         & {\pm \frac{1}{12}}, {\pm \frac{7}{12}}.
    \end{align*}
    适当简化上面的有理数:
    \begin{align*}
         & {\pm 1}, {\pm 2}, {\pm 7}, {\pm 4}, {\pm 14}, {\pm 8}, {\pm 28}, {\pm 56},                                                                                 \\
         & {\pm \frac{1}{3}}, {\pm \frac{2}{3}}, {\pm \frac{7}{3}}, {\pm \frac{4}{3}}, {\pm \frac{14}{3}}, {\pm \frac{8}{3}}, {\pm \frac{28}{3}}, {\pm \frac{56}{3}}, \\
         & {\pm \frac{1}{2}}, {\pm \frac{7}{2}}, {\pm \frac{1}{4}}, {\pm \frac{7}{4}}, {\pm \frac{1}{6}}, {\pm \frac{7}{6}}, {\pm \frac{1}{12}}, {\pm \frac{7}{12}}.
    \end{align*}
    这里有 $48$ 个有理数. 是不是有点太多了? 确实. 我们先把此例放一边.
\end{example}

读者从上个例可感受到一点压力. 理论上, 若 $f(x)$ 有有理根, 则其必在上 $48$ 个数内. 可是, $f(x)$ 的次为 $5$——$f(x)$ 至多有 $5$ 个有理根! 读者可能会想: 要是能再少点就好了! 作者听到了读者的愿望. 喏, 下面就是作者的回应.

\begin{proposition}
    设整数 $u$, $v$ 互素, 且 $u \neq 0$. 这样, $g(x) = ux - v$ 是本原的 $1$ 次多项式. 设 $f(x)$ 是整系数多项式. 若 $g(x)$ 是 $f(x)$ 的因子, 则 $v - u$ 是 $f(1)$ 的因子, 且 $v + u$ 是 $f(-1)$ 的因子.
\end{proposition}

\begin{pf}
    设
    \begin{align*}
        f(x) = a_n x^n + a_{n-1} x^{n-1} + \cdots + a_1 x + a_0
    \end{align*}
    是整系数多项式, 且 $a_n \neq 0$. 因为 $g(x) = ux - v$ 是本原的, $g(x)$ 是 $f(x)$ 的因子, 故存在整系数多项式 $h(x)$ 使 $f(x) = g(x)h(x)$. 所以
    \begin{align*}
         & f(1) = g(1) h(1) = -(v - u)h(1),     \\
         & f(-1) = g(-1) h(-1) = -(v + u)h(-1).
    \end{align*}
    由题设, $f(1)$ 与 $f(-1)$ 都是整数; $v \pm u$ 自然也是整数; 因为 $h(x)$ 是整系数的, 故 $h(1)$ 与 $h(-1)$ 都是整数. 所以 $-h(1)$ 与 $-h(-1)$ 都是整数. 所以 $v - u$ 是 $f(1)$ 的因子, 且 $v + u$ 是 $f(-1)$ 的因子.
\end{pf}

用根的语言描述上个命题, 就是
\begin{proposition}
    设整数 $u$, $v$ 互素, 且 $u \neq 0$. 设 $f(x)$ 是整系数多项式. 若 $\frac{v}{u}$ 是 $f(x)$ 的根, 则 $v - u$ 是 $f(1)$ 的因子, 且 $v + u$ 是 $f(-1)$ 的因子.
\end{proposition}

\begin{pf}
    把上个命题用根的语言说一遍就彳亍了.
\end{pf}

\begin{remark}
    注意到, $\frac{v}{u} = \frac{-v}{-u}$, 且 $-v - (-u) = -(u - v)$, $-v + (-u) = -(v + u)$. 所以, 分子与分母同时差 $\pm 1$ 并不影响结论.
\end{remark}

\begin{remark}
    设 $f(x)$ 是整系数多项式. 若 $1$ (或 $-1$) 是 $f(x)$ 的根, 则 $f(1) = 0$ (或 $f(-1) = 0$). 跟前面的情况类似: 任意整数自然是 $0$ 的因子, 所以又在 ``说废话''. 还是一样, 设 $x - 1$ 是 $f(x)$ 的 $m_1$ 重因子, 且 $x + 1$ 是 $f(x)$ 的 $m_2$ 重因子, 这里 $m_1$ 与 $m_2$ 都是非负整数. 所以存在多项式 $g(x)$, $h(x)$ 使
    \begin{align*}
        f(x) = (x - 1)^{m_1} g(x) = (x + 1)^{m_2} h(x),
    \end{align*}
    且 $x-1$ 不是 $g(x)$ 的因子, $x+1$ 不是 $h(x)$ 的因子. 因为 $x-1$ 与 $x+1$ 互素, 故 $(x-1)^{m_1}$ 与 $(x+1)^{m_2}$ 互素. 所以 $(x-1)^{m_1}$ 是 $h(x)$ 的因子, 且 $(x+1)^{m_2}$ 是 $g(x)$ 的因子. 换句话说, 存在多项式 $\ell_1 (x)$, $\ell_2 (x)$ 使
    \begin{align*}
        g(x) = (x+1)^{m_2} \ell_1 (x), \quad h(x) = (x-1)^{m_1} \ell_2 (x).
    \end{align*}
    所以
    \begin{align*}
        f(x) = (x-1)^{m_1} (x+1)^{m_2} \ell_1 (x) = (x-1)^{m_1} (x+1)^{m_2} \ell_2 (x).
    \end{align*}
    所以 $\ell_1 (x) = \ell_2 (x)$. 记 $\ell (x) = \ell_1 (x) = \ell_2 (x)$, 则
    \begin{align*}
        f(x) = (x-1)^{m_1} (x+1)^{m_2} \ell (x).
    \end{align*}
    $x-1$ 与 $x+1$ 都是本原的, 故 $(x-1)^{m_1}$ 与 $(x+1)^{m_2}$ 也是本原的, 从而 $(x-1)^{m_1} (x+1)^{m_2}$ 是本原的. 所以, $\ell (x)$ 一定是整系数的. 而且, $1$ (或 $-1$) 都不是 $\ell (x)$ 的根. 用反证法. 若 $1$ (或 $-1$) 是 $\ell (x)$ 的根, 故 $1$ (或 $-1$) 也是 $g(x)$ (或 $h(x)$) 的根. 所以 $x-1$ (或 $x+1$) 是 $g(x)$ (或 $h(x)$) 的因子. 矛盾!

    所以, 若 $1$ 或 $-1$ 是 $f(x)$ 的根, 我们总可以先去除所有的 $1$ 与 $-1$, 再寻找 $f(x)$ 的其他的有理根.
\end{remark}

我们回头看前二例.

\begin{example}
    先请出简单的老朋友 $f(x) = 2x^4 - 3x^3 + 3x^2 - 13x + 6$. 我们也知道, 若 $\frac{v}{u}$ 是 $f(x)$ 的有理根, 且 $u$ 与 $v$ 互素, 则 $\frac{v}{u}$ 必形如
    \begin{align*}
        \pm 1, \pm 2, \pm 3, \pm 6, \pm \frac{1}{2}, \pm \frac{3}{2}.
    \end{align*}
    现在可以用 ``$v \pm u$'' 挑出不可能是 $f(x)$ 的根的数. 从前面的综合除法的计算过程, 我们可读出
    \begin{align*}
        f(1) = -5, \quad f(-1) = 27.
    \end{align*}
    $\pm 1$ 自然不是 $f(x)$ 的根, 故不必用 ``$v \pm u$'' 检验 $1$, $-1$ 了. 因为 $-5$ 是不可约的, 故 $-5$ 的因子的数目是 $4$; 而 $27 = 3^3$, 故 $27$ 的因子的数目是 $2 \cdot (1+3) = 8$. 所以, 我们先用没那么多因子的 $f(1) = -5$ 进行 ``$v - u$'' 检验. 这总比综合除法简单吧? 通过运算, 我们可排除 $-2$, $3$, $-3$, $-6$, $-\frac{1}{2}$; 也就是说, $2$, $6$, $\frac{1}{2}$, $\frac{3}{2}$, $-\frac{3}{2}$ 通过 ``$v - u$'' 检验. 我们还要用 $f(-1) = 27$ 进行 ``$v + u$'' 检验. 通过运算, 我们可排除 $6$, $\frac{3}{2}$; 也就是说, $2$, $\frac{1}{2}$, $-\frac{3}{2}$ 还通过 ``$v + u$'' 检验. 我们把 $12$ 个数降为 $3$ 个数. 效果不错. 接下来, 感兴趣的读者可自行再检验 $2$, $\frac{1}{2}$, $-\frac{3}{2}$ 到底是不是 $f(x)$ 的根.
\end{example}

\begin{example}
    再请出不是那么简单的老朋友 $f(x) = 12 x^5-7 x^3+21 x^2+56$. 我们也知道, 若 $\frac{v}{u}$ 是 $f(x)$ 的有理根, 且 $u$ 与 $v$ 互素, 则 $\frac{v}{u}$ 必形如
    \begin{align*}
         & {\pm 1}, {\pm 2}, {\pm 7}, {\pm 4}, {\pm 14}, {\pm 8}, {\pm 28}, {\pm 56},                                                                                 \\
         & {\pm \frac{1}{3}}, {\pm \frac{2}{3}}, {\pm \frac{7}{3}}, {\pm \frac{4}{3}}, {\pm \frac{14}{3}}, {\pm \frac{8}{3}}, {\pm \frac{28}{3}}, {\pm \frac{56}{3}}, \\
         & {\pm \frac{1}{2}}, {\pm \frac{7}{2}}, {\pm \frac{1}{4}}, {\pm \frac{7}{4}}, {\pm \frac{1}{6}}, {\pm \frac{7}{6}}, {\pm \frac{1}{12}}, {\pm \frac{7}{12}}.
    \end{align*}
    现在可以用 $v \pm u$ 挑出不可能是 $f(x)$ 的根的数. 不过, 在此之前, 我们看看 $1$ 或 $-1$ 是否是 $f(x)$ 的根. 先试 $1$. 记 $a = 1$. 临时地, 记
    \begin{align*}
        f(x) = a_5 x^5 + a_4 x^4 + a_3 x^3 + a_2 x^2 + a_1 x + a_0.
    \end{align*}
    我们用综合除法算算:
    \begin{align*}
         & b_4 = a_5 = 12,                      \\
         & b_3 = a b_4 + a_4 = 12 + 0 = 12,     \\
         & b_2 = a b_3 + a_3 = 12 - 7 = 5,      \\
         & b_1 = a b_2 + a_2 = 5 + 21 = 26,     \\
         & b_0 = a b_1 + a_1 = 26 + 0 = 26,     \\
         & b_{-1} = a b_0 + a_0 = 26 + 56 = 82.
    \end{align*}
    很遗憾, $1$ 不是 $f(x)$ 的根. 再看看 $-1$? 记 $a = -1$. 上综合除法!
    \begin{align*}
         & b_4 = a_5 = 12,                      \\
         & b_3 = a b_4 + a_4 = -12 + 0 = -12,   \\
         & b_2 = a b_3 + a_3 = 12 - 7 = 5,      \\
         & b_1 = a b_2 + a_2 = -5 + 21 = 16,    \\
         & b_0 = a b_1 + a_1 = -16 + 0 = -16,   \\
         & b_{-1} = a b_0 + a_0 = 16 + 56 = 72.
    \end{align*}
    很遗憾, $-1$ 不是 $f(x)$ 的根. 不过, 这对我们进行 ``$v \pm u$'' 检验是有利的.

    从前面的综合除法的计算过程, 我们可读出
    \begin{align*}
        f(1) = 82, \quad f(-1) = 72.
    \end{align*}
    $\pm 1$ 自然不是 $f(x)$ 的根, 故不必用 ``$v \pm u$'' 检验 $1$, $-1$ 了. $82 = 2 \cdot 41$, $2$ 与 $41$ 都是不可约的, 故 $82$ 的因子的数目是 $2 \cdot (1+1) \cdot (1+1) = 8$; $72 = 2^3 \cdot 3^2$, $3$ 是不可约的, 故 $72$ 的因子的数目是 $2 \cdot (3+1) \cdot (2+1) = 24$. 所以, 我们先用没那么多因子的 $f(1) = 82$ 进行 ``$v - u$'' 检验. 简单的运算后, 我们发现, 只有 $2$, $\frac{1}{3}$, $\frac{2}{3}$, $\frac{4}{3}$, $\frac{1}{2}$, $\frac{7}{6}$ 通过 ``$v - u$'' 检验. 我们还要用 $f(-1) = 72$ 进行 ``$v + u$'' 检验. 简单的运算后, 我们发现, 只有 $2$, $\frac{1}{3}$, $\frac{1}{2}$ 还通过 ``$v + u$'' 检验. 我们把 $48$ 个数降为 $3$ 个数, 效果不错. 接下来, 感兴趣的读者可自行再检验 $2$, $\frac{1}{3}$, $\frac{1}{2}$ 到底是不是 $f(x)$ 的根. 这里作者给个参考答案: 都不是; 换句话说, $f(x)$ 无有理根. 事实上, 取 $p = 7$, 则 $p$ (与 $f(x)$) 适合 Eisenstein 判别法的三条件, 故 $f(x)$ 是不可约的. 这也说明, $f(x)$ 不可能有有理根.
\end{example}

下例是 ``知乎'' 的一个讨论较多的问题\myFN{感兴趣的读者可进入 \texttt{https://www.zhihu.com/question/357995704} 查看更多情报.}.

\begin{example}
    求 (整系数多项式) $f(x) = x^3 + x^2 - 392$ 的根.

    首先, 作者提醒读者: 这里没说 ``有理根''. 即使 $f(x)$ 是整系数多项式, 难道 $f(x)$ 就不是复系数多项式了吗? 不过, ``十千丈高楼平地起''. 我们看看 $f(x)$ 是否有有理根. 如果没有, 我们就不继续讨论下去——毕竟, 没有有理根的多项式的根求起来不是那么方便; 而且, 它也超 ``纲'' 了.

    $f(x)$ 的首项系数与 $0$ 次系数分别是 $1$ 与 $-392$. $1$ 的全部因子是 $\pm 1$. $-392$ 的全部因子是
    \begin{align*}
        {\pm 1}, {\pm 2}, {\pm 7}, {\pm 4}, {\pm 14}, {\pm 49}, {\pm 8}, {\pm 28}, {\pm 98}, {\pm 56}, {\pm 196}, {\pm 392}.
    \end{align*}
    所以, 若 $\frac{v}{u}$ 是 $f(x)$ 的有理根, 则 $\frac{v}{u}$ 必形如
    \begin{align*}
        {\pm 1}, {\pm 2}, {\pm 7}, {\pm 4}, {\pm 14}, {\pm 49}, {\pm 8}, {\pm 28}, {\pm 98}, {\pm 56}, {\pm 196}, {\pm 392}.
    \end{align*}
    这里有 $24$ 个有理数. 我们先看 $\pm 1$ 是否是 $f(x)$ 的根. 先试 $1$. 记 $a = 1$. 临时地, 记
    \begin{align*}
        f(x) = a_3 x^3 + a_2 x^2 + a_1 x + a_0.
    \end{align*}
    我们用综合除法算算:
    \begin{align*}
         & b_2 = a_3 = 1,                         \\
         & b_1 = a b_2 + a_2 = 1 + 1 = 2,         \\
         & b_0 = a b_1 + a_1 = 2 + 0 = 2,         \\
         & b_{-1} = a b_0 + a_0 = 2 - 392 = -390.
    \end{align*}
    很遗憾, $1$ 不是 $f(x)$ 的根. 再看看 $-1$? 记 $a = -1$. 上综合除法!
    \begin{align*}
         & b_2 = a_3 = 1,                         \\
         & b_1 = a b_2 + a_2 = -1 + 1 = 0,        \\
         & b_0 = a b_1 + a_1 = 0 + 0 = 0,         \\
         & b_{-1} = a b_0 + a_0 = 0 - 392 = -392.
    \end{align*}
    很遗憾, $1$ 不是 $f(x)$ 的根. 不过, 这对我们进行 ``$v \pm u$'' 检验是有利的.

    从前面的综合除法的计算过程, 我们可读出
    \begin{align*}
        f(1) = -390, \quad f(-1) = -392.
    \end{align*}
    $\pm 1$ 自然不是 $f(x)$ 的根, 故不必用 ``$v \pm u$'' 检验 $1$, $-1$ 了. $-390 = (-1) \cdot 2 \cdot 3 \cdot 5 \cdot 13$, $2$, $3$, $5$, $13$ 都是不可约的, 故 $-390$ 的因子的数目是 $2 \cdot (1+1)^4 = 32$; $-392$ 的因子的数目是 $24$. 所以, 我们先用没那么多因子的 $f(-1) = -392$ 进行 ``$v + u$'' 检验. 简单的运算后, 我们发现, 只有 $-2$, $7$, $-8$ 通过 ``$v + u$'' 检验. 我们还要用 $f(1) = -390$ 进行 ``$v - u$'' 检验. 简单的运算后, 我们发现, 只有 $-2$, $7$ 还通过 ``$v - u$'' 检验. 我们把 $24$ 个数降为 $2$ 个数, 效果不错. 接下来, 我们用综合除法检验仅剩的二个数.

    先试 $-2$. 记 $a = -2$. 上综合除法!
    \begin{align*}
         & b_2 = a_3 = 1,                          \\
         & b_1 = a b_2 + a_2 = -2 + 1 = -1,        \\
         & b_0 = a b_1 + a_1 = 2 + 0 = 2,          \\
         & b_{-1} = a b_0 + a_0 = -4 - 392 = -396.
    \end{align*}
    很遗憾, $-2$ 不是 $f(x)$ 的根. 再看看 $7$? 记 $a = 7$. 上综合除法!
    \begin{align*}
         & b_2 = a_3 = 1,                        \\
         & b_1 = a b_2 + a_2 = 7 + 1 = 8,        \\
         & b_0 = a b_1 + a_1 = 56 + 0 = 56,      \\
         & b_{-1} = a b_0 + a_0 = 392 - 392 = 0.
    \end{align*}
    好消息! $7$ 是 $f(x)$ 的根. 不仅如此, 我们还知
    \begin{align*}
        f(x) = (x - 2)\underbrace{(x^2 + 8x + 56)}_{g(x)}.
    \end{align*}

    任务完成了吗? 还没呢. 不过, 我们的任务变轻松了——毕竟, $g(x)$ 的次为 $2$; 少了一点, 是不是? 继续看. $g(x)$ 的首项系数与 $0$ 次系数分别是 $1$ 与 $56$. $1$ 的因子——不必再找一次了. $56$ 的因子——作者从别的例搬过来:
    \begin{align*}
        \pm 1, \pm 2, \pm 7, \pm 4, \pm 14, \pm 8, \pm 28, \pm 56.
    \end{align*}
    所以, 若 $\frac{v}{u}$ 是 $g(x)$ 的有理根, 则 $\frac{v}{u}$ 必形如
    \begin{align*}
        \pm 1, \pm 2, \pm 7, \pm 4, \pm 14, \pm 8, \pm 28, \pm 56.
    \end{align*}
    不过, 我们前面已经排除了除 $7$ 以外的所有数. 现在再看 $7$ 行不行. 读者可以上综合除法. 不过, 读者, $g(x)$ 的次是 $2$ 呀! $g(x) = 0$ 是一元二次方程呀! 这是读者在中学接触的玩意儿. 作者说过, 读者一定不要死板地学算学. 这里, 我们对 $g(x)$ 稍变形:
    \begin{align*}
        g(x) = x^2 + 8x + 16 + 40 = (x + 4)^2 + 40.
    \end{align*}
    由此可见, 若 $t$ 是实数, 则 $g(t) \geq 40$. 所以, $g(t)$ 不可能为 $0$. 所以, $g(x)$ 没有实根. 有理根? 肯定也没有.

    如果我们的任务只是找 $f(x)$ 的有理根, 那 $7$ 就是答案. 若我们想找复根呢? 那我们就要找 $g(x)$ 的复根, 再添上已找到的有理根. 设 $z$ 是复数, 且 $g(z) = 0$. 则
    \begin{align*}
                & z^2 + 8z + 56 = 0                                  \\
        \iff {} & (z + 4)^2 + 40 = 0                                 \\
        \iff {} & \frac{(z + 4)^2}{40} + 1 = 0                       \\
        \iff {} & \left( \frac{z + 4}{2\sqrt{10}} \right)^2 + 1 = 0.
    \end{align*}
    所以, $w = \frac{z + 4}{2\sqrt{10}}$ 是 $x^2 + 1$ 的根.

    读者不难验证
    \begin{align*}
        x^2 + 1 = (x - \ii) (x + \ii).
    \end{align*}
    所以, $x^2 + 1$ 的根恰有二个: $\pm \ii$. 因为 $w$ 是 $x^2 + 1$ 的根, 故
    \begin{align*}
        w = \frac{z + 4}{2\sqrt{10}} = \pm \ii.
    \end{align*}
    由此可知 $z = -4 \pm 2\sqrt{10}\,\ii$.

    综上, $f(x)$ 的复根是 $7$, $-4 \pm 2\sqrt{10}\,\ii$.
\end{example}

\myLine

本文的理论有助于读者判断一个数是否是无理数. 这里, 无理数自然是不是有理数的实数.

设 $f$ 是整系数多项式, 且其首项系数是 $\pm 1$. 设整数 $u$, $v$ 互素. 若 $\frac{v}{u}$ 是 $f$ 的根, 则 $u$ 一定是首项系数 $\pm 1$ 的因子. 所以, $u$ 也必须是 $\pm 1$. 由此可得
\begin{proposition}
    设 $f$ 是整系数多项式, 且其首项系数是 $\pm 1$. 若有理数 $r$ 是 $f$ 的根, 则 $r$ 一定是整数, 且 $r$ 是 $f$ 的 $0$ 次系数的因子.
\end{proposition}

\begin{example}
    设 $n$ 是正整数. 设 $m$ 是整数, 且不存在整数 $s$ 使 $s^n = m$. 我们证明: 不存在有理数 $r$ 使 $r^n = m$.

    用反证法. 若存在有理数 $r$ 使 $r^n = m$, 则有理数 $r$ 是整系数多项式 $f = x^n - m$ 的根. 因为 $f$ 的首项系数是 $1$, 故 $r$ 一定是整数. 可是, $m$ 不是整数的平方, 矛盾!

    读者可能听说过, $\sqrt{2}$ 是无理数. 我们可以这么看: $\sqrt{2}$ 是实数; 实数不是有理数就是无理数; $\sqrt{2}$ 的平方是 $2$; 整数的平方不可能是 $2$.

    类似地, 读者可证明: $\sqrt[3]{2}$ 也是无理数.
\end{example}

感谢读者的阅读. 作者坚信: 会有读者的.
