\subsection*{\FactorsOfHigherDegreeOfPolynomialsOverQ}
\addcontentsline{toc}{subsection}{\FactorsOfHigherDegreeOfPolynomialsOverQ}
\markright{\FactorsOfHigherDegreeOfPolynomialsOverQ}

本文讨论有理系数多项式的高次因子.

在 ``\RationalRootsOfPolynomialsOverQ '' 里, 我们讨论了有理多项式的有理根——也即有理多项式的 $1$ 次因子.

我们先了解一些简单的、有用的事实.

\begin{proposition}
    设 $f$ 既不是 $0$, 也不是单位. 设 $p$ 是 $f$ 的因子, $p$ 不是单位, 且 $N = \deg p$. 若 $f$ 没有次低于 $N$ 且不是单位的因子, 则 $p$ 是不可约的.
\end{proposition}

\begin{pf}
    用反证法. 若 $p$ 是可约的, 则存在多项式 $f_1, f_2$ 使 $p = f_1 f_2$, 且 $f_1$, $f_2$ 不是单位. 所以, $1 \leq \deg f_1 < N$. 既然 $f_1$ 是 $p$ 的因子, $p$ 是 $f$ 的因子, 那么 $f_1$ 自然也是 $f$ 的因子. 所以, $f$ 有一个次低于 $N$ 且不是单位的因子 $f_1$. 矛盾!
\end{pf}

\begin{remark}
    此命题与下述命题很相似:

    设 $f$ 既不是 $0$, 也不是单位. 若 $p$ 是最小的高于 $1$ 的 $f$ 的因子, 则 $p$ 是不可约的.
\end{remark}

\begin{proposition}
    设 $f$ 是多项式. 设多项式 $f_1$, $f_2$ 适合 $f = f_1 f_2$. 设 $\deg f_1 \leq \deg f_2$. 则 $\deg f_1 + \deg f_1 \leq \deg f$.
\end{proposition}

\begin{pf}
    设 $f = 0$. 则 $f_1$ 不可能非零. 用反证法. 若 $f_1 \neq 0$, 则由 $f_1 f_2 = f$ 知 $f_2 = 0$. 所以 $\deg f_2 = -\infty$. 但是 $\deg f_1 \geq 0 > -\infty$, 矛盾! 所以, $f_1 = 0$. 也就是说,
    \begin{align*}
        \deg f_1 + \deg f_1 = -\infty + (-\infty) = -\infty = \deg f \leq \deg f.
    \end{align*}

    设 $f \neq 0$. 则 $f_1$, $f_2$ 都不是零. 所以, $\deg f_1$, $\deg f_2$, $\deg f$ 都是非负整数. 所以
    \begin{align*}
         & \deg f_1 + \deg f_1 \leq \deg f_1 + \deg f_2 = \deg f. \qedhere
    \end{align*}
\end{pf}

\begin{remark}
    本命题的证明稍繁——因为 $-\infty$ 不是数.
\end{remark}

如果我们接受, 对任意正数 $t$, 都有
\begin{align*}
    t (-\infty) = (-\infty) t = -\infty,
\end{align*}
那么 $\frac{1}{2} (-\infty) = -\infty$. 也就是说, 上面的命题可写为
\begin{proposition}
    设 $f$ 是多项式. 设多项式 $f_1$, $f_2$ 适合 $f = f_1 f_2$. 设 $\deg f_1 \leq \deg f_2$. 则 $\deg f_1 \leq \frac{1}{2} \deg f$.
\end{proposition}

\begin{remark}
    此命题与下述命题很相似:

    设 $f$ 是整数. 设整数 $f_1$, $f_2$ 适合 $f = f_1 f_2$. 设 $|f_1| \leq |f_2|$. 则 $|f_1| \leq \sqrt{|f|}$.
\end{remark}

\begin{remark}
    设多项式 $f$ 既不是 $0$, 也不是单位. 设 $N = \deg f$.

    若 $f$ 是不可约的, 则 $f$ 的次不高于 $\frac{1}{2} N$ 的因子只能是单位. 毕竟, 若 $f$ 有次不高于 $\frac{1}{2} N$ 且不是单位的因子 $f_1$, 则 $1 \leq \deg f_1 \leq \frac{1}{2} N$. 若 $f_2$ 适合 $f = f_1 f_2$, 则 $\deg f_2 = \deg f - \deg f_1 \geq N - \frac{1}{2} N = \frac{1}{2} N > 0$. 所以, $f_2$ 也不是单位. 也就是说, $f$ 可写为二个不是单位的多项式的积, 矛盾!

    若 $f$ 是可约的, 则 $f$ 有次不高于 $\frac{1}{2} N$ 且不是单位的因子. 毕竟, 既然 $f$ 是可约的, 则 $f$ 可写为二个不是单位的多项式 $f_1$, $f_2$ 的积. 无妨设 $\deg f_1 \leq \deg f_2$. 由上个命题知, $\deg f_1 \leq \frac{1}{2} N$.
\end{remark}

\begin{remark}
    设 $f$ 的次为 $2$ (或 $3$).

    由上述评注知, 若 $f$ 是可约的, 则 $f$ 有次不高于 $\frac{2}{2}$ (或 $\frac{3}{2}$) 且不是单位的因子. 也就是说, $f$ 有 (系数是有理数的) $1$ 次因子. 也就是说, $f$ 有有理根.

    反过来, 若 $f$ 是不可约的, 则 $f$ 的次不高于 $\frac{2}{2}$ (或 $\frac{3}{2}$) 的因子只能是单位. 也就是说, $f$ 无 (系数是有理数的) $1$ 次因子. 也就是说, $f$ 无有理根.

    综上, 若 $f$ 的次为 $2$ (或 $3$), 则 ``$f$ 是可约的'' 的一个必要与充分条件是 ``$f$ 有有理根''; ``$f$ 是不可约的'' 的一个必要与充分条件是 ``$f$ 无有理根''.
\end{remark}

\begin{example}
    设 $f(x) = ax^2 + bx + c$, 其中 $a$, $b$, $c$ 是有理数, 且 $a \neq 0$. 则
    \begin{align*}
        f(x)
        = {} & \frac{1}{4a} (4a) (ax^2 + bx + c)                                  \\
        = {} & \frac{1}{4a} (4ax^2 + 4abx + 4ac)                                  \\
        = {} & \frac{1}{4a} ((2ax)^2 + 2 \cdot (2ax) \cdot b + 4ac)               \\
        = {} & \frac{1}{4a} ((2ax)^2 + 2 \cdot (2ax) \cdot b + b^2 + (4ac - b^2)) \\
        = {} & \frac{1}{4a} ((2ax + b)^2 + (4ac - b^2)).
    \end{align*}
    记 $\Delta = b^2 - 4ac$. 则
    \begin{align*}
        f(x) = \frac{1}{4a} ((2ax + b)^2 - \Delta).
    \end{align*}

    (i) 若存在有理数 $r$ 使 $r^2 = \Delta$, 则
    \begin{align*}
        f(x)
        = {} & \frac{1}{4a} ((2ax + b)^2 - r^2)          \\
        = {} & \frac{1}{4a} (2ax + b - r) (2ax + b + r).
    \end{align*}
    由此可见, $f(x)$ 是可约的.

    (ii) 若 $\Delta$ 不是有理数的平方, 则对任意有理数 $t$, 因 $2at + b$ 也是有理数, 故 $f(t) \neq 0$ (如果有某有理数 $s$ 使 $f(s) = 0$, 则 $(2as + b)^2 = \Delta$, 即 $\Delta$ 是有理数的平方, 矛盾). 所以, $f(x)$ 无有理根. 根据上述评注, $f(x)$ 是不可约的.

    综上, ``$f(x)$ 是可约的'' 的一个必要与充分条件是 ``$\Delta = b^2 - 4ac$ 是有理数的平方''. 利用此结论, 读者可迅速判断一个系数全为有理数的 $2$ 次多项式是否是可约的.
\end{example}

\begin{example}
    设 $f(x) = x^3 + ax^2 + bx + 1$, 其中 $a$, $b$ 是整数. 我们知道, $f(x)$ 的有理根只能是 $\pm 1$. 所以, 若 $f(1)$ 与 $f(-1)$ 都不是 $0$, 即 $a + b \neq -2$ 且 $a \neq b$, 则 $f(x)$ 无有理根. 也就是说, 此时 $f(x)$ 是不可约的. 这样, 我们可随手写出很多次为 $3$ 的不可约的多项式 (不借助 Eisenstein 判别法).
\end{example}

看到这里, 我们明白了一件事: 欲探 $2$ 次或 $3$ 次多项式的因子, 有理根很方便, 且够用. 那么, 次高一些的多项式 (譬如 $4$ 次) 还能这么简单吗?

设 $f$ 是次不低于 $4$ 的多项式. 若 $f$ 有有理根 $a$, 则 $x-a$ 是 $f$ 的因子. 所以, 若 $g$ 适合 $f = (x-a)g$, 则 $\deg g = \deg f - 1 \geq 3$, 故 $g$ 不是单位. 所以, $f$ 是可约的. 但, 就算 $f$ 无有理根, $f$ 仍可是可约的.

\begin{example}
    设 $h(x) = x^4 + 4$. 我们在 ``\PolynomialsOverZAndOverQ '' 里说过, $h(x)$ 是可约的:
    \begin{align*}
        x^4 + 4
        = {} & x^4 + 4x^2 + 4 - 4x^2          \\
        = {} & (x^2 + 2)^2 - (2x)^2           \\
        = {} & (x^2 + 2x + 2) (x^2 - 2x + 2).
    \end{align*}
    显然 $x^2 \pm 2x + 2$ 不是单位. 所以, $h(x)$ 是可约的.

    可是, $h(x)$ 无有理根: 若 $t$ 是有理数, 则 $h(t) \geq 4$, 当然有 $h(t) \neq 0$.

    顺便一提: 既然 $h(x)$ 无 $1$ 次因子, 而 $h(x)$ 有 $2$ 次因子, 因为 $h(x)$ 的次为 $4$, 故 $h(x)$ 还有一个 $2$ 次因子; 这二个 $2$ 次因子自动地是不可约的\myFN{即 to be automatically irreducible.}.
\end{example}

由上个例, 我们看到: 只考虑有理根 ($1$ 次因子) 是不够的. 也就是说, 没有 $1$ 次因子, 并不代表没有 $2$ 次因子 (当然, 次为 $2$ 或 $3$ 的多项式还是很简单的). 我们需要一种找高次因子的办法.

前面, 我们在找有理根时, 用了整系数多项式, 从而可充分地利用整数的性质找出有理根. 所以, 不失一般性, 设 $f(x)$ 是整系数多项式, 且 $f(x) \neq 0$. 设本原的多项式 $g(x)$ 的次不高于 $n$, 且 $g(x)$ 是 $f(x)$ 的因子. 这样, 存在整系数多项式 $h(x)$ 使 $f(x) = g(x) h(x)$. 取 $n+1$ 个互不相同的整数 $x_0$, $x_1$, $\cdots$, $x_n$. 既然 $g(x)$ 的次不高于 $n$, 且 $g(x)$ 可视为有理系数多项式, 根据插值公式\myFN{在这里, 插值公式的具体的形式不重要; 重要的是插值思想. 读者可翻阅 ``\Interpolation '' 以获取更多情报.}, 有
\begin{align*}
    g(x) = \sum_{i = 0}^{n} g(x_i) \prod_{\begin{smallmatrix}0 \leq \ell \leq n \\\ell \neq i\end{smallmatrix}} \frac{x - x_\ell}{x_i - x_\ell}. \tag*{(L)}
\end{align*}

我们看 $g(x_i)$. 因为 $f(x) = g(x) h(x)$, 故 $f(x_i) = g(x_i) h(x_i)$. 因为 $f(x)$, $g(x)$, $h(x)$ 都是整系数多项式, 故 $f(x_i)$, $g(x_i)$, $h(x_i)$ 都是整数. 也就是说, $g(x_i)$ 一定是 $f(x_i)$ 的因子!

如果每个 $f(x_i)$ 都不是 $0$, 那么每个 $g(x_i)$ 只有有限多个取值! 所以, 无妨假定 $f(x)$ 没有 $1$ 次因子 ($f(x)$ 的所有的 $1$ 次因子均可用 ``\RationalRootsOfPolynomialsOverQ '' 的方法找出). 这样, 对任意整数 $a$, $f(a) \neq 0$.

若 $f(x_i)$ 有 $2d_i$ 个因子 (包括正、负), 由式 (L) 可知, 这样的 $g(x)$ 的数目 (至多) 是
\begin{align*}
    N = (2d_0) (2d_1) \cdots (2d_n) = 2^{n+1} d_0 d_1 \cdots d_n.
\end{align*}
为什么我们加上了 ``至多'' 呢? 因为我们假定 $g(x)$ 是本原的多项式, 所以 $g(x)$ 至少得是整系数多项式; 在整数点有整数值的多项式不一定是整系数多项式 (参考 ``\GeneralizedBinomialCoefficients '')!

既然 $g(x)$ 只有有限多个, 那么我们可列出这些 $g(x)$, 并用它们除 $f(x)$.

这就是本文的最核心的思想.

\begin{remark}
    读者要活学活用算学呀! 很明显, $N \geq 2^{n+1}$; 所以, 当 $n$ 很大时, 由式 (L) 给出的 $g(x)$ 的数目也很多. 我们当然要用一些手段使我们轻松一些.

    设 $f(x)$ 是整系数多项式. 无妨假定 $f(x)$ 没有 $1$ 次因子. 设本原的多项式 $g(x)$ 的次不高于 $n$, 且 $g(x)$ 是 $f(x)$ 的因子.

    (i) 为方便, 我们 (临时地) 简单地记由 $n+1$ 个短句合成的长句 ``$g(x_0) = y_0$, $g(x_1) = y_1$, $\cdots$, $g(x_{n-1}) = y_{n-1}$, 且 $g(x_n) = y_n$'' 为
    \begin{align*}
        (g; y_0, y_1, \cdots, y_{n-1}, y_n).
    \end{align*}
    设 $g_1 (x)$, $g_2 (x)$ 都是次不高于 $n$ 的多项式. 设
    \begin{align*}
         & (g_1; y_0, y_1, \cdots, y_{n-1}, y_n),     \\
         & (g_2; -y_0, -y_1, \cdots, -y_{n-1}, -y_n).
    \end{align*}
    也就是说, $g_2 (x_i) = -g_1 (x_i)$. 由式 (L) 知
    \begin{align*}
        g_2 (x)
        = {} & \sum_{i = 0}^{n} (-y_i) \prod_{\begin{smallmatrix}0 \leq \ell \leq n \\\ell \neq i\end{smallmatrix}} \frac{x - x_\ell}{x_i - x_\ell} \\
        = {} & -\sum_{i = 0}^{n} y_i \prod_{\begin{smallmatrix}0 \leq \ell \leq n \\\ell \neq i\end{smallmatrix}} \frac{x - x_\ell}{x_i - x_\ell}   \\
        = {} & {-g_1 (x)}.
    \end{align*}
    所以, 若 $f(x) = g_1 (x) h_1 (x)$, 则 $f(x) = g_2 (x) (-h_1 (x))$; 反过来, 若 $f(x) = g_2 (x) h_2 (x)$, 则 $f(x) = g_1 (x) (-h_2 (x))$. 既然 $f(x)$ 已被假定无 $1$ 次因子, 那么 $f(x_0) \neq 0$. 既然 $g(x_0)$ 是 $f(x_0)$ 的因子, $g(x_0) \neq 0$. 再结合刚才的讨论, 我们可以假定 $g(x_0) > 0$; 换句话说, 我们可假定 $g(x_0)$ 只取 $f(x_0)$ 的正的因子. 所以, 我们不必列出 $N$ 个 $g(x)$; $\frac{1}{2} N$ 个 $g(x)$ 就够用了.

    (ii) 就算 (i) 让我们少考虑一半的多项式, 但 $\frac{1}{2} \cdot 2^{n+1} = 2^n$ 还是不小 (更别提 $\frac{1}{2} N$) 了. 所以, 我们还要一点技术——比方说, 排除一些不可能是整系数多项式的 $g(x)$.

    设 $a$ 是整数. 设 $g(x)$ 是整系数多项式. 作多项式 $g_a (x) = g(x) - g(a)$. 因为 $g_a (a) = 0$, 且 $g_a (x)$ 与 $x-a$ 都是有理系数多项式, 故存在有理系数多项式 $q(x)$ 使
    \begin{align*}
        g(x) - g(a) = g_a (x) = (x - a) q(x).
    \end{align*}
    因为 $g_a (x)$ 是整系数多项式, 且 $x - a$ 是本原的, 故 $q(x)$ 也是整系数多项式. 设 $b$ 是整数. 则
    \begin{align*}
        g(b) - g(a) = g_a (b) = (b - a) q(b).
    \end{align*}
    因为 $g(b) - g(a)$, $b - a$, $q(b)$ 都是整数, 故 $b - a$ 一定是 $g(b) - g(a)$ 的因子! 此事实是有用的. 比方说, 若 $x_0$, $x_1$, $x_2$ 为 $-2$, $-1$, $3$, 而 $g(x_0)$, $g(x_1)$, $g(x_2)$ 分别是 $2$, $3$, $7$, 则此 $g(x)$ 一定不是整系数多项式.

    (iii) 若 $f(x_0) = 1$ 或 $f(x_0) = -1$, 根据 (i), 我们总可选 $g(x_0) = 1$. 若 $f(x)$ 的首项系数为 $1$ 或 $-1$, 则可考虑 $f(x)$ 的反多项式 $f^{\mathrm{r}} (x)$ (因为已假定 $f(x)$ 无 $1$ 次因子, 故 $f(x)$ 的 $0$ 项系数一定不是 $0$). 若 $f^{\mathrm{r}} (x) = G(x) H(x)$, 则 $f(x) = G^{\mathrm{r}}(x) H^{\mathrm{r}}(x)$.

    (iv) $g(x)$ 的首项系数当然是 $f(x)$ 的首项系数的因子; $g(x)$ 的 $0$ 次系数当然是 $f(x)$ 的 $0$ 次系数的因子. 这一点, 显明地写出 $f(x)$ 与 $g(x)$ 的系数即可看出.

    (v) 设
    \begin{align*}
         & (f_1; w_0, w_1, \cdots, w_{n-1}, w_n),                                                             \\
         & (f_2; \varepsilon_0 w_0, \varepsilon_1 w_1, \cdots, \varepsilon_{n-1} w_{n-1}, \varepsilon_n w_n),
    \end{align*}
    其中 $\varepsilon_0$, $\varepsilon_1$, $\cdots$, $\varepsilon_{n-1}$, $\varepsilon_n$ 是 $n+1$ 个整数的单位 (即 $\pm 1$). 也就是说, $f_2 (x_i) = \varepsilon_i f_1 (x_i)$.\myFN{作者举一个例. 取 $n=2$. 设 $x_0$, $x_1$, $x_2$ 分别是 $0$, $1$, $-1$. 设 $f_1 (x) = 1 - x - x^2$; 设 $f_2 (x) = 2x^2 - 1$. 那么 $f_1 (x_0) = 1$, $f_2 (x_0) = -1$, $f_1 (x_1) = -1$, $f_2 (x_1) = 1$, $f_1 (x_2) = 1$, $f_2 (x_2) = 1$. 这样, $\varepsilon_0$ 就是 $-1$, $\varepsilon_1$ 就是 $-1$, $\varepsilon_2$ 就是 $1$. 通俗地说, $f_1 (x_i)$ 与 $f_2 (x_i)$ 最多差 $\pm 1$.} 因为 $f_1 (x_i)$ 与 $\varepsilon_i f_1 (x_i)$ 有相同的因子, 故用诸 $f_2 (x_i)$ 的因子算出的 $g(x)$ 与用诸 $f_1 (x_i)$ 的因子算出的 $g(x)$ 是一样的.
\end{remark}

下面, 作者举几个例, 帮助读者消化、理解.

\begin{example}
    设 $f(x) = 6 x^7-6 x^6-5 x^5-13 x^4+9 x^3+20 x^2-4 x-6$. 我们试写 $f(x)$ 为若干个不可约的多项式的积.

    老样子, 先试着找 $f(x)$ 的 $1$ 次因子——有理根. $f(x)$ 的首项系数是 $6$, 且 $0$ 次系数是 $-6$. $6$ 的因子有 $\pm 1$, $\pm 2$, $\pm 3$, $\pm 6$; $-6$ 的因子有 $\pm 1$, $\pm 2$, $\pm 3$, $\pm 6$. 所以, 若 $\frac{v}{u}$ 是 $f(x)$ 的有理根, 且 $u$, $v$ 互素, 则 $\frac{v}{u}$ 必形如
    \begin{align*}
         & {\pm \frac{1}{1}}, {\pm \frac{2}{1}}, {\pm \frac{3}{1}}, {\pm \frac{6}{1}}; \\
         & {\pm \frac{1}{2}}, {\pm \frac{3}{2}};                                       \\
         & {\pm \frac{1}{3}}, {\pm \frac{2}{3}};                                       \\
         & {\pm \frac{1}{6}}.
    \end{align*}
    这里有 $18$ 个数; 我们可以用 ``$v \pm u$'' 检验排除一些. 不过, 我们先确定 $\pm 1$ 是不是 $f(x)$ 的根. 利用综合除法, 读者不难算出
    \begin{align*}
         & f(x) = (x-1) (6 x^6-5 x^4-18 x^3-9 x^2+11 x+7) + 1,         \\
         & f(x) = (x+1) (6 x^6-12 x^5+7 x^4-20 x^3+29 x^2-9 x+5) - 11.
    \end{align*}
    所以, $\pm 1$ 都不是 $f(x)$ 的根. 不过, 这并不是很糟; 至少, 我们知道, $f(1) = 1$, $f(-1) = -11$. 因为 $f(1) = 1$, 故 $v - u$ 是 $1$ 的因子——也就是说, $v - u = \pm 1$. 由此, 恰有 $2$, $\frac{1}{2}$, $\frac{3}{2}$, $\frac{2}{3}$ 通过 ``$v - u$'' 检验. 因为 $f(-1) = -11$, 故 $v + u$ 是 $-11$ 的因子. 由此, 这四数都不通过 ``$v + u$'' 检验——换句话说, $f(x)$ 无有理根.

    现在, 我们寻找 $f(x)$ 的次不高于 $2$ 的因子. 因为 $f(x)$ 无 $1$ 次因子, 故 $f(x)$ 的 $2$ 次因子一定是不可约的 (若其真地存在). 不难看出, $f(0) = -6$; 前面的计算告诉我们, $f(1) = 1$, $f(-1) = -11$. 假定本原的多项式 $g(x)$ 是 $f(x)$ 的因子, 且 $g(x)$ 的次不高于 $2$. 根据前面的讨论, $g(0)$ 一定是 $-6$ 的因子, $g(1)$ 一定是 $1$ 的因子, 且 $g(-1)$ 一定是 $-11$ 的因子. 取 $x_0$, $x_1$, $x_2$ 为 $1$, $0$, $-1$. 根据前面的评注, 可假定 $g(1) = 1$. 根据式 (L), 有
    \begin{align*}
        g(x)
        = {} & \sum_{i = 0}^{2} g(x_i) \prod_{\begin{smallmatrix}0 \leq \ell \leq 2 \\\ell \neq i\end{smallmatrix}} \frac{x - x_\ell}{x_i - x_\ell}                            \\
        = {} & g(1) \frac{(x-0)(x-(-1))}{(1-0)(1-(-1))} + g(0) \frac{(x-1)(x-(-1))}{(0-1)(0-(-1))}                                   \\
             & \qquad + g(-1) \frac{(x-1)(x-0)}{(-1-1)(-1-0)}                                                                        \\
        = {} & g(1) \left( \frac{1}{2}x^2 + \frac{1}{2}x \right) + g(0) (1-x^2) + g(-1) \left( \frac{1}{2}x^2 - \frac{1}{2}x \right) \\
        = {} & g(0) + \frac{g(1) - g(-1)}{2} x + \frac{g(1) - 2g(0) + g(-1)}{2} x^2.
    \end{align*}
    从上面的公式可看出, 若 $g(x)$ 是整系数多项式, 则 $g(0)$ 是整数, 且 $2$ 是 $g(1) - g(-1)$ 的因子\myFN{为什么这里没说 $2$ 次系数呢? 因为 $g(1) - 2g(0) + g(-1)$ 可写为 $g(1) - g(-1) + 2(g(0) + g(-1))$. 现在读者不难看出, 若 $g(0)$, $g(1)$, $g(-1)$, $\frac{g(1) - g(-1)}{2}$ 都是整数, 则 $g(x)$ 是整系数多项式.}——这也是评注里提到过的点. 因为现在我们假定 $g(x)$ 的次不高于 $2$, 故我们可以直接写出 $g(x)$ 的显明的式. 当 $n$ 较高时, 显明地写出 $g(x)$ 的公式就不方便了.

    现在, 我们具体地写出 $g(x)$. 我们限定 $g(1) = 1$; 又因 $f(0)$ 有 $8$ 个因子, 且 $f(-1)$ 有 $4$ 个因子, 故这样的 $g(x)$ 至多有 $32$ 个. 下面, 我们一个一个地写出这些 $g(x)$. 还是老样子, 我们 (临时地) 用 ``$(g; y_0, y_1, y_2)$'' 表示长句 ``$g(x_0) = y_0$, $g(x_1) = y_1$, 且 $g(x_2) = y_2$''. 当然, 作者再说一次: $x_0$, $x_1$, $x_2$ 分别是 $1$, $0$, $-1$. 这里, $g_0 (x)$, $g_1 (x)$, $\cdots$, $g_{31} (x)$ 的次都不超过 $2$.

    \begin{quotation}
        若 $(g_0; 1, 1, 1)$, 则 $g_0 (x) = 1$.

        若 $(g_1; 1, 1, -1)$, 则 $g_1 (x) = -x^2+x+1$.

        若 $(g_2; 1, 1, 11)$, 则 $g_2 (x) = 5 x^2-5 x+1$.

        若 $(g_3; 1, 1, -11)$, 则 $g_3 (x) = -6 x^2+6 x+1$.

        若 $(g_4; 1, -1, 1)$, 则 $g_4 (x) = 2 x^2-1$.

        若 $(g_5; 1, -1, -1)$, 则 $g_5 (x) = x^2+x-1$.

        若 $(g_6; 1, -1, 11)$, 则 $g_6 (x) = 7 x^2-5 x-1$.

        若 $(g_7; 1, -1, -11)$, 则 $g_7 (x) = -4 x^2+6 x-1$.

        若 $(g_8; 1, 2, 1)$, 则 $g_8 (x) = -x^2+2$.

        若 $(g_9; 1, 2, -1)$, 则 $g_9 (x) = -2 x^2+x+2$.

        若 $(g_{10}; 1, 2, 11)$, 则 $g_{10} (x) = 4 x^2-5 x+2$.

        若 $(g_{11}; 1, 2, -11)$, 则 $g_{11} (x) = -7 x^2+6 x+2$.

        若 $(g_{12}; 1, -2, 1)$, 则 $g_{12} (x) = 3 x^2-2$.

        若 $(g_{13}; 1, -2, -1)$, 则 $g_{13} (x) = 2 x^2+x-2$.

        若 $(g_{14}; 1, -2, 11)$, 则 $g_{14} (x) = 8 x^2-5 x-2$.

        若 $(g_{15}; 1, -2, -11)$, 则 $g_{15} (x) = -3 x^2+6 x-2$.

        若 $(g_{16}; 1, 3, 1)$, 则 $g_{16} (x) = -2 x^2+3$.

        若 $(g_{17}; 1, 3, -1)$, 则 $g_{17} (x) = -3 x^2+x+3$.

        若 $(g_{18}; 1, 3, 11)$, 则 $g_{18} (x) = 3 x^2-5 x+3$.

        若 $(g_{19}; 1, 3, -11)$, 则 $g_{19} (x) = -8 x^2+6 x+3$.

        若 $(g_{20}; 1, -3, 1)$, 则 $g_{20} (x) = 4 x^2-3$.

        若 $(g_{21}; 1, -3, -1)$, 则 $g_{21} (x) = 3 x^2+x-3$.

        若 $(g_{22}; 1, -3, 11)$, 则 $g_{22} (x) = 9 x^2-5 x-3$.

        若 $(g_{23}; 1, -3, -11)$, 则 $g_{23} (x) = -2 x^2+6 x-3$.

        若 $(g_{24}; 1, 6, 1)$, 则 $g_{24} (x) = -5 x^2+6$.

        若 $(g_{25}; 1, 6, -1)$, 则 $g_{25} (x) = -6 x^2+x+6$.

        若 $(g_{26}; 1, 6, 11)$, 则 $g_{26} (x) = -5 x+6$.

        若 $(g_{27}; 1, 6, -11)$, 则 $g_{27} (x) = -11 x^2+6 x+6$.

        若 $(g_{28}; 1, -6, 1)$, 则 $g_{28} (x) = 7 x^2-6$.

        若 $(g_{29}; 1, -6, -1)$, 则 $g_{29} (x) = 6 x^2+x-6$.

        若 $(g_{30}; 1, -6, 11)$, 则 $g_{30} (x) = 12 x^2-5 x-6$.

        若 $(g_{31}; 1, -6, -11)$, 则 $g_{31} (x) = x^2+6 x-6$.
    \end{quotation}

    我们不难看到, 这 $32$ 个多项式里, 有二个 ``坏蛋'': $g_0 (x)$ 与 $g_{26} (x)$. 它们的次低于 $2$. 为什么会混入这些 ``坏蛋'' 呢? 因为 (L) 式不保证求出的 $g(x)$ 的次必须是 $n$ 呀! 它只保证 $g(x)$ 的次不高于 $n$. $g_0 (x)$ 当然没什么用; 事实上, 它总是会出现的 (毕竟, $1$ 是每一个整数的因子). $g_{26} (x)$ 在此处也没什么用, 因为我们知道 $f(x)$ 无 $1$ 次因子.\myFN{但有一点很有意思. 设 $f^{\prime} (x)$ 有本原的 $1$ 次因子 $\ell (x)$ (这里的 $f^{\prime}$ 自然是为与 $f$ 区分). 因为 $\ell (x)$ 的次不高于 $2$, 故 $\ell (x)$ 也会出现在 $g^{\prime} (x)$ 中. 这也是找 $f^{\prime} (x)$ 的 $1$ 次因子的办法——不过, 没有 ``\RationalRootsOfPolynomialsOverQ '' 里的好用、方便, 是不?}

    我们可排除一些 $g(x)$. 我们刚才就排除了 $g_0 (x)$ 与 $g_{26} (x)$. 还能再排除一些吗?

    这些 $g(x)$ 是本原的吗? 的确, 每一个都是本原的 (也包括 $g_0 (x)$ 与 $g_{26} (x)$).

    评注说过, $g(x)$ 的首项系数必是 $f(x)$ 的首项系数 $6$ 的因子——所以, 我们不必考虑以下 $13$ 个多项式: $g_2 (x)$, $g_6 (x)$, $g_7 (x)$, $g_{10} (x)$, $g_{11} (x)$, $g_{14} (x)$, $g_{19} (x)$, $g_{20} (x)$, $g_{22} (x)$, $g_{24} (x)$, $g_{27} (x)$, $g_{28} (x)$, $g_{30} (x)$. 还有 $17$ 个多项式.

    还能再排除一些吗? 当然可以. $g(2)$ 是不是 $f(2)$ 的因子? 是吧? 利用综合除法, 我们可算出 $f(2) = 154$. 还可进一步地写 $f(2)$ 为 $2 \cdot 7 \cdot 11$. 我们现在就看, 在剩下的 $17$ 个 $g_i (x)$ 里, 哪些 $g_i (2)$ 是 $154$ 的因子. 运算后, 我们又排除了 $9$ 个多项式: $g_5 (x)$, $g_9 (x)$, $g_{12} (x)$, $g_{13} (x)$, $g_{16} (x)$, $g_{18} (x)$, $g_{25} (x)$, $g_{29} (x)$, $g_{31} (x)$. 还有 $8$ 个多项式.

    我们当然可以再想办法排除一些; 不过, $f(-2) = -1\,190$, $f(3) = 6\,885$……这些数有点大呢. 那就到此为止吧! 我们开始带余除法了. 我们挺幸运——$g_1 (x)$ 就是 $f(x)$ 的因子:
    \begin{align*}
        f(x) = g_1 (x) \underbrace{(-6 x^5-x^3+12 x^2+2 x-6)}_{f_1 (x)}.
    \end{align*}
    因为 $f(x)$ 无 $1$ 次因子, 故 $g_1 (x)$ 自动地是不可约的. $f_1 (x)$ 的次是 $5$; 所以, 我们的任务轻松了一点. 因为 $g_1 (1)$, $g_1 (0)$, $g_1 (-1)$ 分别是 $1$, $1$, $-1$, 故 $f_1 (1)$, $f_1 (0)$, $f_1 (-1)$ 分别是 $1$, $-6$, $11$. $f_1 (x)$ 当然没有 $1$ 次因子. $f_1 (x)$ 的次不高于 $2$ 的因子可能有哪些呢? 正如评注所言, $g_0 (x)$, $g_1 (x)$, $\cdots$, $g_{31} (x)$ 恰好是用诸 $f_1 (x_i)$ 的因子算出的 $g(x)$. 我们淘汰了 $24$ 个多项式, 是吧? 因为我们用一些简单的事实, 确定了那些 $g_i (x)$ 一定不是 $f(x)$ 的因子——所以它们也肯定不是 $f_1 (x)$ 的因子! 所以, $f_1 (x)$ 的次不高于 $2$ 的因子仍在剩下的 $8$ 个多项式里. 继续从 $g_1 (x)$ 开始. 不过, 这一次, $g_1 (x)$ 不是 $f_1 (x)$ 的因子了. $g_3 (x)$ 也不是 $f_1 (x)$ 的因子. 不过, $g_4 (x)$ 是 $f_1 (x)$ 的因子:
    \begin{align*}
        f_1 (x) = g_4 (x) \underbrace{(-3 x^3-2 x+6)}_{f_2 (x)}.
    \end{align*}
    因为 $f_1 (x)$ 无 $1$ 次因子, 故 $g_4 (x)$ 自动地是不可约的. $f_2 (x)$ 的次是 $3$; 所以, 我们的任务轻松了一点. $f_2 (x)$ 也没有 $1$ 次因子! 这样, $f_2 (x)$ 是不可约的.

    综上, 我们有
    \begin{align*}
        f(x) = (-x^2+x+1) (2x^2-1) (-3x^3-2x+6).
    \end{align*}
\end{example}

或许, 读者会觉得上面的那个 $7$ 次式 ``怪'' ``偏'' ``难''. 那我们看一个简单的例?

\begin{example}
    设 $f(x) = x^{15}-1$. 我们试写 $f(x)$ 为若干个不可约的多项式的积.

    读者可能还记得 ``\SyntheticDivision '' 里的乘法公式\myFN{为避免与这里的 $f$, $g$ 混淆, 此公式的字母被改写了.}:
    \begin{align*}
        F^m - G^m = (F - G)(F^{m-1} + F^{n-2} G + \cdots + F^{m-i} G^{i-1} + \cdots + G^{m-1}).
    \end{align*}
    这里, $F$, $G$ 是任意的二个多项式, $m$ 是正整数. 取 $G = 1$, 有
    \begin{align*}
        F^m - 1 = (F - 1)(F^{m-1} + F^{m-2} + \cdots + F + 1). \tag*{(D)}
    \end{align*}
    所以, 取 $m = 15$, $F = x$, 有
    \begin{align*}
        f(x) = (x - 1)(x^{14} + x^{13} + \cdots + x + 1).
    \end{align*}
    这个公式是正确的; 不过, $x^{14} + x^{13} + \cdots + x + 1$ 是不是不可约的呢? 现在, 还不好说. 我们直接拿本文的方法讨论这个 $14$ 次多项式吗? 似乎不太方便呢.

    所以, 这个时候, 读者需要灵活地使用算学知识破解此题! 怎么 ``灵活地'' 呀? 作者给一个思路.

    首先, $15 = 3 \cdot 5 = 5 \cdot 3$; 读者应该对此不陌生吧? 取式 (D) 的 $F = x^3$, $m = 5$, 有
    \begin{align*}
        f(x)
        = {} & F^5 - 1                                                     \\
        = {} & (F - 1) (F^4 + F^3 + F^2 + F + 1)                           \\
        = {} & (x^3 - 1) \underbrace{(x^{12} + x^9 + x^6 + x^3 + 1)}_{q_1} \\
        = {} & (x - 1) \underbrace{(x^2 + x + 1)}_{p_1} q_1.
    \end{align*}
    取式 (D) 的 $F = x^5$, $m = 3$, 有
    \begin{align*}
        f(x)
        = {} & F^3 - 1                                                   \\
        = {} & (F - 1) (F^2 + F + 1)                                     \\
        = {} & (x^5 - 1) \underbrace{(x^{10} + x^5 + 1)}_{q_2}           \\
        = {} & (x - 1) \underbrace{(x^4 + x^3 + x^2 + x + 1)}_{p_2} q_2.
    \end{align*}
    (提醒读者: 这里的二个 ``$F$'' 当然不是同一个 $F$.) 所以
    \begin{align*}
        (x - 1) p_1 q_1 = (x - 1) p_2 q_2.
    \end{align*}
    因为 $x-1$ 是非零的多项式, 故
    \begin{align*}
        p_1 q_1 = p_2 q_2.
    \end{align*}
    我们回想, 在 ``\PolynomialsOverZAndOverQ '' 里, 我们利用 Eisenstein 判别法与 ``$\alpha x + \beta$'' 法得出, 当 $q$ 是正的不可约的整数 (素数) 时, $1 + x + \cdots + x^{q-2} + x^{q-1}$ 是不可约的. 所以, $p_1$ (对应 $q = 3$), $p_2$ (对应 $q = 5$) 都是不可约的. 所以, $p_1$ 一定是 $q_2$ 的因子, 且 $p_2$ 一定是 $q_1$ 的因子. 所以, 存在多项式 $r_1$, $r_2$ 使 $q_2 = p_1 r_2$, 且 $q_1 = p_2 r_1$. 所以
    \begin{align*}
        p_1 p_2 r_1 = p_2 p_1 r_2.
    \end{align*}
    因为 $p_1$, $p_2$ 是非零的, 故 $p_1 p_2$ 是非零的; 由此可知 $r_1 = r_2$. 记 $r_2 = r(x)$. 则
    \begin{align*}
        f(x) = (x - 1) p_2 q_2 = (x - 1) p_2 p_1 r(x).
    \end{align*}
    此 $r(x)$ 可用带余除法确定. 既然 $q_2 = p_1 r(x)$, 那么我们用 $p_1$ 除 $q_2$:
    \begin{align*}
             & x^{10} + x^5 + 1                                                      \\
        = {} & x^8 \cdot x^2 + x^5 + 1                                               \\
        = {} & x^8 (x^2 + x + 1) - x^8 (x + 1) + x^5 + 1                             \\
        = {} & x^8 p_1 - x^9 - x^8 + x^5 + 1                                         \\
        = {} & x^8 p_1 - x^7 \cdot x^2 - x^8 + x^5 + 1                               \\
        = {} & x^8 p_1 - x^7 \cdot (x^2 + x + 1) + x^7 \cdot (x + 1) - x^8 + x^5 + 1 \\
        = {} & (x^8 - x^7) p_1 + x^7 + x^5 + 1                                       \\
        = {} & (x^8 - x^7) p_1 + x^5 \cdot x^2 + x^5 + 1                             \\
        = {} & (x^8 - x^7) p_1 + x^5 \cdot (x^2 + x + 1) - x^5 (x + 1) + x^5 + 1     \\
        = {} & (x^8 - x^7 + x^5) p_1 - (x^6 - 1).
    \end{align*}
    注意到
    \begin{align*}
        x^6 - 1
        = {} & (x^3)^2 - 1              \\
        = {} & (x^3 + 1) (x^3 - 1)      \\
        = {} & (x^3 + 1) (x - 1) p_1    \\
        = {} & (x^4 - x^3 + x - 1) p_1,
    \end{align*}
    故
    \begin{align*}
        x^{10} + x^5 + 1
        = {} & (x^8 - x^7 + x^5) p_1 - (x^4 - x^3 + x - 1) p_1  \\
        = {} & (x^8 - x^7 + x^5) p_1 + (-x^4 + x^3 - x + 1) p_1 \\
        = {} & (x^8 - x^7 + x^5 - x^4 + x^3 - x + 1) p_1.
    \end{align*}
    也就是说,
    \begin{align*}
        r(x) = x^8 - x^7 + x^5 - x^4 + x^3 - x + 1.
    \end{align*}
    所以
    \begin{align*}
        f(x) = (x-1) (x^2+x+1) (x^4+x^3+x^2+x+1) r(x),
    \end{align*}
    且在 $r(x)$ 前的三个多项式都是不可约的. 我们还要继续研究 $r(x)$.

    $r(x)$ 的首项系数与 $0$ 次系数都是 $1$, 故 $r(x)$ 的有理根只能是 $\pm 1$. 利用综合除法, 有
    \begin{align*}
         & r(x) = (x-1) (x^7+x^4+x^2+x) + 1,             \\
         & r(x) = (x+1) (x^7-2 x^6+2 x^5-x^4+x^2-x) + 1.
    \end{align*}
    所以, $r(x)$ 无 $1$ 次因子. 因为 $r(x)$ 的次为 $8$, 故若 $r(x)$ 是可约的, 则 $r(x)$ 必有次不高于 $4$ 的本原的因子 $g(x)$. 取 $x_0$, $x_1$, $x_2$, $x_3$, $x_4$ 为 $0$, $1$, $-1$, $2$, $-2$. 显然, $r(x_0) = 1$; 根据前面的计算, $r(x_1) = r(x_2) = 1$. 不难用综合除法算出 $r(x_3) = 151$, $r(x_4) = 331$. 这二个数看上去很大; 其实, 并没有特别可怕! 读者可用 ``\FactorizationOfIntegers '' 的知识得出: $151$ 与 $331$ 都是不可约的整数!\myFN{因为 $12^2 = 144$, $13^2 = 169$, 故用 $2$, $3$ 及 ``$6k \pm 1$ 数'' $5$, $7$, $11$ 试除可知, $151$ 是不可约的; 因为 $18^2 = 324$, $19^2 = 361$, 故用 $2$, $3$ 及 ``$6k \pm 1$ 数'' $5$, $7$, $11$, $13$, $17$ 试除可知, $331$ 也是不可约的.} $g(x_i)$ 是 $r(x_i)$ 的因子. 因为 $r(x_0) = 1$, 故我们可限定 $g(x_0) = 1$. $r(x_1)$, $r(x_2)$ 都恰有 $2$ 个因子; $r(x_3)$, $r(x_4)$ 都恰有 $4$ 个因子. 所以, 这样的 $g(x)$ 至多有 $2 \cdot 2 \cdot 4 \cdot 4 = 64$ 个. 因为 $g(x)$ 至少是整系数的, 故 $3$ 一定是 $g(2) - g(-1)$ 与 $g(1) - g(-2)$ 的因子, 且 $4$ 一定是 $g(2) - g(-2)$ 的因子. 所以, $(g(2), g(-1))$ 的组合只能是 $(1, 1)$, $(151, 1)$, $(-1, -1)$, $(-151, -1)$; $(g(1), g(-2))$ 的组合只能是 $(1, 1)$, $(1, 331)$, $(-1, -1)$, $(-1, -331)$; $(g(2), g(-2))$ 的组合只能是 $(1, 1)$, $(1, -331)$, $(-1, -1)$, $(-1, 331)$, $(151, -1)$, $(151, 331)$, $(-151, 1)$, $(-151, -331)$. 如何方便地写出一组 $(g(0), g(1), g(-1), g(2), g(-2))$ 呢? 考虑到 $g(0)$, $g(1)$, $g(-1)$ 较为简单, 故我们用它们入手.

    设 $g(0)$, $g(1)$, $g(-1)$ 分别是 $1$, $1$, $1$. 所以 $g(2)$ 能取 $1$, $151$. 若 $g(2)$ 取 $1$, 则 $g(-2)$ 能取 $1$, $-331$; 但考虑到 $g(1) = 1$, 故 $g(-2) = 1$. 若 $g(2)$ 取 $151$, 则 $g(-2)$ 能取 $-1$, $331$; 但考虑到 $g(1) = 1$, 故 $g(-2) = 331$. 这给出二组可能:
    \begin{align*}
        (1,1,1,1,1), \quad (1,1,1,151,331).
    \end{align*}
    其对应的 $g(x)$ 分别是
    \begin{align*}
        1, \quad 20 x^4-15 x^3-20 x^2+15 x+1.
    \end{align*}

    设 $g(0)$, $g(1)$, $g(-1)$ 分别是 $1$, $-1$, $1$. 所以 $g(2)$ 能取 $1$, $151$. 若 $g(2)$ 取 $1$, 则 $g(-2)$ 能取 $1$, $-331$; 但考虑到 $g(1) = -1$, 故 $g(-2) = -331$. 若 $g(2)$ 取 $151$, 则 $g(-2)$ 能取 $-1$, $331$; 但考虑到 $g(1) = -1$, 故 $g(-2) = -1$. 这给出二组可能:
    \begin{align*}
        (1,-1,1,1,-331), \quad (1,-1,1,151,-1).
    \end{align*}
    其对应的 $g(x)$ 分别是
    \begin{align*}
        -\frac{27}{2} x^4+28 x^3+\frac{25}{2} x^2-29 x+1, \quad \frac{13}{2} x^4+13 x^3-\frac{15}{2} x^2-14 x+1.
    \end{align*}
    这二个多项式都不是整系数的, 故被舍去.\myFN{事实上, 若用 Newton 插值公式, 那么不用具体地算出多项式即可知道它一定不是整系数的. 这里, 作者具体地写出多项式来, 仅供读者参考.}

    设 $g(0)$, $g(1)$, $g(-1)$ 分别是 $1$, $-1$, $-1$. 所以 $g(2)$ 能取 $-1$, $-151$. 若 $g(2)$ 取 $-1$, 则 $g(-2)$ 能取 $-1$, $331$; 但考虑到 $g(1) = -1$, 故 $g(-2) = -1$. 若 $g(2)$ 取 $-151$, 则 $g(-2)$ 能取 $1$, $-331$; 但考虑到 $g(1) = -1$, 故 $g(-2) = -331$. 这给出二组可能:
    \begin{align*}
        (1,-1,-1,-1,-1), \quad (1,-1,-1,-151,-331).
    \end{align*}
    其对应的 $g(x)$ 分别是
    \begin{align*}
        \frac{1}{2} x^4-\frac{5}{2} x^2+1, \quad -\frac{39}{2} x^4+15 x^3+\frac{35}{2} x^2-15 x+1.
    \end{align*}
    这二个多项式都不是整系数的, 故被舍去.

    设 $g(0)$, $g(1)$, $g(-1)$ 分别是 $1$, $1$, $-1$. 所以 $g(2)$ 能取 $-1$, $-151$. 若 $g(2)$ 取 $-1$, 则 $g(-2)$ 能取 $-1$, $331$; 但考虑到 $g(1) = 1$, 故 $g(-2) = 331$. 若 $g(2)$ 取 $-151$, 则 $g(-2)$ 能取 $1$, $-331$; 但考虑到 $g(1) = 1$, 故 $g(-2) = 1$. 这给出二组可能:
    \begin{align*}
        (1,1,-1,-1,331), \quad (1,1,-1,-151,1).
    \end{align*}
    其对应的 $g(x)$ 分别是
    \begin{align*}
        14 x^4-28 x^3-15 x^2+29 x+1, \quad -6 x^4-13 x^3+5 x^2+14 x+1.
    \end{align*}

    读者可能会想: 刚才没有检验 $2$ 是否为 $g(1) - g(-1)$ 的因子; 为什么呢? 因为 $g(1)$ 与 $g(-1)$ 都是奇数, 而 $2$ 一定是二个奇数的因子\myFN{若读者不相信这一点, 可自行用反证法证明. 当然, 明确定义是有必要的: 若 $2$ 是整数 $t$ 的因子, 则 $t$ 是偶数; 否则, $t$ 是奇数.}.

    总之, 通过稍细致的考虑, 我们把 $64$ 个 $g(x)$ 降低到 $8$ 个; 用插值法具体地算出 $g(x)$ 后, 此数又减半. 作为对比, 我们回想起, 在上个例 ($f(x)$ 的次为 $7$) 中, 由于 $f(1) = 1$, $f(-1) = -11$, 故 $g(1)$, $g(-1)$ 都是奇数, 从而 ``$2$ 必是 $g(1) - g(-1)$ 的因子'' 帮不上什么忙.

    再看看仅剩的 $4$ 个整系数多项式. 首先, $1$ 没什么用, 故被舍去. 剩下的 $3$ 个多项式都是 $4$ 次的. 不过, 它们的首项系数都不是 $r(x)$ 的首项系数的因子, 故它们也不是 $r(x)$ 的因子. 所以, $r(x)$ 没有 $1$, $2$, $3$, $4$ 次因子. 所以, $r(x)$ 是不可约的.

    综上, 我们有
    \begin{align*}
        f(x)
        = {} & (x-1) (x^2+x+1) (x^4+x^3+x^2+x+1)       \\
             & \qquad \cdot (x^8-x^7+x^5-x^4+x^3-x+1).
    \end{align*}
\end{example}

\begin{remark}
    一般地, 欲写 $x^n - 1$ ($n$ 为正整数) 为若干个不可约的多项式的积, 我们有更好、更系统的方法——可惜, 它需要更深、更抽象的知识, 故作者无法在本文讲述这些精彩的理论. 感兴趣的读者可查找分圆多项式 \term{cyclotomic polynomials} 的相关理论.
\end{remark}

最后, 读者可自行思考这个小问题:

\begin{quotation}
    试找出整数 $u$, $v$, 使有理系数多项式 $(ux - v)(x^3 - x) - 1$ 可写为二个系数是有理数且次均为正整数的多项式的积.
\end{quotation}

这里, 作者稍解释本题:

(i) 本题不要求找出所有的 $u$, $v$. 解谜者写出一组 $u$, $v$ 就很不错了. 当然, 作者并不是说不欢迎找出了所有的 $u$, $v$ 的解答.

(ii) 可以用任何 ``有道理的'' 知识求解本题. 也就是说, 就算不用本文的知识, 也是可以的——解谜者不要学死算学!

(iii) 如果解谜者理解了本题在说什么, 解谜者马上就会意识到, ``可写为二个系数是有理数且次均为正整数的多项式的积'' 只不过是 ``是可约的'' 的一种稍复杂的说法罢了. 不过, 作者避开术语的目的就是不想限制解谜者的思维. 这也解释了 (i).

参考解答:
\begin{align*}
     & \texttt{https://chaoli.club/index.php/6727}                         \\
     & \texttt{https://www.zhihu.com/question/485329457/answer/2108231297}
\end{align*}

感谢读者读到这里. 读者辛苦了! 休息一下吧!
