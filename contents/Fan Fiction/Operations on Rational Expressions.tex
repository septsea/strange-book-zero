\subsection*{\OperationsOnRationalExprsssions}
\addcontentsline{toc}{subsection}{\OperationsOnRationalExprsssions}
\markright{\OperationsOnRationalExprsssions}

本文讲述如何进行有理式的运算.

在 ``\DefinitionOfRationalExpressions'' 里, 我们定义了有理式与其最基本的三事物: 相等、加法、乘法. 利用整式的运算律, 我们得到了有理式的运算律. 借助相反元, 减法归结为加法; 借助倒元, 除法归结为乘法. 并且, 整式也可视为有理式, 并保持相等、加法、乘法的结果. 在这种等同下, $\frac{\ast}{\ast}$ 可认为是 $\ast \div \ast$. 所以, 我们放宽了 $\frac{\ast}{\ast}$ 的使用范围, $\ast$ 也可是有理式. 利用有理式的运算律, 我们得到跟定义长得很像的公式. 具体地说, 我们有如下定义与命题\myFN{读者可乘此机会复习前文的内容.}:

\begin{definition}
    设 $f$, $g$ 是整式, 且 $f \neq 0$. 形如 $\frac{g}{f}$ 的文字是有理式.
\end{definition}

\begin{definition}
    设 $f$, $g$, $u$, $v$ 是整式, 且 $f$, $u$ 不是零.

    (i) $\frac{g}{f} = \frac{v}{u}$ 定义为 $gu = fv$.

    (ii) 有理式的加法定义为
    \begin{align*}
        \frac{g}{f} + \frac{v}{u} = \frac{gu + fv}{fu}.
    \end{align*}

    (iii) 有理式的乘法定义为
    \begin{align*}
        \frac{g}{f} \cdot \frac{v}{u} = \frac{gv}{fu}.
    \end{align*}
\end{definition}

\begin{definition}
    设 $r$, $s$ 是有理式.

    (i) 有理式的减法定义为
    \begin{align*}
        r - s = r + (-s).
    \end{align*}

    (ii) 若 $s \neq 0$, 则有理式的除法定义为
    \begin{align*}
        \frac{r}{s} = rs^{-1}.
    \end{align*}
\end{definition}

\begin{proposition}
    设 $r$, $s$, $t$, $w$ 是有理式.

    (I)\myFN{\textit{I} stands for \textit{identification}.} 整式都是有理式. 具体地说, 若 $f$, $d$ 是整式, 且 $d \neq 0$, 则 $f = \frac{fd}{d}$.

    有理式的相等适合如下性质:

    (R) 反射律: $r = r$.

    (S) 对称律: 若 $r = s$, 则 $s = r$.

    (T) 推移律: 若 $r = s$, 且 $s = t$, 则 $r = t$.

    有理式的加法适合如下性质:

    (A0) $r + s$ 也是有理式.

    (A1) 若 $r = t$, $s = w$, 则 $r + s = t + w$.

    (A2) 加法交换律: $r + s = s + r$.

    (A3) 加法结合律: $(r + s) + t = r + (s + t)$.

    (A4) 零: 存在一个称为 ``$0$'' 的有理式, 使 $0 + r = r$.

    (A5) 相反元: 存在一个称为 ``$-r$'' 的有理式, 使 $-r + r = 0$.

    有理式的减法适合如下性质:

    (S0)\myFN{This \textit{S} stands for \textit{subtraction}.} $r - s$ 也是有理式.

    (S1) 若 $r = t$, $s = w$, 则 $r - s = t - w$.

    有理式的乘法适合如下性质:

    (M0) $rs$ 也是有理式.

    (M1) 若 $r = t$, $s = w$, 则 $rs = tw$.

    (M2) 乘法交换律: $rs = sr$.

    (M3) 乘法结合律: $(rs)t = r(st)$.

    (M4) 幺: 存在一个称为 ``$1$'' 的有理式, 使 $1r = r$.

    (M5)$^{\prime}$ 倒元: 若 $r \neq 0$, 则存在一个称为 ``$r^{-1}$'' 的有理式, 使 $r^{-1} r = 1$.

    加法与乘法还有一座桥:

    (D) 分配律: $r (s + t) = rs + rt$.

    根据分配律, 我们有 $0r = r0 = 0$, $-rs = (-r)s$, $-s = (-1)s$.

    下设 $s \neq 0$, $w \neq 0$. 有理式的除法适合如下性质:

    (D0)\myFN{This \textit{D} stands for \textit{division}.} $\frac{r}{s}$ 也是有理式.

    (D1) 若 $r = t$, $s = w$, 则 $\frac{r}{s} = \frac{t}{w}$.

    (D2) $\frac{r}{s} = \frac{t}{w}$ 的一个必要与充分条件是 $rw = st$.

    (D3) $\frac{r}{s} = \frac{rw}{sw} = \frac{wr}{ws}$.

    (D4) $\frac{r}{s} \pm \frac{t}{w} = \frac{rw \pm st}{sw}$.

    (D5) $-\frac{r}{s} = \frac{-r}{s}$.

    (D6) $\frac{r}{s} \cdot \frac{t}{w} = \frac{rt}{sw}$.

    (D7) $\left( \frac{s}{w} \right)^{-1} = \frac{w}{s}$.

    (D8) $\frac{s}{s} = 1$.
\end{proposition}

\begin{remark}
    改上个命题的 ``整式'' ``有理式'' 为 ``整数'' ``有理数'', 命题仍是对的.
\end{remark}

在正式进入有理式的运算前, 我们先介绍有理式的最简形.

设 $r$, $s$ 是有理式, 且 $s \neq 0$. 所以存在整式 $f$, $g$, $u$, $v$ ($f$, $u$, $v$ 均不为零) 使 $r = \frac{g}{f}$, $s = \frac{v}{u}$. 所以
\begin{align*}
    \frac{r}{s} = rs^{-1} = \frac{g}{f} \cdot \frac{u}{v} = \frac{gu}{fv}.
\end{align*}
$gu$, $fv$ 都是整式, 且 $fv \neq 0$. 我们可求出 $gu$, $fv$ 的一个最大公因子 $d$. 因为 $fv \neq 0$, 故 $d \neq 0$. 设整式 $F$, $G$ 使 $gu = dG$, $fv = dF$. 则
\begin{align*}
    \frac{r}{s} = \frac{gu}{fv} = \frac{dG}{dF} = \frac{G}{F}.
\end{align*}
$F$, $G$ 都是整式, 且 $F \neq 0$, $F$ 与 $G$ 互素. 换句话说, 我们有
\begin{proposition}
    设 $r$, $s$ 是有理式, 且 $s \neq 0$. 存在二个互素的整式 $F$, $G$ 使 $F \neq 0$, 且
    \begin{align*}
        \frac{r}{s} = \frac{G}{F}.
    \end{align*}
\end{proposition}

这样的 $F$ 与 $G$ 不是唯一的, 因为最大公因子不是唯一的. 不过, 因为整式的任意二个最大公因子相伴, 故这样的 $F$ 与 $G$ 也顶多差一个单位.

不正式地说, $\frac{G}{F}$ 是 $\frac{r}{s}$ 的一个最简形.

顺便一提: 若 $F$ 是整式的单位, 则 $\frac{G}{F}$ 就是整式. 我们通常写它为 $GF^{-1} = F^{-1} G$, 这里 $F^{-1}$ 当然也是整式的单位.

\begin{example}
    设 $r = \frac{4x^2 - 4}{-2}$. 显然, $4x^2 - 4$ 与 $-2$ 互素 (因为 $-2$ 是整式的单位). 所以
    \begin{align*}
        r = (-2)^{-1} (4x^2 - 4) = (-2)^{-1} \cdot 4(x^2 - 1) = -2(x^2 - 1).
    \end{align*}
\end{example}

\begin{remark}
    我们写有理数 $\frac{b}{a}$ 时, 也会将它写为 $\frac{g}{f}$ 的形式, 其中 $f \neq 0$, 且整数 $f$ 与 $g$ 互素. 当然, $\frac{g}{\pm 1}$ 就是 $\pm g$. 写有理数为其最简形的过程是 ``约分''.

    为什么我们需要约分? 古代中国算学家刘徽曾为《九章算术》作注\myFN{亦即 《九章算术注》.}. 下面是他在《九章算术注》里给出的理由\myFN{此话的一个 (现代) 汉语翻译可以是: ``约分的原因是物品的数量不可能全部是整数, 这时必须用分数表示. 分数作为一个数来说, 如果太繁琐就难用. 例如 $\frac{2}{4}$, 繁琐的表示形式有 $\frac{4}{8}$, 简约的表示形式有 $\frac{1}{2}$. 虽然表示形式不同, 但数值上是相同的. 分母分子互相推算, 经常有不同的情况, 所以计算前要进行约分.''}:
    \begin{quotation}
        约分者, 物之数量, 不可悉全, 必以分言之. 分之为数, 繁则难用. 设有四分之二者, 繁而言之, 亦可为八分之四; 约而言之, 则二分之一也. 虽则异辞, 至于为数, 亦同归尔. 法实相推, 动有参差, 故为术者先治诸分.
    \end{quotation}

    作为对比, 我们试算 $a = \frac{18}{4} + \frac{233}{699}$ 与 $a^{\prime} = \frac{9}{2} + \frac{1}{3}$. 因为 $\frac{18}{4} = \frac{9}{2}$, 且 $\frac{233}{699} = \frac{1}{3}$, 故 $a = a^{\prime}$. 不过, 如果我们不使用最简形, 我们有
    \begin{align*}
        a = \frac{18}{4} + \frac{233}{699} = \frac{18 \cdot 699 + 4 \cdot 233}{4 \cdot 699} = \frac{13\,514}{2\,796}.
    \end{align*}
    注意到 $9$ 与 $2$ 互素, 且 $1$ 与 $3$ 互素, 故出现在 $a^{\prime}$ 中的二个有理数已为最简形. 不难算出
    \begin{align*}
        a^{\prime} = \frac{9}{2} + \frac{1}{3} = \frac{9 \cdot 3 + 2 \cdot 1}{2 \cdot 3} = \frac{29}{6}.
    \end{align*}
    或许读者怀疑 $a = a^{\prime}$. 我们可直接验证:
    \begin{align*}
        13\,514 \cdot 6 = 81\,084 = 2\,796 \cdot 29.
    \end{align*}
    所以, $a = a^{\prime}$ 是正确的.

    作者希望本评注能够帮助读者理解为什么我们需要最简形.
\end{remark}

\begin{example}
    设 $s = \frac{x^{m} - 1}{x^{m+1} - 1}$, 其中 $m$ 是正整数. 我们试找 $s$ 的一个最简形.

    根据乘法公式, 对任意正整数 $n$, 有
    \begin{align*}
        x^n - 1 = (x - 1) \underbrace{(x^{n-1} + x^{n-2} + \cdots + 1)}_{g_n}.
    \end{align*}
    所以
    \begin{align*}
        s = \frac{x^{m} - 1}{x^{m+1} - 1} = \frac{(x - 1) g_{m}}{(x - 1) g_{m+1}} = \frac{g_{m}}{g_{m+1}}.
    \end{align*}
    注意到
    \begin{align*}
        g_{m+1} = xg_{m} + 1 \implies 1 \cdot g_{m+1} + (-x) \cdot g_{m} = 1,
    \end{align*}
    故 $g_{m+1}$ 与 $g_{m}$ 互素. 所以, $s$ 的一个最简形是 $\frac{g_{m}}{g_{m+1}}$.
\end{example}

\begin{example}
    设 $s = \frac{x^4 - 1}{x^3 - 2x^2 - x + 2}$. 我们试找 $s$ 的一个最简形.

    求最大公因子的一个好方法是辗转相除法. 以小除大 (以低次的整式除高次的整式), 直到余式为零:
    \begin{align*}
         & \underbrace{x^4 - 1}_{r_{-1}} = {(x + 2)\underbrace{(x^3 - 2x^2 - x + 2)}_{r_0}} + {5\underbrace{(x^2 - 1)}_{r_1}}, \\
         & r_0 = (x - 2) r_1.
    \end{align*}
    所以 $r_1 = x^2 - 1$ 就是 $r_{-1}$ 与 $r_0$ 的一个最大公因子. 再作二次带余除法 (当然, 事实上, 只要再作一次):
    \begin{align*}
         & r_{-1} = (x^2 + 1) r_1, \\
         & r_0 = (x - 2) r_1.
    \end{align*}
    所以
    \begin{align*}
        s = \frac{r_{-1}}{r_0} = \frac{(x^2 + 1) r_1}{(x - 2) r_1} = \frac{x^2 + 1}{x - 2}.
    \end{align*}
\end{example}

\begin{remark}
    事实上, 上二个例无本质区别: 有的可直接看出来, 而有的需要稍繁的计算.
\end{remark}

现在, 我们考虑有理式的加、减法.

\begin{proposition}
    若 $r$, $s$, $t$, $w$ 是有理式, 且 $s \neq 0$, $w \neq 0$, 则
    \begin{align*}
        \frac{r}{s} \pm \frac{t}{w} = \frac{rw \pm st}{sw}.
    \end{align*}
\end{proposition}

\begin{pf}
    作者不必再证一遍吧?
\end{pf}

当然, 我们在运算前, 一般先分别写 $\frac{r}{s}$ 与 $\frac{t}{w}$ 为最简形——这可使运算容易一些.

\begin{example}
    设 $r = \frac{2x^3 - 3x^2 + 1}{x^2 - 2x + 1}$, $s = \frac{x^2 - 4}{x^3 - 8}$, 求 $r + s$ 与 $r - s$.

    首先, 化 $r$, $s$ 为最简形:
    \begin{align*}
         & r = \frac{2x^3 - 3x^2 + 1}{x^2 - 2x + 1} = \frac{(x^2 - 2x + 1)(2x + 1)}{(x^2 - 2x + 1)} = \frac{2x + 1}{1}, \\
         & s = \frac{x^2 - 4}{x^3 - 8} = \frac{(x - 2)(x + 2)}{(x - 2)(x^2 + 2x + 4)} = \frac{x + 2}{x^2 + 2x + 4}.
    \end{align*}
    所以
    \begin{align*}
        r \pm s
        = {} & \frac{(2x + 1)(x^2 + 2x + 4) \pm 1 (x + 2)}{1(x^2 + 2x + 4)} \\
        = {} & \frac{(2x^3 + 5x^2 + 10x + 4) \pm (x + 2)}{x^2 + 2x + 4}.
    \end{align*}
    所以
    \begin{align*}
         & r + s = \frac{(2x^3 + 5x^2 + 10x + 4) + (x + 2)}{x^2 + 2x + 4} = \frac{2x^3 + 5x^2 + 11x + 6}{x^2 + 2x + 4}, \\
         & r - s = \frac{(2x^3 + 5x^2 + 10x + 4) - (x + 2)}{x^2 + 2x + 4} = \frac{2x^3 + 5x^2 + 9x + 2}{x^2 + 2x + 4}.
    \end{align*}
\end{example}

\begin{example}
    设 $r = \frac{x}{x^2 - 4}$, $s = \frac{3}{x^2 - 4}$. 求 $r + s$ 与 $r - s$.

    因为 $x$ 与 $x^2 - 4$ 互素, 且 $3$ 与 $x^2 - 4$ 互素, 故我们不必写 $r$ 或 $s$ 为最简形. 所以
    \begin{align*}
        r \pm s = \frac{x (x^2 - 4) \pm (x^2 - 4) 3}{(x^2 - 4)(x^2 - 4)} = \frac{(x \pm 3)(x^2 - 4)}{(x^2 - 4)(x^2 - 4)} = \frac{x \pm 3}{x^2 - 4}.
    \end{align*}
\end{example}

一般地, 下面的命题成立.
\begin{proposition}
    若 $r$, $s$, $t$ 是有理式, 且 $s \neq 0$, 则
    \begin{align*}
        \frac{r}{s} \pm \frac{t}{s} = \frac{r \pm t}{s}.
    \end{align*}
\end{proposition}

\begin{pf}
    直接计算即可:
    \begin{align*}
        \frac{r}{s} \pm \frac{t}{s}
        = {} & \frac{r}{s} + \frac{\pm t}{s} \\
        = {} & \frac{rs + s(\pm t)}{ss}      \\
        = {} & \frac{sr + s(\pm t)}{ss}      \\
        = {} & \frac{s(r \pm t)}{ss}         \\
        = {} & \frac{r \pm t}{s}. \qedhere
    \end{align*}
\end{pf}

\begin{remark}
    有时, 此公式可使计算稍容易一些.
\end{remark}

\begin{example}
    设 $r = \frac{1}{(x-1)^2 (x+1)}$, $s = \frac{1}{(x-1) (x+1)^2}$, 求 $r - s$.

    我们可写
    \begin{align*}
         & r = \frac{1}{(x-1)^2 (x+1)} = \frac{x+1}{(x-1)^2 (x+1)^2}, \\
         & s = \frac{1}{(x-1) (x+1)^2} = \frac{x-1}{(x-1)^2 (x+1)^2}.
    \end{align*}
    所以
    \begin{align*}
        r - s = \frac{(x+1) - (x-1)}{(x-1)^2 (x+1)^2} = \frac{2}{(x-1)^2 (x+1)^2}.
    \end{align*}
\end{example}

接下来, 我们考虑有理式的乘、乘法.

相比加、减法, 乘法与除法的公式简单一些.

\begin{proposition}
    若 $r$, $s$, $t$, $w$ 是有理式, 且 $s \neq 0$, $w \neq 0$, 则
    \begin{align*}
         & \frac{r}{s} \cdot \frac{t}{w} = \frac{rt}{sw}, \\
         & \frac{r/s}{t/w} = \frac{rw}{st}.
    \end{align*}
\end{proposition}

\begin{remark}
    $r/s$ 是 $\frac{r}{s}$ 的一种写法.
\end{remark}

\begin{pf}
    请读者自行证明这二个公式. 不过, 作者可给一个提示:
    \begin{align*}
         & \frac{r/s}{t/w} = (r/s) (t/w)^{-1} = (r/s) (w/t). \qedhere
    \end{align*}
\end{pf}

\begin{example}
    若 $r = \frac{x}{x - 1}$, $s = \frac{x + 1}{x}$, 求 $rs$ 与 $\frac{r}{s}$.

    直接套用公式即可:
    \begin{align*}
         & rs = \frac{x (x - 1)}{(x + 1) x} = \frac{x - 1}{x + 1},         \\
         & \frac{r}{s} = \frac{x x}{(x - 1)(x + 1)} = \frac{x^2}{x^2 - 1}.
    \end{align*}
\end{example}

\begin{example}
    若 $r = \frac{x}{x^2 - 1}$, $s = x + 1$, 求 $rs$.

    视 $s = \frac{x+1}{1}$, 再套用公式:
    \begin{align*}
        rs = \frac{x (x + 1)}{(x^2 - 1) \cdot 1} = \frac{x (x + 1)}{(x - 1)(x + 1)} = \frac{x}{x - 1}.
    \end{align*}
\end{example}

一般地, 下面的命题成立.
\begin{proposition}
    若 $r$, $s$, $t$ 是有理式, 且 $s \neq 0$, 则
    \begin{align*}
        \frac{r}{s} \cdot t = t \cdot \frac{r}{s} = \frac{rt}{s} = \frac{tr}{s}.
    \end{align*}
\end{proposition}

\begin{pf}
    因为乘法是交换的, 证明 $\frac{r}{s} \cdot t = \frac{rt}{s}$ 就够了. 因为 $t = \frac{t}{1}$, 故
    \begin{align*}
         & \frac{r}{s} \cdot t = \frac{r}{s} \cdot \frac{t}{1} = \frac{rt}{s1} = \frac{rt}{s}. \qedhere
    \end{align*}
\end{pf}

\begin{remark}
    这个命题也是省事的——乘法时, 不必每次都写 $t$ 为 $\frac{t}{1}$ 了.
\end{remark}

\myLine

最后我们介绍幂与一些乘法公式.

\begin{definition}
    设 $r$ 是有理式. 设 $n$ 是正整数.

    (i) $r^0$ 是 $r$ 的 $0$ 次幂. 定义 $r^0 = 1$.

    (ii) $r^n$ 是 $n$ 个 $r$ 的积.

    (iii) $r^{-n}$ 是 $n$ 个 $r^{-1}$ 的积.
\end{definition}

\begin{proposition}
    设 $r$, $s$ 是有理式, 且 $m$, $n$ 是非负整数. 则

    (i) $r^{m+n} = r^m r^n$.

    (ii) $(r^m)^n = r^{mn}$.

    (iii) $r^m s^m = (rs)^m$.
\end{proposition}

\begin{pf}
    (i) 根据结合律, $m + n$ 个 $r$ 的积是 $m$ 个 $r$ 的积与 $n$ 的 $r$ 的积的积. (读者也可对 $n$ 用算学归纳法. 当然, 因为我们约定 $r^0 = 1$, ``$0$ 个 $r$ 的积'' 当然是 $1$.)

    (ii) $(r^m)^n$ 按定义, 是 $n$ 个 $r^m$ 的积. $r^m$ 是 $m$ 个 $r$ 的积. 这里有 $mn$ 个 $r$, 根据结合律, 这个积就是 $r^{mn}$.

    (iii) $m = 0$ 或 $m = 1$ 时, 显然. 这里, 就需要交换律与结合律了. 读者可体会一下:
    \begin{align*}
        r^2 s^2
        = {} & (rr) (ss) = ((rr)s) s = (r(rs)) s \\
        = {} & ((rs) r) s = (rs) (rs) = (rs)^2.
    \end{align*}
    结合 (i), 读者可用算学归纳法 (起步的 $m$ 可以选 $0$):
    \begin{align*}
        r^{m+1} s^{m+1}
        = {} & (r^m r) (s^m s) = ((r^m r) s^m) s = (r^m (r s^m)) s \\
        = {} & (r^m (s^m r)) s = ((r^m s^m) r) s = (rs)^m (rs)     \\
        = {} & (rs)^{m+1}. \qedhere
    \end{align*}
\end{pf}

\begin{proposition}
    设 $r$, $s$ 是非零的有理式, 且 $m$, $n$ 是整数. 则

    (i) $r^{m+n} = r^m r^n$.

    (ii) $(r^m)^n = r^{mn}$.

    (iii) $r^m s^m = (rs)^m$.
\end{proposition}

\begin{pf}
    这里 $m$ 与 $n$ 可以是整数了. 不过, 对 $r$, $s$ 的要求也高了.

    (i) 假如 $m$, $n$ 都是非负整数, 这是显然的.

    假如 $m$, $n$ 都是负整数, 那么 $r^{m+n}$, $r^m$, $r^n$ 就是 $(r^{-1})^{-m-n}$, $(r^{-1})^{-m}$, $(r^{-1})^{-n}$. $-m$, $-n$ 都是非负整数, 且 $-m + (-n)$ 就是 $-m-n$.

    假如 $m$ 或 $n$ 的某一个是 $0$, 此事也是显然的. 麻烦的事情是 $m$ 与 $n$ 有正有负. 不失一般性, 设 $m > 0 > n$.

    若 $m + n = 0$, 则 $n = -m$. 所以
    \begin{align*}
        r^m r^n
        = {} & r^m r^{-m} = r^m (r^{-1})^m \\
        = {} & (rr^{-1})^m = 1^m           \\
        = {} & 1 = r^{m+n}.
    \end{align*}

    若 $m + n > 0$, 则 $m > -n$. 所以
    \begin{align*}
        r^m r^n = (r^{m+n} r^{-n}) r^n = r^{m+n} (r^{-n} r^n) = r^{m+n}.
    \end{align*}

    若 $m + n < 0$, 则 $m < -n$. 所以
    \begin{align*}
        r^m r^n
        = {} & r^m (r^{-1})^{-n} = r^m (r^{-1})^{m + (-m - n)}                         \\
        = {} & r^m ((r^{-1})^m (r^{-1})^{-m - n}) = (r^m (r^{-1})^m) (r^{-1})^{-m - n} \\
        = {} & (r^{-1})^{-m - n} = r^{m+n}.
    \end{align*}

    (ii) 若 $m$ 或 $n$ 为 $0$, 这是显然的——左、右二侧都是 $1$. 若 $m$, $n$ 是正整数, 就不必再证了.

    若 $m < 0$, $n > 0$, 则 $r^m = (r^{-1})^{-m}$. $((r^{-1})^{-m})^n = (r^{-1})^{(-m)n} = (r^{-1})^{-mn} = r^{mn}$.

    若 $m > 0$, $n < 0$, 则 $(r^m)^n = ((r^m)^{-1})^{-n}$. $(r^m)^{-1}$ 是什么呢? 因为 $(r^m)^{-1} r^m = 1$, 且 $(r^{-1} r)^m = (r^{-1})^m r^m = 1$, 故 $(r^m)^{-1} = (r^{-1})^m$. 所以 $(r^m)^n = ((r^{-1})^m)^{-n} = (r^{-1})^{m(-n)} = (r^{-1})^{-mn} = r^{-(-mn)} = r^{mn}$.

    若 $m < 0$, $n < 0$, 则 $r^m = (r^{-1})^{-m}$. 所以 $(r^m)^n = ((r^{-1})^{-m})^n = (r^{-1})^{-mn} = ((r^{-1})^{-1})^{-(-mn)} = ((r^{-1})^{-1})^{mn}$. $(r^{-1})^{-1}$ 是什么呢? 因为 $r r^{-1} = r^{-1} r = 1$, 且 $(r^{-1})^{-1} r^{-1} = 1$, 故 $(r^{-1})^{-1} = r$. 综上, $(r^m)^n = r^{mn}$.

    (iii) 若 $m \geq 0$, 则不必证了. 若 $m < 0$, 则 $r^m = (r^{-1})^{-m}$, $s^m = (s^{-1})^{-m}$. 所以 $r^m s^m = (r^{-1} s^{-1})^{-m}$. 不过, 因为 $(rs)^{-1} (rs) = 1$, 且 $(r^{-1} s^{-1}) (rs) = (r^{-1} r) (s^{-1} s) = 1$, 故 $(rs)^{-1} = r^{-1} s^{-1}$. 所以, $r^m s^m = ((rs)^{-1})^{-m} = (rs)^m$.
\end{pf}

总结一下上述二个命题, 就是
\begin{proposition}
    设 $r$, $s$ 是有理式, 且 $m$, $n$ 是非负整数. 有理式的幂适合如下规则:

    (i) $r^{m+n} = r^m r^n$.

    (ii) $(r^m)^n = r^{mn}$.

    (iii) $r^m s^m = (rs)^m$.

    若 $r$, $s$ 均不为 $0$, 则 $m$, $n$ 可取全体整数.
\end{proposition}

\begin{remark}
    上个命题的 ``有理式'' 可替换为 ``复数'' ``实数'' 或 ``有理数''——这些数的乘法都适合结合律与交换律, 且非零的数有倒数. 推理过程完全一致——改 ``有理式'' 为 ``复数'' ``实数'' 或 ``有理数'' 即可.
\end{remark}

现在, 我们看一些乘法公式. 不过, 在此之前, 我们先了解有理式与整式的复合.

\begin{definition}
    设
    \begin{align*}
        f(x) = a_0 + a_1 x + \cdots + a_n x^n
    \end{align*}
    是整式. 设 $r(x)$ 是有理式. 定义
    \begin{align*}
        (f \circ r)(x) = f(r(x)) = a_0 + a_1 r(x) + \cdots + a_n (r(x))^n.
    \end{align*}
    这称为有理式 $r(x)$ 与整式 $f(x)$ 的复合. 显然, $(f \circ r)(x)$ 仍为有理式.

    在语境明确的情况下, ``$(x)$'' 可不写出.
\end{definition}

\begin{example}
    设 $f = 1 + 2x + 3x^2$, $r = \frac{1}{x}$. 则
    \begin{align*}
        f \circ r = 1 + \frac{2}{x} + \frac{3}{x^2} = \frac{x^2 + 2x + 3}{x^2}.
    \end{align*}
\end{example}

\begin{proposition}
    设 $f$, $g$ 是整式, 且 $r$ 是有理式.

    (i) 若 $f = g$, 则 $f \circ r = g \circ r$.

    (ii) 若 $f + g = s$, 则 $f \circ r + g \circ r = s \circ r$. 这里, $\circ$ 的优先级高于 $+$; 所以, $f \circ r + g \circ r$ 是 $(f \circ r) + (g \circ r)$ 的简写.

    (iii) 若 $fg = p$, 则 $(f \circ r) \cdot (g \circ r) = p \circ r$.
\end{proposition}

\begin{pf}
    设
    \begin{align*}
         & f = a_0 + a_1 x + \cdots + a_n x^n, \\
         & g = b_0 + b_1 x + \cdots + b_n x^n.
    \end{align*}

    (i) 若 $f = g$, 则对 $0$ 至 $n$ 间的任意整数 $i$, 都有 $a_i = b_i$. 所以
    \begin{align*}
        f \circ r
        = {} & a_0 + a_1 r + \cdots + a_n r^n \\
        = {} & b_0 + b_1 r + \cdots + b_n r^n \\
        = {} & g \circ r.
    \end{align*}

    (ii) 若 $s = f + g$, 则
    \begin{align*}
        s = c_0 + c_1 x + \cdots + c_n x^n,
    \end{align*}
    其中
    \begin{align*}
        c_i = a_i + b_i, \quad i = 0,1,\cdots,n.
    \end{align*}
    所以
    \begin{align*}
             & f \circ r + g \circ r                                               \\
        = {} & (a_0 + a_1 r + \cdots + a_n r^n) + (b_0 + b_1 r + \cdots + b_n r^n) \\
        = {} & (a_0 + b_0) + (a_1 + b_1) r + \cdots + (a_n + b_n) r^n              \\
        = {} & c_0 + c_1 r + \cdots + c_n r^n                                      \\
        = {} & s \circ r.
    \end{align*}

    (iii) 若 $p = fg$, 则
    \begin{align*}
        p = d_0 + d_1 x + \cdots + d_{2n} x^{2n},
    \end{align*}
    其中
    \begin{align*}
        d_i = a_0 b_i + a_1 b_{i-1} + \cdots + a_i b_0, \quad i = 0,1,\cdots,2n.
    \end{align*}
    所以
    \begin{align*}
             & (f \circ r) \cdot (g \circ r)                                     \\
        = {} & (a_0 + a_1 r + \cdots + a_n r^n) (b_0 + b_1 r + \cdots + b_n r^n) \\
        = {} & (a_0 b_0) + (a_0 b_1 + a_1 b_0) r + \cdots + (a_n b_n) r^{2n}     \\
        = {} & d_0 + d_1 r + \cdots + d_{2n} r^{2n}                              \\
        = {} & p \circ r. \qedhere
    \end{align*}
\end{pf}

我们得到了如下命题:
\begin{proposition}
    若整式 $f_0 (x)$, $f_1 (x)$, $\cdots$, $f_{n-1} (x)$ 之间有一个由加法与乘法计算得到的关系, 那么将 $x$ 换为有理式 $r(x)$, 这样的关系仍成立.
\end{proposition}

从而, 我们有
\begin{proposition}
    设 $r$, $s$, $t$ 是有理式, $n$ 是正整数. 下面的乘法公式成立:
    \begin{align*}
         & (r + s)^{n} = r^n + \binom{n}{1} r^{n-1} s + \cdots + \binom{n}{i} r^{n-i} s^i + \cdots + s^n, \\
         & r^n - s^n = (r - s)(r^{n-1} + r^{n-2} s + \cdots + r^{n-i} s^{i-1} + \cdots + s^{n-1}),        \\
         & r^2 - s^2 = (r - s)(r + s),                                                                    \\
         & r^3 - s^3 = (r - s)(r^2 + rs + s^2),                                                           \\
         & r^3 + s^3 = (r + s)(r^2 - rs + s^2),                                                           \\
         & r^3 + s^3 + t^3 - 3rst = (r + s + t)(r^2 + s^2 + t^2 - rs - rt - st).
    \end{align*}
\end{proposition}

\begin{remark}
    事实上, 上面的乘法公式都可以直接验证. 不过, 不正式地说, 因为有理式与整式的加法与乘法的运算律是完全一致的 (当然, 涉及倒元的除外), 且这些乘法公式都是运用运算律推出的, 故它们自动地在有理式里也成立.
\end{remark}

本文就到这里. 读者辛苦了!
