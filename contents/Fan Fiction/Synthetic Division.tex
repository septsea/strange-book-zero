\subsection*{\SyntheticDivision}
\addcontentsline{toc}{subsection}{\SyntheticDivision}
\markright{\SyntheticDivision}

本文的目标是为读者介绍带余除法的一个特殊情况——综合除法 \term{synthetic division}. 当然, 细心的读者一定不会只学到综合除法.

还是老样子: ``数'' 一定是复数 (或实数、有理数); ``多项式'' 的系数一定是数.

前面, 我们讨论了多项式的一些性质. 我们没有在 ``\HEADING '' 里讨论那些性质, 是因为当时我们不需要 ``因子'' ``公因子'' ``最大公因子'' 等概念. 读者应该还记得, 多项式的微商、多项式的根、插值、广义二项系数、求和公式等内容是我们讨论的重点. 现在, 我们的方向变了很多.

在讨论多项式的根时, 我们曾经为读者介绍过这个命题:
\begin{proposition}
    设 $f(x)$ 是 $n$ 次多项式 ($n \geq 1$), $a$ 是数. 则存在 $n-1$ 次多项式 $q(x)$ 使
    \begin{align*}
        f(x) = q(x) (x-a) + f(a).
    \end{align*}
    根据带余除法, 这样的 $q(x)$ 一定是唯一的.
\end{proposition}

这是带余除法的推论. 我们当时并不关心 $q(x)$ 是什么; 我们只关心这个 $q(x)$ 不但存在, 且唯一. 我们用它建立了多项式与多项式函数的联系: (系数为数的) 多项式与多项式函数没有本质区别. 但现在, 我们不但关心 $q(x)$ 到底是什么, 我们还要给出一种方便计算 $q(x)$ 的方法——这就是综合除法所干的事情.

综合除法, 原则上, 当然也可以放在 ``\SomePropertiesOfPolynomials '' 里讨论. 不过, 作者为了让 ``\SomePropertiesOfIntegers '' 与 ``\SomePropertiesOfPolynomials '' 的结构一致, 作者决定专门写二篇文讨论多项式独有的东西: 综合除法与重因子. 这么安排, 还有一个好处: 消除了过长的文给读者带来的压力.

\begin{example}
    设 $f(x) = x^6 + x^3 + 1$. 我们计算 $x - 2$ 除 $f(x)$.

    我们先用普通的带余除法试试看. 显然, $\deg (x-2) = 1$. 这里, $x - 2$ 的首项系数为 $1$, 所以我们的计算并不会很复杂. 取
    \begin{align*}
        q_1 (x) = 1 \cdot 1^{-1} \cdot x^{6-1} = x^5.
    \end{align*}
    则
    \begin{align*}
        r_1 (x)
        = {} & f(x) - q_1 (x) (x - 2)         \\
        = {} & (x^6 + x^3 + 1) - x^5 (x - 2)  \\
        = {} & (x^6 + x^3 + 1) - (x^6 - 2x^5) \\
        = {} & 2x^5 + x^3 + 1.
    \end{align*}
    $r_1 (x)$ 的次仍不低于 $1$. 因此, 再来一次. 取
    \begin{align*}
        q_2 (x) = 2 \cdot 1^{-1} \cdot x^{5-1} = 2x^4.
    \end{align*}
    则
    \begin{align*}
        r_2 (x)
        = {} & r_1 (x) - q_2 (x) (x - 2)        \\
        = {} & (2x^5 + x^3 + 1) - 2x^4 (x - 2)  \\
        = {} & (2x^5 + x^3 + 1) - (2x^5 - 4x^4) \\
        = {} & 4x^4 + x^3 + 1.
    \end{align*}
    $r_2 (x)$ 的次仍不低于 $1$. 因此, 再来一次. 取
    \begin{align*}
        q_3 (x) = 4 \cdot 1^{-1} \cdot x^{4-1} = 4x^3.
    \end{align*}
    则
    \begin{align*}
        r_3 (x)
        = {} & r_2 (x) - q_3 (x) (x - 2)        \\
        = {} & (4x^4 + x^3 + 1) - 4x^3 (x - 2)  \\
        = {} & (4x^4 + x^3 + 1) - (4x^4 - 8x^3) \\
        = {} & 9x^3 + 1.
    \end{align*}
    $r_3 (x)$ 的次仍不低于 $1$. 因此, 再来一次. 取
    \begin{align*}
        q_4 (x) = 9 \cdot 1^{-1} \cdot x^{3-1} = 9x^2.
    \end{align*}
    则
    \begin{align*}
        r_4 (x)
        = {} & r_3 (x) - q_4 (x) (x - 2)   \\
        = {} & (9x^3 + 1) - 9x^2 (x - 2)   \\
        = {} & (9x^3 + 1) - (9x^3 - 18x^2) \\
        = {} & 18x^2 + 1.
    \end{align*}
    $r_4 (x)$ 的次仍不低于 $1$. 因此, 再来一次. 取
    \begin{align*}
        q_5 (x) = 18 \cdot 1^{-1} \cdot x^{2-1} = 18x.
    \end{align*}
    则
    \begin{align*}
        r_5 (x)
        = {} & r_4 (x) - q_5 (x) (x - 2)   \\
        = {} & (18x^2 + 1) - 18x (x - 2)   \\
        = {} & (18x^2 + 1) - (18x^2 - 36x) \\
        = {} & 36x + 1.
    \end{align*}
    $r_5 (x)$ 的次仍不低于 $1$. 因此, 再来一次. 取
    \begin{align*}
        q_6 (x) = 36 \cdot 1^{-1} \cdot x^{1-1} = 36.
    \end{align*}
    则
    \begin{align*}
        r_6 (x)
        = {} & r_5 (x) - q_6 (x) (x - 2) \\
        = {} & (36x + 1) - 36 (x - 2)    \\
        = {} & (36x + 1) - (36x - 72)    \\
        = {} & 73.
    \end{align*}
    $r_6 (x)$ 的次低于 $1$. 这样
    \begin{align*}
             & f(x)                                                                        \\
        = {} & q_1 (x) (x-2) + r_1 (x)                                                     \\
        = {} & q_1 (x) (x-2) + q_2 (x) (x-2) + r_2 (x)                                     \\
        = {} & q_1 (x) (x-2) + q_2 (x) (x-2) + q_3 (x) (x-2) + r_3 (x)                     \\
        = {} & q_1 (x) (x-2) + q_2 (x) (x-2) + q_3 (x) (x-2) + q_4 (x) (x-2) + r_4 (x)     \\
        = {} & q_1 (x) (x-2) + q_2 (x) (x-2) + q_3 (x) (x-2) + q_4 (x) (x-2)               \\
             & \qquad + q_5 (x) (x-2) + r_5 (x)                                            \\
        = {} & q_1 (x) (x-2) + q_2 (x) (x-2) + q_3 (x) (x-2) + q_4 (x) (x-2)               \\
             & \qquad + q_5 (x) (x-2) + q_6 (x) (x-2) + r_6 (x)                            \\
        = {} & (q_1 (x) + q_2 (x) + q_3 (x) + q_4 (x) + q_5 (x) + q_6 (x)) (x-2) + r_6 (x) \\
        = {} & (x^5 + 2x^4 + 4x^3 + 9x^2 + 18x + 36) (x-2) + 73.
    \end{align*}
    也就是说,
    \begin{align*}
        q(x) = x^5 + 2x^4 + 4x^3 + 9x^2 + 18x + 36, \quad f(2) = r_6 (x) = 73.
    \end{align*}
\end{example}
