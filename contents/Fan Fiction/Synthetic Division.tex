\subsection*{\SyntheticDivision}
\addcontentsline{toc}{subsection}{\SyntheticDivision}
\markright{\SyntheticDivision}

本文的目标是为读者介绍带余除法的一个特殊情况——综合除法 \term{synthetic division}. 当然, 细心的读者一定不会只学到综合除法.

还是老样子: ``数'' 一定是复数 (或实数、有理数); ``整式'' 的系数一定是数.

前面, 我们讨论了整式的一些性质. 我们没有在 ``\HEADING'' 里讨论那些性质, 是因为当时我们不需要 ``因子'' ``公因子'' ``最大公因子'' 等概念. 读者应该还记得, 整式的流数、整式的根、插值、广义二项系数、求和公式等内容是我们讨论的重点. 现在, 我们的方向变了很多.

在讨论整式在一点的取值时, 我们曾经为读者介绍过这个命题:
\begin{proposition}
    设 $f(x)$ 是 $n$ 次整式 ($n \geq 1$), $a$ 是数. 则存在 $n-1$ 次整式 $q(x)$ 使
    \begin{align*}
        f(x) = q(x) (x-a) + f(a).
    \end{align*}
    根据带余除法, 这样的 $q(x)$ 一定是唯一的.
\end{proposition}

这是带余除法的推论. 我们当时并不关心 $q(x)$ 是什么; 我们只关心这个 $q(x)$ 不但存在, 且唯一. 但现在, 我们不但关心 $q(x)$ 到底是什么, 我们还要给出一种方便计算 $q(x)$ 的方法——这就是综合除法所干的事情.

综合除法, 原则上, 当然也可以放在 ``\SomePropertiesOfPolynomials'' 里讨论. 不过, 作者为了让 ``\SomePropertiesOfIntegers'' 与 ``\SomePropertiesOfPolynomials'' 的结构一致, 作者决定专门写二篇文讨论整式独有的东西: 综合除法与重因子. 这么安排, 还有一个好处: 消除了过长的文给读者带来的压力.

\begin{example}
    设 $f(x) = x^6 + x^3 + 1$. 我们计算 $x - 2$ 除 $f(x)$.

    我们先用普通的带余除法试试看. 显然, $\deg (x-2) = 1$. 这里, $x - 2$ 的首项系数为 $1$, 所以我们的计算并不会很复杂. 取
    \begin{align*}
        q_1 (x) = 1 \cdot 1^{-1} \cdot x^{6-1} = x^5.
    \end{align*}
    则
    \begin{align*}
        r_1 (x)
        = {} & f(x) - q_1 (x) (x - 2)         \\
        = {} & (x^6 + x^3 + 1) - x^5 (x - 2)  \\
        = {} & (x^6 + x^3 + 1) - (x^6 - 2x^5) \\
        = {} & 2x^5 + x^3 + 1.
    \end{align*}
    $r_1 (x)$ 的次仍不低于 $1$. 因此, 再来一次. 取
    \begin{align*}
        q_2 (x) = 2 \cdot 1^{-1} \cdot x^{5-1} = 2x^4.
    \end{align*}
    则
    \begin{align*}
        r_2 (x)
        = {} & r_1 (x) - q_2 (x) (x - 2)        \\
        = {} & (2x^5 + x^3 + 1) - 2x^4 (x - 2)  \\
        = {} & (2x^5 + x^3 + 1) - (2x^5 - 4x^4) \\
        = {} & 4x^4 + x^3 + 1.
    \end{align*}
    $r_2 (x)$ 的次仍不低于 $1$. 因此, 再来一次. 取
    \begin{align*}
        q_3 (x) = 4 \cdot 1^{-1} \cdot x^{4-1} = 4x^3.
    \end{align*}
    则
    \begin{align*}
        r_3 (x)
        = {} & r_2 (x) - q_3 (x) (x - 2)        \\
        = {} & (4x^4 + x^3 + 1) - 4x^3 (x - 2)  \\
        = {} & (4x^4 + x^3 + 1) - (4x^4 - 8x^3) \\
        = {} & 9x^3 + 1.
    \end{align*}
    $r_3 (x)$ 的次仍不低于 $1$. 因此, 再来一次. 取
    \begin{align*}
        q_4 (x) = 9 \cdot 1^{-1} \cdot x^{3-1} = 9x^2.
    \end{align*}
    则
    \begin{align*}
        r_4 (x)
        = {} & r_3 (x) - q_4 (x) (x - 2)   \\
        = {} & (9x^3 + 1) - 9x^2 (x - 2)   \\
        = {} & (9x^3 + 1) - (9x^3 - 18x^2) \\
        = {} & 18x^2 + 1.
    \end{align*}
    $r_4 (x)$ 的次仍不低于 $1$. 因此, 再来一次. 取
    \begin{align*}
        q_5 (x) = 18 \cdot 1^{-1} \cdot x^{2-1} = 18x.
    \end{align*}
    则
    \begin{align*}
        r_5 (x)
        = {} & r_4 (x) - q_5 (x) (x - 2)   \\
        = {} & (18x^2 + 1) - 18x (x - 2)   \\
        = {} & (18x^2 + 1) - (18x^2 - 36x) \\
        = {} & 36x + 1.
    \end{align*}
    $r_5 (x)$ 的次仍不低于 $1$. 因此, 再来一次. 取
    \begin{align*}
        q_6 (x) = 36 \cdot 1^{-1} \cdot x^{1-1} = 36.
    \end{align*}
    则
    \begin{align*}
        r_6 (x)
        = {} & r_5 (x) - q_6 (x) (x - 2) \\
        = {} & (36x + 1) - 36 (x - 2)    \\
        = {} & (36x + 1) - (36x - 72)    \\
        = {} & 73.
    \end{align*}
    $r_6 (x)$ 的次低于 $1$. 这样
    \begin{align*}
             & f(x)                                                                        \\
        = {} & q_1 (x) (x-2) + r_1 (x)                                                     \\
        = {} & q_1 (x) (x-2) + q_2 (x) (x-2) + r_2 (x)                                     \\
        = {} & q_1 (x) (x-2) + q_2 (x) (x-2) + q_3 (x) (x-2) + r_3 (x)                     \\
        = {} & q_1 (x) (x-2) + q_2 (x) (x-2) + q_3 (x) (x-2) + q_4 (x) (x-2) + r_4 (x)     \\
        = {} & q_1 (x) (x-2) + q_2 (x) (x-2) + q_3 (x) (x-2) + q_4 (x) (x-2)               \\
             & \qquad + q_5 (x) (x-2) + r_5 (x)                                            \\
        = {} & q_1 (x) (x-2) + q_2 (x) (x-2) + q_3 (x) (x-2) + q_4 (x) (x-2)               \\
             & \qquad + q_5 (x) (x-2) + q_6 (x) (x-2) + r_6 (x)                            \\
        = {} & (q_1 (x) + q_2 (x) + q_3 (x) + q_4 (x) + q_5 (x) + q_6 (x)) (x-2) + r_6 (x) \\
        = {} & (x^5 + 2x^4 + 4x^3 + 9x^2 + 18x + 36) (x-2) + 73.
    \end{align*}
    也就是说,
    \begin{align*}
        q(x) = x^5 + 2x^4 + 4x^3 + 9x^2 + 18x + 36, \quad f(2) = r_6 (x) = 73.
    \end{align*}

    读者可能感到疲劳. 的确, 作者自己都快要睡着了. 这些文字打出来, 作者可再算九遍了吧.
\end{example}

设 $a$ 为数. 我们用 $x-a$ 除 $f(x)$. 设 $f(x)$ 的次为 $n$, 且 $n \geq 1$ (若 $n < 1$, 则 $x-a$ 除 $f(x)$ 的商与余式分别是 $0$ 与 $f(x)$). 所以, 商的次是 $n-1$, 且余式 (可认为) 是数. 这样, 我们可以待定系数. 具体地说, 设
\begin{align*}
    f(x) = a_n x^n + a_{n-1} x^{n-1} + \cdots + a_1 x + a_0,
\end{align*}
且
\begin{align*}
    f(x) = (x - a) (b_{n-1} x^{n-1} + b_{n-2} x^{n-2} + \cdots + b_0) + b_{-1}.
\end{align*}
上式可写为
\begin{align*}
    f(x)
    = {} & b_{n-1} x^n + (b_{n-2} - ab_{n-1}) x^{n-1} + \cdots + (b_{j-1} - ab_j) x^j \\
         & \qquad + \cdots + (b_0 - ab_1) x + (b_{-1} - ab_0).
\end{align*}
比较系数, 有
\begin{align*}
     & a_n = b_{n-1},                                           \\
     & a_{n-1} = b_{n-2} - ab_{n-1},                            \\
     & \cdots \cdots \cdots \cdots \cdots \cdots \cdots \cdots, \\
     & a_j = b_{j-1} - ab_j \quad (0 \leq j < n),               \\
     & \cdots \cdots \cdots \cdots \cdots \cdots \cdots \cdots, \\
     & a_1 = b_0 - ab_1,                                        \\
     & a_0 = b_{-1} - ab_0.
\end{align*}
由此解出
\begin{align*}
    \tag*{(R)} \begin{aligned}
         & b_{n-1} = a_n,                                     \\
         & b_{j-1} = ab_j + a_j \quad (j = n-1,n-2,\cdots,0).
    \end{aligned}
\end{align*}

\begin{remark}
    或许, 读者觉得 $b_{j-1} = ab_j + a_j$ 的右侧还有 ``未知的'' $b_j$, 因此作者 ``并没有解出 $b_{n-1}$, $b_{n-2}$, $\cdots$, $b_0$, $b_{-1}$''. 事实上, 作者在后面也写了, $j$ 取 $n-1$, $n-2$, $\cdots$, $0$. 因为 $b_{n-1}$ 已知 (它就是 $a_n$), 故可求出 $b_{n-2} = ab_{n-1} + a_{n-1} = a_n a + a_{n-1}$. 所以, 读者可接着求出 $b_{n-3} = ab_{n-2} + a_{n-2} = a_n a^2 + a_{n-1} a + a_{n-2}$. 也就是说, $b_{n-1}$, $b_{n-2}$, $\cdots$, $b_0$, $b_{-1}$ 是按次序被求出的数. 当然, 作者知道, 肯定有读者不服. 作为参考, 作者也给出一个直接的表达式.

    一般地, $b_{n-1}$, $b_{n-2}$, $\cdots$, $b_0$, $b_{-1}$ 的具体的表达式如下:
    \begin{align*}
        \tag*{(E)} \begin{aligned}
             & b_{n-1} = a_n,                                                                        \\
             & b_{n-2} = a_n a + a_{n-1},                                                            \\
             & b_{n-3} = a_n a^2 + a_{n-1} a + a_{n-2},                                              \\
             & \cdots \cdots \cdots \cdots \cdots \cdots \cdots \cdots \cdots \cdots \cdots \cdots,  \\
             & b_{j} = a_n a^{n-1-j} + a_{n-1} a^{n-1-j-1} + \cdots + a_{j+1} \quad (-1 \leq j < n), \\
             & \cdots \cdots \cdots \cdots \cdots \cdots \cdots \cdots \cdots \cdots \cdots \cdots,  \\
             & b_{0} = a_n a^{n-1} + a_{n-1} a^{n-2} + \cdots + a_1,                                 \\
             & b_{-1} = a_n a^n + a_{n-1} a^{n-1} + \cdots + a_1 a + a_0.
        \end{aligned}
    \end{align*}
\end{remark}

\begin{example}
    还是取 $f(x) = x^6 + x^3 + 1$. 我们计算 $x - 2$ 除 $f(x)$. 这里, $a = 2$.

    如果利用公式 (R), 则
    \begin{align*}
         & b_5 = a_6 = 1,            \\
         & b_4 = ab_5 + a_5 = 2,     \\
         & b_3 = ab_4 + a_4 = 4,     \\
         & b_2 = ab_3 + a_3 = 9,     \\
         & b_1 = ab_2 + a_2 = 18,    \\
         & b_0 = ab_1 + a_1 = 36,    \\
         & b_{-1} = ab_0 + a_0 = 73.
    \end{align*}
    故
    \begin{align*}
        f(x)
        = {} & (x - a)(b_5 x^5 + b_4 x^4 + b_3 x^3 + b_2 x^2 + b_1 x + b_0) + b_{-1} \\
        = {} & (x^5 + 2x^4 + 4x^3 + 9x^2 + 18x + 36) (x-2) + 73.
    \end{align*}

    可是, 如果用公式 (E), 则
    \begin{align*}
         & b_5 = a_6 = 1,                                                        \\
         & b_4 = a_6 a + a_5 = a = 2,                                            \\
         & b_3 = a_6 a^2 + a_5 a + a_4 = a^2 = 4,                                \\
         & b_2 = a_6 a^3 + a_5 a^2 + a_4 a + a_3 = a^3 + 1 = 9,                  \\
         & b_1 = a_6 a^4 + a_5 a^3 + a_4 a^2 + a_3 a + a_2 = a^4 + a = 18,       \\
         & b_0 = a_6 a^5 + a_5 a^4 + a_4 a^3 + a_3 a^2 + a_2 a = a^5 + a^2 = 36, \\
         & \begin{aligned}
            b_{-1}
            = {} & a_6 a^6 + a_5 a^5 + a_4 a^4 + a_3 a^3 + a_2 a^2 + a_1 a + a_0 \\
            = {} & a^6 + a^3 + 1                                                 \\
            = {} & 73.
        \end{aligned}
    \end{align*}
    结果当然是一样的. 不过, 读者是否感觉, 公式 (E) 不如公式 (R) 简单? 公式 (R) 里, 后一个数 ($b_{j-1}$) 都是 $f(x)$ 的某个系数 ($a_j$) 加前一个数 ($b_j$) 乘 $a$; 公式 (E) 里, 越到后面, 表达式越长. 作者挑选的 $f(x)$ 的 $1$, $2$, $4$, $5$ 次系数都是 $0$, 所以还不是那么可怕. 但如果作者挑选的整式的系数全都不是 $0$ 呢?

    这就是作者推荐公式 (R) 的理由.
\end{example}

\myLine

下面我们来看看综合除法的应用.

读者可能已经注意到了, $b_{-1} = f(a)$. 这是正确的: 因为商与余式是唯一的. 所以, 如果不关心 $b_{-1}$ 之前的数 $b_{n-1}$, $b_{n-2}$, $\cdots$, $b_1$, $b_0$ 的意义, 我们可得到计算整式在点 $a$ 的值的秦九韶算法\myFN{在西方, 一般用不列颠算学家 William George Horner 的名字命名此算法; 在中国, 一般用中国算学家秦九韶的名字命名此算法.}:
\begin{proposition}
    设 $a$ 是数. 设 $n$ 是正整数. 设
    \begin{align*}
        f(x) = a_n x^n + a_{n-1} x^{n-1} + \cdots + a_1 x + a_0.
    \end{align*}
    按如下规则作 $n$ 个数 $b_{n-1}$, $b_{n-2}$, $\cdots$, $b_0$, $b_{-1}$:
    \begin{align*}
         & b_{n-1} = a_n,                                     \\
         & b_{j-1} = ab_j + a_j \quad (j = n-1,n-2,\cdots,0).
    \end{align*}
    则 $b_{-1} = f(a)$.
\end{proposition}

\begin{example}
    设
    \begin{align*}
        f(x) = 8 x^8 - 12 x^7 - 2 x^6 + 43 x^5 - 78 x^4 + 77 x^3 - 46 x^2 + 15 x - 2.
    \end{align*}
    设 $a = 3$. 求 $f(a)$.

    读者可以试试直接将 $x$ 替换为 $a$. 不难看出, 计算 $a_j a^j$ 需 $j + 1$ 次乘法 ($j \geq 1$), 故直接将 $x$ 替换为 $a$, 需 $9 + 8 + 7 + \cdots + 2 + 0 = 35$ 次乘法. 最后, 把 $9$ 个数相加, 需 $8$ 次加法. 挑战有点大; 请有兴趣的读者这么算一算.

    再试试上个命题所说的方法:
    \begin{align*}
         & b_7 = a_8 = 8,                 \\
         & b_6 = ab_7 + a_7 = 12,         \\
         & b_5 = ab_6 + a_6 = 34,         \\
         & b_4 = ab_5 + a_5 = 145,        \\
         & b_3 = ab_4 + a_4 = 357,        \\
         & b_2 = ab_3 + a_3 = 1\,148,     \\
         & b_1 = ab_2 + a_2 = 3\,398,     \\
         & b_0 = ab_1 + a_1 = 10\,209,    \\
         & b_{-1} = ab_0 + a_0 = 30\,625.
    \end{align*}
    由此可见, 每步
    \begin{align*}
        b_{j-1} = ab_j + a_j \quad (j = 7, 6, \cdots, 0)
    \end{align*}
    需 $1$ 次乘法与 $1$ 次加法. $j$ 从 $7$ 降到 $0$, 故有 $8$ 步. 所以, 用此方法, 需 $8$ 次乘法与 $8$ 次加法.

    现在, 读者应该能体会到此法的威力了.
\end{example}

\myLine

我们还可利用综合除法得到一个很有用的乘法公式.

设 $n$ 是正整数. 设
\begin{align*}
    f(x) = x^n = a_n x^n + a_{n-1} x^{n-1} + \cdots + a_1 x + a_0.
\end{align*}
则
\begin{align*}
    a_n = 1, \quad a_{n-1} = a_{n-2} = \cdots = a_1 = a_0 = 0.
\end{align*}
设 $a$ 是某个非零数. 设
\begin{align*}
    f(x) = (x - a) (b_{n-1} x^{n-1} + b_{n-2} x^{n-2} + \cdots + b_0) + b_{-1}.
\end{align*}
根据综合除法, 知
\begin{align*}
     & b_{n-1} = a_n,                                                      \\
     & b_{j-1} = ab_j + a_j \quad (j = n-1,n-2,\cdots,0). \tag*{(\myStar)}
\end{align*}
由于 $0 \leq j < n$ 时 $a_j = 0$, 故 (\myStar) 变为
\begin{align*}
    b_{j-1} = ab_j.
\end{align*}
二侧同乘 $a^{j-1}$, 得
\begin{align*}
    a^j b_j = a^{j-1} b_{j-1}, \quad 0 \leq j < n.
\end{align*}
由此可知
\begin{align*}
    a^{-1} b_{-1} = \cdots = a^{j-1} b_{j-1} = a^j b_j = a^{j+1} b_{j+1} = \cdots = a^{n-1} b_{n-1} = a^{n-1}.
\end{align*}
所以
\begin{align*}
    a^j b_j = a^{n-1}, \quad -1 \leq j < n.
\end{align*}
所以
\begin{align*}
    b_j = a^{n-1-j}, \quad -1 \leq j < n.
\end{align*}
所以
\begin{align*}
    x^n = (x - a)(x^{n-1} + ax^{n-2} + \cdots + a^{n-2}x + a^{n-1}) + a^n.
\end{align*}
此式也可写为
\begin{align*}
    x^n - a^n = (x - a)(x^{n-1} + ax^{n-2} + \cdots + a^{n-2}x + a^{n-1}).
\end{align*}
我们得到了重要的乘法公式:
\begin{proposition}
    设 $n$ 是正整数. 设 $a$ 是数, 且 $a \neq 0$. 则
    \begin{align*}
        x^n - a^n = (x - a)(x^{n-1} + ax^{n-2} + \cdots + a^{n-2}x + a^{n-1}).
    \end{align*}
\end{proposition}

值得一提的是, 这个公式可推广为
\begin{proposition}
    设 $f$, $g$ 是整式. 设 $n$ 是正整数. 则
    \begin{align*}
        f^n - g^n = (f - g)(f^{n-1} + f^{n-2} g + \cdots + f^{n-i} g^{i-1} + \cdots + g^{n-1}).
    \end{align*}
\end{proposition}

\begin{pf}
    记
    \begin{align*}
        P = f^{n-1} + f^{n-2} g + \cdots + f^{n-i} g^{i-1} + \cdots + g^{n-1}.
    \end{align*}
    则
    \begin{align*}
         & fP = f^n + f^{n-1} g + f^{n-2} g^2 + \cdots + fg^{n-1},                    \\
         & gP = \hphantom{f^n + {}} f^{n-1}g + f^{n-2} g^2 + \cdots + fg^{n-1} + g^n.
    \end{align*}
    从而
    \begin{align*}
         & (f - g)P = fP - gP = f^n - g^n. \qedhere
    \end{align*}
\end{pf}

\begin{example}
    设 $n = 2$. 则
    \begin{align*}
        f^2 - g^2 = (f - g)(f + g).
    \end{align*}
    这就是平方差公式. 类似地, 取 $n = 3$. 则
    \begin{align*}
        f^3 - g^3 = (f - g)(f^2 + fg + g^2).
    \end{align*}
    这就是立方差公式. 若把 $g$ 换为 $-g$, 则
    \begin{align*}
         & f^3 - (-g)^3 = f^3 + g^3,                                   \\
         & (f - (-g))(f^2 + f(-g) + (-g)^2) = (f + g)(f^2 - fg + g^2).
    \end{align*}
    由此可得立方和公式:
    \begin{align*}
        f^3 + g^3 = (f + g)(f^2 - fg + g^2).
    \end{align*}
\end{example}

最后, 我们以一个稍复杂的 (但有用的) 例结束本文.

\begin{example}
    设 $f$, $g$, $h$ 都是整式. 则
    \begin{align*}
             & f^3 + g^3 + h^3 - 3fgh                                        \\
        = {} & (f + g)(f^2 - fg + g^2) + h^3 - 3fgh                          \\
        = {} & (f + g)(f^2 + 2fg + g^2 - 3fg) + h^3 - 3fgh                   \\
        = {} & (f + g)(f^2 + 2fg + g^2) - (f + g)(3fg) + h^3 - 3fgh          \\
        = {} & (f + g)^3 + h^3 - (3fg(f + g) + 3fgh)                         \\
        = {} & (f + g + h)((f + g)^2 - (f + g)h + h^2) - (f + g + h)(3fg)    \\
        = {} & (f + g + h)(f^2 + 2fg + g^2 - fh - gh + h^2 - 3fg)            \\
        = {} & (f + g + h)\underbrace{(f^2 + g^2 + h^2 - fg - fh - gh)}_{P}.
    \end{align*}
    由此, 我们得到了新的公式:
    \begin{align*}
        f^3 + g^3 + h^3 - 3fgh = (f + g + h)(f^2 + g^2 + h^2 - fg - fh - gh).
    \end{align*}

    若假定 $f$, $g$, $h$ 都是复系数整式, 我们还可以对 $P$ 下手:
    \begin{align*}
             & f^2 + g^2 + h^2 - fg - fh - gh                                                                                                                                                                                \\
        = {} & f^2 - 2f \cdot \frac{g+h}{2} + (g^2 - gh + h^2)                                                                                                                                                               \\
        = {} & f^2 - 2f \cdot \frac{g+h}{2} + \frac{(g+h)^2}{4} + (g^2 - gh + h^2) - \left( \frac{g^2}{4} + \frac{2gh}{4} + \frac{h^2}{4} \right)                                                                            \\
        = {} & \left( f - \frac{g}{2} - \frac{h}{2} \right)^2 + \frac{3}{4} (g^2 - 2gh + h^2)                                                                                                                                \\
        = {} & \left( f - \frac{g}{2} - \frac{h}{2} \right)^2 - \left( \frac{\ii \sqrt{3}}{2} (g - h) \right)^2                                                                                                              \\
        = {} & \left( f - \frac{g}{2} - \frac{h}{2} + \ii \sqrt{3}\, \frac{g}{2} - \ii \sqrt{3}\, \frac{h}{2} \right) \left( f - \frac{g}{2} - \frac{h}{2} - \ii \sqrt{3}\, \frac{g}{2} + \ii \sqrt{3}\, \frac{h}{2} \right) \\
        = {} & \left( f + \frac{-1 + \ii \sqrt{3}}{2} g + \frac{-1 - \ii \sqrt{3}}{2} h \right) \left( f + \frac{-1 - \ii \sqrt{3}}{2} g + \frac{-1 + \ii \sqrt{3}}{2} h \right).
    \end{align*}
    记
    \begin{align*}
        \omega = \frac{-1 + \ii \sqrt{3}}{2}.
    \end{align*}
    则
    \begin{align*}
        \omega^2 = \frac{(-1)^2 + 2(-1) \ii \sqrt{3} + 3\ii^2}{4} = \frac{-1 - \ii \sqrt{3}}{2}.
    \end{align*}
    故
    \begin{align*}
             & f^2 + g^2 + h^2 - fg - fh - gh                                                                                                                                    \\
        = {} & \left( f + \frac{-1 + \ii \sqrt{3}}{2} g + \frac{-1 - \ii \sqrt{3}}{2} h \right) \left( f + \frac{-1 - \ii \sqrt{3}}{2} g + \frac{-1 + \ii \sqrt{3}}{2} h \right) \\
        = {} & (f + \omega g + \omega^2 h) (f + \omega^2 g + \omega h).
    \end{align*}
    所以
    \begin{align*}
        f^3 + g^3 + h^3 - 3fgh = (f + g + h) (f + \omega g + \omega^2 h) (f + \omega^2 g + \omega h).
    \end{align*}
\end{example}

感谢读者的阅读.
