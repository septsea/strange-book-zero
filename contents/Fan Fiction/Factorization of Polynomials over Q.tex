\subsection*{\FactorizationOfPolynomialsOverQ}
\addcontentsline{toc}{subsection}{\FactorizationOfPolynomialsOverQ}
\markright{\FactorizationOfPolynomialsOverQ}

作者将在本文为读者介绍有理系数整式的因子分解.

读者可能还能想起这个命题:
\begin{proposition}
    设整式 $f$ 既不是 $0$, 也不是单位.

    (i) 存在不可约的整式 $p_1$, $p_2$, $\cdots$, $p_m$ 使
    \begin{align*}
        f = p_1 p_2 \cdots p_m;
    \end{align*}

    (ii) 若 $q_1$, $q_2$, $\cdots$, $q_m$, $s_1$, $s_2$, $\cdots$, $s_n$ 是不可约的整式, 且
    \begin{align*}
        f = q_1 q_2 \cdots q_m = s_1 s_2 \cdots s_n,
    \end{align*}
    则 $m = n$, 且可以适当地调换 $s_1$, $s_2$, $\cdots$, $s_m$ 的顺序, 使任取 $1$ 至 $m$ 间的整数 $\ell$, $q_\ell$ 与 $s_\ell$ 相伴 (注意: 调换顺序后的 $s_\ell$ 不一定跟原来的 $s_\ell$ 相等!).
\end{proposition}

如果读者还能回忆起此命题的证明, 读者就会发现: 我们只要知道 $1$ 次整式是不可约的就够了 (算学归纳法的始条件: 命题对次为 $1$ 的整式成立). 换句话说, 虽然此命题断言, 我们可写既不是 $0$, 也不是单位的整式为若干个不可约的整式的积, 但它可没告诉我们怎么写. 本文就是要告诉读者一个具体的写法.

作者先提醒读者: 或许, 本文跟 ``\FactorizationOfIntegers'' 类似; 读者可对比阅读二文.

在进入正题前, 作者先给出因子分解的定义.

\begin{example}
    事实上, 我们在初中就学习了如何 ``因式分解'' 整式\myFN{这里, ``因式'' 就是我们常说的 ``因子''. 在算学里, 整数的 ``因子'' 是 ``因数'', 整式的 ``因子'' 是 ``因式''. 不过, 我们在此统一使用 ``因子''.}. 作者查阅了初中算学课本. 在人民教育出版社出版的算学课本\myFN{感兴趣的读者可进入 \texttt{https://mp.weixin.qq.com/s/YBX0oYzlADELmtjOM-wJaw}; 八年级, 上册; 114 页.}里, 因式分解被描述为:
    \begin{quotation}
        (探究) 请把下列整式写成整式的积的形式:

        (1) $x^2 + x$;

        (2) $x^2 - 1$.

        ~\

        根据整式的乘法, 可以联想得到
        \begin{align*}
             & x^2 + x = x(x+1),         \\
             & x^2 - 1 = (x + 1)(x - 1).
        \end{align*}

        上面我们把一个整式化成了几个整式的积的形式, 像这样的式的变形叫作整式的因式分解 \term{factorization}, 也叫作把这个整式分解因式.
    \end{quotation}

    现在, 作者提出一个问题: $1$ 是不是整式? 如果是, 那么任意的整式 $f$ 均可写为几个整式的积: $f = 1 \cdot f$. 这样, 我们根本就不必讨论因式分解了! 所以, 看上去, $1$ 不是整式! 不过, 如果 $1$ 不是整式, 那么 $1$ 是什么呢? 仅仅是一个数? 又或者说, 什么都不是?

    读者可能会说: 分解因式, 必须进行到每一个整式因式都不能再分解为止. 彳亍. 那么, 作者再排出几个问题: 什么是 ``不能再分解''? $3x + 2$ 能不能再分解? $4x - 8$ 能不能再分解? $-x - 1$ 能不能再分解? $x^2 - 2$ 能不能再分解? 或许, 读者认为 $3x + 2$ 不能再分解, 而 $4x - 8$ 还能再分解, 是吧? 作者知道读者怎么想: $4x - 8 = 4(x - 2)$, 对不? 可这个 $4$ 能不能再分解? $4 = 2 \cdot 2$, 对吧? 读者为什么不写 $4x - 8 = 2 \cdot 2 \cdot (x-2)$ 呢? $-x - 1$ 能不能写为 $(-1) \cdot (x+1)$? $x^2 - 2$ 能不能写为 $(x - \sqrt{2}) \cdot (x + \sqrt{2})$?

    由此可见, 读者在中学算学里学到的 ``因式分解'' 不是十分清楚. 作者的任务就是: 清楚地写出本文的 (显然作者无法让别的地方的) ``有理系数整式的因子分解'' 的含义.

    作者顺便提一下, 作者在互联网上看到很多关于 $0.\dot{9}$ (零点九, 九循环) 与 $1$ 的大小关系的讨论了. 由于不明确 $0.\dot{9}$ 的定义, 大家就在说 ``这不对!'' ``这有大问题!'' 等评论. 由于此问题的详尽的讨论需要分析学, 作者就不在这里讨论这个问题了. 作者的目的是: 告诉读者 ``定义很重要!''.
\end{example}

跟整数的因子分解类似, 作者给出
\begin{definition}
    设 (有理系数) 整式 $f \neq 0$. 那么, $f$ 一定可写为 (至多一个; 因为有限多个单位的积还是单位) (有理系数整式的) 单位与有限多个 (可以是零个) (有理系数整式的) 不可约的整式的积, 即: 存在单位 $\varepsilon$ 与不可约的整式 $p_1$, $p_2$, $\cdots$, $p_s$ ($s$ 可为 $0$; 此时, $f$ 是单位) 使
    \begin{align*}
        f = \varepsilon p_1 p_2 \cdots p_s.
    \end{align*}
    上式右侧即为 $f$ 的因子分解 \term{factorization of $f$}. 动词短语 ``写 $f$ 为单位与有限多个不可约的整式的积'' 的一个简单的称呼是 ``因子分解 $f$'' \term{to factorize $f$}.
\end{definition}

\begin{remark}
    依此定义, $3x + 2$, $4x - 8$, $-x - 1$ 的因子分解可以是自身. 当然, $2(2x - 4)$ 也是 $4x - 8$ 的一个因子分解; 甚至, $\frac{1}{5} (15x + 10)$ 也是 $3x + 2$ 的一个因子分解. 这些差异, 在 ``单位'' ``相伴'' 下, 不是差异!

    $x^2 - 2$ (作为有理系数整式) 是不可约的 (Eisenstein 判别法). 而且, 我们在本文里, 不考虑系数不全是有理数的整式. 所以, $(x - \sqrt{2}) \cdot (x + \sqrt{2})$ 自然是 ``跑远了''.
\end{remark}

\begin{remark}
    有时, 为书写方便, 允许在因子分解里出现幂. 比如说, 设
    \begin{align*}
        f = (x-1) \cdot (x-1) \cdot (x-1) \cdot (5x-20) \cdot (5x-20).
    \end{align*}
    上式右侧已经是 $f$ 的因子分解了. 不过, 为了方便, 我们可以认为,
    \begin{align*}
        (x-1)^3 (5x-20)^2
    \end{align*}
    也是 $f$ 的因子分解. 此时, 我们视 $(x-1)^3$ (或 $(5x-20)^2$) 为 $3$ 个 $x-1$ 的积 (或 $2$ 个 $5x-20$ 的积), 而不是一个不可约的整式!
\end{remark}

事实上, 我们已经在 ``\RationalRootsOfPolynomialsOverQ'' ``\FactorsOfHigherDegreeOfPolynomialsOverQ'' 里讨论过因子分解的方法了——我们讨论了如何寻找有理系数整式的不可约的因子. 仔细的读者可能注意到, 我们一直在说 ``试写某整式为 (若干个) 不可约的整式的积'', 而不是直接说 ``因子分解某整式''——这是因为我们还未明确 ``因子分解'' 的确切含义. 毕竟, 在中学, ``因式分解'' 只要求将整式写为几个整式的积, 并没有明确地要求这几个整式适合哪些条件 (顶多说 ``不可再分解'').

作者将在本文介绍更多的因子分解整式的方法.

事实上, 有二个基本的方法, 我们一直在用——不过, 我们没有正式地提到它们.

一个方法是提取公因子. 设 $f$, $g_1$, $g_2$, $\cdots$, $g_n$. 因为整式适合乘法分配律, 故
\begin{align*}
    f \cdot (g_1 + g_2 + \cdots + g_n) = fg_1 + fg_2 + \cdots + fg_n.
\end{align*}
将此等式的左、右二侧互换, 有
\begin{align*}
    fg_1 + fg_2 + \cdots + fg_n = f(g_1 + g_2 + \cdots + g_n).
\end{align*}

\begin{example}
    设 $f(x) = x^4 - 2x^2$. 不难看出
    \begin{align*}
        f(x)
        = {} & x^2 \cdot x^2 - x^2 \cdot 2 \\
        = {} & x^2 (x^2 - 2).
    \end{align*}
    我们说, 视 $x^2$ 为 $2$ 个 $x$ 的积 $x \cdot x$ 的简写. 根据 Eisenstein 判别法, $x^2 - 2$ 是不可约的. 所以 $x^2 (x^2 - 2)$ 是 $f(x)$ 的因子分解. 当然, 若读者喜欢使 $0$ 次系数非负, $-x^2 (2 - x^2)$ 也是可以的.
\end{example}

另一个方法是套用公式. 我们在中学, 学过 ``平方差公式'' ``完全平方公式'':
\begin{align*}
     & f^2 - g^2 = (f + g) (f - g),      \\
     & f^2 \pm 2 fg + g^2 = (f \pm g)^2,
\end{align*}
其中 $f$, $g$ 是任意的整式. 在 ``\GeneralizedBinomialCoefficients'', 我们接触了二项展开:
\begin{align*}
    (f + g)^n = f^n + \binom{n}{1} f^{n-1} g + \cdots + \binom{n}{i} f^{n-i} g^i + \cdots + g^n.
\end{align*}
``完全平方公式'' 就是取二项展开的 $n$ 为 $2$ 所得的式. 在 ``\SyntheticDivision'', 我们还接触了
\begin{align*}
    f^n - g^n = (f - g)(f^{n-1} + f^{n-2} g + \cdots + f^{n-i} g^{i-1} + \cdots + g^{n-1}).
\end{align*}
``平方差公式'' 就是取上式的 $n$ 为 $2$ 所得的式.

\begin{example}
    设 $f(x) = 4x^2 - 4x + 1$. 则
    \begin{align*}
        f(x)
        = {} & (2x)^2 + 2 (2x) (-1) + (-1)^2 \\
        = {} & (2x + (-1))^2                 \\
        = {} & (2x - 1)^2.
    \end{align*}
    我们当然也可以用找有理根的方法因子分解 $f(x)$.
\end{example}

\begin{example}
    设 $f(x) = 18 - 2x^2$. 则
    \begin{align*}
        f(x)
        = {} & 2 \cdot 9 + 2 \cdot (-x^2) \\
        = {} & 2(9 + (-x^2))              \\
        = {} & 2(3^2 - x^2)               \\
        = {} & 2(3 + x)(3 - x).
    \end{align*}
    我们当然也可以用找有理根的方法因子分解 $f(x)$. 喜欢使非单位的因子的首项系数为正数的读者也可写
    \begin{align*}
        f(x) = (-2) (x + 3) (x - 3).
    \end{align*}

    注意到, 我们不但套用了公式, 还提取了公因子.
\end{example}

\begin{example}
    设 $f(x) = x^4 + x^2 + 1$. 则
    \begin{align*}
        f(x)
        = {} & x^4 + 2x^2 + 1 - x^2               \\
        = {} & (x^2)^2 + 2x^2 \cdot 1 + 1^2 - x^2 \\
        = {} & (x^2 + 1)^2 - x^2                  \\
        = {} & (x^2 + 1 + x) (x^2 + 1 - x).
    \end{align*}
    不难判断, $x^2 \pm x + 1$ 是不可约的.

    注意到, 我们同时套用了二个乘法公式.
\end{example}

\begin{example}
    设 $f(x) = x^4 - 16$. 则
    \begin{align*}
        f(x)
        = {} & (x^2)^2 - (4)^2            \\
        = {} & (x^2 + 4) (x^2 - 4)        \\
        = {} & (x^2 + 4) (x^2 - 2^2)      \\
        = {} & (x^2 + 4) (x + 2) (x - 2).
    \end{align*}
    不难判断, $x^2 + 4$ 是不可约的. 当然, 我们也可以这样:
    \begin{align*}
        f(x)
        = {} & x^4 - 2^4                                       \\
        = {} & (x - 2) (x^3 + x^2 \cdot 2 + x \cdot 2^2 + 2^3) \\
        = {} & (x - 2) (x^3 + 2x^2 + 4x + 8)                   \\
        = {} & (x - 2) (x^2 (x + 2) + 4 (x + 2))               \\
        = {} & (x - 2) (x^2 + 4) (x + 2).
    \end{align*}
\end{example}

\begin{example}
    设 $f(x) = ax^2 + bx + c$, 且 $a \neq 0$. 我们可以配方:
    \begin{align*}
        f(x)
        = {} & a \left( x^2 + \frac{b}{a} x \right) + c                                          \\
        = {} & a \left( x^2 + 2x \frac{b}{2a} \right) + c                                        \\
        = {} & a \left( x^2 + 2x \frac{b}{2a} + \frac{b^2}{4a^2} \right) - \frac{ab^2}{4a^2} + c \\
        = {} & a \left( x + \frac{b}{2a} \right)^2 - a \frac{b^2 - 4ac}{4a^2}                    \\
        = {} & a \left( \left( x + \frac{b}{2a} \right)^2 - \frac{b^2 - 4ac}{4a^2} \right).
    \end{align*}
    若 $b^2 - 4ac$ 是有理数 $r$ 的平方, 则
    \begin{align*}
        f(x)
        = {} & a \left( \left( x + \frac{b}{2a} \right)^2 - \frac{r^2}{4a^2} \right)              \\
        = {} & a \left( \left( x + \frac{b}{2a} \right)^2 - \left( \frac{r}{2a} \right)^2 \right) \\
        = {} & a \left( x + \frac{b + r}{2a} \right) \left( x + \frac{b - r}{2a} \right)          \\
        = {} & a \left( x - \frac{-b - r}{2a} \right) \left( x - \frac{-b + r}{2a} \right).
    \end{align*}
    由此可见, $f(x)$ 的二个根是
    \begin{align*}
        \frac{-b \pm r}{2a} = \frac{-b \pm \sqrt{b^2 - 4ac}}{2a}.
    \end{align*}

    这是 $2$ 次式的一种通用的因子分解法. 当然, 读者可能也还记得, 在 ``\FactorsOfHigherDegreeOfPolynomialsOverQ'' 里, 我们有
    \begin{align*}
        ax^2 + bx + c = \frac{1}{4a} ((2ax + b)^2 + (4ac - b^2)).
    \end{align*}
    感兴趣的读者可自行比较此式与
    \begin{align*}
        ax^2 + bx + c = a \left( \left( x + \frac{b}{2a} \right)^2 - \frac{b^2 - 4ac}{4a^2} \right).
    \end{align*}
\end{example}

\begin{example}
    设 $f(x) = 4x^2 - 4x - 3$. 则
    \begin{align*}
        f(x)
        = {} & (2x)^2 + 2 (2x) (-1) - 3                   \\
        = {} & (2x)^2 + 2 (2x) (-1) + (-1)^2 - (-1)^2 - 3 \\
        = {} & (2x)^2 + 2 (2x) (-1) + (-1)^2 - 2^2        \\
        = {} & (2x + (-1))^2 - 2^2                        \\
        = {} & (2x + (-1) + 2) (2x + (-1) - 2)            \\
        = {} & (2x + 1) (2x - 3).
    \end{align*}

    ``具体的式, 具体地分析.'' 显然, 我们没有必要再将 $4x^2$ 的 $4$ 写出来了——毕竟, $4 \left(x^2 - x - \frac{3}{4} \right)$ 涉及稍繁的不是整数的有理数的运算.
\end{example}

还有一个公式值得一提:
\begin{align*}
    f^3 + g^3 + h^3 - 3fgh = (f + g + h)(f^2 + g^2 + h^2 - fg - fh - gh),
\end{align*}
其中, $f$, $g$, $h$ 都是整式. 感兴趣的读者可自行再\myFN{见 ``\SyntheticDivision''.}推出此式.

读者可以看到, 提取公因子法与套用公式法可简化因子分解. 读者应灵活地运用算学知识因子分解整式. 我们在因子分解 $x^{15} - 1$ 时就用到了乘法公式. 当然, $x^{n+1} + x^n - 2$ 也是.

接下来, 作者将为读者展现一个可选的方法. 为什么说 ``可选'' 呢? 因为运算不少. 请读者休息片刻.

\myLine

设 $f(x) \neq 0$. 读者可能还能记起, 若 $M(x)$ 是 $f(x)$ 与 $Df(x)$ 的最大公因子, 则 $f(x)$ 有重因子的一个必要与充分条件是 $M(x)$ 不是单位. 进一步, 若 $h(x)$ 适合 $f(x) = h(x)M(x)$, 则 $h(x)$ 与 $f(x)$ 有相同的不可约的因子, 但 $h(x)$ 无重因子. 这有什么好处呢? 当时作者在 ``\MultipleFactors'' 里没说此性质的好处. 现在, 请读者看下面的例.

\begin{example}
    设 $f(x) = x^6-2 x^5-8 x^4+14 x^3+11 x^2-28 x+12$. 我们先试着找 $1$ 次因子 (也就是有理根). 首先, $x$ 不是 $f(x)$ 的因子. 因为 $f(x)$ 的首项系数是单位, 故 $f(x)$ 的有理根一定是整数 $v$, 且 $v$ 还是 $0$ 次系数 $12$ 的因子. 所以, $v$ 可能是
    \begin{align*}
        \pm 1, \pm 2, \pm 3, \pm 4, \pm 6, \pm 12.
    \end{align*}
    还是老样子, 我们打算用 ``$v \pm u$'' 检验排除一些不可能是 $f(x)$ 的根的数 (这里 $u = 1$). 我们希望 $f(1) \neq 0$. 利用综合除法, 我们有
    \begin{align*}
        f(x) = (x - 1) \underbrace{(x^5-x^4-9 x^3+5 x^2+16 x-12)}_{g_1 (x)}.
    \end{align*}
    $1$ 居然是 $f(x)$ 的根! 不过, 这也不是坏事. $g_1 (x)$ 的首项系数仍为单位, 且其 $0$ 次系数是 $f(x)$ 的 $0$ 次系数的相伴, 故 $g_1 (x)$ 的有理根仍可能是上面的 $12$ 个整数的一个. 还是用 $1$ 试; 我们希望 $g_1 (1) \neq 0$. 不过
    \begin{align*}
        g_1 (x) = (x - 1) \underbrace{(x^4-9 x^2-4 x+12)}_{g_2 (x)}.
    \end{align*}
    $1$ 居然也是 $g_1 (x)$ 的根! 也就是说, $(x-1)^2$ 是 $f(x)$ 的因子! 事实上, 读者可继续用 $x-1$ 除 $g_2 (x)$:
    \begin{align*}
        g_2 (x) = (x - 1) \underbrace{(x^3+x^2-8 x-12)}_{g_3 (x)}.
    \end{align*}
    再用 $x-1$ 除 $g_3 (x)$:
    \begin{align*}
        g_3 (x) = (x - 1) (x^2 + 2x - 6) - 18.
    \end{align*}
    由此可见, $f(x) = (x-1)^3 g_3 (x)$, 且 $x-1$ 不再是 $g_3 (x)$ 的因子. 还有一点请读者注意: 我们还没试 $-1$ 呢. 当然, 读者可用综合除法算出
    \begin{align*}
        g_3 (x) = (x + 1)(x^2 - 8) - 4.
    \end{align*}
    也就是说, $x+1$ 不是 $g_3 (x)$ 的因子. 现在, 总算可以用 ``$v \pm u$'' 检验了. $-4 = (-1) \cdot 2^2$, 故 $-4$ 的因子数为 $2 \cdot (1+2) = 6$; $-18 = (-2) \cdot 3^2$, 故 $-18$ 的因子数为 $2 \cdot (1+1) \cdot (1+2) = 12$. 我们从 $-4$ 入手. 因为 $v + u = v + 1$ 一定是 $g_3 (-1) = -4$ 的因子, 故恰有 $3$ 个数通过 ``$v + u$'' 检验: $-2$, $3$, $-3$. 再看 $-18$. 因为 $v - u = v - 1$ 一定是 $g_3 (1) = -18$ 的因子, 故恰有 $2$ 个数又通过 ``$v - u$'' 检验: $-2$, $3$. 继续进行综合除法, 发现 $-2$ 是 $g_3 (x)$ 的根:
    \begin{align*}
        g_3 (x) = (x + 2) \underbrace{(x^2 - x - 6)}_{g_4 (x)}.
    \end{align*}
    我们可直接对 $g_4 (x)$ 作如下变形:
    \begin{align*}
        g_4 (x)
        = {} & \frac{1}{4} (4x^2 - 4x - 24)                   \\
        = {} & \frac{1}{4} ((2x)^2 - 2 (2x) \cdot 1 - 24)     \\
        = {} & \frac{1}{4} ((2x)^2 - 2 (2x) \cdot 1 + 1 - 25) \\
        = {} & \frac{1}{4} ((2x - 1)^2 - 5^2)                 \\
        = {} & \frac{1}{4} (2x - 1 - 5) (2x - 1 + 5)          \\
        = {} & (x - 3) (x + 2).
    \end{align*}
    换句话说,
    \begin{align*}
         & g_4 (x) = (x + 2) (x - 3),          \\
         & g_3 (x) = (x + 2)^2 (x - 3),        \\
         & f(x) = (x - 1)^3 (x + 2)^2 (x - 3).
    \end{align*}
    我们发现, $x-1$ 与 $x+2$ 都是重因子. $x-1$ 一直影响我们的发挥——我们连除三次才消去了 $x-1$. $x+2$ 呢? 好在 $f(x)$ 的次不是很高, 我们用 $2$ 次式的技巧, 避开了除法. 但读者也不难看出: 若我们 ``机械地'' 判断 $g_4 (x)$ 的有理根, 那么 $-2$ 与 $3$ 仍为 ``候选根''. 我们用综合除法可判断一个数是不是整式的根, 但却不知道它是否会重复出现.
\end{example}

\begin{example}
    还是用 $f(x) = x^6-2 x^5-8 x^4+14 x^3+11 x^2-28 x+12$. 不过, 这次, 我们先看 $f(x)$ 有无重因子. 读者应该还记得求流数的法则吧? (如果不记得了, 就再往前看看.) 不难写出
    \begin{align*}
        Df(x) = 6 x^5-10 x^4-32 x^3+42 x^2+22 x-28.
    \end{align*}
    用 $Df(x)$ 除 $f(x)$:
    \begin{align*}
        f(x) = \frac{3x-1}{18} Df(x) - \frac{1}{9} \underbrace{(29 x^4-47 x^3-87 x^2+199 x-94)}_{r_0 (x)}.
    \end{align*}
    用 $r_0 (x)$ 除 $Df(x)$:
    \begin{align*}
        Df(x) = \frac{2}{841} (87x-4) r_0 (x) - \frac{12\,150}{841} \underbrace{(x^3-3 x+2)}_{r_1 (x)}.
    \end{align*}
    用 $r_1 (x)$ 除 $r_0 (x)$:
    \begin{align*}
        r_0 (x) = (29x-47) r_1 (x).
    \end{align*}
    由此可见, $r_1 (x)$ 就是 $f(x)$ 与 $Df(x)$ 的最大公因子. 因为 $r_1 (x)$ 不是单位, 故 $f(x)$ 有重因子. 用 $r_1 (x)$ 除 $f(x)$:
    \begin{align*}
        f(x) = r_1 (x) \underbrace{(x^3-2 x^2-5 x+6)}_{h(x)}.
    \end{align*}
    接下来该怎么办呢? 我们可以先找 $h(x)$ 的有理根, 从而得到 $h(x)$ 的 $1$ 次因子. $h(x)$ 的 $1$ 次因子是 $f(x)$ 的 $1$ 次因子. 不仅如此, $r_1 (x)$ 的不可约的因子也一定是 $h(x)$ 的因子\myFN{因为 $r_1 (x)$ 的不可约的因子是 $f(x)$ 的重因子, 而 $f(x)$ 的重因子, 至少, 按定义, 是不可约的. $f(x)$ 的不可约的因子都是 $h(x)$ 的因子.}. 所以, $h(x)$ 的因子分解可帮助我们找出 $r_1 (x)$ 的因子分解. $f(x) = r_1 (x) h(x)$, 故 $f(x)$ 的因子分解就是 $h(x)$ 与 $r_1 (x)$ 的因子分解的合并. 因为 $h(x)$ 无重因子, 故找出 $h(x)$ 的一个 (不是单位的) 因子 $p(x)$, 得 $h(x) = p(x) \ell(x)$ 后, 就不必判断 $p(x)$ 是否为 $\ell (x)$ 的因子了.

    我们就拿上面的 $h(x)$ 举例. 老样子, $h(x)$ 的有理根 $v$ 一定是 ($6$ 的因子):
    \begin{align*}
        \pm 1, \pm 2, \pm 3, \pm 6.
    \end{align*}
    先试 $\pm 1$. 利用综合除法, 可得
    \begin{align*}
        h(x) = (x - 1) \underbrace{(-6 - x + x^2)}_{h_1 (x)}.
    \end{align*}
    老样子, $h_1 (x)$ 的有理根 $v$ 一定是 ($-6$ 的因子):
    \begin{align*}
        \pm 1, \pm 2, \pm 3, \pm 6.
    \end{align*}
    不过, $x-1$ 不再是 $h_1 (x)$ 的因子了. 利用综合除法, 我们有
    \begin{align*}
        h_1 (x) = (x + 1) (x - 2) - 4.
    \end{align*}
    所以, $x+1$ 也不是 $h_1 (x)$ 的因子. 当然, 就算 $x-1$ 不是 $h_1 (x)$ 的因子, 我们还是要知道 $h_1 (1)$:
    \begin{align*}
        h_1 (x) = (x - 1) x - 6.
    \end{align*}
    由此可见, $v - u = v - 1$ 一定是 $h_1 (1) = -6$ 的因子, 且 $v + u = v + 1$ 一定是 $h_1 (-1) = -4$ 的因子. 由此, 我们可排除 $2$, $-3$, $6$, $-6$, 从而还剩 $-2$, $3$. 接下来不难检验, $-2$ 与 $3$ 都是 $h_1 (x)$ 的根:
    \begin{align*}
        h_1 (x) = (x + 2) (x - 3).
    \end{align*}
    由此可知
    \begin{align*}
        h(x) = (x - 1) (x + 2) (x - 3).
    \end{align*}
    还剩 $r_1 (x)$. 我们可以用 $h(x)$ 帮助我们. $r_1 (x) = x^3 - 3x + 2$, 故 $3$ 不可能是 $r_1 (x)$ 的根. 也就是说, $x - 3$ 不是 $r_1 (x)$ 的因子 (故 $x - 3$ 不是 $f(x)$ 的重因子!). 因为 $-1$, $2$ 不是 $h(x)$ 的根, 故 $x+1$, $x-2$ 不可能是 $r_1 (x)$ 的因子. 还有 $x - 1$ 与 $x + 2$. 根据综合除法, 我们得到
    \begin{align*}
        r_1 (x) = (x - 1) (x^2 + x - 2) = (x - 1) (x - 1) (x + 2).
    \end{align*}
    所以
    \begin{align*}
        f(x) = h(x) r_1 (x) = (x-1)^3 (x+2)^2 (x-3).
    \end{align*}
\end{example}

\begin{example}
    再看 $f(x) = x^{15} - 1$. 还是老样子, 先看看 $f(x)$ 是否有重因子. 事实上, 对任意正整数 $n$, $x^n - 1$ 都不会有重因子:
    \begin{align*}
        (-1) (x^n - 1) + \frac{x}{n} (nx^{n-1}) = 1.
    \end{align*}
    也就是说, $x^n - 1$ 与其流数 $nx^{n-1}$ 是互素的.

    此事实有什么用呢? 读者可能还记得: 当时, 在 ``\FactorsOfHigherDegreeOfPolynomialsOverQ'' 里, 我们因子分解 $f(x)$ 时, 用了二遍乘法公式, 得
    \begin{align*}
        f(x)
        = {} & (x-1) (x^2+x+1) (x^4+x^3+x^2+x+1)                           \\
             & \qquad \cdot \underbrace{(x^8-x^7+x^5-x^4+x^3-x+1)}_{r(x)}.
    \end{align*}
    在 $r(x)$ 前的三个整式都是不可约的; 我们还得继续因子分解 $r(x)$, 是不? 我们先从 $1$ 次因子开始寻找, 是吧? 我们知道, 若 $r(x)$ 有有理根, 则 $r(x)$ 的有理根只能是 $\pm 1$. 我们当初作了二遍综合除法, 对不? 现在, 我们换一个视角判断. 因为 $f(x)$ 无重因子, 故 $1$ 不可能是 $r(x)$ 的有理根; 至于 $-1$, 可以用 $f(x)$ 判断: $f(-1) = -2$, 故 $-1$ 也不是 $r(x)$ 的有理根. 至于 $r(1)$ 与 $r(-1)$, 直接将 $r(x)$ 的 $x$ 换为 $\pm 1$ 即可——我们都知道, $1$ 的整数次幂是 $1$, $-1$ 的奇数次幂是 $-1$, 且 $-1$ 的偶数次幂是 $1$——也就是说, 不需要综合除法, 也能算出 $r(1)$ 与 $r(-1)$.
\end{example}

综上, 欲因子分解 $f(x)$, 可以这样:

(i) 求 $Df(x)$.

(ii) 找 $f(x)$ 与 $Df(x)$ 的一个最大公因子 $M(x)$.

(iii) 设 $f(x) = h(x) M(x)$, 则 $f(x)$ 的不可约的因子都是 $h(x)$ 的因子, 且 $h(x)$ 不再有重因子. 因为 $M(x)$ 包含 $f(x)$ 的重因子, 故 $M(x)$ 的不可约的因子也是 $h(x)$ 的因子.

(iv) 因子分解 $h(x)$. 好消息: 若不是单位的 $p(x)$ 使 $h(x) = p(x) \ell(x)$, 则 $p(x)$ 一定不是 $\ell (x)$ 的因子.

(v) 借助 $h(x)$ 的因子分解, 并结合其他事实, 写出 $M(x)$ 的因子分解. 当然, 我们也可以利用 $M(x)$ 与 $DM(x)$ 的最大公因子来判断 $M(x)$ 有无重因子.

(vi) 合并 $h(x)$ 与 $M(x)$ 的因子分解, 即得 $f(x)$ 的因子分解.

对于特殊的 $f(x)$, 其流数可能很简单 (如 $x^n - 1$), 故我们很快就能判断出 $f(x)$ 是否有重因子; 对于稍一般的 $f(x)$, 计算 $f(x)$ 与 $Df(x)$ 的一个最大公因子将是一件麻烦的事——所以, 这是可选的. 当然, 对于计算机而言, 找最大公因子不是难题——不过, 这就不是我们所讨论的话题了.

感谢读者能读到这里!
