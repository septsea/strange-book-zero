\section*{\Preface}
% Use the following line to add the unnumbered section to the table of contents
\addcontentsline{toc}{section}{\Preface}

本文是瞎写的. 我给本文的另一个名字是 ``Re: ゼロ から 始める ポリノミアル の イントロダクション''. 不过想了想, 算了算了. 龙鸣日语, 不好意思直接说出来.

本文用尽可能朴素的语言讨论了多项式及其部分应用.

总是可以去这儿得到本文的最新版本:
\begin{align*}
     & \texttt{https://gitee.com/septsea/strange-book-zero}  \\
     & \texttt{https://github.com/septsea/strange-book-zero}
\end{align*}

您可以自由地阅读、修改、再分发本文.

如果您发现本文有什么地方不对, 那么您就毫不犹豫地告诉我. 当然, 任何意见与建议也是可以的.

(记得先看看最新版本改过来没有哟. 不过就算没看最新版本也没关系啦. 我一定会处理您的消息的! 嘿嘿.)

就先说到这里.

\ \

\providecommand{\appendDate}{}
\renewcommand{\appendDate}[1]{\par \hfill {\itshape \sffamily #1}}

\begin{remark}
    总算写完 ``\Prerequisites '' 了. 我写这玩意儿花了好久好久啊. 先发布再说吧.
    \appendDate{June 3, 2021}
\end{remark}

\begin{remark}
    忘记介绍域是什么东西了. 我真是笨蛋啊.
    \appendDate{June 3, 2021}
\end{remark}

\begin{remark}
    前几日意识到, 我不能又写得严谨, 又指望着中学生都能读懂. 不过本文业已成形, ``改'' 不如 ``重写''. 不过本文是开源的 (主要是无版权), 您可以随意重写.
    \appendDate{June 17, 2021}
\end{remark}

\begin{remark}
    6 月 6 日, 我在\hyperref{https://chaoli.club/index.php/6396}{}{}{这里}发了贴, 目的是让更多的人看到我写的 \ding{200} 文. 我得到了很多意见与建议. 今日, 我写完了我想写的东西. 我维护本文就好. 我觉得超理太棒了!
    \appendDate{June 20, 2021}
\end{remark}

\begin{remark}
    我总算在改错了. 感谢超理读者 ``没啥好叫的'' 指出本文的一个错误! 然后我自己发现了一堆印刷错误. 啊啦啊啦. 看多了视觉小说, 我的大脑生锈了呢. 顺便一提, 看本文看累了的时候, 不妨看看小说哦! \hyperref{https://gitee.com/septsea/ss}{}{}{这里! 这里! I am sharing my copies of visual novels with my readers!}
    \appendDate{July 29, 2021}
\end{remark}
