\subsection*{Derivatives Revisited}
\addcontentsline{toc}{subsection}{Derivatives Revisited}
\markright{Derivatives Revisited}

本节将再讨论多项式的导数\period

在讨论导数前, 让我们捡起在 Generalized Binomial Coefficients 节里没用过的二项展开:

\begin{proposition}
    设 $r$, $s \in \FF[x]$\period 设 $n$ 是非负整数\period 则
    \begin{align*}
        (r + s)^{n} = \sum_{k = 0}^{n} \binom{n}{k} r^{n - k} s^{k} \period
    \end{align*}
    此式称为二项展开\period
\end{proposition}

\begin{remark}
    等式右侧的 $\binom{n}{k}$ 称为二项系数 \term{binomial coefficient}\period 事实上, $\binom{n}{k}$ 一开始就是为讨论 $(r+s)^n$ 的展开而生的\period
\end{remark}

\begin{example}
    在中学, 我们学过完全平方和公式:
    \begin{align*}
        (r + s)^2 = r^2 + 2rs + s^2 \period
    \end{align*}
    在二项展开里, 取 $n = 2$, 就可以得到这个公式:
    \begin{align*}
         & \binom{2}{0} = 1, \quad \binom{2}{1} = 2, \quad \binom{2}{2} = 1, \\
         & \begin{aligned}
            (r + s)^2
            = {} & 1r^2 s^0 + 2r^1 s^1 + 1r^0 s^2 \\
            = {} & r^2 + 2rs + s^2 \period
        \end{aligned}
    \end{align*}

    在上节, 我们用分配律拆开了 $(1+x)^3$:
    \begin{align*}
        (1+x)^3
        = {} & (1+x)^2 (1+x)         \\
        = {} & (1+2x+x^2) (1+x)      \\
        = {} & 1+2x+x^2 + x+2x^2+x^3 \\
        = {} & 1+3x+3x^2+x^3 \period
    \end{align*}
    在二项展开里, 取 $n = 3$:
    \begin{align*}
         & \binom{3}{0} = 1 = \binom{3}{3}, \quad \binom{3}{1} = 3 = \binom{3}{2}, \\
         & \begin{aligned}
            (r+s)^3
            = {} & 1r^3 s^0 + 3r^2 s^1 + 3r^1 s^2 + 1r^0 s^3 \\
            = {} & r^3 + 3r^2 s + 3r s^2 + s^3 \period
        \end{aligned}
    \end{align*}
    用 $1$, $x$ 替换 $r$, $s$, 有
    \begin{align*}
        (1+x)^3
        = {} & 1^3 + 3 \cdot 1^2 x + 3 \cdot 1 x^2 + x^3 \\
        = {} & 1 + 3x + 3x^2 + x^3 \period
    \end{align*}
\end{example}

设 $c \in \FF$\period 在 Polynomial Equality 节, 我们用 $1$, $x-c$, $(x-c)^2$, $\cdots$, $(x-c)^n$ 引出线性无关, 并证明了

\begin{proposition}
    设 $a_0$, $b_0$, $a_1$, $b_1$, $\cdots$, $a_n$, $b_n \in \FF$\period 设 $c \in \FF$\period 再设
    \begin{align*}
        f(x) = \sum_{i = 0}^n a_i (x-c)^i, \quad g(x) = \sum_{i = 0}^n b_i (x-c)^i \period
    \end{align*}
    则 $f(x)=g(x)$ 的一个必要与充分条件是
    \begin{align*}
        a_0 = b_0, \quad a_1 = b_1, \quad \cdots, \quad a_n = b_n \period
    \end{align*}
    并且, 任取
    \begin{align*}
        f(x) = \sum_{i = 0}^n u_i x^i \in \FF[x],
    \end{align*}
    必存在 $v_0$, $v_1$, $\cdots$, $v_n \in \FF$ 使
    \begin{align*}
        f(x) = \sum_{i = 0}^n v_i (x-c)^i \period
    \end{align*}
\end{proposition}

利用二项展开, 有
\begin{align*}
    x^i
    = {} & (c + (x - c))^i                                           \\
    = {} & \sum_{j = 0}^{i} \binom{n}{k} c^{n - k} (x-c)^{k} \period
\end{align*}
由此, 我们可以把任意多项式
\begin{align*}
    f(x) = \sum_{i = 0}^n u_i x^i \in \FF[x]
\end{align*}
写为
\begin{align*}
    f(x) = \sum_{i = 0}^n v_i (x-c)^i \in \FF[x] \period
\end{align*}

\begin{example}
    设
    \begin{align*}
        f(x) = x^3 - 6x^2 + 15x - 12 \period
    \end{align*}
    取 $c = 2$\period 利用二项展开, 有
    \begin{align*}
         & \begin{aligned}
            x^3
            = {} & (2 + (x - 2))^3                                                             \\
            = {} & 1 \cdot 2^3 + 3 \cdot 2^2 (x-2) + 3 \cdot 2^1 (x-2)^2 + 1 \cdot 2^0 (x-2)^3 \\
            = {} & 8 + 12(x-2) + 6(x-2)^2 + (x-2)^3,
        \end{aligned} \\
         & \begin{aligned}
            x^2
            = {} & (2 + (x - 2))^2                                       \\
            = {} & 1 \cdot 2^2 + 2 \cdot 2^1 (x-2) + 1 \cdot 2^0 (x-2)^2 \\
            = {} & 4 + 4(x-2) + (x-2)^2,
        \end{aligned} \\
         & x = 2 + (x-2) \period
    \end{align*}
    所以
    \begin{align*}
        f(x)
        = {} & x^3 - 6x^2 + 15x - 12              \\
        = {} & 8 + 12(x-2) + 6(x-2)^2 + (x-2)^3   \\
             & \qquad - 6(4 + 4(x-2) + (x-2)^2)   \\
             & \qquad \qquad + 15(2 + (x-2)) - 12 \\
        = {} & (x-2)^3 + 3(x-2) + 2 \period
    \end{align*}
\end{example}
