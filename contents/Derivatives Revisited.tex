\subsection*{Derivatives Revisited}
\addcontentsline{toc}{subsection}{Derivatives Revisited}
\markright{Derivatives Revisited}

本节将再讨论多项式的导数\period

在讨论导数前, 让我们捡起在 Generalized Binomial Coefficients 节里没用过的二项展开:

\begin{proposition}
    设 $r$, $s \in \FF[x]$\period 设 $n$ 是非负整数\period 则
    \begin{align*}
        (r + s)^{n} = \sum_{k = 0}^{n} \binom{n}{k} r^{n - k} s^{k} \period
    \end{align*}
    此式称为二项展开\period
\end{proposition}

\begin{remark}
    等式右侧的 $\binom{n}{k}$ 称为二项系数 \term{binomial coefficient}\period 事实上, $\binom{n}{k}$ 一开始就是为讨论 $(r+s)^n$ 的展开而生的\period
\end{remark}

\begin{example}
    在中学, 我们学过完全平方和公式:
    \begin{align*}
        (r + s)^2 = r^2 + 2rs + s^2 \period
    \end{align*}
    在二项展开里, 取 $n = 2$, 就可以得到这个公式:
    \begin{align*}
         & \binom{2}{0} = 1, \quad \binom{2}{1} = 2, \quad \binom{2}{2} = 1, \\
         & \begin{aligned}
            (r + s)^2
            = {} & 1r^2 s^0 + 2r^1 s^1 + 1r^0 s^2 \\
            = {} & r^2 + 2rs + s^2 \period
        \end{aligned}
    \end{align*}

    在上节, 我们用分配律拆开了 $(1+x)^3$:
    \begin{align*}
        (1+x)^3
        = {} & (1+x)^2 (1+x)         \\
        = {} & (1+2x+x^2) (1+x)      \\
        = {} & 1+2x+x^2 + x+2x^2+x^3 \\
        = {} & 1+3x+3x^2+x^3 \period
    \end{align*}
    在二项展开里, 取 $n = 3$:
    \begin{align*}
         & \binom{3}{0} = 1 = \binom{3}{3}, \quad \binom{3}{1} = 3 = \binom{3}{2}, \\
         & \begin{aligned}
            (r+s)^3
            = {} & 1r^3 s^0 + 3r^2 s^1 + 3r^1 s^2 + 1r^0 s^3 \\
            = {} & r^3 + 3r^2 s + 3r s^2 + s^3 \period
        \end{aligned}
    \end{align*}
    用 $1$, $x$ 替换 $r$, $s$, 有
    \begin{align*}
        (1+x)^3
        = {} & 1^3 + 3 \cdot 1^2 x + 3 \cdot 1 x^2 + x^3 \\
        = {} & 1 + 3x + 3x^2 + x^3 \period
    \end{align*}
\end{example}

设 $c \in \FF$\period 在 Polynomial Equality 节, 我们用 $1$, $x-c$, $(x-c)^2$, $\cdots$, $(x-c)^n$ 引出线性无关, 并证明了

\begin{proposition}
    设 $a_0$, $b_0$, $a_1$, $b_1$, $\cdots$, $a_n$, $b_n \in \FF$\period 设 $c \in \FF$\period 再设
    \begin{align*}
        f(x) = \sum_{i = 0}^n a_i (x-c)^i, \quad g(x) = \sum_{i = 0}^n b_i (x-c)^i \period
    \end{align*}
    则 $f(x)=g(x)$ 的一个必要与充分条件是
    \begin{align*}
        a_0 = b_0, \quad a_1 = b_1, \quad \cdots, \quad a_n = b_n \period
    \end{align*}
    并且, 任取
    \begin{align*}
        f(x) = \sum_{i = 0}^n u_i x^i \in \FF[x],
    \end{align*}
    必存在 $v_0$, $v_1$, $\cdots$, $v_n \in \FF$ 使
    \begin{align*}
        f(x) = \sum_{i = 0}^n v_i (x-c)^i \period
    \end{align*}
\end{proposition}

利用二项展开, 有
\begin{align*}
    x^i
    = {} & (c + (x - c))^i                                           \\
    = {} & \sum_{j = 0}^{i} \binom{i}{j} c^{i - j} (x-c)^{j} \period
\end{align*}
由此, 我们可以把任意多项式
\begin{align*}
    f(x) = \sum_{i = 0}^n u_i x^i \in \FF[x]
\end{align*}
写为
\begin{align*}
    f(x) = \sum_{i = 0}^n v_i (x-c)^i \in \FF[x] \period
\end{align*}

\begin{example}
    设
    \begin{align*}
        f(x) = x^3 - 6x^2 + 15x - 12 \period
    \end{align*}
    取 $c = 2$\period 利用二项展开, 有
    \begin{align*}
         & \begin{aligned}
            x^3
            = {} & (2 + (x - 2))^3                                                             \\
            = {} & 1 \cdot 2^3 + 3 \cdot 2^2 (x-2) + 3 \cdot 2^1 (x-2)^2 + 1 \cdot 2^0 (x-2)^3 \\
            = {} & 8 + 12(x-2) + 6(x-2)^2 + (x-2)^3,
        \end{aligned} \\
         & \begin{aligned}
            x^2
            = {} & (2 + (x - 2))^2                                       \\
            = {} & 1 \cdot 2^2 + 2 \cdot 2^1 (x-2) + 1 \cdot 2^0 (x-2)^2 \\
            = {} & 4 + 4(x-2) + (x-2)^2,
        \end{aligned} \\
         & x = 2 + (x-2) \period
    \end{align*}
    所以
    \begin{align*}
        f(x)
        = {} & x^3 - 6x^2 + 15x - 12              \\
        = {} & 8 + 12(x-2) + 6(x-2)^2 + (x-2)^3   \\
             & \qquad - 6(4 + 4(x-2) + (x-2)^2)   \\
             & \qquad \qquad + 15(2 + (x-2)) - 12 \\
        = {} & (x-2)^3 + 3(x-2) + 2 \period
    \end{align*}
\end{example}

现在, 读者可能不再那么不熟悉二项展开了\period 我们正式重述导数\period 不过, 我们并不会完全照搬 Derivatives 节\period

\begin{definition}
    设
    \begin{align*}
        f(x) = a_0 + a_1 x + a_2 x^2 + \cdots + a_{n-1} x^{n-1} + a_n x^n \in \FF[x] \period
    \end{align*}
    $f(x)$ 的导数是多项式
    \begin{align*}
        Df(x) = 0 + 1a_1 + 2a_2 x + \cdots + (n-1)a_{n-1} x^{n-2} + na_n x^{n-1} \in \FF[x] \period
    \end{align*}
    设 $t \in \FF$\period 我们把
    \begin{align*}
        0 + 1a_1 + 2a_2 t + \cdots + (n-1)a_{n-1} t^{n-2} + na_n t^{n-1} \in \FF
    \end{align*}
    简单地写为 $Df(t)$\period
\end{definition}

\begin{remark}
    若 $f(x) = c$, $c \in \FF$, 则 $Df(x)$ 为零多项式\period
\end{remark}

\begin{remark}
    读者可能会注意到我们在这里换了个记号\period 之前, 我们用 $f^{\prime} (x)$ 或 $(f(x))^{\prime}$ 表示多项式 $f(x)$ 的导数——那个时候, 我们还是在抽象的整环 $D$ 上讨论问题\period 现在, 我们在熟悉的 $\FF$ 里讨论问题\period 读者已经很久都没见到 $D$ 了吧? 从此节开始, 我们用 $D$ 记号表示导数\period 所以, $D$ 将不表示整环\period
\end{remark}

\begin{example}
    取 $f(x) = x^6 - x^3 + 1 \in \FF[x]$\period 则
    \begin{align*}
        Df(x) = 6x^5 - 3x^2 \period
    \end{align*}
\end{example}

下面的命题也是老朋友了\period

\begin{proposition}
    设 $f(x)$, $g(x) \in \FF[x]$, $c \in \FF$\period 则

    (i) $D(cf(x)) = c Df(x)$;

    (ii) $D(f(x) \pm g(x)) = Df(x) \pm Dg(x)$\period

    由 (i) (ii) 与数学归纳法可知: 当 $c_0$, $c_1$, $\cdots$, $c_{k-1} \in \FF$, 且 $f_0 (x)$, $f_1 (x)$, $\cdots$, $f_{k-1} (x) \in \FF[x]$ 时,
    \begin{align*}
        D \left( \sum_{\ell = 0}^{k-1} c_\ell f_\ell (x) \right) = \sum_{\ell = 0}^{k-1} c_\ell Df_\ell (x) \period
    \end{align*}
\end{proposition}

\begin{pf}
    本来我们不必重复证明这些命题\period 不过, 为了让读者更好地熟悉 $D$ 记号, 我们还是在此处证明 (i) (ii), 并将剩下的推论留给读者作练习\period 设
    \begin{align*}
         & f(x) = a_0 + a_1 x + a_2 x^2 + \cdots + a_{n-1} x^{n-1} + a_n x^n, \\
         & g(x) = b_0 + b_1 x + b_2 x^2 + \cdots + b_{n-1} x^{n-1} + b_n x^n
    \end{align*}
    是 $\FF[x]$ 中二个元\period

    (i) $cf(x)$ 就是多项式
    \begin{align*}
        ca_0 + ca_1 x + ca_2 x^2 + \cdots + ca_{n-1} x^{n-1} + ca_n x^n,
    \end{align*}
    故
    \begin{align*}
        D(cf(x))
        = {} & D(ca_0 + ca_1 x + ca_2 x^2 + \cdots + ca_{n-1} x^{n-1} + ca_n x^n) \\
        = {} & ca_1 + 2ca_2 x + \cdots + (n-1)ca_{n-1} x^{n-2} + nca_n x^{n-1}    \\
        = {} & ca_1 + c2a_2 x + \cdots + c(n-1)a_{n-1} x^{n-2} + cna_n x^{n-1}    \\
        = {} & c(a_1 + 2a_2 x + \cdots + (n-1)a_{n-1} x^{n-2} + na_n x^{n-1})     \\
        = {} & cDf(x) \period
    \end{align*}

    (ii) $f(x) \pm g(x)$ 就是多项式
    \begin{align*}
         & (a_0 \pm b_0) + (a_1 \pm b_1) x + (a_2 \pm b_2) x^2 + \cdots       \\
         & \qquad \qquad + (a_{n-1} \pm b_{n-1}) x^{n-1} + (a_n \pm b_n) x^n,
    \end{align*}
    故
    \begin{align*}
             & D(f(x) \pm g(x))                                                                \\
        = {} & D((a_0 \pm b_0) + (a_1 \pm b_1) x + (a_2 \pm b_2) x^2 + \cdots                  \\
             & \qquad \qquad + (a_{n-1} \pm b_{n-1}) x^{n-1} + (a_n \pm b_n) x^n)              \\
        = {} & (a_1 \pm b_1) + 2(a_2 \pm b_2) x + \cdots + (n-1) (a_{n-1} \pm b_{n-1}) x^{n-2} \\
             & \qquad \qquad + n (a_n \pm b_n) x^{n-1}                                         \\
        = {} & (a_1 \pm b_1) + (2a_2 x \pm 2b_2 x) + \cdots + ((n-1)a_{n-1} x^{n-2}            \\
             & \qquad \qquad \pm (n-1)b_{n-1} x^{n-2})
        + (na_n x^{n-1} \pm nb_n x^{n-1})                                                      \\
        = {} & (a_1 + 2a_2 x + \cdots + (n-1)a_{n-1} x^{n-2} + na_n x^{n-1})                   \\
             & \qquad \qquad \pm (b_1 + 2b_2 x + \cdots + (n-1)b_{n-1} x^{n-2} + nb_n x^{n-1}) \\
        = {} & Df(x) \pm Dg(x) \period \qedhere
    \end{align*}
\end{pf}

\begin{example}
    取
    \begin{align*}
        f(x) = x^3 + 2, \quad g(x) = x^2 + x - 1 \period
    \end{align*}
    不难得到
    \begin{align*}
        Df (x) = 3x^2, \quad Dg (x) = 2x + 1 \period
    \end{align*}

    (i) $4g(x)$ 也是多项式, 当然可以有导数\period 因为
    \begin{align*}
        4g(x) = 4x^2 + 4x - 4,
    \end{align*}
    故
    \begin{align*}
        D(4g(x)) = 8x + 4,
    \end{align*}
    这刚好是 $4Dg(x)$:
    \begin{align*}
        4Dg(x) = 4(2x + 1) = 8x + 4 \period
    \end{align*}

    (ii) $f(x) + g(x)$ 也是多项式\period 因为
    \begin{align*}
        f(x) + g(x) = x^3 + 2 + x^2 + x - 1 = x^3 + x^2 + x + 1,
    \end{align*}
    故
    \begin{align*}
        D(f(x) + g(x)) = 3x^2 + 2x + 1,
    \end{align*}
    而这刚好是 $Df(x) + Dg(x)$:
    \begin{align*}
        Df(x) + Dg(x) = 3x^2 + 2x + 1 \period
    \end{align*}
\end{example}

前面讲差商与差分时, 我们引入了高级差商与高级差分\period 类似地, 我们引入高级导数\period

\begin{definition}
    设 $f(x) \in \FF[x]$\period 记
    \begin{align*}
        D^0 f(x) = f(x) \in \FF[x],
    \end{align*}
    并称其为 $f(x)$ 的 $0$ 级导数 \term{zeroth-order derivative}\period $1$ 级导数就是导数:
    \begin{align*}
        D^1 f(x) = D f(x) = D (D^0 f(x)) \in \FF[x] \period
    \end{align*}
    $1$ 级导数的导数是 $2$ 级导数:
    \begin{align*}
        D^2 f(x) = D (D^1 f(x)) \in \FF[x] \period
    \end{align*}
    $2$ 级导数的导数是 $3$ 级导数:
    \begin{align*}
        D^3 f(x) = D (D^2 f(x)) \in \FF[x] \period
    \end{align*}
    一般地, $e$ 级导数就是 $e - 1$ 级导数的导数:
    \begin{align*}
        D^e f(x) = D (D^{e-1} f(x)) \in \FF[x] \period
    \end{align*}
    高级导数可指代任意 $e$ 级导数, 此处 $e > 1$\period

    设 $t \in \FF$\period 既然 $D^e f(x)$ 是某个多项式
    \begin{align*}
        v_0 + v_1 x + \cdots + v_s x^s \in \FF[x],
    \end{align*}
    我们将
    \begin{align*}
        v_0 + v_1 t + \cdots + v_s t^s \in \FF
    \end{align*}
    简单地写为 $D^e f(t)$\period
\end{definition}

\begin{example}
    设
    \begin{align*}
        f(x) = 2x^3 + 3x^2 + 5x + 7 \period
    \end{align*}
    根据定义, $f(x)$ 的 $0$ 级导数就是自己:
    \begin{align*}
        D^0 f(x) = 2x^3 + 3x^2 + 5x + 7 \period
    \end{align*}
    $f(x)$ 的 $1$ 级导数是
    \begin{align*}
        D^1 f(x) = Df(x) = 6x^2 + 6x + 5 \period
    \end{align*}
    $f(x)$ 的 $2$ 级导数是
    \begin{align*}
        D^2 f(x) = D(D^1 f(x)) = 12x + 6 \period
    \end{align*}
    $f(x)$ 的 $3$ 级导数是
    \begin{align*}
        D^3 f(x) = D(D^2 f(x)) = 12 \period
    \end{align*}
    $f(x)$ 的 $4$ 级导数是
    \begin{align*}
        D^4 f(x) = D(D^3 f(x)) = 0 \period
    \end{align*}
    读者不难验证: 对任意超出 $3$ 的整数 $e$, 必有
    \begin{align*}
        D^e f(x) = 0 \period
    \end{align*}
\end{example}

类似地, 高级导数适合如下性质:

\begin{proposition}
    设 $e$ 是非负整数\period 当 $c_0$, $c_1$, $\cdots$, $c_{k-1} \in \FF$, 且 $f_0 (x)$, $f_1 (x)$, $\cdots$, $f_{k-1} (x) \in \FF[x]$ 时,
    \begin{align*}
        D^e \left( \sum_{\ell = 0}^{k-1} c_\ell f_\ell (x) \right)
        = \sum_{\ell = 0}^{k-1} c_\ell D^e f_\ell (x) \period
    \end{align*}
\end{proposition}

\begin{pf}
    用数学归纳法\period 我们把具体过程留给读者当练习\period
\end{pf}

% cSpell: disable-next-line
当初我们为得到 Vandermonde 恒等式与二项展开, 我们引入了临时工具 $r^{[k]}$\period 现在, 类似地, 为了更方便地讨论多项式的高级导数, 我们引入

\begin{definition}
    设 $m$ 为整数\period 设 $r \in \FF[x]$\period 定义
    \begin{align*}
        q_m (r) = \begin{cases}
            \frac{1}{m!} r^m, & \quad m > 0;        \\
            1,                & \quad m = 0;        \\
            0,                & \quad m < 0 \period
        \end{cases}
    \end{align*}
\end{definition}

设 $k$ 是整数\period 我们知道
\begin{align*}
    \Delta \binom{x}{k} = \binom{x}{k - 1} \period
\end{align*}
类似地, 我们有

\begin{proposition}
    设 $m$ 为整数\period 则
    \begin{align*}
        D q_m (x) = q_{m-1} (x) \period
    \end{align*}
\end{proposition}

\begin{pf}
    $m > 0$ 时,
    \begin{align*}
        D q_m (x) = m \cdot \frac{x^{m-1}}{m!} = \frac{x^{m-1}}{(m-1)!} = q_{m-1} (x) \period
    \end{align*}
    $m \leq 0$ 时, $q_m (x) = a$, 这里 $a \in \FF$\period 故
    \begin{align*}
         & D q_m (x) = 0 = q_{m-1} (x) \period \qedhere
    \end{align*}
\end{pf}

由此可得

\begin{proposition}
    设 $e$ 是非负整数\period 设 $m$ 为整数\period 则
    \begin{align*}
        D^e q_m (x) = q_{m-e} (x) \period
    \end{align*}
\end{proposition}

\begin{pf}
    用数学归纳法\period 我们把具体过程留给读者当练习\period
\end{pf}

现在, 我们看高级导数与二项展开的关系\period

固定某 $c \in \FF$\period 固定某非负整数 $n$\period 任取不高于 $n$ 的非负整数 $i$\period 则
\begin{align*}
    q_{i} (x)
    = {} & \frac{1}{i!} \sum_{j = 0}^{i} \binom{i}{j} c^{i - j} (x-c)^{j}           \\
    = {} & \frac{1}{i!} \sum_{j = 0}^{i} \frac{i!}{(i - j)! j!} c^{i - j} (x-c)^{j} \\
    = {} & \frac{1}{i!} \sum_{j = 0}^{i} i! q_{i-j} (c) q_{j} (x-c)                 \\
    = {} & \sum_{j = 0}^{i} q_{i-j} (c) q_{j} (x-c)                                 \\
    = {} & \sum_{j = 0}^{n} q_{i-j} (c) q_{j} (x-c) \period
\end{align*}
任取次不高于 $n$ 的多项式
\begin{align*}
    f(x) = a_0 + a_1 x + \cdots + a_n x^n \in \FF[x] \period
\end{align*}
设
\begin{align*}
    b_\ell = \ell! a_\ell \quad (\ell = 0,1,\cdots,n) \period
\end{align*}
则
\begin{align*}
    f(x) = b_0 q_0 (x) + b_1 q_1 (x) + \cdots + b_n q_n (x) \period
\end{align*}
不难看出, 当 $j$ 是非负整数时,
\begin{align*}
    D^j f(x) = b_0 q_{0-j} (x) + b_1 q_{1-j} (x) + \cdots + b_n q_{n-j} (x) \period
\end{align*}
所以
\begin{align*}
    f(x)
    = {} & \sum_{i = 0}^{n} b_i q_i (x)                                     \\
    = {} & \sum_{i = 0}^{n} b_i \sum_{j = 0}^{n} q_{i-j} (c) q_{j} (x-c)    \\
    = {} & \sum_{i = 0}^{n} \sum_{j = 0}^{n} b_i q_{i-j} (c) q_{j} (x-c)    \\
    = {} & \sum_{j = 0}^{n} \sum_{i = 0}^{n} b_i q_{i-j} (c) q_{j} (x-c)    \\
    = {} & \sum_{j = 0}^{n} \left( \sum_{i = 0}^{n} b_i q_{i-j} (c) \right)
    q_{j} (x-c)                                                             \\
    = {} & \sum_{j = 0}^{n} D^j f(c) q_{j} (x-c)                            \\
    = {} & \sum_{j = 0}^{n} \frac{D^j f(c)}{j!} (x-c)^j \period
\end{align*}

我们已经证明了

\begin{proposition}
    设 $n$ 是非负整数\period 设 $f(x)$ 是次不高于 $n$ 的多项式\period 设 $c \in \FF$\period 则 Taylor 公式 \term{Taylor's formula} 成立:
    \begin{align*}
        f(x) = \sum_{j = 0}^{n} \frac{D^j f(c)}{j!} (x-c)^j \period
    \end{align*}
\end{proposition}

\begin{remark}
    我们可以说, Taylor 公式是二项展开的推广\period 也可以说, 二项展开是 Taylor 公式的特例\period
\end{remark}

\begin{remark}
    取 $c = 0$, 有
    \begin{align*}
        f(x) = \sum_{j = 0}^{n} \frac{D^j f(0)}{j!} x^j \period
    \end{align*}
    读者可能会注意到, 上式的形式与
    \begin{align*}
        f(x) = \sum_{k = 0}^{n} \Delta^k f(0) \binom{x}{k}
    \end{align*}
    的形式十分相似\period
\end{remark}

\begin{remark}
    以后我们不用 $q_m (r)$ 记号了\period
\end{remark}

\begin{remark}
    我们提一个读者可能已经注意到的事实\period 设 $n$ 是非负整数\period 则 $n$ 次多项式的 $n$ 级导数不是 $0$, 但 $n+1$ 级导数是 $0$\period 这也解释了为什么在 Taylor 公式里, 我们只要求 $n$ 不低于 $f(x)$ 的次\period
\end{remark}

\begin{example}
    取 $n = 3$\period 设
    \begin{align*}
        f(x) = x^3 - 6x^2 + 15x - 12 \period
    \end{align*}
    则 $f(x)$ 的次不高于 $n$, 且
    \begin{align*}
         & D^0 f(x) = f(x) = x^3 - 6x^2 + 15x - 12, \\
         & D^1 f(x) = D f(x) = 3x^2 - 12x + 15,     \\
         & D^2 f(x) = D (D f(x)) = 6x - 12,         \\
         & D^3 f(x) = D (D^2 f(x)) = 6 \period
    \end{align*}
    取 $c = 2$\period 则
    \begin{align*}
         & D^0 f(2) = 2^3 - 6 \cdot 2^2 + 15 \cdot 2 - 12 = 2, \\
         & D^1 f(2) = 3 \cdot 2^2 - 12 \cdot 2 + 15 = 3,       \\
         & D^2 f(2) = 6 \cdot 2 - 12 = 0,                      \\
         & D^3 f(2) = 6 \period
    \end{align*}
    根据 Taylor 公式,
    \begin{align*}
        f(x)
        = {} & 2 + \frac{3}{1!} (x-2) + \frac{0}{2!} (x-2)^2 + \frac{6}{3!} (x-2)^3 \\
        = {} & 2 + 3(x-2) + (x-2)^3 \period
    \end{align*}
\end{example}

Taylor 公式一个用途是证明

\begin{proposition}
    设 $f(x)$, $g(x) \in \FF[x]$\period 则
    \begin{align*}
        D(f(x) g(x)) = Df(x) \cdot g(x) + f(x) \cdot Dg(x) \period \tag*{(\myStar)}
    \end{align*}
\end{proposition}

\begin{pf}
    设 $h(x) = f(x)g(x)$\period 取整数 $n$ 使 $\deg f(x) \leq n$, $\deg g(x) \leq n$, $\deg h(x) \leq n$, 且 $1 \leq n$\period 任取 $c \in \FF$\period 则
    \begin{align*}
         & f(x) = \sum_{i = 0}^{n} \frac{D^i f(c)}{i!} (x-c)^i,        \\
         & g(x) = \sum_{j = 0}^{n} \frac{D^j g(c)}{j!} (x-c)^j,        \\
         & h(x) = \sum_{k = 0}^{n} \frac{D^k h(c)}{k!} (x-c)^k \period
    \end{align*}
    不过, 既然 $h(x)$ 是 $f(x)$ 与 $g(x)$ 的积, 也应有
    \begin{align*}
        h(x) = \sum_{k = 0}^{n+n} s_k (x-c)^k = \sum_{k = 0}^{n} s_k (x-c)^k,
    \end{align*}
    其中
    \begin{align*}
        s_k
        = {} & \sum_{i = 0}^{k} \frac{D^i f(c)}{i!} \cdot \frac{D^{k-i} g(c)}{(k-i)!}   \\
        = {} & \frac{1}{k!} \sum_{i = 0}^{k} \binom{k}{i} D^i f(c) D^{k-i} g(c) \period
    \end{align*}
    所以, 任取不超过 $n$ 的非负整数 $k$, 必有
    \begin{align*}
                 & s_k = \frac{D^k h(c)}{k!}                                              \\
        \implies & D^k h(c) = \sum_{i = 0}^{k} \binom{k}{i} D^i f(c) D^{k-i} g(c) \period
    \end{align*}
    作多项式
    \begin{align*}
        E(x) = D^k h(x) - \sum_{i = 0}^{k} \binom{k}{i} D^i f(x) D^{k-i} g(x) \period
    \end{align*}
    上面的推理告诉我们, 任取 $c \in \FF$, 必有 $E(c) = 0$\period 所以 $E(x)$ 一定是零多项式, 即
    \begin{align*}
        D^k h(x) = \sum_{i = 0}^{k} \binom{k}{i} D^i f(x) D^{k-i} g(x) \period
    \end{align*}
    取 $k=1$, 有
    \begin{align*}
             & D (f(x)g(x))                                          \\
        = {} & D^1 h(x)                                              \\
        = {} & \sum_{i = 0}^{1} \binom{1}{i} D^i f(x) D^{1-i} g(x)   \\
        = {} & 1 \cdot D^0 f(x) D^1 g(x) + 1 \cdot D^1 f(x) D^0 g(x) \\
        = {} & Df(x) \cdot g(x) + f(x) \cdot Dg(x) \period \qedhere
    \end{align*}
\end{pf}

\begin{remark}
    事实上, 我们得到了高级导数的 Leibniz 公式 \term{Leibniz's formula}: 若 $k$ 是非负整数, 且 $f(x)$, $g(x) \in \FF[x]$, 则
    \begin{align*}
        D^k (f(x)g(x)) = \sum_{i = 0}^{k} \binom{k}{i} D^i f(x) D^{k-i} g(x) \period
    \end{align*}
    不过, 在本文里, 我们用不到这个公式\period
\end{remark}

\begin{example}
    取
    \begin{align*}
        f(x) = x^3 + 2, \quad g(x) = x^2 + x - 1 \period
    \end{align*}
    不难得到
    \begin{align*}
        Df (x) = 3x^2, \quad Dg (x) = 2x + 1 \period
    \end{align*}

    $f(x)$ 与 $g(x)$ 的积
    \begin{align*}
        f(x) g(x) = x^5 + x^4 - x^3 + 2x^2 + 2x - 2
    \end{align*}
    的导数是
    \begin{align*}
        D(f(x) g(x)) = 5x^4 + 4x^3 - 3x^2 + 4x + 2 \period
    \end{align*}
    如果用上面的 (\myStar) 计算, 就是
    \begin{align*}
             & Df (x) g(x) + f(x) Dg (x)                \\
        = {} & 3x^2 (x^2 + x - 1) + (x^3 + 2) (2x + 1)  \\
        = {} & 3x^4 + 3x^3 - 3x^2 + 2x^4 + x^3 + 4x + 2 \\
        = {} & 5x^4 + 4x^3 - 3x^2 + 4x + 2 \period
    \end{align*}
\end{example}

下面的二个命题是正确的:

\begin{proposition}
    当 $f_0 (x)$, $f_1 (x)$, $\cdots$, $f_{k-1} (x) \in \FF[x]$ 时,
    \begin{align*}
             & D(f_0 (x) f_1 (x) \cdots f_{k-1} (x))                                              \\
        = {} & Df_0 (x) f_1 (x) \cdots f_{k-1} (x) + f_0 (x) Df_1 (x) \cdots f_{k-1} (x) + \cdots \\
             & \qquad \qquad + f_0 (x) f_1 (x) \cdots Df_{k-1} (x) \period
    \end{align*}
    取 $f_0 (x) = f_1 (x) = \cdots = f_{k-1} (x) = f(x)$ 知
    \begin{align*}
        D((f(x))^k) = k(f(x))^{k-1} Df(x) \period
    \end{align*}
\end{proposition}

\begin{pf}
    用数学归纳法\period 我们把具体过程留给读者当练习\period
\end{pf}

\begin{example}
    设 $f(x) = (x^2 + x - 1)^{666}$\period 求 $Df(x)$\period

    取 $g(x) = x^2 + x - 1$\period 显然, $f(x) = (g(x))^{666}$\period 所以
    \begin{align*}
        Df(x)
        = {} & D((g(x))^{666})                          \\
        = {} & 666 (g(x))^{666 - 1} Dg(x)               \\
        = {} & 666 (x^2 + x - 1)^{665} (2x + 1)         \\
        = {} & 666 (2x + 1) (x^2 + x - 1)^{665} \period
    \end{align*}
\end{example}

\begin{proposition}
    设 $f(x)$, $g(x) \in \FF[x]$\period 则 $f(x)$ 与 $g(x)$ 的复合的导数适合链规则:
    \begin{align*}
        D(g \circ f) (x) = (Dg \circ f)(x) Df (x) \period
    \end{align*}
\end{proposition}

\begin{pf}
    可看 Derivatives 节的相应内容\period
\end{pf}

\begin{example}
    设 $f(x) = (x^2 + x - 1)^5 + 3 (x^2 + x - 1)^4 - 1$\period 求 $Df(x)$\period

    取 $g(x) = x^5 + 3x^4 - 1$ 与 $h(x) = x^2 + x - 1$\period 则
    \begin{align*}
         & Dg(x) = 5x^4 + 12x^3 = x^3 (5x + 12), \\
         & Dh(x) = 2x + 1 \period
    \end{align*}
    显然,
    \begin{align*}
        f(x) = g(h(x)) = (g \circ h) (x) \period
    \end{align*}
    所以
    \begin{align*}
        Df(x)
        = {} & (Dg \circ h)(x) Dh(x)                            \\
        = {} & (x^2 + x - 1)^3 (5(x^2 + x - 1) + 12) (2x + 1)   \\
        = {} & (2x + 1) (5x^2 + 5x + 7) (x^2 + x - 1)^3 \period
    \end{align*}
\end{example}
