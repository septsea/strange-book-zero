\subsection*{\Derivatives}
\addcontentsline{toc}{subsection}{\Derivatives}
\markright{\Derivatives}

本节讨论多项式的微商.

在本节, 我们会将一些容易证明的命题留给读者练习. 读者可乘此机会让自己熟悉证明命题的过程与数学归纳法.

\begin{definition}
    设
    \begin{align*}
        f(x) = a_0 + a_1 x + a_2 x^2 + \cdots + a_{n-1} x^{n-1} + a_n x^n \in D[x].
    \end{align*}
    $f(x)$ 的微商 \term{derivative} 是多项式
    \begin{align*}
        f^{\prime}(x) = 0 + 1a_1 + 2a_2 x + \cdots + (n-1)a_{n-1} x^{n-2} + na_n x^{n-1} \in D[x].
    \end{align*}
    $f^{\prime} (x)$ 也可写为 $(f(x))^{\prime}$.
\end{definition}

\begin{remark}
    整环 $D$ 里不一定有名为 ${} \pm 2$, ${}\pm 3$, $\cdots$ 的元. 回忆一下, 若 $a \in D$, $n \in \NN$, 则
    \begin{align*}
        na = n \cdot a = \underbrace{a + a + \cdots + a}_{\text{$n$ $a$'s}}.
    \end{align*}
    若 $-n \in \NN$, 则
    \begin{align*}
        na = -((-n)a).
    \end{align*}
    当然, 在 $\ZZ$ (或 $\FF$) 里, $na$ 可以认为是 $\ZZ$ (或 $\FF$) 的二个元 $n$ 与 $a$ 的积.
\end{remark}

\begin{example}
    取 $f(x) = x^6 - x^3 + 1 \in D[x]$. 若 $D = \FF$, 则
    \begin{align*}
        f^{\prime}(x) = 6x^5 - 3x^2 + 0 = 6x^5 - 3x^2.
    \end{align*}
    若 $D$ 是 $4$ 元集 $V$, 则
    \begin{align*}
        f^{\prime}(x) = (6 \cdot 1)x^5 + (3 \cdot (-1)) x^2 + 0 = x^2.
    \end{align*}
    这里, $V = \{\, 0,1,\tau,\tau^2 \,\}$. 它的加法与乘法如下:
    \begin{align*}
        \arraycolsep=0.25cm
        \begin{array}{c|cccc}
            +      & 0      & 1      & \tau   & \tau^2 \\ \hline
            0      & 0      & 1      & \tau   & \tau^2 \\
            1      & 1      & 0      & \tau^2 & \tau   \\
            \tau   & \tau   & \tau^2 & 0      & 1      \\
            \tau^2 & \tau^2 & \tau   & 1      & 0      \\
        \end{array}
        \qquad \qquad
        \begin{array}{c|cccc}
            \cdot  & 0 & 1      & \tau   & \tau^2 \\ \hline
            0      & 0 & 0      & 0      & 0      \\
            1      & 0 & 1      & \tau   & \tau^2 \\
            \tau   & 0 & \tau   & \tau^2 & 1      \\
            \tau^2 & 0 & \tau^2 & 1      & \tau   \\
        \end{array}
    \end{align*}
    在前面 (``\Prerequisites '' 节的 ``\Domains '' 小节), 我们知道, $V$ 是整环. 任取 $a \in V$, 都有
    \begin{align*}
        2 \cdot a = a + a = 0.
    \end{align*}
    所以 $a = -a$. 这样,
    \begin{align*}
         & 6 \cdot 1 = 2 \cdot (3 \cdot 1) = (3 \cdot 1) + (3 \cdot 1) = 0, \\
         & 3 \cdot (-1) = (-1) + (-1) + (-1) = 1 + 1 + 1 = 0 + 1 = 1.
    \end{align*}
    所以, 当我们把 $f(x)$ 视为 $V[x]$ 中元时, 它的微商 ``有点奇怪''. 同样的道理, 在 $V$ 与 $V[x]$ 中,
    \begin{align*}
        (x^{2k})^{\prime} = (2k \cdot 1) x^{2k-1} = 0 x^{2k-1} = 0.
    \end{align*}
\end{example}

\begin{remark}
    微商就是 $D[x]$ 到 $D[x]$ 的函数 (也就是 $D[x]$ 的变换):
    \begin{align*}
         & D[x] \to D[x], \tag*{$\prime \colon$}                                        \\
         & a_0 + a_1 x + \cdots + a_n x^n \mapsto a_1 + 2a_2 x + \cdots + na_n x^{n-1}.
    \end{align*}
\end{remark}

\begin{definition}
    设
    \begin{align*}
         & f(x) = a_0 + a_1 x + \cdots + a_m x^m, \\
         & g(x) = b_0 + b_1 x + \cdots + b_n x^n
    \end{align*}
    为 $D[x]$ 中的二个元. 我们称
    \begin{align*}
        (g \circ f)(x) = g(f(x)) = b_0 + b_1 f(x) + \cdots + b_n (f(x))^n
    \end{align*}
    为 $f(x)$ 与 $g(x)$ 的复合 \term{composition}.
\end{definition}

\begin{remark}
    可以看到, $f(x)$ 与 $g(x)$ 的复合仍为多项式. 设
    \begin{align*}
        h(x) = d_0 + d_1 x + \cdots + d_s x^s \in D[x].
    \end{align*}
    记
    \begin{align*}
        \ell(x)
        = {} & (h \circ g) (x)                                         \\
        = {} & d_0 + d_1 (b_0 + b_1 x + \cdots + b_n x^n) + \cdots     \\
             & \qquad \qquad + d_s (b_0 + b_1 x + \cdots + b_n x^n)^s,
    \end{align*}
    则
    \begin{align*}
        ((h \circ g) \circ f) (x)
        = {} & (\ell \circ f) (x)                                             \\
        = {} & d_0 + d_1 (b_0 + b_1 f(x) + \cdots + b_n (f(x))^n) + \cdots    \\
             & \qquad \qquad + d_s (b_0 + b_1 f(x) + \cdots + b_n (f(x))^n)^s \\
        = {} & d_0 + d_1 (g \circ f) (x) + \cdots + d_s ((g \circ f) (x))^s   \\
        = {} & (h \circ (g \circ f)) (x).
    \end{align*}
    换句话说, 多项式的复合适合结合律.
\end{remark}

\begin{example}
    取
    \begin{align*}
        g(x) = b_0 + b_1 x + \cdots + b_n x^n, \quad f(x) = x - c \in D[x].
    \end{align*}
    那么
    \begin{align*}
         & (g \circ f) (x) = g(f(x)) = b_0 + b_1 (x-c) + \cdots + b_n (x-c)^n, \\
         & (f \circ g) (x) = f(g(x)) = -c + b_0 + b_1 x + \cdots + b_n x^n.
    \end{align*}
    这表明: 多项式的复合一般不交换.
\end{example}

下面的命题相当显然了.

\begin{proposition}
    设 $f(x)$, $g(x)$, $h(x) \in D[x]$.

    (i) 设 $p(x) = f(x) + g(x)$. 则
    \begin{align*}
        p(h(x)) = f(h(x)) + g(h(x)).
    \end{align*}

    (ii) 设 $q(x) = f(x) g(x)$. 则
    \begin{align*}
        q(h(x)) = f(h(x)) g(h(x)).
    \end{align*}
\end{proposition}

\begin{pf}
    设
    \begin{align*}
         & f(x) = a_0 + a_1 x + \cdots + a_n x^n, \\
         & g(x) = b_0 + b_1 x + \cdots + b_n x^n
    \end{align*}
    是 $D[x]$ 中二个元. 这样,
    \begin{align*}
         & f(h(x)) = a_0 + a_1 h(x) + \cdots + a_n (h(x))^n, \\
         & g(h(x)) = b_0 + b_1 h(x) + \cdots + b_n (h(x))^n.
    \end{align*}

    (i) 根据加法的定义, 有
    \begin{align*}
        p(x) = f(x) + g(x) = c_0 + c_1 x + \cdots + c_n x^n,
    \end{align*}
    其中
    \begin{align*}
        c_i = a_i + b_i, \quad i = 0,1,\cdots,n.
    \end{align*}
    所以
    \begin{align*}
        p(h(x)) = c_0 + c_1 h(x) + \cdots + c_n (h(x))^n.
    \end{align*}
    根据多项式的运算律, 有
    \begin{align*}
             & f(h(x)) + g(h(x))                                                                   \\
        = {} & (a_0 + a_1 h(x) + \cdots + a_n (h(x))^n) + (b_0 + b_1 h(x) + \cdots + b_n (h(x))^n) \\
        = {} & (a_0 + b_0) + (a_1 + b_1) h(x) + \cdots + (a_n + b_n) (h(x))^n                      \\
        = {} & c_0 + c_1 h(x) + \cdots + c_n (h(x))^n                                              \\
        = {} & p(h(x)).
    \end{align*}

    (ii) 根据乘法的定义, 有
    \begin{align*}
        q(x) = f(x) g(x) = d_0 + d_1 x + \cdots + d_{2n} x^{2n},
    \end{align*}
    其中
    \begin{align*}
        d_i = a_0 b_i + a_1 b_{i-1} + \cdots + a_i b_0, \quad i = 0,1,\cdots,2n.
    \end{align*}
    所以
    \begin{align*}
        q(h(x)) = d_0 + d_1 h(x) + \cdots + d_{2n} (h(x))^{2n}.
    \end{align*}
    根据多项式的运算律, 有
    \begin{align*}
             & f(h(x)) g(h(x))                                                                               \\
        = {} & (a_0 + a_1 h(x) + \cdots + a_n (h(x))^n) (b_0 + b_1 h(x) + \cdots + b_n (h(x))^n)             \\
        = {} & (a_0 b_0) + (a_0 b_1 + a_1 b_0) h(x) + \cdots + (a_n b_n) (h(x))^{2n}                         \\
        = {} & (a_0 b_0) + (a_0 b_1 + a_1 b_0) h(x) + \cdots + (a_0 b_{2n} + a_1 b_{2n-1} + \cdots + a_n b_n \\
             & \qquad + a_{n+1} b_{n-1} + \cdots + a_{2n} b_0) (h(x))^{2n}                                   \\
        = {} & c_0 + c_1 h(x) + \cdots + c_{2n} (h(x))^{2n}                                                  \\
        = {} & q(h(x)). \qedhere
    \end{align*}

\end{pf}

\begin{example}
    考虑 $\ZZ$ 与 $\ZZ[x]$. 取
    \begin{align*}
        f(x) = x^3 + 2, \quad g(x) = x^2 + x - 1.
    \end{align*}
    不难得到
    \begin{align*}
        f^{\prime} (x) = 3x^2, \quad g^{\prime} (x) = 2x + 1.
    \end{align*}

    (i) $4g(x)$ 也是多项式, 当然可以有微商. 因为
    \begin{align*}
        4g(x) = 4x^2 + 4x - 4,
    \end{align*}
    故
    \begin{align*}
        (4g(x))^{\prime} = 8x + 4,
    \end{align*}
    这刚好是 $4g^{\prime} (x)$:
    \begin{align*}
        4g^{\prime} (x) = 4(2x + 1) = 8x + 4.
    \end{align*}

    (ii) $f(x) + g(x)$ 也是多项式. 因为
    \begin{align*}
        f(x) + g(x) = x^3 + 2 + x^2 + x - 1 = x^3 + x^2 + x + 1,
    \end{align*}
    故
    \begin{align*}
        (f(x) + g(x))^{\prime} = 3x^2 + 2x + 1,
    \end{align*}
    而这刚好是 $f^{\prime} (x) + g^{\prime} (x)$:
    \begin{align*}
        f^{\prime} (x) + g^{\prime} (x) = 3x^2 + 2x + 1.
    \end{align*}
\end{example}

一般地, 我们有

\begin{proposition}
    设 $f(x)$, $g(x) \in D[x]$, $c \in D$. 则

    (i) $(cf(x))^{\prime} = c f^{\prime} (x)$;

    (ii) $(f(x) \pm g(x))^{\prime} = f^{\prime} (x) \pm g^{\prime} (x)$.

    由 (i) (ii) 与数学归纳法可知: 当 $c_0$, $c_1$, $\cdots$, $c_{k-1} \in D$, 且 $f_0 (x)$, $f_1 (x)$, $\cdots$, $f_{k-1} (x) \in D[x]$ 时,
    \begin{align*}
             & (c_0 f_0(x) + c_1 f_1(x) + \cdots + c_{k-1} f_{k-1}(x))^{\prime}                  \\
        = {} & c_0 f^{\prime}_0(x) + c_1 f^{\prime}_1(x) + \cdots + c_{k-1} f^{\prime}_{k-1}(x).
    \end{align*}
\end{proposition}

\begin{pf}
    我们证明 (i) (ii), 将剩下的推论留给读者作练习. 设
    \begin{align*}
         & f(x) = a_0 + a_1 x + a_2 x^2 + \cdots + a_{n-1} x^{n-1} + a_n x^n, \\
         & g(x) = b_0 + b_1 x + b_2 x^2 + \cdots + b_{n-1} x^{n-1} + b_n x^n
    \end{align*}
    是 $D[x]$ 中二个元.

    (i) $cf(x)$ 就是多项式
    \begin{align*}
        ca_0 + ca_1 x + ca_2 x^2 + \cdots + ca_{n-1} x^{n-1} + ca_n x^n,
    \end{align*}
    故
    \begin{align*}
        (cf(x))^{\prime}
        = {} & (ca_0 + ca_1 x + ca_2 x^2 + \cdots + ca_{n-1} x^{n-1} + ca_n x^n)^{\prime} \\
        = {} & ca_1 + 2ca_2 x + \cdots + (n-1)ca_{n-1} x^{n-2} + nca_n x^{n-1}            \\
        = {} & ca_1 + c2a_2 x + \cdots + c(n-1)a_{n-1} x^{n-2} + cna_n x^{n-1}            \\
        = {} & c(a_1 + 2a_2 x + \cdots + (n-1)a_{n-1} x^{n-2} + na_n x^{n-1})             \\
        = {} & cf^{\prime} (x).
    \end{align*}

    (ii) $f(x) \pm g(x)$ 就是多项式
    \begin{align*}
         & (a_0 \pm b_0) + (a_1 \pm b_1) x + (a_2 \pm b_2) x^2 + \cdots       \\
         & \qquad \qquad + (a_{n-1} \pm b_{n-1}) x^{n-1} + (a_n \pm b_n) x^n,
    \end{align*}
    故
    \begin{align*}
             & (f(x) \pm g(x))^{\prime}                                                        \\
        = {} & ((a_0 \pm b_0) + (a_1 \pm b_1) x + (a_2 \pm b_2) x^2 + \cdots                   \\
             & \qquad \qquad + (a_{n-1} \pm b_{n-1}) x^{n-1} + (a_n \pm b_n) x^n)^{\prime}     \\
        = {} & (a_1 \pm b_1) + 2(a_2 \pm b_2) x + \cdots + (n-1) (a_{n-1} \pm b_{n-1}) x^{n-2} \\
             & \qquad \qquad + n (a_n \pm b_n) x^{n-1}                                         \\
        = {} & (a_1 \pm b_1) + (2a_2 x \pm 2b_2 x) + \cdots + ((n-1)a_{n-1} x^{n-2}            \\
             & \qquad \qquad \pm (n-1)b_{n-1} x^{n-2})
        + (na_n x^{n-1} \pm nb_n x^{n-1})                                                      \\
        = {} & (a_1 + 2a_2 x + \cdots + (n-1)a_{n-1} x^{n-2} + na_n x^{n-1})                   \\
             & \qquad \qquad \pm (b_1 + 2b_2 x + \cdots + (n-1)b_{n-1} x^{n-2} + nb_n x^{n-1}) \\
        = {} & f^{\prime} (x) \pm g^{\prime} (x). \qedhere
    \end{align*}
\end{pf}

\begin{proposition}
    设 $f(x)$, $g(x) \in D[x]$. 则
    \begin{align*}
        (f(x) g(x))^{\prime} = f^{\prime} (x) g(x) + f(x) g^{\prime} (x). \tag*{(\myStar)}
    \end{align*}
    由 (\myStar) 与数学归纳法可知: 当 $f_0 (x)$, $f_1 (x)$, $\cdots$, $f_{k-1} (x) \in D[x]$ 时,
    \begin{align*}
             & (f_0 (x) f_1 (x) \cdots f_{k-1} (x))^{\prime}                                                      \\
        = {} & f_0^{\prime} (x) f_1 (x) \cdots f_{k-1} (x) + f_0 (x) f_1^{\prime} (x) \cdots f_{k-1} (x) + \cdots \\
             & \qquad \qquad + f_0 (x) f_1 (x) \cdots f_{k-1}^{\prime} (x).
    \end{align*}
    取 $f_0 (x) = f_1 (x) = \cdots = f_{k-1} (x) = f(x)$ 知
    \begin{align*}
        ((f(x))^k)^{\prime} = k(f(x))^{k-1} f^{\prime} (x).
    \end{align*}
\end{proposition}

\begin{pf}
    我们证明 (\myStar), 将剩下的二个式留给读者作练习. 首先, 任取 $i$, $j \in \NN$, $p$, $q \in D$, 有
    \begin{align*}
        px^i \cdot qx^j = pqx^{i+j}.
    \end{align*}
    这样,
    \begin{align*}
        (px^i \cdot qx^j)^{\prime}
        = {} & (pqx^{i+j})^{\prime}                             \\
        = {} & (i+j)pq x^{i+j-1}                                \\
        = {} & ipq x^{(i-1)+j} + jpq x^{i+(j-1)}                \\
        = {} & ipq x^{i-1} x^j + jpq x^i x^{j-1}                \\
        = {} & (ipx^{i-1}) (qx^j) + (px^i) (jqx^{j-1})          \\
        = {} & (px^i)^{\prime} (qx^j) + (px^i) (qx^j)^{\prime}.
    \end{align*}

    设
    \begin{align*}
         & f(x) = a_0 + a_1 x + \cdots + a_m x^m, \\
         & g(x) = b_0 + b_1 x + \cdots + b_n x^n
    \end{align*}
    为 $D[x]$ 中的二个元. 取 $px^i$ 为 $a_0$, $a_1 x$, $\cdots$, $a_m x^m$, 有
    \begin{align*}
        (a_0 \cdot qx^j)^{\prime}     = {} & (a_0)^{\prime} (qx^j) + (a_0) (qx^j)^{\prime},           \\
        (a_1 x \cdot qx^j)^{\prime}   = {} & (a_1 x)^{\prime} (qx^j) + (a_1 x) (qx^j)^{\prime},       \\
        \cdots \cdots \cdots \cdots        & \cdots \cdots \cdots \cdots \cdots \cdots \cdots \cdots, \\
        (a_m x^m \cdot qx^j)^{\prime} = {} & (a_m x^m)^{\prime} (qx^j) + (a_m x^m) (qx^j)^{\prime}.
    \end{align*}
    所以
    \begin{align*}
             & (f(x) \cdot qx^j)^{\prime}                                                                             \\
        = {} & (a_0 \cdot qx^j + a_1 x \cdot qx^j + \cdots + a_m x^m \cdot qx^j)^{\prime}                             \\
        = {} & (a_0 \cdot qx^j)^{\prime} + (a_1 x \cdot qx^j)^{\prime} + \cdots
        + (a_m x^m \cdot qx^j)^{\prime}                                                                               \\
        = {} & ((a_0)^{\prime} (qx^j) + (a_0) (qx^j)^{\prime}) + ((a_1 x)^{\prime} (qx^j)
        + (a_1 x) (qx^j)^{\prime})                                                                                    \\
             & \qquad \qquad + \cdots + ((a_m x^m)^{\prime} (qx^j) + (a_m x^m) (qx^j)^{\prime})                       \\
        = {} & ((a_0)^{\prime} (qx^j) + (a_1 x)^{\prime} (qx^j) + \cdots + (a_m x^m)^{\prime} (qx^j))                 \\
             & \qquad \qquad + ((a_0) (qx^j)^{\prime} + (a_1 x) (qx^j)^{\prime} + \cdots + (a_m x^m) (qx^j)^{\prime}) \\
        = {} & ((a_0)^{\prime} + (a_1 x)^{\prime} + \cdots + (a_m x^m)^{\prime}) (qx^j)                               \\
             & \qquad \qquad + (a_0 + a_1 x + \cdots + a_m x^m) (qx^j)^{\prime}                                       \\
        = {} & (a_0 + a_1 x + \cdots + a_m x^m)^{\prime} (qx^j) + f(x) (qx^j)^{\prime}                                \\
        = {} & f^{\prime} (x) (qx^j) + f(x) (qx^j)^{\prime}.
    \end{align*}
    再取 $qx^j$ 为 $b_0$, $b_1 x$, $\cdots$, $b_n x^n$, 有
    \begin{align*}
        (f(x) \cdot b_0)^{\prime}     = {} & f^{\prime} (x) (b_0) + f(x) (b_0)^{\prime},              \\
        (f(x) \cdot b_1 x)^{\prime}   = {} & f^{\prime} (x) (b_1 x) + f(x) (b_1 x)^{\prime},          \\
        \cdots \cdots \cdots \cdots        & \cdots \cdots \cdots \cdots \cdots \cdots \cdots \cdots, \\
        (f(x) \cdot b_n x^n)^{\prime} = {} & f^{\prime} (x) (b_n x^n) + f(x) (b_n x^n)^{\prime}.
    \end{align*}
    所以
    \begin{align*}
             & (f(x) g(x))^{\prime}                                                                           \\
        = {} & (f(x) \cdot b_0 + f(x) \cdot b_1 x + \cdots + f(x) \cdot b_n x^n)^{\prime}                     \\
        = {} & (f(x) \cdot b_0)^{\prime} + (f(x) \cdot b_1 x)^{\prime} + \cdots
        + (f(x) \cdot b_n x^n)^{\prime}                                                                       \\
        = {} & (f^{\prime} (x) (b_0) + f(x) (b_0)^{\prime}) + (f^{\prime} (x) (b_1 x)
        + f(x) (b_1 x)^{\prime})                                                                              \\
             & \qquad \qquad + \cdots + (f^{\prime} (x) (b_n x^n) + f(x) (b_n x^n)^{\prime})                  \\
        = {} & (f^{\prime} (x) (b_0) + (f^{\prime} (x) (b_1 x) + \cdots + f^{\prime} (x) (b_n x^n))           \\
             & \qquad \qquad + (f(x) (b_0)^{\prime} + f(x) (b_1 x)^{\prime} + \cdots f(x) (b_n x^n)^{\prime}) \\
        = {} & f^{\prime} (x) (b_0 + b_1 x + \cdots + b_n x^n)                                                \\
             & \qquad \qquad + f(x) ((b_0)^{\prime} + (b_1 x)^{\prime} + \cdots + (b_n x^n)^{\prime})         \\
        = {} & f^{\prime} (x) g(x) + f(x) (b_0 + b_1 x + \cdots + b_n x^n)^{\prime}                           \\
        = {} & f^{\prime} (x) g(x) + f(x) g^{\prime} (x). \qedhere
    \end{align*}
\end{pf}

\begin{example}
    考虑 $\ZZ$ 与 $\ZZ[x]$. 取
    \begin{align*}
        f(x) = x^3 + 2, \quad g(x) = x^2 + x - 1.
    \end{align*}
    不难得到
    \begin{align*}
        f^{\prime} (x) = 3x^2, \quad g^{\prime} (x) = 2x + 1.
    \end{align*}

    $f(x)$ 与 $g(x)$ 的积
    \begin{align*}
        f(x) g(x) = x^5 + x^4 - x^3 + 2x^2 + 2x - 2
    \end{align*}
    的微商是
    \begin{align*}
        (f(x) g(x))^{\prime} = 5x^4 + 4x^3 - 3x^2 + 4x + 2.
    \end{align*}
    如果用上面的 (\myStar) 计算, 就是
    \begin{align*}
             & f^{\prime} (x) g(x) + f(x) g^{\prime} (x) \\
        = {} & 3x^2 (x^2 + x - 1) + (x^3 + 2) (2x + 1)   \\
        = {} & 3x^4 + 3x^3 - 3x^2 + 2x^4 + x^3 + 4x + 2  \\
        = {} & 5x^4 + 4x^3 - 3x^2 + 4x + 2.
    \end{align*}
\end{example}

也许这不太能体现 (\myStar) 的作用: 算二个多项式积的微商时, 先拆再算好像没什么不方便的. 的确如此. 可是 (\myStar) 的推论
\begin{align*}
    ((f(x))^k)^{\prime} = k(f(x))^{k-1} f^{\prime}(x)
\end{align*}
很有用. 看下面的例.

\begin{example}
    还是考虑 $\ZZ$ 与 $\ZZ[x]$. 计算
    \begin{align*}
         & p(x) = (g \circ f) (x) = g(f(x)) = (x^3 + 2)^2 + (x^3 + 2) - 1, \\
         & q(x) = (f \circ g) (x) = f(g(x)) = (x^2 + x - 1)^3 + 2
    \end{align*}
    的微商.

    用定义写出 $p(x)$ 的微商并不是很难. 因为
    \begin{align*}
        p(x) = (x^6 + 4x^3 + 4) + x^3 + 2 - 1 = x^6 + 5x^3 + 5,
    \end{align*}
    故
    \begin{align*}
        p^{\prime} (x) = 6x^5 + 15x^2.
    \end{align*}
    不过用定义写出 $q(x)$ 就有点麻烦了: 三项的立方不是那么好算. 但是, 我们利用这个推论, 可直接写出
    \begin{align*}
        q^{\prime} (x) = 3(x^2 + x - 1)^2 (2x + 1).
    \end{align*}
\end{example}

记 $g(x) = x^k$. 取 $f(x) \in D[x]$. 不难看出,
\begin{align*}
    (f(x))^k = (g \circ f)(x).
\end{align*}
所以
\begin{align*}
    (g \circ f)^{\prime}(x) = ((f(x))^k)^{\prime} = k(f(x))^{k-1} f^{\prime} (x) = (g^{\prime} \circ f)(x) f^{\prime} (x).
\end{align*}

这告诉我们什么呢? 如果我们把 $f(x)$ 看成文字 $y$, 那么 $y^k \in D[y]$ 的微商是 $ky^{k-1}$. 将此结果乘 $y = f(x) \in D[x]$ 的微商 $f^{\prime} (x)$, 就是 $(g \circ f) (x) \in D[x]$ 的微商.

取 $h(x) = x \in D[x]$. 那么 $(f \circ h)(x)$ 就是 $f(x)$. 因为 $(x)^{\prime} = 1$, 所以
\begin{align*}
    (f \circ h)^{\prime} (x) = f^{\prime} (x) = (f^{\prime} \circ h) (x) h^{\prime} (x).
\end{align*}

我们作出猜想: 任取 $f(x)$, $g(x) \in D[x]$, 必有
\begin{align*}
    (g \circ f)^{\prime} (x) = (g^{\prime} \circ f)(x) f^{\prime} (x).
\end{align*}
幸运的事儿是, 这个猜想是正确的.

\begin{proposition}
    设 $f(x)$, $g(x) \in D[x]$. 则 $f(x)$ 与 $g(x)$ 的复合的微商适合链规则 \term{the chain rule}:
    \begin{align*}
        (g \circ f)^{\prime} (x) = (g^{\prime} \circ f)(x) f^{\prime} (x).
    \end{align*}
    链规则也可写为
    \begin{align*}
        (g(f(x)))^{\prime} = g^{\prime} (f(x)) f^{\prime} (x).
    \end{align*}
\end{proposition}

\begin{pf}
    设
    \begin{align*}
        g(x) = b_0 + b_1 x + b_2 x^2 + \cdots + b_{n-1} x^{n-1} + b_n x^n \in D[x],
    \end{align*}
    则
    \begin{align*}
        (g \circ f) (x) = b_0 + b_1 f(x) + b_2 (f(x))^2 + \cdots + b_{n-1} (f(x))^{n-1} + b_n (f(x))^n.
    \end{align*}
    所以
    \begin{align*}
             & (g \circ f)^{\prime} (x)                                \\
        = {} & b_1 f^{\prime} (x) + b_2 ((f(x))^2)^{\prime} + \cdots
        + b_{n-1} ((f(x))^{n-1})^{\prime} + b_n ((f(x))^n)^{\prime}    \\
        = {} & b_1 f^{\prime} (x) + b_2 \cdot 2 f(x) f^{\prime} (x)
        + \cdots + b_{n-1} \cdot (n-1) (f(x))^{n-2} f^{\prime} (x)     \\
             & \qquad \qquad + b_n \cdot n (f(x))^{n-1} f^{\prime} (x) \\
        = {} & b_1 f^{\prime} (x) + 2b_2 f(x) f^{\prime} (x)
        + \cdots + (n-1) b_{n-1} (f(x))^{n-2} f^{\prime} (x)           \\
             & \qquad \qquad + n b_n (f(x))^{n-1} f^{\prime} (x)       \\
        = {} & (b_1 + 2b_2 f(x) + \cdots + (n-1) b_{n-1} (f(x))^{n-2}
        + n b_n (f(x))^{n-1}) f^{\prime} (x)                           \\
        = {} & (g^{\prime} \circ f) (x) f^{\prime} (x). \qedhere
    \end{align*}
\end{pf}

\begin{example}
    我们用链规则计算 $p(x)$ 的微商:
    \begin{align*}
        p^{\prime} (x) = (g^{\prime} \circ f)(x) f^{\prime} (x) = (2(x^3+2) + 1)(3x^2) = 3x^2(2x^3 + 5).
    \end{align*}
    这跟前面算出的 $6x^5 + 15x^2$ 是一致的.
\end{example}
