\subsection*{\PolynomialsOverF}
\addcontentsline{toc}{subsection}{\PolynomialsOverF}
\markright{\PolynomialsOverF}

我们在前几节讨论的都是整环 $D$ 上的整式, 所以它们看上去是有些抽象的. 从现在开始, 我们不讨论抽象的 $D$ 与 $D[x]$, 而是讨论 $\FF$ 与 $\FF[x]$, 其中 $\FF$ 可代指 $\QQ$, $\RR$, $\CC$ 的任意一个. 细心的读者会注意到我们在前几节未使用 $\sum$ 符号: 这是为了让读者没那么困难地适应整式理论. 从本节起, 我们会较多地使用这个 $\sum$. 读者也可以乘此机会让自己熟悉它. 当然, 我们偶尔也会使用 $\prod$ 符号.

本节并没有什么新的知识 (除了最后几个关于 $\FF[x]$ 的命题). 读者可以乘此机会温习一下所学内容. 我们将重述一些定义与命题. 我们在学校学算学的时候, 也会有复习课. 就当本节就是 ``复习节'' 吧!

先从整式的定义与运算开始.

\begin{definition}
    设 $x$ 是不在 $\FF$ 里的任意一个文字. 形如
    \begin{align*}
        f(x)
        = {} & \sum_{i=0}^n a_i x^i                                                                       \\
        = {} & a_0 + a_1 x + \cdots + a_n x^n \quad (n \in \NN,\ a_0,a_1,\cdots,a_n \in \FF,\ a_n \neq 0)
    \end{align*}
    的表达式称为 $\FF$ 上 $x$ 的一个整式. $n$ 称为其次, $a_i$ 称为其 $i$ 次系数, $a_i x^i$ 称为其 $i$ 次项. $f(x)$ 的次可写为 $\deg f(x)$.

    若二个整式的次与各同次系数均相等, 则二者相等.

    整式的系数为 $0$ 的项可以不写.

    约定 $0 \in \FF$ 也是整式, 称为零整式. 零整式的次是 $-\infty$. 任取整数 $m$, 约定
    \begin{align*}
         & -\infty = -\infty, \quad -\infty < m,                        \\
         & -\infty + m = m + (-\infty) = -\infty + (-\infty) = -\infty.
    \end{align*}
    当然, 还约定, 零整式只跟自己相等. 换句话说,
    \begin{align*}
        \sum_{i=0}^n a_i x^i = 0
    \end{align*}
    的一个必要与充分条件是
    \begin{align*}
        a_0 = a_1 = \cdots = a_n = 0.
    \end{align*}

    $\FF$ 上 $x$ 的所有整式作成的集是 $\FF[x]$:
    \begin{align*}
        \FF[x] = \left\{\, \sum_{i=0}^n a_i x^i \,\,\middle|\,\, n \in \NN,\ a_0,a_1,\cdots,a_n \in \FF \,\right\}.
    \end{align*}

    文字 $x$ 只是一个符号, 它与 $\FF$ 的元的和与积都是形式的. 我们说, $x$ 是不定元.
\end{definition}

\begin{definition}
    设
    \begin{align*}
        f(x) = \sum_{i=0}^n a_i x^i, \quad g(x) = \sum_{i=0}^n b_i x^i \in \FF[x].
    \end{align*}
    规定加法如下:
    \begin{align*}
        f(x) + g(x) = \sum_{i=0}^n (a_i + b_i) x^i.
    \end{align*}
\end{definition}

\begin{proposition}
    设 $f(x)$, $g(x)$, $h(x) \in \FF[x]$. $\FF[x]$ 的加法适合如下性质:

    (i) $f(x) + g(x) \in \FF[x]$;

    (ii) $(f(x) + g(x)) + h(x)$ = $f(x) + (g(x) + h(x))$;

    (iii) 存在整式 $0$ 使 $0 + f(x) = f(x) + 0 = f(x)$;

    (iv) 存在整式 $-f(x)$ 使 $-f(x) + f(x) = f(x) + (-f(x)) = 0$;

    (v) $f(x) + g(x) = g(x) + f(x)$.
\end{proposition}

\begin{definition}
    设
    \begin{align*}
        f(x) = \sum_{i=0}^n a_i x^i, \quad g(x) = \sum_{i=0}^n b_i x^i \in \FF[x].
    \end{align*}
    则
    \begin{align*}
        -g(x) = \sum_{i=0}^n (-b_i) x^i.
    \end{align*}
    规定减法如下:
    \begin{align*}
        f(x) - g(x) = f(x) + (-g(x)).
    \end{align*}
\end{definition}

\begin{proposition}
    设 $f(x)$, $g(x) \in \FF[x]$. 则
    \begin{align*}
        \deg {(f(x) \pm g(x))} \leq \max \{\, \deg f(x), \deg g(x) \,\}.
    \end{align*}
    若 $\deg f(x) > \deg g(x)$, 则
    \begin{align*}
        \deg {(f(x) \pm g(x))} = \deg f(x).
    \end{align*}
    类似地, 若 $\deg f(x) < \deg g(x)$, 则
    \begin{align*}
        \deg {(f(x) \pm g(x))} = \deg g(x).
    \end{align*}
\end{proposition}

\begin{definition}
    设
    \begin{align*}
        f(x) = \sum_{i=0}^n a_i x^i = a_0 + a_1 x + \cdots + a_n x^n \in \FF[x].
    \end{align*}
    这称为 $f(x)$ 的升次排列. 下面的写法称为 $f(x)$ 的降次排列:
    \begin{align*}
        \sum_{j=0}^{n} a_{n-j} x^{n-j} = a_n x^n + a_{n-1} x^{n-1} + \cdots + a_0.
    \end{align*}

    (非零) 整式的最高次非零项是首项. 它的系数是此整式的首项系数.
\end{definition}

\begin{definition}
    设
    \begin{align*}
        f(x) = \sum_{i=0}^m a_i x^i, \quad g(x) = \sum_{j=0}^n b_j x^j \in \FF[x].
    \end{align*}
    规定乘法如下:
    \begin{align*}
        f(x) g(x) = \sum_{k=0}^{m+n} \left( \sum_{i=0}^k a_i b_{k-i} \right) x^k.
    \end{align*}
\end{definition}

\begin{proposition}
    设 $m$, $n \in \NN$, $p$, $q \in \FF$. 则
    \begin{align*}
        px^i \cdot qx^j = (px^i) (qx^j) = (pq)x^{i + j}.
    \end{align*}
\end{proposition}

\begin{proposition}
    设 $f(x)$, $g(x) \in \FF[x]$. 则
    \begin{align*}
        \deg f(x) g(x) = \deg f(x) + \deg g(x).
    \end{align*}
\end{proposition}

\begin{proposition}
    设 $f(x)$, $g(x)$, $h(x) \in \FF[x]$. $\FF[x]$ 的加法与乘法适合 (i) 至 (v) 及如下性质:

    (vi) $f(x) g(x) \in \FF[x]$;

    (vii) $(f(x) g(x)) h(x) = f(x) (g(x) h(x))$;

    (viii) 存在整式 $1$ 使 $1f(x) = f(x)1 = f(x)$;

    (ix) $(-1)f(x) = -f(x)$;

    (x) $f(x) g(x) = g(x) f(x)$;

    (xi) 若 $f(x) \neq 0$, 则
    \begin{align*}
         & f(x) g(x) = f(x) h(x) \implies g(x) = h(x), \\
         & g(x) f(x) = h(x) f(x) \implies g(x) = h(x);
    \end{align*}

    (xii) 二个分配律都对:
    \begin{align*}
         & f(x) (g(x) + h(x)) = f(x) g(x) + f(x) h(x), \\
         & (g(x) + h(x)) f(x) = g(x) f(x) + h(x) f(x).
    \end{align*}
\end{proposition}

\begin{remark}
    $\FF[x]$ 的一个名字就是 (域) $\FF$ 上 ($x$) 的整式环.
\end{remark}

\begin{definition}
    设 $m \in \NN$. 整式 $f(x)$ 的 $m$ 次幂就是 $m$ 个 $f(x)$ 的积:
    \begin{align*}
        (f(x))^m = \underbrace{f(x) \cdot f(x) \cdot \cdots \cdot f(x)}_{\text{$m$ $f(x)$'s}} = \prod_{\ell = 0}^{m-1} f(x).
    \end{align*}
    设 $m$, $n \in \NN$, $f(x)$, $g(x) \in \FF[x]$, 则整式的幂适合如下规则:
    \begin{align*}
         & (f(x))^m (f(x))^n = (f(x))^{m+n},  \\
         & ((f(x))^m)^n = (f(x))^{mn},        \\
         & (f(x) g(x))^m = (f(x))^m (g(x))^m.
    \end{align*}
\end{definition}

\begin{proposition}
    设 $f(x) \in \FF[x]$. 非零的 $c \in \FF$ 是 $0$ 次整式, 那么
    \begin{align*}
        \deg cf(x) = \deg f(x).
    \end{align*}
\end{proposition}

再来看整式的带余除法. 因为 $\FF$ 的每个非零元都是 $\FF$ 的单位, 所以有

\begin{proposition}
    设 $f(x) \in \FF[x]$ 是非零整式. 对任意 $g(x) \in \FF[x]$, 存在唯一的 $q(x), r(x) \in \FF[x]$ 使
    \begin{align*}
        g(x) = q(x) f(x) + r(x), \quad \deg r(x) < \deg f(x).
    \end{align*}
    一般称其为带余除法: $q(x)$ 就是商; $r(x)$ 就是余式. 并且, 当 $f(x)$ 的次不高于 $g(x)$ 的次时, $f(x)$, $g(x)$, $q(x)$ 间还有如下的次关系:
    \begin{align*}
        \deg g(x) = \deg {(g(x) - r(x))} = \deg q(x) + \deg f(x).
    \end{align*}
\end{proposition}

可以看到, 在 $\FF[x]$ 里, 带余除法的适用范围更广了. 我们还得到了一个有用的事实.

\begin{proposition}
    整式的带余除法不因系数的范围变大而改变. 具体地说, 设 $E$, $K$ 是三文字 $\QQ$, $\RR$, $\CC$ 的任意二个, 且 $K \subset E$. 设 $f(x)$, $g(x) \in K[x]$, 且 $f(x) \neq 0$. 设 $K$ 上的整式 $q_1 (x)$, $r_1 (x)$ 适合
    \begin{align*}
        g(x) = q_1 (x) f(x) + r_1 (x), \quad \deg r_1 (x) < \deg f(x).
    \end{align*}
    因为 $K$ 的元都是 $E$ 的元, 故 $f(x)$, $g(x)$ 当然可认为是 $E$ 上的整式. 设 $E$ 上的整式 $q_2 (x)$, $r_2 (x)$ 适合
    \begin{align*}
        g(x) = q_2 (x) f(x) + r_2 (x), \quad \deg r_2 (x) < \deg f(x).
    \end{align*}
    则 $q_1 (x) = q_2 (x)$, 且 $r_1 (x) = r_2 (x)$.
\end{proposition}

下面回顾整式的相等. 我们借助 ``线性无关'' 讨论相等问题.

\begin{definition}
    设 $p_0 (x)$, $p_1 (x)$, $\cdots$, $p_n (x) \in \FF[x]$. 设 $c_0$, $c_1$, $\cdots$, $c_n \in \FF$. 我们说
    \begin{align*}
        \sum_{i = 0}^{n} c_i p_i (x)
    \end{align*}
    是整式 $p_0 (x)$, $p_1 (x)$, $\cdots$, $p_n (x)$ 的一个线性组合. $c_0$, $c_1$, $\cdots$, $c_n$ 就是此线性组合的系数.

    若不存在一组不全为 $0$ 的 $\FF$ 中元 $d_0$, $d_1$, $\cdots$, $d_n$ 使
    \begin{align*}
        \sum_{i = 0}^{n} d_i p_i (x) = 0,
    \end{align*}
    则说 $p_0 (x)$, $p_1 (x)$, $\cdots$, $p_n (x)$ 是线性无关的. 换句话说, ``$p_0 (x)$, $p_1 (x)$, $\cdots$, $p_n (x)$ 是线性无关的'' 意味着: 若 $\FF$ 中元 $r_0$, $r_1$, $\cdots$, $r_n$ 使
    \begin{align*}
        \sum_{i = 0}^{n} r_i p_i (x) = 0,
    \end{align*}
    则 $r_0 = r_1 = \cdots = r_n = 0$.
\end{definition}

\begin{proposition}
    设 $p_0 (x)$, $p_1 (x)$, $\cdots$, $p_n (x) \in \FF[x]$ 分别是 $0$, $1$, $\cdots$, $n$ 次整式. 则:

    (i) $p_0 (x)$, $p_1 (x)$, $\cdots$, $p_n (x)$ 是线性无关的;

    (ii) 任意次不高于 $n$ 的整式都可唯一地写为 $p_0 (x)$, $p_1 (x)$, $\cdots$, $p_n (x)$ 的线性组合.
\end{proposition}

由于 $\FF$ 的每个非零元都是单位, 上面的命题的结论变强了. 下面的例体现了这一点.

\begin{example}
    考虑 $\FF$ 与 $\FF[x]$. 取 $n=2$, 及
    \begin{align*}
        p_0 (x) = -1, \quad p_1 (x) = 2x, \quad p_2 (x) = 3x^2.
    \end{align*}
    这三个整式是线性无关的. 考虑 $f(x) = 3 + x - 2x^2$. 设 $c_0$, $c_1$, $c_2 \in \FF$ 使
    \begin{align*}
        3 + x - 2x^2 = c_0 \cdot (-1) + c_1 \cdot 2x + c_2 \cdot 3x^2.
    \end{align*}
    这相当于
    \begin{align*}
        3 = -c_0, \quad 1 = 2c_1, \quad -2 = 3c_2.
    \end{align*}
    由此可得
    \begin{align*}
        c_0 = -3, \quad c_1 = \frac12, \quad c_2 = -\frac23.
    \end{align*}
    可以看到, 在 $\ZZ$ 与 $\ZZ[x]$ 里 $p_0 (x)$, $p_1 (x)$, $p_2 (x)$ 的线性组合还不能表示这个 $f(x)$, 但当我们在 ``大环境'' $\FF$ 与 $\FF[x]$ 下讨论问题时就可以了.
\end{example}

\begin{remark}
    我们常常把 $D$ 的元分为三类: 零、单位、非零且不是单位的元. 但是在 $\FF$, 只要分为二类即可: 零与非零.
\end{remark}

\begin{proposition}
    设 $a_0$, $b_0$, $a_1$, $b_1$, $\cdots$, $a_n$, $b_n \in \FF$. 设 $c \in \FF$. 再设
    \begin{align*}
        f(x) = \sum_{i = 0}^n a_i (x-c)^i, \quad g(x) = \sum_{i = 0}^n b_i (x-c)^i.
    \end{align*}
    则 $f(x)=g(x)$ 的一个必要与充分条件是
    \begin{align*}
        a_0 = b_0, \quad a_1 = b_1, \quad \cdots, \quad a_n = b_n.
    \end{align*}
    并且, 任取
    \begin{align*}
        f(x) = \sum_{i = 0}^n u_i x^i \in \FF[x],
    \end{align*}
    必存在 $v_0$, $v_1$, $\cdots$, $v_n \in \FF$ 使
    \begin{align*}
        f(x) = \sum_{i = 0}^n v_i (x-c)^i.
    \end{align*}
\end{proposition}

我们看看整式的流数.

\begin{definition}
    设
    \begin{align*}
        f(x) = \sum_{i = 0}^{n} a_i x^i \in \FF[x].
    \end{align*}
    $f(x)$ 的流数是整式
    \begin{align*}
        f^{\prime}(x) = \sum_{i = 1}^{n} i a_i x^{i - 1} \in \FF[x].
    \end{align*}
    $f^{\prime} (x)$ 也可写为 $(f(x))^{\prime}$.
\end{definition}

\begin{remark}
    若 $f(x) = c$, $c \in \FF$, 则 $f^{\prime} (x)$ 为零整式.
\end{remark}

\begin{definition}
    设
    \begin{align*}
        f(x) = \sum_{i = 0}^{m} a_i x^i, \quad g(x) = \sum_{j = 0}^{n} b_j x^j
    \end{align*}
    为 $\FF[x]$ 中的二个元. 我们称
    \begin{align*}
        (g \circ f)(x) = g(f(x)) = \sum_{j = 0}^{n} b_j (f(x))^j
    \end{align*}
    为 $f(x)$ 与 $g(x)$ 的复合.
\end{definition}

\begin{proposition}
    整式的复合适合结合律. 具体地说, 设 $f(x)$, $g(x)$, $h(x) \in \FF[x]$, 则
    \begin{align*}
        ((h \circ g) \circ f)(x) = (h \circ (g \circ f))(x).
    \end{align*}
\end{proposition}

\begin{proposition}
    设 $f(x)$, $g(x)$, $h(x) \in \FF[x]$.

    (i) 若 $f(x) = g(x)$, 则 $(f \circ h)(x) = (g \circ h)(x)$.

    (ii) 若 $p(x) = f(x) + g(x)$, 则 $(p \circ h)(x) = (f \circ h)(x) + (g \circ h)(x)$.

    (iii) 若 $q(x) = f(x) g(x)$, 则 $(q \circ h)(x) = (f \circ h)(x) \cdot (g \circ h)(x)$.
\end{proposition}

\begin{proposition}
    若整式 $f_0 (x)$, $f_1 (x)$, $\cdots$, $f_{n-1} (x)$ 之间有一个由加法与乘法计算得到的关系, 那么将 $x$ 换为整式 $h(x)$, 这样的关系仍成立.
\end{proposition}

\begin{proposition}
    设 $f(x)$, $g(x) \in \FF[x]$, $c \in \FF$. 则

    (i) $(cf(x))^{\prime} = c f^{\prime} (x)$;

    (ii) $(f(x) \pm g(x))^{\prime} = f^{\prime} (x) \pm g^{\prime} (x)$.

    由 (i) (ii) 与算学归纳法可知: 当 $c_0$, $c_1$, $\cdots$, $c_{k-1} \in \FF$, 且 $f_0 (x)$, $f_1 (x)$, $\cdots$, $f_{k-1} (x) \in \FF[x]$ 时,
    \begin{align*}
        \left( \sum_{\ell = 0}^{k-1} c_\ell f_\ell (x) \right)^{\prime} = \sum_{\ell = 0}^{k-1} c_\ell f_\ell^{\prime} (x).
    \end{align*}
\end{proposition}

\begin{proposition}
    设 $f(x)$, $g(x) \in \FF[x]$. 则
    \begin{align*}
        (f(x) g(x))^{\prime} = f^{\prime} (x) g(x) + f(x) g^{\prime} (x). \tag*{(\ding{72})}
    \end{align*}
    由 (\ding{72}) 与算学归纳法可知: 当 $f_0 (x)$, $f_1 (x)$, $\cdots$, $f_{k-1} (x) \in \FF[x]$ 时,
    \begin{align*}
             & (f_0 (x) f_1 (x) \cdots f_{k-1} (x))^{\prime}                                                      \\
        = {} & f_0^{\prime} (x) f_1 (x) \cdots f_{k-1} (x) + f_0 (x) f_1^{\prime} (x) \cdots f_{k-1} (x) + \cdots \\
             & \qquad \qquad + f_0 (x) f_1 (x) \cdots f_{k-1}^{\prime} (x).
    \end{align*}
    取 $f_0 (x) = f_1 (x) = \cdots = f_{k-1} (x) = f(x)$ 知
    \begin{align*}
        ((f(x))^k)^{\prime} = k(f(x))^{k-1} f^{\prime} (x).
    \end{align*}
\end{proposition}

\begin{proposition}
    设 $f(x)$, $g(x) \in \FF[x]$. 则 $f(x)$ 与 $g(x)$ 的复合的流数适合链规则:
    \begin{align*}
        (g \circ f)^{\prime} (x) = (g^{\prime} \circ f)(x) f^{\prime} (x).
    \end{align*}
    % 链规则也可写为
    % \begin{align*}
    %     (g(f(x)))^{\prime} = g^{\prime} (f(x)) f^{\prime} (x).
    % \end{align*}
\end{proposition}

有时, 我们会把整式的不定元 $x$ 改为数. 这就引出了整式在一点的值.

\begin{definition}
    设
    \begin{align*}
        f(x) = \sum_{i=0}^n a_i x^i \in \FF[x].
    \end{align*}
    设 $t \in \FF$. 我们把 $\FF$ 的元
    \begin{align*}
        \sum_{i=0}^n a_i t^i
    \end{align*}
    简单地写为 $f(t)$, 并称其为整式 $f(x)$ 在点 $t$ 的值.

    顺便一提, $f(x)$ 的流数也是整式:
    \begin{align*}
        f^{\prime} (x) = \sum_{i = 1}^{n} i a_i x^{i - 1}.
    \end{align*}
    我们把
    \begin{align*}
        \sum_{i = 1}^{n} i a_i t^{i - 1} \in \FF
    \end{align*}
    简单地写为 $f^{\prime} (t)$.
\end{definition}

整式在一点的值适合如下性质:
\begin{proposition}
    设 $t \in \FF$. 设 $f(x)$, $g(x) \in \FF[x]$.

    (i) 若 $f(x) = g(x)$, 则 $f(t) = g(t)$.

    (ii) 若 $p(x) = f(x) + g(x)$, 则 $p(t) = f(t) + g(t)$.

    (iii) 若 $q(x) = f(x) g(x)$, 则 $q(t) = f(t) g(t)$.
\end{proposition}

我们得到了如下命题:
\begin{proposition}
    若整式 $f_0 (x)$, $f_1 (x)$, $\cdots$, $f_{n-1} (x)$ 之间有一个由加法与乘法计算得到的关系, 那么将 $x$ 换为 $\FF$ 的元 $t$, 这样的关系仍成立.
\end{proposition}

\begin{definition}
    设 $f(x) \in \FF[x]$. 称函数
    \begin{align*}
         & \FF \to \FF, \tag*{$f_\mathrm{f} \colon$} \\
         & t \mapsto f(t)
    \end{align*}
    为 $\FF$ 的整式函数. 我们也说, 这个函数是由 $\FF$ 上 $x$ 的整式 $f(x)$ 诱导的整式函数.
\end{definition}

\begin{definition}
    设 $f_\mathrm{f}$ 与 $g_\mathrm{f}$ 是 $\FF$ 的二个整式函数. 二者的和 $f_\mathrm{f} + g_\mathrm{f}$ 定义为
    \begin{align*}
         & \FF \to \FF, \tag*{$f_\mathrm{f} + g_\mathrm{f} \colon$} \\
         & t \mapsto f_\mathrm{f} (t) + g_\mathrm{f} (t).
    \end{align*}
    二者的积 $f_\mathrm{f} \, g_\mathrm{f}$ 定义为
    \begin{align*}
         & \FF \to \FF, \tag*{$f_\mathrm{f} \, g_\mathrm{f} \colon$} \\
         & t \mapsto f_\mathrm{f} (t) g_\mathrm{f} (t).
    \end{align*}
\end{definition}

\begin{proposition}
    设 $f(x)$, $g(x) \in \FF[x]$. 设 $f_\mathrm{f}$, $g_\mathrm{f}$ 分别是 $f(x)$, $g(x)$ 诱导的整式函数.

    (i) 若 $f(x) = g(x)$, 则 $f_\mathrm{f} = g_\mathrm{f}$ (函数的相等).

    (ii) $f(x) + g(x)$ 诱导整式函数 $f_\mathrm{f} + g_\mathrm{f}$.

    (iii) $f(x) g(x)$ 诱导整式函数 $f_\mathrm{f} \, g_\mathrm{f}$.
\end{proposition}

下面是带余除法的推论. 它在根的讨论里起了重要的作用.
\begin{proposition}
    设 $f(x) \in \FF[x]$ 是 $n$ 次整式 ($n \geq 1$), $a \in \FF$. 则存在 $n-1$ 次整式 $q(x)$ ($\in \FF[x]$) 使
    \begin{align*}
        f(x) = q(x) (x-a) + f(a).
    \end{align*}
    根据带余除法, 这样的 $q(x)$ 一定是唯一的.
\end{proposition}

最后, 我们回顾整式的根.

\begin{definition}
    设 $f(x)$ 是 $\FF$ 上 $x$ 的整式. 若有 $a \in \FF$ 使 $f(a) = 0$, 则说 $a$ 是 (整式) $f(x)$ 的根.
\end{definition}

\begin{proposition}
    设 $f(x) \in \FF[x]$ 是 $n$ 次整式 ($n \geq 1$). $a$ 是 $f(x)$ 的根的一个必要与充分条件是: 存在 $n-1$ 次整式 $q(x)$ ($\in \FF[x]$) 使
    \begin{align*}
        f(x) = q(x) (x-a).
    \end{align*}
    根据带余除法, 这样的 $q(x)$ 一定是唯一的.
\end{proposition}

\begin{proposition}
    设 $f(x) \in \FF[x]$ 是 $n$ 次整式 ($n \in \NN$). 则 $f(x)$ 至多有 $n$ 个不同的根.
\end{proposition}

\begin{remark}
    在上节, 我们知道, 整环 $D$ 上的整式 $f(x) = ax + b$ ($a \neq 0$) 不一定有根. 可是, 在域 $\FF$ 里, $f(x)$ 就有根 $-\frac{b}{a}$.
\end{remark}

\begin{proposition}
    设 $a_0$, $a_1$, $\cdots$, $a_n$ 是 $\FF$ 的元. 设 $n$ 是非负整数. 设
    \begin{align*}
        f(x) = \sum_{i = 0}^n a_i x^i.
    \end{align*}
    若 $t_0$, $t_1$, $\cdots$, $t_n$ 是 $n+1$ 个互不相同的 $\FF$ 的元, 且
    \begin{align*}
        f(t_0) = f(t_1) = \cdots = f(t_n) = 0,
    \end{align*}
    则 $f(x)$ 必为零整式. 通俗地说, 次不高于 $n$ (且系数为 $\FF$ 的元) 的整式不可能有 $n$ 个以上的互不相同的根, 除非这个整式是零.
\end{proposition}

\begin{proposition}
    设 $a_0$, $b_0$, $a_1$, $b_1$, $\cdots$, $a_n$, $b_n$ 是 $\FF$ 的元. 设 $n$ 是非负整数. 设
    \begin{align*}
        f(x) = \sum_{i = 0}^n a_i x^i, \quad g(x) = \sum_{i = 0}^n b_i x^i.
    \end{align*}
    若 $t_0$, $t_1$, $\cdots$, $t_n$ 是 $n+1$ 个互不相同的 $\FF$ 的元, 且
    \begin{align*}
        f(t_0) = g(t_0), \quad f(t_1) = g(t_1), \quad \cdots, \quad f(t_n) = g(t_n),
    \end{align*}
    则 $f(x)$ 必等于 $g(x)$. 通俗地说, 若次不高于 $n$ (且系数为 $\FF$ 的元) 的二个整式若在多于 $n$ 处取一样的值, 则这二个整式相等.
\end{proposition}

\begin{definition}
    设 $a \in \FF$ 是整式 $f(x) \in \FF[x]$ 的根. 设 $f(x) \neq 0$. 那么, 存在唯一的整式 $q(x) \in \FF[x]$ 使
    \begin{align*}
        f(x) = (x - a) q(x),
    \end{align*}
    且 $q(x) \neq 0$. 若 $q(a) = 0$, 则说 $a$ 是 $f(x)$ 的一个重根. 若 $q(a) \neq 0$, 则说 $a$ 是 $f(x)$ 的一个单根.
\end{definition}

\begin{proposition}
    设 $a \in \FF$ 是整式 $f(x) \in \FF[x]$ 的根. 设 $f(x) \neq 0$. 则:

    (i) 若 $a$ 是 $f(x)$ 的重根, 则 $a$ 是 $f^{\prime} (x)$ 的根;

    (ii) 若 $a$ 是 $f(x)$ 的单根, 则 $a$ 不是 $f^{\prime} (x)$ 的根.

    所以, $f(x)$ 有重根的一个必要与充分条件是: $f(x)$ 与 $f^{\prime} (x)$ 有公共根.
\end{proposition}

下面是一些新命题. 由于 $\FF$ 里有无限多个元, 所以

\begin{proposition}
    设 $f(x) \in \FF[x]$. 设 $S \subset \FF$, 且 $S$ 有无限多个元. 若任取 $t \in S$, 必有 $f(t) = 0$, 则 $f(x)$ 必为零整式. 通俗地说, 系数为 $\FF$ 的元的整式不可能有无限多个根, 除非这个整式是零.
\end{proposition}

\begin{pf}
    $f(x)$ 的次不可能是非负整数. 所以 $f(x)$ 只能是 $0$.
\end{pf}

由此立得

\begin{proposition}
    设 $f(x)$, $g(x) \in \FF[x]$. 设 $S \subset \FF$, 且 $S$ 有无限多个元. 若任取 $t \in S$, 必有 $f(t) = g(t)$, 则 $f(x)$ 与 $g(x)$ 是二个相同的整式. 通俗地说, 若系数为 $\FF$ 的元的二个整式在无限多个地方有相同的取值, 则这二个整式必相等.
\end{proposition}

\begin{pf}
    考虑 $h(x) = f(x) - g(x)$, 并利用上个命题.
\end{pf}

前面已经知道, 整式确定整式函数. 利用上面的命题, 我们有

\begin{proposition}
    $\FF$ 上的整式与 $\FF$ 的整式函数是一一对应的: 不但二个不同的 $\FF$ 上的整式给出二个不同的 $\FF$ 的整式函数, 而且二个不同的 $\FF$ 的整式函数给出二个不同的 $\FF$ 上的整式.
\end{proposition}

\begin{remark}
    这个命题说明, 整式 (本文讨论的 ``整式'') 与整式函数 (中学算学的 ``整式'') 可被认为是同一事物. 尽管如此, 除非特别说明, 我们仍视形如 ``$a_0 + a_1 x + \cdots + a_n x^n$'' 的文字为整式, 而不是整式函数. 或者这么说——在本文的讨论里, $x$ 不再是数.
\end{remark}
