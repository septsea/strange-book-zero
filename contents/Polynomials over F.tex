\subsection*{Polynomials over $\FF$}
\addcontentsline{toc}{subsection}{Polynomials over \texorpdfstring{$\FF$}{𝔽}}
\markright{Polynomials over $\FF$}

我们在前几节讨论的都是整环 $D$ 上的多项式, 所以看上去是有些抽象的\period 从现在开始, 我们讨论 $\FF$ 与 $\FF[x]$, 其中 $\FF$ 可代指 $\QQ$, $\RR$, $\CC$ 的任意一个\period

先回顾一下我们在前几节里证明的结论\period

\begin{proposition}
    $\FF[x]$ 作成整环\period 所以, $\FF[x]$ 的一个名字就是 (域) $\FF$ 上 ($x$) 的多项式环\period
\end{proposition}

因为 $\FF$ 的每个非零元都是 $\FF$ 的单位, 所以有

\begin{proposition}
    设 $f(x) \in \FF[x]$ 是非零多项式\period 对任意 $g(x) \in \FF[x]$, 存在唯一的 $q(x), r(x) \in \FF[x]$ 使
    \begin{align*}
        g(x) = q(x) f(x) + r(x), \quad \deg r(x) < \deg f(x) \period
    \end{align*}
    一般称其为带余除法: $q(x)$ 就是商; $r(x)$ 就是余式\period 并且, 当 $f(x)$ 的次不高于 $g(x)$ 的次时, $f(x)$, $g(x)$, $q(x)$ 间还有如下的次关系:
    \begin{align*}
        \deg g(x) = \deg (g(x) - r(x)) = \deg q(x) + \deg f(x) \period
    \end{align*}
\end{proposition}

可以看到, 在 $\FF[x]$ 里, 带余除法的适用范围更广了\period

由于 $\FF$ 里有无数多个元, 所以

\begin{proposition}
    设 $f(x) \in \FF[x]$\period 设 $S \subset \FF$, 且 $S$ 有无数多个元\period 若任取 $t \in S$, 必有 $f(t) = 0$, 则 $f(x)$ 必为零多项式\period 通俗地说, 就是系数为 $\FF$ 的元的多项式不可能有无数多个根, 除非这个多项式是零\period
\end{proposition}

\begin{pf}
    $f(x)$ 的次不可能是非负整数\period 所以 $f(x)$ 只能是 $0$\period
\end{pf}

由此立得

\begin{proposition}
    设 $f(x),g(x) \in \FF[x]$\period 设 $S \subset \FF$, 且 $S$ 有无数多个元\period 若任取 $t \in S$, 必有 $f(t) = g(t)$, 则 $f(x)$ 与 $g(x)$ 是二个相同的多项式\period 通俗地说, 就是若系数为 $\FF$ 的元的二个多项式在无数多个地方有相同的取值, 则这二个多项式必相等\period
\end{proposition}

\begin{pf}
    考虑 $h(x) = f(x) - g(x)$, 并利用上个命题\period
\end{pf}

前面已经知道, 多项式确定多项式函数\period 利用上面的命题, 我们有

\begin{proposition}
    $\FF$ 上的多项式与 $\FF$ 的多项式函数是一一对应的: 不但二个不同的 $\FF$ 上的多项式给出二个不同的 $\FF$ 的多项式函数, 而且二个不同的 $\FF$ 的多项式函数给出二个不同的 $\FF$ 上的多项式\period
\end{proposition}
