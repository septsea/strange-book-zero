\subsection*{\DivisionAlgorithm}
\addcontentsline{toc}{subsection}{\DivisionAlgorithm}
\markright{\DivisionAlgorithm}

我们知道, 非负整数有这样的性质:

\begin{proposition}
    设 $f$ 是正整数, $g$ 是非负整数. 则必有一对非负整数 $q$, $r$ 使
    \begin{align*}
        g = qf + r, \quad 0 \leq r < f.
    \end{align*}
\end{proposition}

例如, 取 $f=5$, $g=23$. 不难看出,
\begin{align*}
    23 = 4 \cdot 5 + 3.
\end{align*}

整式也有类似的性质哟.

\begin{proposition}
    设
    \begin{align*}
        f(x) = a_n x^n + a_{n-1} x^{n-1} + \cdots + a_0 \in D[x],
    \end{align*}
    且 $a_n$ 是 $D$ 的单位. 对任意 $g(x) \in D[x]$, 存在 $q(x), r(x) \in D[x]$ 使
    \begin{align*}
        g(x) = q(x) f(x) + r(x), \quad \deg r(x) < n.
    \end{align*}
    一般称其为带馀除法: $q(x)$ 就是商 \term{quotient}; $r(x)$ 就是馀式 \term{remainder}.
\end{proposition}

\begin{pf}
    用算学归纳法. 记 $\deg g(x) = m$. 若 $m<n$, 则 $q(x) = 0$, $r(x) = g(x)$ 适合要求. 所以, 命题对不高于 $n-1$ 的 $m$ 都成立.

    设 $m \leq \ell$ ($\ell \geq n-1$) 时, 命题成立. 考虑 $m=\ell+1$ 的情形. 此时, 设
    \begin{align*}
        g(x) = b_{\ell + 1} x^{\ell} + b_{\ell} x^{\ell} + \cdots + b_0 \in D[x].
    \end{align*}
    作一个跟 $g(x)$ 有着共同首项的整式:
    \begin{align*}
        s(x) = {} & b_{\ell + 1} a_n^{-1} x^{\ell + 1 - n} f(x)                                                                               \\
        = {}      & b_{\ell + 1} a_n^{-1} x^{\ell + 1 - n} (a_n x^n + a_{n-1} x^{n-1} + \cdots + a_0)                                         \\
        = {}      & b_{\ell + 1} a_n^{-1} (a_n x^{\ell + 1} + a_{n-1} x^{\ell} + \cdots + a_0 x^{\ell + 1 - n})                               \\
        = {}      & b_{\ell + 1} (x^{\ell + 1} + a_n^{-1} a_{n-1} x^{\ell} + \cdots + a_n^{-1} a_0 x^{\ell + 1 - n})                          \\
        = {}      & b_{\ell + 1} x^{\ell + 1} + b_{\ell + 1} a_n^{-1} a_{n-1} x^{\ell} + \cdots + b_{\ell + 1} a_n^{-1} a_0 x^{\ell + 1 - n}.
    \end{align*}
    因为 $a_n$ 是单位, 故 $s(x) \in D[x]$. 设 $r_1 (x) = g(x) - s(x) \in D[x]$. 这样, $r_1 (x)$ 的次不高于 $\ell$. 根据归纳假设, 有 $q_2 (x)$, $r_2(x) \in D[x]$ 使
    \begin{align*}
        r_1 (x) = q_2 (x) f(x) + r_2 (x), \quad \deg r_2 (x) < n.
    \end{align*}
    所以
    \begin{align*}
        g(x)
        = {} & b_{\ell+1} a_n^{-1} x^{\ell+1-n} f(x) + r_1 (x)                \\
        = {} & b_{\ell+1} a_n^{-1} x^{\ell+1-n} f(x) + q_2 (x) f(x) + r_2 (x) \\
        = {} & (b_{\ell+1} a_n^{-1} x^{\ell+1-n} + q_2 (x)) f(x) + r_2 (x).
    \end{align*}
    记 $q(x) = b_{\ell+1} a_n^{-1} x^{\ell+1-n} + q_2 (x)$, $r(x) = r_2 (x)$, 则 $q(x)$, $r(x)$ 适合要求. 所以, $m \leq \ell + 1$ 时, 命题成立. 根据算学归纳法, 命题成立.
\end{pf}

\begin{example}
    取 $\FF[x]$ 的二元 $f(x) = 2(x-1)^2 (x+2)$, $g(x) = 8x^6 + 1$. 我们来找一对整式 $q(x), r(x) \in \FF[x]$ 使
    \begin{align*}
        g(x) = q(x) f(x) + r(x), \quad \deg r(x) < \deg f(x).
    \end{align*}
    不难看出, $f(x)$ 的次是 3, 且
    \begin{align*}
        f(x) = 2(x^2 - 2x + 1)(x+2) = 2x^3 - 6x + 4.
    \end{align*}

    我们按上面证明的方法寻找 $q(x)$ 与 $r(x)$. $a_3 = 2$ 是 $\FF$ 的单位, 且 $a_3^{-1} = \frac12$. 取
    \begin{align*}
        q_1 (x) = 8\cdot \frac12 \cdot x^{6-3} = 4x^3.
    \end{align*}
    则
    \begin{align*}
        r_1 (x)
        = {} & g(x) - q_1(x) f(x)                  \\
        = {} & (8x^6 + 1) - 4x^3 (2x^3 - 6x + 4)   \\
        = {} & (8x^6 + 1) - (8x^6 - 24x^4 + 16x^3) \\
        = {} & 24x^4 - 16x^3 + 1.
    \end{align*}
    $r_1 (x)$ 的次仍不低于 $3$. 因此, 再来一次. 取
    \begin{align*}
        q_2 (x) = 24 \cdot \frac12 \cdot x^{4-3} = 12x.
    \end{align*}
    则
    \begin{align*}
        r_2 (x)
        = {} & r_1 (x) - q_2(x) f(x)                     \\
        = {} & (24x^4 - 16x^3 + 1) - 12x (2x^3 - 6x + 4) \\
        = {} & (24x^4 - 16x^3 + 1) - (24x^4 - 72x + 48x) \\
        = {} & -16x^3 + 72x^2 - 48x + 1.
    \end{align*}
    $r_2 (x)$ 的次仍不低于 $3$. 因此, 再来一次. 取
    \begin{align*}
        q_3 (x) = -16 \cdot \frac12 \cdot x^{3-3} = -8.
    \end{align*}
    则
    \begin{align*}
        r_3 (x)
        = {} & r_2 (x) - q_3(x) f(x)                             \\
        = {} & (-16x^3 + 72x^2 - 48x + 1) - (-8) (2x^3 - 6x + 4) \\
        = {} & (-16x^3 + 72x^2 - 48x + 1) - (-16x^3 + 48x - 32)  \\
        = {} & 72x^2 - 96x + 33.
    \end{align*}
    $r_3 (x)$ 的次低于 $3$. 这样
    \begin{align*}
        g(x)
        = {} & q_1 (x) f(x) + r_1 (x)                               \\
        = {} & q_1 (x) f(x) + q_2 (x) f(x) + r_2 (x)                \\
        = {} & q_1 (x) f(x) + q_2 (x) f(x) + q_3 (x) f(x) + r_3 (x) \\
        = {} & (q_1 (x) + q_2 (x) + q_3 (x)) f(x) + r_3 (x)         \\
        = {} & (4x^3 + 12x - 8) f(x) + (72x^2 - 96x + 33).
    \end{align*}
    也就是说,
    \begin{align*}
        q(x) = 4x^3 + 12x - 8, \quad r(x) = 72x^2 - 96x + 33.
    \end{align*}
\end{example}

\begin{remark}
    带馀除法要求 $f(x)$ 的首项系数是单位是有必要的.

    在上面的例里, $f(x)$ 与 $g(x)$ 可以看成 $\ZZ[x]$ 的元, 但 $2$ 不是 $\ZZ$ 的单位. 虽然最终所得 $q(x)$, $r(x)$ 也是 $\ZZ[x]$ 的元, 但这并不是一定会出现的. 我们看下面的简单例.

    考虑 $\ZZ[x]$ 的整式 $f(x)=2x$. 设
    \begin{align*}
         & r(x) = r_0,                            \\
         & q(x) = q_0 + q_1 x + \cdots + q_p x^p, \\
         & g(x) = g_0 + g_1 x + \cdots + g_s x^s,
    \end{align*}
    且 $r_0$, $q_0$, $\cdots$, $q_p$, $g_0$, $\cdots$, $g_s \in \ZZ$, $q_p$, $g_s \neq 0$. 若 $g(x) = q(x)f(x) + r(x)$, 则
    \begin{align*}
        g_0 + g_1 x + \cdots + g_s x^s = r_0 + 2q_0 x + 2q_1 x^2 + \cdots + 2q_p x^{p+1}.
    \end{align*}
    所以
    \begin{align*}
         & p = s - 1,                          \\
         & r_0 = g_0,                          \\
         & 2q_{i-1} = g_i, \quad i=1,\cdots,s.
    \end{align*}
    这说明, $g(x)$ 的 $i$ 项系数 ($i=1,\cdots,s$) 必须是偶数. 所以, 不存在 $q(x)$, $r(x) \in \ZZ[x]$ 使
    \begin{align*}
        1 + 3x + x^2 = q(x) \cdot 2x + r(x), \quad \deg r(x) < 1.
    \end{align*}
\end{remark}

我们知道, 用一个正整数除\myFN{$f$ 除 $g$ 意味着 $g$ 除以 $f$, 也就是 $g \div f$.}非负整数, 所得的馀数与商是唯一的. 比方说, $5$ 除 $23$ 的馀数只能是 $3$.

整式也有类似的性质哟. 不过, 我们需要借助另一个命题的帮助.

\begin{proposition}
    设 $f(x) \in D[x]$, 且 $f(x) \neq 0$. 若 $D$ 上 $x$ 的 $2$ 个整式 $q(x)$, $r(x)$ 适合
    \begin{align*}
        q(x) f(x) + r(x) = 0, \quad \deg r(x) < \deg f(x),
    \end{align*}
    则必有
    \begin{align*}
        q(x) = r(x) = 0.
    \end{align*}
    通俗地说, 二个非零整式的积的次不可能变低.
\end{proposition}

\begin{pf}
    题设条件即
    \begin{align*}
        -q(x) f(x) = r(x).
    \end{align*}
    反证法. 若 $-q(x) \neq 0$, 则 $\deg {(-q(x))} \geq 0$. 从而
    \begin{align*}
        \deg r(x) = \deg {(-q(x))} + \deg f(x) \geq \deg f(x).
    \end{align*}
    可是,
    \begin{align*}
        \deg r(x) < \deg f(x),
    \end{align*}
    矛盾! 故 $-q(x) = 0$. 这样, $r(x) = 0$.
\end{pf}

\begin{proposition}
    设 $f(x) \in D[x]$, 且 $f(x) \neq 0$. 若 $D$ 上 $x$ 的 $4$ 个整式 $q_1 (x)$, $r_1 (x)$, $q_2 (x)$, $r_2 (x)$ 适合
    \begin{align*}
         & q_1 (x) f(x) + r_1 (x) = q_2 (x) f(x) + r_2 (x),          \\
         & \deg r_1 (x) < \deg f(x), \quad \deg r_2 (x) < \deg f(x),
    \end{align*}
    则必有
    \begin{align*}
        q_1 (x) = q_2 (x), \quad r_1 (x) = r_2 (x).
    \end{align*}
\end{proposition}

\begin{pf}
    记
    \begin{align*}
        Q(x) = q_1 (x) - q_2 (x), \quad R(x) = r_1 (x) - r_2 (x).
    \end{align*}
    题设条件即
    \begin{align*}
        (q_1 (x) - q_2 (x)) f(x) + (r_1 (x) - r_2 (x)) = 0,
    \end{align*}
    也就是
    \begin{align*}
        Q(x) f(x) + R(x) = 0.
    \end{align*}
    注意到
    \begin{align*}
        \deg R(x)
        =    {} & \deg {(r_1 (x) - r_2 (x))}                \\
        \leq {} & \max \{\, \deg r_1 (x), \deg r_2 (x) \,\} \\
        <    {} & \deg f(x).
    \end{align*}
    根据上个命题, $Q(x) = R(x) = 0$. 所以
    \begin{align*}
         & q_1 (x) = q_2 (x), \quad r_1 (x) = r_2 (x). \qedhere
    \end{align*}
\end{pf}

这样, 我们得到了这个命题:

\begin{proposition}
    设
    \begin{align*}
        f(x) = a_n x^n + a_{n-1} x^{n-1} + \cdots + a_0 \in D[x],
    \end{align*}
    且 $a_n$ 是 $D$ 的单位. 对任意 $g(x) \in D[x]$, 存在唯一的 $q(x), r(x) \in D[x]$ 使
    \begin{align*}
        g(x) = q(x) f(x) + r(x), \quad \deg r(x) < n.
    \end{align*}
    一般称其为带馀除法: $q(x)$ 就是商; $r(x)$ 就是馀式. 并且, 当 $f(x)$ 的次不高于 $g(x)$ 的次时, $f(x)$, $g(x)$, $q(x)$ 间还有如下的次关系:
    \begin{align*}
        \deg g(x) = \deg {(g(x) - r(x))} = \deg q(x) + \deg f(x).
    \end{align*}
\end{proposition}

最后, 我们指出一个有用的事实.

设 $D$, $E$ 是整环, 且 $D \subset E$ (也就是说, $D$ 是 $E$ 的子整环). 设 $f(x)$, $g(x) \in D[x]$, 且 $f(x) \neq 0$. 设 $D$ 上的整式 $q_1 (x)$, $r_1 (x)$ 适合
\begin{align*}
    g(x) = q_1 (x) f(x) + r_1 (x), \quad \deg r_1 (x) < \deg f(x).
\end{align*}
因为 $D$ 的元都是 $E$ 的元, 故 $f(x)$, $g(x)$ 当然可认为是 $E$ 上的整式. 设 $E$ 上的整式 $q_2 (x)$, $r_2 (x)$ 适合
\begin{align*}
    g(x) = q_2 (x) f(x) + r_2 (x), \quad \deg r_2 (x) < \deg f(x).
\end{align*}
我们说, $q_1 (x) = q_2 (x)$, 且 $r_1 (x) = r_2 (x)$. 为什么呢? 因为 $D$ 的元都是 $E$ 的元, 故 $q_1 (x)$, $r_1 (x)$ 当然可认为是 $E$ 上的整式. 这样, 根据前面的命题, 必有
\begin{align*}
    q_2 (x) = q_1 (x), \quad r_2 (x) = r_1 (x).
\end{align*}
由此可见: $q_2 (x)$, $r_2 (x) \in D[x]$. 这有什么用处呢?

在前面的例里, 我们知道, 不存在 $q(x)$, $r(x) \in \ZZ[x]$, 使
\begin{align*}
    1 + 3x + x^2 = q(x) \cdot 2x + r(x), \quad \deg r(x) < 1.
\end{align*}
当初, 我们根据整式的相等的定义, 再根据奇偶性, 得出了这个判断. 现在我们可以简便地作出推理. 用反证法. 假设这样的 $q(x)$, $r(x) \in \ZZ[x]$ 是存在的, 也就是
\begin{align*}
    1 + 3x + x^2 = q(x) \cdot 2x + r(x), \quad \deg r(x) < 1.
\end{align*}
我们考虑 $\QQ[x]$. $\ZZ$, $\QQ$ 都是整环, 且 $\ZZ \subset \QQ$. $1+3x+x^2$ 与 $2x$ 当然是 $\QQ$ 上的整式. 我们不难算出
\begin{align*}
    1 + 3x + x^2 = {\underbrace{\left( \frac{3}{2} + \frac{1}{2}x \right)}_{v(x)}} \cdot 2x + \underbrace{1}_{u(x)}, \quad \deg u(x) = 0 < 1.
\end{align*}
由此可见, $v(x) = q(x)$, 且 $r(x) = u(x)$. 但是, $v(x) \notin \ZZ[x]$. 矛盾! 所以, 这样的 $q(x)$, $r(x)$ 不存在!

不过, 对 $1 + 4x + 2x^2$ 与 $2x$ 而言, 我们有
\begin{align*}
    1 + 4x + 2x^2 = (2 + x) \cdot 2x + 1, \quad \deg 1 < 1.
\end{align*}
我们知道, 若视 $1 + 4x + 2x^2$ 与 $2x$ 是 $\ZZ$ 上的整式, 则因为 $2x$ 的首项系数不是 $\ZZ$ 的单位, 我们无法直接用 ($\ZZ[x]$ 的) 带馀除法算 $2x$ 除 $1 + 4x + 2x^2$. 若考虑 $\QQ$ 与 $\QQ[x]$, 我们可以算出上面的式. 很凑巧地, $2 + x$ 与 $1$ 都是 $\ZZ$ 上的整式, 故, 具体地说, 我们可在 $\ZZ[x]$ 算 $2x$ 除 $1 + 4x + 2x^2$; 作为对比, 我们不可在 $\ZZ[x]$ 算 $2x$ 除 $1 + 3x + x^2$.

不正式地说, 若 $D$, $E$ 是整环, 且 $D \subset E$, 则 $E$ 是 $D$ 的 ``扩大'' \term{extension}. 由此, 我们有: 整式的带馀除法不因系数的范围变大而改变.
