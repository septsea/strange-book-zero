\subsection*{\IntroductionToMethodOfFluxionsOnPolynomials}
\addcontentsline{toc}{subsection}{\IntroductionToMethodOfFluxionsOnPolynomials}
\markright{\IntroductionToMethodOfFluxionsOnPolynomials}

本节讨论整式的流数术 \term{method of fluxions} 初步\myFN{学过 $1$ 元函数流数术的读者可能会觉得本节废话连篇. 不过, 为照顾不熟悉三角不等式及相关知识的读者, 作者也没什么更好的写作思路了.}. 这也是本文的最终节.

How time flies! 一开始, 我们在 ``\Prerequisites'' 给读者介绍预备知识. 然后, 我们给读者介绍了系数为整环的元的整式. 当初, 整式还是有点抽象的. 我们利用带馀除法推出了几个很重要的命题, 并指出: 当整式的系数为 $\mathbb{F}$ 的元时, 整式与中学学的整式 (函数) 没有根本上的区别. 我们在 ``\Interpolation'' 节开始介绍整式的应用. 后面的 ``\SummationFormulae'' 节告诉读者一种方便的求和法. 在上节, 我们捡起很久未出场的流数, 并把它重讲了一遍. 我们利用高级流数推广了二项展开, 得到了 Taylor 公式, 并用它重证整式的积的流数规则.

之前, 流数都是形式的——有点 ``空降'' 的味道. 现在, 我们要用 Taylor 公式给流数一种含义. 如果读者知道一点分析学 \term{calculus}, 读者将不会对本节感到特别陌生; 如果读者没有学过分析学, 无妨视本节为 ``入分作''.

我们在 ``\PolynomialsOverF'' 节说过, 我们不再讨论抽象的整环或系数为整环的元的整式, 而是讨论 $\mathbb{F}$ 与 $\mathbb{F}[x]$. 现在, 我们再具体一点——讨论老朋友 $\mathbb{R}$ 与 $\mathbb{R}[x]$.

不过, 我们需要承认一个事实是对的. 为什么说 ``承认'' 呢? 因为它的证明需要超出本文的知识很多很多的工具. 但是, 它并不是什么 ``牵强附会'' 的命题. 是什么命题呢? 我们不必现在就说; 我们在用到它的时候再说.

我们先带读者熟悉实数.

读者也许还记得实数 $a$ 的绝对值:
\begin{align*}
    |a| = \begin{cases}
        a,  & \quad a \geq 0; \\
        -a, & \quad a < 0.
    \end{cases}
\end{align*}

请读者尝试自行证明下面四个命题. 当然, 熟悉这四个命题的读者可以不证. 我们把它们写在这里供读者参考.

\begin{proposition}
    设 $a \in \mathbb{R}$. 则 $|a| \geq 0$.
\end{proposition}

\begin{proposition}
    设 $a \in \mathbb{R}$. 则
    \begin{align*}
        a = \begin{cases}
            |a|,  & \quad a \geq 0; \\
            -|a|, & \quad a < 0.
        \end{cases}
    \end{align*}
\end{proposition}

\begin{proposition}
    设 $a \in \mathbb{R}$. 则 $-|a| \leq a \leq |a|$.
\end{proposition}

\begin{proposition}
    设 $a \in \mathbb{R}$, 且 $b > 0$. 则
    \begin{align*}
         & |a| \leq b \iff -b \leq a \leq b; \\
         & |a| < b \iff -b < a < b.
    \end{align*}
\end{proposition}

读者也许还记得平方的性质:
\begin{align*}
    a^2 = (-a)^2.
\end{align*}
并且, 若 $a$, $b$ 都是非负数, 则
\begin{align*}
    a = b \iff a^2 = b^2.
\end{align*}

利用这些性质, 我们有

\begin{proposition}
    设 $a_0$, $a_1$, $\cdots$, $a_{n-1} \in \mathbb{R}$. 则
    \begin{align*}
        |a_0 a_1 \cdots a_{n-1}| = |a_0| \cdot |a_1| \cdot \cdots \cdot |a_{n-1}|.
    \end{align*}
    特别地, 若 $a_0 = a_1 = \cdots = a_{n-1} = a$, 则
    \begin{align*}
        |a^n| = |a|^n.
    \end{align*}
\end{proposition}

\begin{pf}
    请读者尝试自行证明此命题. 不过, 我们愿意提示读者: (i) 等式的左右二侧都是非负的; (ii) 等式的左右二侧的平方是一样的.
\end{pf}

下面是一个十分重要的不等式:

\begin{proposition}
    设 $a_0$, $a_1$, $\cdots$, $a_{n-1} \in \mathbb{R}$. 则
    \begin{align*}
        |a_0 + a_1 + \cdots + a_{n-1}| \leq |a_0| + |a_1| + \cdots + |a_{n-1}|.
    \end{align*}
    这个不等式的一个名字是三角不等式 \term{triangle inequality}.
\end{proposition}

\begin{pf}
    易知
    \begin{align*}
         & -|a_0| \leq a_0 \leq |a_0|,             \\
         & -|a_1| \leq a_1 \leq |a_1|,             \\
         & \cdots \cdots \cdots \cdots
        \cdots \cdots \cdots \cdots
        \cdots \cdots \cdots \cdots,               \\
         & -|a_{n-1}| \leq a_{n-1} \leq |a_{n-1}|.
    \end{align*}
    记
    \begin{align*}
        b = |a_0| + |a_1| + \cdots + |a_{n-1}|.
    \end{align*}
    易知 $b \geq 0$, 且
    \begin{align*}
        -b \leq a_0 + a_1 + \cdots + a_{n-1} \leq b.
    \end{align*}
    所以
    \begin{align*}
         & |a_0 + a_1 + \cdots + a_{n-1}| \leq b = |a_0| + |a_1| + \cdots + |a_{n-1}|. \qedhere
    \end{align*}
\end{pf}

\begin{definition}
    设 $a$, $b$ 是实数, 且 $a < b$. 称
    \begin{align*}
        [a,b] = \{\, t \in \mathbb{R} \mid a \leq t \leq b \,\}
    \end{align*}
    为闭区间 \term{closed interval}; 称
    \begin{align*}
        (a,b) = \{\, t \in \mathbb{R} \mid a < t < b \,\}
    \end{align*}
    为开区间 \term{open interval}. 类似地, 有半闭区间 \term{half-closed interval}:
    \begin{align*}
         & [a,b) = \{\, t \in \mathbb{R} \mid a \leq t < b \,\}, \\
         & (a,b] = \{\, t \in \mathbb{R} \mid a < t \leq b \,\}.
    \end{align*}
    $[a,b]$, $(a,b)$, $[a,b)$, $(a,b]$ 都是有限区间 \term{finite interval}. 此名暗示着, 还有无限区间 \term{infinite interval}:
    \begin{align*}
         & (-\infty, a) = \{\, t \in \mathbb{R} \mid t < a \,\},    \\
         & (-\infty, a] = \{\, t \in \mathbb{R} \mid t \leq a \,\}, \\
         & (b, +\infty) = \{\, t \in \mathbb{R} \mid t > b \,\},    \\
         & [b, +\infty) = \{\, t \in \mathbb{R} \mid t \geq b \,\}, \\
         & (-\infty, +\infty) = \mathbb{R}.
    \end{align*}
    有限区间与无限区间都是区间 \term{interval}.
\end{definition}

\begin{proposition}
    设 $a$, $b$ 是实数, 且 $a < b$. 若 $r > |a|$ 且 $r > |b|$, 则 $[a,b]$, $(a,b)$, $[a,b)$, $(a,b]$ 都是 $[-r, r]$ 的真子集.
\end{proposition}

\begin{pf}
    请读者尝试自行证明此命题.
\end{pf}

% 下面的定义指出, 利用 $S$ 的非空子集, 点可被分为三类.

% \begin{definition}
%     设 $S$ 是 $\mathbb{R}$ 的非空子集. 设 $a \in \mathbb{R}$. 若存在正数 $\delta_0$ 使
%     \begin{align*}
%         |t - a| < \delta_0 \implies t \in S,
%     \end{align*}
%     则说 $a$ 是 $S$ 的内点 \term{interior point}. 显然, $S$ 的内点都是 $S$ 的元.

%     若存在正数 $\delta_1$ 使
%     \begin{align*}
%         |t - a| < \delta_1 \implies t \notin S,
%     \end{align*}
%     则说 $a$ 是 $S$ 的外点 \term{exterior point}. 显然, $S$ 的外点一定不是 $S$ 的元.

%     若 $a$ 既不是内点, 也不是外点, 就说 $a$ 是界点 \term{boundary point}.
% \end{definition}

% \begin{example}
%     考虑
%     \begin{align*}
%         S = (0, 4] = \{\, t \in \mathbb{R} \mid 0 < t \leq 4 \,\}.
%     \end{align*}

%     $2$ 是 $S$ 的内点. 这是因为, 存在正数 $\delta_0 = 1$, 使
%     \begin{align*}
%         |t - 2| < \delta_0 \implies t \in S.
%     \end{align*}

%     $666$ 是 $S$ 的外点. 这是因为, 存在正数 $\delta_1 = 233$, 使
%     \begin{align*}
%         |t - 666| < \delta_1 \implies t \notin S.
%     \end{align*}

%     $0$ 是 $S$ 的界点. 反证法. 若 $0$ 是 $S$ 的内点, 则应有某正数 $\delta_2$ 使
%     \begin{align*}
%         |t - 0| < \delta_2 \implies t \in S.
%     \end{align*}
%     可是, 这是矛盾: 当 $t = -\frac{\delta_2}{2}$ 时, 此 $t$ 适合 $|t - 0| < \delta_2$ 却不适合 $t \in S$. 所以 $0$ 一定不是 $S$ 的内点. 若 $t$ 是 $S$ 的外点, 则应有正数 $\delta_3$ 使
%     \begin{align*}
%         |t - 0| < \delta_3 \implies t \notin S.
%     \end{align*}
%     可是, 这仍是矛盾. 若 $\frac12 < \delta_3$, 取 $t = \frac12$. 此 $t$ 适合 $|t - 0| < \delta_3$ 与 $t \in S$. 若 $\frac12 \geq \delta_3$, 取 $t = \frac{\delta_3}{2}$. 此 $t$ 也适合 $|t - 0| < \delta_3$ 与 $t \in S$. 所以 $0$ 也不是 $S$ 的外点. 这样, $0$ 必为 $S$ 的界点.

%     类似地, 读者可以用反证法证明 $4$ 也是 $S$ 的界点. 注意到界点既可以是 $S$ 的元, 也可以不是 $S$ 的元.
% \end{example}

% \begin{remark}
%     % Something wrong with the formatter
%     设 $a$, $b$ 是实数, 且 $a < b$. 我们有如下命题:

%     (i) $a$ 是 $[a,b]$, $(a,b)$, $[a,b)$, $(a,b]$, $(-\infty,a)$, $(-\infty,a]$ 的界点;

%     (ii) $b$ 是 $[a,b]$, $(a,b)$, $[a,b)$, $(a,b]$, $(b,+\infty)$, $[b,+\infty)$ 的界点;

%                     (iii) 若 $t \in (a,b)$, 则 $t$ 是 $[a,b]$, $(a,b)$, $[a,b)$, $(a,b]$ 的内点;

%                     (iv) 若 $t \notin [a,b]$, 则 $t$ 是 $[a,b]$, $(a,b)$, $[a,b)$, $(a,b]$ 的外点;

%                     (v) 若 $t \in (-\infty,a)$, 则 $t$ 是 $(-\infty,a)$, $(-\infty,a]$ 的内点, 且 $t$ 是 $(a,+\infty)$, $[a,+\infty)$ 的外点;

%     (vi) 若 $t \in (b,+\infty)$, 则 $t$ 是 $(b,+\infty)$, $[b,+\infty)$ 的内点, 且 $t$ 是 $(-\infty,b)$, $(-\infty,b]$ 的外点;

%     (vii) 若 $t \in \mathbb{R}$, 则 $t$ 是 $\mathbb{R}$ 的内点;

%     (viii) 若 $r > |a|$ 且 $r > |b|$, 则 $[a,b]$, $(a,b)$, $[a,b)$, $(a,b]$ 都是 $[-r, r]$ 的真子集.
% \end{remark}

下面建立一些关于整式的不等式.

\begin{proposition}
    设 $n$ 是非负整数. 设 $a_0, a_1$, $\cdots$, $a_n \in \mathbb{R}$. 任取正数 $r$, 必存在正数 $M$ 使
    \begin{align*}
        |u| \leq r \implies |a_0 + a_1 u + \cdots + a_n u^n| \leq M.
    \end{align*}
\end{proposition}

\begin{pf}
    设正数 $C$ 不低于 $|a_0|$, $|a_1|$, $\cdots$, $|a_n|$ 的任意一个. 则 $|u| \leq r$ 时,
    \begin{align*}
                & |a_0 + a_1 u + \cdots + a_n u^n|         \\
        \leq {} & |a_0| + |a_1 u| + \cdots + |a_n u^n|     \\
        = {}    & |a_0| + |a_1| |u| + \cdots + |a_n| |u|^n \\
        \leq {} & C (1 + |u| + \cdots + |u|^n)             \\
        \leq {} & C (1 + r + \cdots + r^n).
    \end{align*}
    记
    \begin{align*}
        M = C (1 + r + \cdots + r^n) > 0.
    \end{align*}
    由此,
    \begin{align*}
         & |u| \leq r \implies |a_0 + a_1 u \cdots + a_n u^n| \leq M. \qedhere
    \end{align*}
\end{pf}

\begin{proposition}
    设 $I$ 是有限区间. 设 $f(x) \in \mathbb{R}[x]$. 存在正数 $M$ 使
    \begin{align*}
        u \in I \implies |f(u)| \leq M.
    \end{align*}
    用文字描述这句话, 就是: 整式 (函数) 在任意有限区间上都是有界的 \term{to be bounded}.
\end{proposition}

\begin{pf}
    取 $r > 0$ 使 $I \subset [-r, r]$. 设
    \begin{align*}
        f(x) = a_0 + a_1 x + \cdots + a_n x^n \in \mathbb{R}[x].
    \end{align*}
    根据上个命题, 存在正数 $M$ 使
    \begin{align*}
        |u| \leq r \implies |f(u)| \leq M.
    \end{align*}
    所以
    \begin{align*}
         & u \in I \implies |f(u)| \leq M. \qedhere
    \end{align*}
\end{pf}

我们有时称 $\mathbb{R}$ 的元为点.

\begin{proposition}
    设 $f(x) \in \mathbb{R}[x]$. 设 $t_0 \in \mathbb{R}$. 任取 $\varepsilon > 0$, 必有 $\delta > 0$, 使
    \begin{align*}
        |t - t_0| < \delta \implies |f(t) - f(t_0)| < \varepsilon.
    \end{align*}
    通俗地说, 当点 $t$ 与点 $t_0$ 足够近时, 整式在二点的值可任意接近.
\end{proposition}

\begin{pf}
    若 $f(x) = c$, $c \in \mathbb{R}$, 则
    \begin{align*}
        |f(t) - f(t_0)| = |c - c| = 0 < \varepsilon
    \end{align*}
    总是成立的. 下设 $f(x)$ 的次高于 $0$.

    根据 ``\ValueOfAPolynomialAtAPoint'' 节的结论, 存在整式 $q(x)$ 使
    \begin{align*}
        f(x) = (x - t_0) q(x) + f(t_0).
    \end{align*}
    所以
    \begin{align*}
        |f(t) - f(t_0)| = |q(t)| |t - t_0|.
    \end{align*}
    设 $I = [t_0 - 1, t_0 + 1]$. 不难看出,
    \begin{align*}
        I
        = {} & \{\, t \in \mathbb{R} \mid t_0 - 1 \leq t \leq t_0 + 1 \,\} \\
        = {} & \{\, t \in \mathbb{R} \mid -1 \leq t - t_0 \leq 1 \,\}      \\
        = {} & \{\, t \in \mathbb{R} \mid |t - t_0| \leq 1 \,\}.
    \end{align*}
    利用上个命题, 存在 $M > 0$ 使
    \begin{align*}
        |t - t_0| \leq 1 \implies |q(t)| \leq M.
    \end{align*}
    这样,
    \begin{align*}
        |t - t_0| \leq 1 \implies |f(t) - f(t_0)| \leq M |t - t_0|.
    \end{align*}
    任取 $\varepsilon > 0$. 取一个既低于 $1$ 也低于 $\frac{\varepsilon}{M}$ 的正数 $\delta$. 这样, $|t - t_0| < \delta$ 时, 必有
    \begin{align*}
         & |f(t) - f(t_0)| \leq M |t - t_0| < M \cdot \frac{\varepsilon}{M} = \varepsilon. \qedhere
    \end{align*}
\end{pf}

\begin{proposition}
    设 $t_0 \in \mathbb{R}$. 设 $\ell$ 是非负整数. 设
    \begin{align*}
         & f(x) = a_\ell (x - t_0)^\ell + a_{\ell+1} (x - t_0)^{\ell+1}
        + \cdots + a_n (x - t_0)^n,                                     \\
         & g(x) = a_\ell (x - t_0)^\ell,
    \end{align*}
    且 $a_\ell \neq 0$. 则存在 $\delta > 0$, 使 $0 < |t - t_0| < \delta$ 时, 必有 $f(t)$ 与 $g(t)$ 同号.
\end{proposition}

\begin{pf}
    我们说, 二个不为 $0$ 的数 $a$, $b$ 同号, 相当于 $ab > 0$. 记
    \begin{align*}
        p(x) = \frac{1}{a_\ell} \sum_{j = \ell + 1}^{n} a_j (x - t_0)^{j - (\ell + 1)}.
    \end{align*}
    则
    \begin{align*}
        f(x)
        = {} & a_\ell (x - t_0)^\ell + a_\ell (x - t_0)^{\ell} (x - t_0) p(x) \\
        = {} & a_\ell (x - t_0)^\ell (1 + (x - t_0) p(x))                     \\
        = {} & g(x) (1 + (x - t_0) p(x)).
    \end{align*}
    所以
    \begin{align*}
        f(x) g(x) = (g(x))^2 (1 + (x - t_0) p(x)).
    \end{align*}
    记 $q(x) = 1 + (x - t_0) p(x)$. 取 $\varepsilon = \frac12$. 由上个命题, 存在 $\delta > 0$ 使
    \begin{align*}
        |t - t_0| < \delta \implies |q(t) - 1| = |q(t) - q(t_0)| < \varepsilon = \frac12.
    \end{align*}
    所以
    \begin{align*}
        |t - t_0| < \delta \implies q(t) - 1 > -\frac12.
    \end{align*}
    所以
    \begin{align*}
        0 < |t - t_0| < \delta \implies q(t) > \frac12.
    \end{align*}
    因为
    \begin{align*}
        0 < |t - t_0| \implies (g(t))^2 > 0,
    \end{align*}
    故
    \begin{align*}
         & 0 < |t - t_0| < \delta \implies f(t) g(t) > \frac12 (g(t))^2 > 0. \qedhere
    \end{align*}
\end{pf}

下面讨论流数与变率的关系.

\begin{definition}
    设 $a$, $b$ 是实数, 且 $a < b$. 设 $f(x) \in \mathbb{R}[x]$. 我们说整式 $f(x)$ 在区间 $[a, b]$ 的平均变率 \term{average rate of change} 是
    \begin{align*}
        \frac{f(b) - f(a)}{b - a}.
    \end{align*}
\end{definition}

\begin{example}
    设 $f(x) = a_0 + a_1 x$. 则
    \begin{align*}
        \frac{f(b) - f(a)}{b - a} = \frac{(a_0 + a_1 b) - (a_0 + a_1 a)}{b - a} = a_1.
    \end{align*}
    可以看到, $f(x)$ 在 $[a, b]$ 的平均变率与具体区间无关. 反过来, 若整式 $f(x)$ 适合: 任取 $c$, $d \in \mathbb{R}$, $c < d$, 都有
    \begin{align*}
        \frac{f(d) - f(c)}{d - c}
    \end{align*}
    为常数 $A$, 则
    \begin{align*}
        d > 0 \implies f(d) = f(0) + Ad.
    \end{align*}
    作整式
    \begin{align*}
        E(x) = f(0) + Ax - f(x),
    \end{align*}
    则任取 $d > 0$, 都有 $E(d) = 0$. 这样, $E(x)$ 是零整式, 即
    \begin{align*}
        f(x) = f(0) + Ax.
    \end{align*}
\end{example}

从上面的例可知: 次低于 $2$ 的整式 $f(x) = a_0 + a_1 x$ 在任意闭区间 $[a, b]$ 的平均变率都是常数. 我们说, 任取 $t \in \mathbb{R}$, $f(x)$ 在点 $t$ 的变率 \term{rate of change} 是 $a_1$.

不过, 次高于 $1$ 的整式有着不一样的平均变率.

\begin{example}
    设 $f(x) = x^2 + x - 1$. 取 $a = 0$, $b = 1$, $c = 2$. 易知
    \begin{align*}
        f(a) = -1, \quad f(b) = 1, \quad f(c) = 5.
    \end{align*}
    所以, $f(x)$ 在 $[a, b]$ 的平均变率是
    \begin{align*}
        \frac{f(b) - f(a)}{b - a} = \frac{1 - (-1)}{1 - 0} = 2.
    \end{align*}
    而 $f(x)$ 在 $[b, c]$ 的平均变率是
    \begin{align*}
        \frac{f(c) - f(b)}{c - b} = \frac{5 - 1}{2 - 1} = 4.
    \end{align*}
    顺便一提, $f(x)$ 在 $[a, c]$ 的平均变率是
    \begin{align*}
        \frac{f(c) - f(a)}{c - a} = \frac{5 - (-1)}{2 - 0} = 3.
    \end{align*}
\end{example}

虽然我们现在还不知道任意整式 $f(x)$ 在点 $t$ 的变率, 但我们还是能作出一些定性判断的.

\begin{example}
    设 $f_1 (x)$, $f_2 (x)$ 是整式. 设想 $P_1$, $P_2$ 二人同时同地在一条笔直的路上骑车单向前进. 设 $f_1 (t)$, $f_2 (t)$ 分别表示 $t$ s 后 $P_1$, $P_2$ 距始点的距离. 所以, $f_1 (x)$ (或 $f_2 (x)$) 在 $[a, b]$ 的平均变率代表 $a$ s 至 $b$ s 这一段的平均速率. 如果 $f_1 (x)$ (或 $f_2 (x)$) 的次为 $1$, 则平均变率 $A$ 不变, 也就是我们常说的 ``匀速直线运动''. 我们也说, 任取 $a > 0$, $P_1$ (或 $P_2$) 在 $a$ s 的速率都是 $A$.

    设 $t_0$ s 后, $P_1$ 从后赶上 $P_2$ 并超越之. 这相当于, 存在 $\delta > 0$ 使
    \begin{align*}
        t_0 - \delta < t < t_0 & \implies f_1 (t) < f_2 (t), \\
        t = t_0                & \implies f_1 (t) = f_2 (t), \\
        t_0 < t < t_0 + \delta & \implies f_1 (t) > f_2 (t).
    \end{align*}
    经验告诉我们, $P_1$ 在 $t_0$ s 的速率一定不低于 $P_2$ 在 $t_0$ s 的速率. 若不然, $P_1$ 是不可能赶上 $P_2$ 后并超越之, 是不是?
\end{example}

抽象上面的例, 我们可得到

\begin{proposition}
    虽然我们还不能准确地定义变率, 但生活经验告诉我们, 变率应适合如下特性:

    设 $f (x)$, $g (x) \in \mathbb{R}[x]$. 若 $f (t_0) = g (t_0)$, 且存在 $\delta > 0$ 使
    \begin{align*}
        t_0 - \delta < t < t_0 & \implies f (t) < g (t), \\
        t_0 < t < t_0 + \delta & \implies f (t) > g (t).
    \end{align*}
    我们说, $f (x)$ 在点 $t_0$ 的变率不低于 $g (x)$ 在点 $t_0$ 的变率.
\end{proposition}

为充分利用此特性, 我们特化之.

设 $A \in \mathbb{R}$. 取 $g(x) = f(t_0) + A (x - t_0)$, 则 $f(t_0) = g(t_0)$. 因为 $g(x)$ 在 $t_0$ 的变率是 $A$, 故

\begin{proposition}
    变率应适合如下特性:

    设 $f (x) \in \mathbb{R}[x]$. 若存在 $\delta_0 > 0$ 使
    \begin{align*}
        t_0 - \delta_0 < t < t_0 & \implies f (t) < f(t_0) + A (t - t_0), \\
        t_0 < t < t_0 + \delta_0 & \implies f (t) > f(t_0) + A (t - t_0).
    \end{align*}
    我们说, $f (x)$ 在点 $t_0$ 的变率不低于 $A$.

    若存在 $\delta_1 > 0$ 使
    \begin{align*}
        t_0 - \delta_1 < t < t_0 & \implies f (t) > f(t_0) + A (t - t_0), \\
        t_0 < t < t_0 + \delta_1 & \implies f (t) < f(t_0) + A (t - t_0).
    \end{align*}
    我们说, $f (x)$ 在点 $t_0$ 的变率不高于 $A$.
\end{proposition}

若 $t_0 - \delta < t < t_0$, 则 $t_0 - t > 0$. 所以
\begin{align*}
    f (t) < f(t_0) + A (t - t_0)
    \iff {} & f (t) < f(t_0) - A (t_0 - t)       \\
    \iff {} & A (t_0 - t) < f(t_0) - f(t)        \\
    \iff {} & A < \frac{f(t_0) - f(t)}{t_0 - t}  \\
    \iff {} & \frac{f(t) - f(t_0)}{t - t_0} > A.
\end{align*}
若 $t_0 < t < t_0 + \delta$, 则 $t - t_0 > 0$. 所以
\begin{align*}
    f (t) > f(t_0) + A (t - t_0)
    \iff {} & f(t) - f(t_0) > A (t - t_0)        \\
    \iff {} & \frac{f(t) - f(t_0)}{t - t_0} > A.
\end{align*}
$t_0 - \delta < t < t_0$ 与 $t_0 < t < t_0 + \delta$ 相当于
\begin{align*}
    0 < |t - t_0| < \delta.
\end{align*}
这样, 我们有

\begin{proposition}
    变率应适合如下特性:

    设 $f (x) \in \mathbb{R}[x]$. 若存在 $\delta > 0$ 使
    \begin{align*}
        0 < |t - t_0| < \delta & \implies \frac{f(t) - f(t_0)}{t - t_0} > A,
    \end{align*}
    我们说, $f (x)$ 在点 $t_0$ 的变率不低于 $A$.
\end{proposition}

同理可得

\begin{proposition}
    变率应适合如下特性:

    设 $f (x) \in \mathbb{R}[x]$. 若存在 $\delta > 0$ 使
    \begin{align*}
        0 < |t - t_0| < \delta & \implies \frac{f(t) - f(t_0)}{t - t_0} < A,
    \end{align*}
    我们说, $f (x)$ 在点 $t_0$ 的变率不高于 $A$.
\end{proposition}

现在, 让我们揭秘变率.

设 $t_0 \in \mathbb{R}$. 设 $f(x)$ 的次不高于 $n$. 根据 Taylor 公式,
\begin{align*}
    f(x) = f(t_0) + Df(t_0) (x - t_0) + \sum_{j = 2}^{n} \frac{D^j f(t_0)}{j!} (x - t_0)^j.
\end{align*}
所以, $t \neq t_0$ 时,
\begin{align*}
    \frac{f(t) - f(t_0)}{t - t_0} = Df(t_0) + \sum_{j = 2}^{n} \frac{D^j f(t_0)}{j!} (t - t_0)^{j - 1}.
\end{align*}
设 $A \in \mathbb{R}$. 则
\begin{align*}
    \frac{f(t) - f(t_0)}{t - t_0} - A = (Df(t_0) - A) + \sum_{j = 2}^{n} \frac{D^j f(t_0)}{j!} (t - t_0)^{j - 1}.
\end{align*}
记
\begin{align*}
    q(x) = (Df(t_0) - A) + \sum_{j = 2}^{n} \frac{D^j f(t_0)}{j!} (x - t_0)^{j - 1}.
\end{align*}
若 $Df(t_0) - A \neq 0$, 则存在 $\delta > 0$, 使 $0 < |t - t_0| < \delta$ 时, $q(t)$ 与 $Df(t_0) - A$ 同号.

设 $f(x)$ 在点 $t_0$ 的变率为 $r$. 任取 $A < Df(t_0)$, 必有
\begin{align*}
    0 < |t - t_0| < \delta \implies \frac{f(t) - f(t_0)}{t - t_0} > A \implies r \geq A.
\end{align*}
任取 $A > Df(t_0)$, 必有
\begin{align*}
    0 < |t - t_0| < \delta \implies \frac{f(t) - f(t_0)}{t - t_0} < A \implies r \leq A.
\end{align*}

我们证明: $r = Df(t_0)$. 反证法. 若 $r < Df(t_0)$, 作
\begin{align*}
    A_0 = \frac{Df(t_0) + r}{2}.
\end{align*}
不难看出
\begin{align*}
    A_0 < \frac{Df(t_0) + Df(t_0)}{2} = Df(t_0),
\end{align*}
故
\begin{align*}
    r \geq A_0 = \frac{Df(t_0) + r}{2} \implies r \geq Df(t_0),
\end{align*}
矛盾! 若 $r > Df(t_0)$, 作
\begin{align*}
    A_1 = \frac{Df(t_0) + r}{2}.
\end{align*}
不难看出
\begin{align*}
    A_1 > \frac{Df(t_0) + Df(t_0)}{2} = Df(t_0),
\end{align*}
故
\begin{align*}
    r \leq A_1 = \frac{Df(t_0) + r}{2} \implies r \leq Df(t_0),
\end{align*}
矛盾! 所以, $r$ 必为 $Df(t_0)$.

我们得到了本节最重要的命题:

\begin{proposition}
    设 $f (x) \in \mathbb{R}[x]$. 设 $t_0 \in \mathbb{R}$. 则 $Df(t_0)$ 是 $f(x)$ 在点 $t_0$ 的变率.
\end{proposition}

至此, 我们找到了流数的一种含义, 本节的任务终了. 我们就讨论到这里吧. 再见, 读者朋友!
