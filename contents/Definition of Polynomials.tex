\subsection*{\DefinitionOfPolynomials}
\addcontentsline{toc}{subsection}{\DefinitionOfPolynomials}
\markright{\DefinitionOfPolynomials}

现在开始介绍多项式.

\begin{definition}
    设 $D$ 是整环. 设 $x$ 是不在 $D$ 里的任意一个文字. 形如
    \begin{align*}
        f(x) = a_0 x^0 + a_1 x^1 + \cdots + a_n x^n \quad (n \in \NN,\ a_0,a_1,\cdots,a_n \in D,\ a_n \neq 0)
    \end{align*}
    的表达式称为 $D$ 上 $x$ 的一个多项式 \term{polynomial in $x$ over $D$}. $n$ 称为其次 \term{degree}, $a_i$ 称为其 $i$ 次系数 \term{the $i^\text{th}$ coefficient}, $a_i x^i$ 称为其 $i$ 次项 \term{the $i^\text{th}$ term}. $f(x)$ 的次可写为 $\deg f(x)$.

    若二个多项式的次与各同次系数均相等, 则二者相等.

    多项式的系数为 $0$ 的项可以不写.

    约定 $0 \in D$ 也是多项式, 称为零多项式. 零多项式的次是 $-\infty$. 任取整数 $m$, 约定
    \begin{align*}
         & -\infty = -\infty, \quad -\infty < m,                        \\
         & -\infty + m = m + (-\infty) = -\infty + (-\infty) = -\infty.
    \end{align*}
    当然, 还约定, 零多项式只跟自己相等. 换句话说,
    \begin{align*}
        a_0 x^0 + a_1 x^1 + \cdots + a_n x^n = 0
    \end{align*}
    的一个必要与充分条件是
    \begin{align*}
        a_0 = a_1 = \cdots = a_n = 0.
    \end{align*}

    $D$ 上 $x$ 的所有多项式作成的集是 $D[x]$:
    \begin{align*}
        D[x] = \{\, a_0 x^0 + a_1 x^1 + \cdots + a_n x^n \mid n \in \NN,\ a_0,a_1,\cdots,a_n \in D \,\}.
    \end{align*}

    文字 $x$ 只是一个符号, 它与 $D$ 的元的和与积都是形式的. 我们说, $x$ 是不定元 \term{indeterminate}.
\end{definition}

\begin{example}
    $0y^0 + 1y^1 + (-1)y^2 + 0y^3 + (-7)y^4 \in \ZZ[y]$ 是一个 $4$ 次多项式. 顺便一提, 一般把 $y^1$ 写为 $y$. 这个多项式的一个更普通的写法是
    \begin{align*}
        y - y^2 - 7y^4.
    \end{align*}

    也许 $y^0$ 看起来有些奇怪. 如上所言, 这只是一个形式上的表达式. 我们之后再处理这个小细节.
\end{example}

\begin{example}
    $z^0 + z + z^{\frac{3}{2}}$ 不是 $z$ 的多项式.
\end{example}

\begin{example}
    考虑 $\ZZ$ 与 $\ZZ[x]$. 设
    \begin{align*}
        f(x) = ax^0 + x + 2x^2 - x^4 - bx^5, \quad g(x) = cx + dx^2 - x^4 - 3x^5,
    \end{align*}
    其中 $a$, $b$, $c$, $d$ 都是整数. 那么, $f(x) = g(x)$ 相当于
    \begin{align*}
        a = 0, \quad 1 = c, \quad 2 = d, \quad 0 = 0, \quad -1 = -1, \quad -b = -3,
    \end{align*}
    也就是
    \begin{align*}
        a = 0, \quad b = 3, \quad c = 1, \quad d = 2.
    \end{align*}
\end{example}

\begin{remark}
    文字 $x$ 的意义在算学中是不断进化的 \term{evolving}. 在中小学里, $x$ 是未知元 \term{unknown}: 虽然它是待求的, 但是它是一个具体的数. 后来在函数里, $x$ 表示变元 \term{variable}, 不过它的取值范围是确定的. 在上面的定义里, $x$ 仅仅是一个文字, 成为不定元.
\end{remark}

下面考虑多项式的运算. 先从加法开始.

\begin{definition}
    设
    \begin{align*}
        f(x) = a_0 x^0 + a_1 x + \cdots + a_n x^n, \quad g(x) = b_0 x^0 + b_1 x + \cdots + b_n x^n
    \end{align*}
    是 $D[x]$ 的元. 规定加法如下:
    \begin{align*}
        f(x) + g(x) = (a_0 + b_0) x^0 + (a_1 + b_1) x + \cdots + (a_n + b_n) x^n.
    \end{align*}
\end{definition}

\begin{example}
    取 $\ZZ[x]$ 的二个元 $f(x)=x^0 + 2x^2$, $g(x)=-3x^0 + 4x - x^3$. 先改写一下:
    \begin{align*}
        f(x) = 1x^0 + 0x + 2x^2 + 0x^3, \quad g(x) = -3x^0 + 4x + 0x^2 + (-1)x^3.
    \end{align*}
    所以
    \begin{align*}
        f(x) + g(x) = -2x^0 + 4x + 2x^2 - x^3.
    \end{align*}
\end{example}

\begin{proposition}
    $D[x]$ 作成加群.
\end{proposition}

\begin{pf}
    设
    \begin{align*}
        f(x) = {} & a_0 x^0 + a_1 x + \cdots + a_n x^n, \\
        g(x) = {} & b_0 x^0 + b_1 x + \cdots + b_n x^n, \\
        h(x) = {} & c_0 x^0 + c_1 x + \cdots + c_n x^n
    \end{align*}
    是 $D[x]$ 的元. 根据加法的定义, $+$ 显然是 $D[x]$ 的二元运算. 因为 $D$ 的加法适合交换律, 故
    \begin{align*}
        g(x) + f(x)
        = {} & (b_0 + a_0) x^0 + (b_1 + a_1) x + \cdots + (b_n + a_n) x^n \\
        = {} & (a_0 + b_0) x^0 + (a_1 + b_1) x + \cdots + (a_n + b_n) x^n \\
        = {} & f(x) + g(x).
    \end{align*}
    也就是说, $D[x]$ 的加法适合交换律.

    注意到
    \begin{align*}
             & (f(x) + g(x)) + h(x)                                                               \\
        = {} & ((a_0 + b_0) x^0 + (a_1 + b_1) x + \cdots + (a_n + b_n) x^n)                       \\
             & \qquad + (c_0 x^0 + c_1 x + \cdots + c_n x^n)                                      \\
        = {} & ((a_0 + b_0) + c_0) x^0 + ((a_1 + b_1) + c_1) x + \cdots + ((a_n + b_n) + c_n) x^n \\
        = {} & (a_0 + b_0 + c_0) x^0 + (a_1 + b_1 + c_1) x + \cdots + (a_n + b_n + c_n) x^n.
    \end{align*}
    类似地, 计算 $f(x) + (g(x) + h(x))$ 也可以得到一样的结果. 也就是说, $D[x]$ 的加法适合结合律.

    零多项式可以写为
    \begin{align*}
        0 = 0x^0 + 0x + \cdots + 0x^n.
    \end{align*}
    这样
    \begin{align*}
        0 + f(x)
        = {} & (0 + a_0) x^0 + (0 + a_1) x + \cdots + (0 + a_n) x^n \\
        = {} & a_0 x^0 + a_1 x + \cdots + a_n x^n                   \\
        = {} & f(x).
    \end{align*}
    类似地, $f(x) + 0 = f(x)$.

    记
    \begin{align*}
        \underline{f}(x) = (-a_0) x^0 + (-a_1) x + \cdots + (-a_n) x^n.
    \end{align*}
    这样
    \begin{align*}
        \underline{f}(x) + f(x)
        = {} & (-a_0 + a_0) x^0 + (-a_1 + a_1) x + \cdots + (-a_n + a_n) x^n \\
        = {} & 0 x^0 + 0 x + \cdots + 0 x^n                                  \\
        = {} & 0.
    \end{align*}
    类似地, $f(x) + \underline{f}(x) = 0$. 以后, 我们把这个 $\underline{f}(x)$ 用普通的符号写为
    \begin{align*}
        -f(x) = -a_0 x^0 - a_1 x - \cdots - a_n x^n.
    \end{align*}

    综上, $D[x]$ 是加群.
\end{pf}

\begin{definition}
    设 $f(x)$, $g(x) \in D[x]$. 规定减法如下:
    \begin{align*}
        f(x) - g(x) = f(x) + (-g(x)).
    \end{align*}
\end{definition}

\begin{remark}
    可以看出, $f(x) \pm g(x)$ 的次既不会超出 $f(x)$ 的次, 也不会超出 $g(x)$ 的次. 用符号写出来, 就是
    \begin{align*}
        \deg (f(x) \pm g(x)) \leq \max \{\, \deg f(x), \deg g(x) \,\}.
    \end{align*}
    若 $\deg f(x) > \deg g(x)$, 则
    \begin{align*}
        \deg (f(x) \pm g(x)) = \deg f(x).
    \end{align*}
    类似地, 若 $\deg f(x) < \deg g(x)$, 则
    \begin{align*}
        \deg (f(x) \pm g(x)) = \deg g(x).
    \end{align*}
\end{remark}

\begin{remark}
    既然 $D[x]$ 是加群, 且每个 $a_i x^i$ ($i = 0,1,\cdots,n$) 都可以看成是多项式, 那么多项式的项的次序是不重要的. 前面的写法称为升次排列 \term{ascending order}. 下面的写法称为降次排列 \term{descending order}:
    \begin{align*}
        a_n x^n + a_{n-1} x^{n-1} + \cdots + a_0 x^0.
    \end{align*}
    这跟中学里接触的多项式是一样的.

    (非零) 多项式的最高次非零项是首项 \term{leading term}. 它的系数是此多项式的首项系数 \term{the coefficient of the leading term}.
\end{remark}

\begin{example}
    $y - y^2 - 7y^4 \in \ZZ[x]$ 可以写为 $-7y^4 - y^2 + y$, 其首项是 $-7y^4$, 且其首项系数是 $-7$.
\end{example}

现在考虑乘法.

\begin{definition}
    设
    \begin{align*}
        f(x) = a_0 x^0 + a_1 x + \cdots + a_m x^m, \quad g(x) = b_0 x^0 + b_1 x + \cdots + b_n x^n
    \end{align*}
    是 $D[x]$ 的元. 规定乘法如下:
    \begin{align*}
        f(x) g(x) = c_0 x^0 + c_1 x + \cdots + c_{m+n} x^{m+n},
    \end{align*}
    其中
    \begin{align*}
        c_k = a_0 b_k + a_1 b_{k-1} + \cdots + a_k b_0.
    \end{align*}
    且约定 $i > m$ 时 $a_i = 0$, $j > n$ 时 $b_j = 0$. 在这个约定下, 不难看出, $\ell > m+n$ 时, $c_\ell = 0$. 所以, 我们至少有
    \begin{align*}
        \deg f(x)g(x) \leq \deg f(x) + \deg g(x).
    \end{align*}
\end{definition}

\begin{example}
    取 $\ZZ[x]$ 的二个元 $f(x)=x^0 + 2x^2$, $g(x)=-3x^0 + 4x - x^3$. 先改写一下:
    \begin{align*}
        f(x) = 1x^0 + 0x + 2x^2, \quad g(x) = -3x^0 + 4x + 0x^2 + (-1)x^3.
    \end{align*}
    所以
    \begin{align*}
         & c_0 = 1 \cdot (-3) = -3,                         \\
         & c_1 = 1 \cdot 4 + 0 \cdot (-3) = 4,              \\
         & c_2 = 1 \cdot 0 + 0 \cdot 4 + 2 \cdot (-3) = -6, \\
         & c_3 = 1 \cdot (-1) + 0 \cdot 0 + 2 \cdot 4 = 7,  \\
         & c_4 = 0 \cdot (-1) + 2 \cdot 0 = 0,              \\
         & c_5 = 2 \cdot (-1) = -2.
    \end{align*}
    所以
    \begin{align*}
        f(x) g(x) = -3x^0 + 4x - 6x^2 + 7x^3 - 2x^5.
    \end{align*}
\end{example}

\begin{example}
    设
    \begin{align*}
        f(x) = a_0 x^0 + a_1 x + \cdots + a_m x^m.
    \end{align*}
    是 $D[x]$ 的元. 零多项式可以写为
    \begin{align*}
        0 = 0x^0,
    \end{align*}
    由此易知
    \begin{align*}
        0f(x) = f(x)0 = 0.
    \end{align*}
\end{example}

\begin{remark}
    设
    \begin{align*}
        f(x) = a_0 x^0 + a_1 x + \cdots + a_m x^m, \quad g(x) = b_0 x^0 + b_1 x + \cdots + b_n x^n
    \end{align*}
    是 $D[x]$ 的元, 且 $a_m \neq 0$, $b_n \neq 0$. 这样, $f(x)g(x)$ 的 $m+n$ 次项就是 $cx^{m+n}$, 其中
    \begin{align*}
        c
        = {} & a_0 b_{m+n} + \cdots + a_{m-1} b_{n+1} + a_m b_n + a_{m+1} b_{n-1} + \cdots + a_{m+n}b_n \\
        = {} & 0 + \cdots + 0 + a_m b_n + 0 + \cdots + 0                                                \\
        = {} & a_m b_n.
    \end{align*}
    因为 $a_m \neq 0$, $b_n \neq 0$, 所以 $a_m b_n \neq 0$ (反证法: 若 $a_m b_n = 0 = a_m 0$, 因为 $a_m \neq 0$, 根据 $D$ 的消去律, 得 $b_n = 0$, 矛盾!). 所以
    \begin{align*}
        \deg f(x) g(x) = \deg f(x) + \deg g(x).
    \end{align*}
    可以验证, 若 $f$ 或 $g$ 的任意一个是 $0$, 这个关系也对.
\end{remark}

\begin{remark}
    设
    \begin{align*}
         & f(x) = px^m = a_0 + a_1 x + \cdots + a_m x^m, \\
         & g(x) = qx^n = b_0 + b_1 x + \cdots + b_n x^n.
    \end{align*}
    当 $i \neq m$ 时, $a_i = 0$; 当 $i=m$ 时, $a_i = p \neq 0$. 当 $j \neq n$ 时, $b_j = 0$; 当 $j=n$ 时, $b_j = q \neq 0$. 现在考虑这二个多项式的积
    \begin{align*}
        f(x)g(x) = c_0 + c_1 x + \cdots + c_{m+n} x^{m+n},
    \end{align*}
    其中
    \begin{align*}
        c_k = a_0 b_k + a_1 b_{k-1} + \cdots + a_k b_0.
    \end{align*}
    我们来看什么时候 $a_\ell b_{k - \ell}$ 不是 $0$. 这相当于要求 $a_\ell$ 跟 $b_{k - \ell}$ 都不是 $0$, 所以
    \begin{align*}
        \ell = m, \quad k - \ell = n,
    \end{align*}
    也就是
    \begin{align*}
        \ell = m, \quad k = m + n.
    \end{align*}
    所以, 当 $k \neq m+n$ 时, $c_k = 0$; 当 $k = m+n$ 时,
    \begin{align*}
        c_{m+n} = a_m b_n = pq \neq 0.
    \end{align*}
    所以, 任取 $m$, $n \in \NN$, 必有
    \begin{align*}
        (px^m) (qx^n) = (pq) x^{m+n}.
    \end{align*}
    特别地, 取 $p=q=1$, 有
    \begin{align*}
        x^m x^n = x^{m+n}.
    \end{align*}
    这里提醒读者: 这个式是形式上的表达式, 其内涵与中学的 ``同底数幂相乘, 底数不变, 指数相加'' 的内涵是不一样的!

    顺便一提, 若 $p$ 跟 $q$ 的一个是 $0$, 则每个 $c_k$ 全为 $0$, 故此时积是零多项式, 此式仍成立.
\end{remark}

\begin{proposition}
    $D[x]$ 作成整环. 所以, $D[x]$ 的一个名字就是 (整环) $D$ 上 ($x$) 的多项式 (整) 环.
\end{proposition}

\begin{pf}
    已经知道, $D[x]$ 是加群. 下面先说明 $D[x]$ 是交换环.

    根据定义, 多项式的乘法还是多项式, 也就是说, 乘法是二元运算.

    设
    \begin{align*}
         & f(x) = a_0 x^0 + a_1 x + \cdots + a_m x^m, \\
         & g(x) = b_0 x^0 + b_1 x + \cdots + b_n x^n, \\
         & h(x) = u_0 x^0 + u_1 x + \cdots + u_s x^s
    \end{align*}
    是 $D[x]$ 的元. 则
    \begin{align*}
         & f(x) g(x) = c_0 x^0 + c_1 x + \cdots + c_{m+n} x^{m+n}, \\
         & g(x) f(x) = d_0 x^0 + d_1 x + \cdots + d_{n+m} x^{n+m},
    \end{align*}
    其中
    \begin{align*}
         & c_k = a_0 b_k + a_1 b_{k-1} + \cdots + a_k b_0, \\
         & d_k = b_0 a_k + b_1 a_{k-1} + \cdots + b_k a_0.
    \end{align*}
    因为 $D$ 的乘法适合交换律, 加法适合交换律与结合律, 故 $c_k = d_k$. 这样, $D[x]$ 的乘法适合交换律.

    不难算出
    \begin{align*}
             & (f(x) g(x)) h(x)                                                                  \\
        = {} & (c_0 x^0 + c_1 x + \cdots + c_{m+n} x^{m+n}) (u_0 x^0 + u_1 x + \cdots + u_s x^s) \\
        = {} & v_0 x^0 + v_1 x + \cdots + v_{m+n+s} x^{m+n+s},
    \end{align*}
    其中
    \begin{align*}
        v_t
        = {} & (\text{the sum of all $a_i b_j u_r$'s with $i+j+r=t$})                           \\
        = {} & a_0 b_0 u_t + a_0 b_1 u_{t-1} + \cdots + a_0 b_t u_0 + a_1 b_0 u_{t-1} + \cdots.
    \end{align*}
    类似地, 计算 $f(x) (g(x) h(x))$ 也可以得到一样的结果. 也就是说, $D[x]$ 的乘法适合结合律.

    现在验证分配律. 前面已经看到, 多项式的乘法是交换的, 所以只要验证一个分配律即可. 不失一般性, 设 $s=n$. 这样
    \begin{align*}
        g(x) + h(x) = (b_0 + u_0) x^0 + (b_1 + u_1) x + \cdots + (b_n + u_n) x^n.
    \end{align*}
    所以
    \begin{align*}
        f(x) (g(x) + h(x)) = p_0 x^0 + p_1 x^1 + \cdots + p_{m+n} x^{m+n},
    \end{align*}
    其中
    \begin{align*}
        p_k
        = {} & a_0 (b_k + c_k) + a_1 (b_{k-1} + c_{k-1}) + \cdots + a_k (b_0 + c_0)                     \\
        = {} & (a_0 b_k + a_0 c_k) + (a_1 b_{k-1} + a_1 c_{k-1}) + \cdots + (a_k b_0 + a_k c_0)         \\
        = {} & (a_0 b_k + a_1 b_{k-1} + \cdots + a_k b_0) + (a_0 c_k + a_1 c_{k-1} + \cdots + a_k c_0).
    \end{align*}
    不难看出, 这就是 $f(x)g(x)$ 的 $k$ 次系数与 $f(x)h(x)$ 的 $k$ 次系数的和. 这样, $D[x]$ 的加法与乘法适合分配律. 至此, 我们知道, $D[x]$ 是交换环.

    交换环离整环还差二步: 一是乘法幺, 二是消去律. 先看消去律. 若 $f(x)g(x) = f(x)h(x)$, $f(x) \neq 0$, 根据分配律,
    \begin{align*}
        0 = f(x)g(x) - f(x)h(x) = f(x) (g(x) - h(x)).
    \end{align*}
    如果 $g(x) - h(x) \neq 0$, 则 $g(x) - h(x)$ 的次不是 $-\infty$. $f(x)$ 的次不是 $-\infty$, 故 $f(x)(g(x) - h(x))$ 的次不是 $-\infty$. 换句话说, $f(x)(g(x) - h(x)) \neq 0$, 矛盾!

    再看乘法幺. 设
    \begin{align*}
        e(x) = x^0.
    \end{align*}
    不难算出
    \begin{align*}
        e(x) f(x) = f(x) e(x) = f(x).
    \end{align*}

    综上, $D[x]$ 是整环.
\end{pf}

\begin{example}
    在前面, 我们直接用定义计算了下面二个多项式的积:
    \begin{align*}
        f(x) = x^0 + 2x^2, \quad g(x) = -3x^0 + 4x - x^3.
    \end{align*}
    现在, 我们利用
    \begin{align*}
        (px^m) (qx^n) = (pq) x^{m+n} \quad (p,q \in D, \, m,n \in \NN)
    \end{align*}
    与运算律再算一次:
    \begin{align*}
        f(x) g(x)
        = {} & (x^0 + 2x^2) (-3x^0 + 4x - x^3)                                 \\
        = {} & x^0 (-3x^0 + 4x - x^3) + 2x^2 (-3x^0 + 4x - x^3)                \\
        = {} & -3x^{0+0} + 4x^{0+1} - x^{0+3} - 6x^{2+0} + 8x^{2+1} - 2x^{2+3} \\
        = {} & -3x^0 + 4x - x^3 - 6x^2 + 8x^3 - 2x^5                           \\
        = {} & -3x^0 + 4x - 6x^2 + 7x^3 - 2x^5.
    \end{align*}
    这跟之前的结果是一致的.
\end{example}

\begin{definition}
    设 $m \in \NN$. 多项式 $f(x)$ 的 $m$ 次幂就是 $m$ 个 $f(x)$ 的积:
    \begin{align*}
        (f(x))^m = \underbrace{f(x) \cdot f(x) \cdot \cdots \cdot f(x)}_{\text{$m$ $f(x)$'s}}.
    \end{align*}
    既然 $D[x]$ 是整环, 那么前面的幂规则都适用. 具体地说, 设 $m$, $n \in \NN$, $f(x)$, $g(x) \in D[x]$, 则
    \begin{align*}
         & (f(x))^m (f(x))^n = (f(x))^{m+n},  \\
         & ((f(x))^m)^n = (f(x))^{mn},        \\
         & (f(x) g(x))^m = (f(x))^m (g(x))^m.
    \end{align*}
    前面, 我们知道
    \begin{align*}
        x^m x^n = x^{m+n}.
    \end{align*}
    当时, 我们还说, 这跟中学的 ``同底数幂相乘, 底数不变, 指数相加'' 有着不一样的内涵. 有了 ``幂'' 这个概念后, 我们发现, $x^m$ 的确可以视为 $m$ 个 $x$ 的积.
\end{definition}

\begin{remark}
    以后, 我们把 $x^0$ 写为 $1$. 换句话说, 代替
    \begin{align*}
        a_0 x^0 + a_1 x + \cdots + a_n x^n,
    \end{align*}
    我们写
    \begin{align*}
        a_0 + a_1 x + \cdots + a_n x^n.
    \end{align*}

    这儿还有一件事儿值得一提. 考虑
    \begin{align*}
        D_0 = \{\, ax^0 \mid a \in D \,\} \subset D[x].
    \end{align*}
    任取 $D_0$ 的二元 $ax^0$, $bx^0$. 首先, $ax^0 = bx^0$ 的一个必要与充分条件是 $a=b$. 然后, 不难看出,
    \begin{align*}
        ax^0 + bx^0 = (a+b)x^0, \quad (ax^0)(bx^0) = (ab)x^0.
    \end{align*}
    由此可以看出, $D_0$ 与 $D$ ``几乎完全一样''. 用摩登 \term{modern} 算学的话来说, ``$D_0$ 与 $D$ 是天然同构的 \term{naturally isomorphic}''.

    我们不打算深究这一点. 上面, 我们把 $x^0$ 写为 $1$; 反过来, $D$ 的元 $a$ 也可以理解为是多项式 $ax^0$. 这跟中学的习惯是一致的.

    最后, 我们指出: 既然非零的 $c \in D$ 可视为 $0$ 次多项式, 那么 $cf(x)$ 也是多项式. 如果
    \begin{align*}
        f(x) = a_0 + a_1 x + \cdots + a_n x^n,
    \end{align*}
    那么
    \begin{align*}
        cf(x) = ca_0 + ca_1 x + \cdots + ca_n x^n,
    \end{align*}
    且
    \begin{align*}
        \deg cf(x) = \deg f(x).
    \end{align*}
\end{remark}
