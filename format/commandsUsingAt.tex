\makeatletter
\newcommand{\leqnomode}{\tagsleft@true\let\veqno\@@leqno}
\newcommand{\reqnomode}{\tagsleft@false\let\veqno\@@eqno}

% Clear even-numbered pages
\renewcommand{\cleardoublepage}{\relax \clearpage
    \if@twoside \ifodd\c@page\relax
        \else \thispagestyle{empty} \ \clearpage\fi\fi}

% For two-sided documents
\renewcommand\ps@headings{%
    \let\@oddfoot\@empty\let\@evenfoot\@empty
    \def\@evenhead{\sffamily\thepage\hfil\slshape\leftmark}%
    \def\@oddhead{\sffamily{\slshape\rightmark}\hfil\thepage}%
    \let\@mkboth\markboth
    \def\sectionmark##1{%
        \markboth {{%
                    \ifnum \c@secnumdepth >\z@
                        \thesection\quad
                    \fi
                    ##1}}{}}%
    \def\mySubSectmark##1{%
        \markright {%
            \ifnum \c@secnumdepth >\@ne
                \thesubsection\quad
            \fi
            ##1}}}

\renewcommand\ps@plain{
    \let\@oddhead\@empty
    \def\@oddfoot{\hfil\sffamily\thepage\hfil}%
    \let\@evenhead\@empty
    \let\@evenfoot\@oddfoot}

\makeatother

% Taken from https://tex.stackexchange.com/questions/412815/double-bar-overline
%%%%%%%%%%%%%%%%%%%%%%%%%%%%%%%%%%%%%%%%%%%%%%%%%%%%%%%%%%%%%%%%%%%
%% This code is a slight modification of Hendrik Vogt's \widebar %%
%% See: https://tex.stackexchange.com/questions/16337            %%
%%%%%%%%%%%%%%%%%%%%%%%%%%%%%%%%%%%%%%%%%%%%%%%%%%%%%%%%%%%%%%%%%%%
\makeatletter
\let\save@mathaccent\mathaccent
\newcommand*\if@single[3]{%
\setbox0\hbox{${\mathaccent"0362{#1}}^H$}%
\setbox2\hbox{${\mathaccent"0362{\kern0pt#1}}^H$}%
\ifdim\ht0=\ht2 #3\else #2\fi
}
%The bar will be moved to the right by a half of \macc@kerna, which is computed by amsmath:
\newcommand*\rel@kern[1]{\kern#1\dimexpr\macc@kerna}
%If there's a superscript following the bar, then no negative kern may follow the bar;
%an additional {} makes sure that the superscript is high enough in this case:
\newcommand*\wideaccent[2]{\@ifnextchar^{{\wide@accent{#1}{#2}{0}}}{\wide@accent{#1}{#2}{1}}}
%Use a separate algorithm for single symbols:
\newcommand*\wide@accent[3]{\if@single{#2}{\wide@accent@{#1}{#2}{#3}{1}}{\wide@accent@{#1}{#2}{#3}{2}}}
\newcommand*\wide@accent@[4]{%
    \begingroup
    \def\mathaccent##1##2{%
        %Enable nesting of accents:
        \let\mathaccent\save@mathaccent
        %If there's more than a single symbol, use the first character instead (see below):
        \if#42 \let\macc@nucleus\first@char \fi
        %Determine the italic correction:
        \setbox\z@\hbox{$\macc@style{\macc@nucleus}_{}$}%
        \setbox\tw@\hbox{$\macc@style{\macc@nucleus}{}_{}$}%
        \dimen@\wd\tw@
        \advance\dimen@-\wd\z@
        %Now \dimen@ is the italic correction of the symbol.
        \divide\dimen@ 3
        \@tempdima\wd\tw@
        \advance\@tempdima-\scriptspace
        %Now \@tempdima is the width of the symbol.
        \divide\@tempdima 10
        \advance\dimen@-\@tempdima
        %Now \dimen@ = (italic correction / 3) - (Breite / 10)
        \ifdim\dimen@>\z@ \dimen@0pt\fi
        %The bar will be shortened in the case \dimen@<0 !
        \rel@kern{0.6}\kern-\dimen@
        \if#41
            #1{\rel@kern{-0.6}\kern\dimen@\macc@nucleus\rel@kern{0.4}\kern\dimen@}%
            \advance\dimen@0.4\dimexpr\macc@kerna
            %Place the combined final kern (-\dimen@) if it is >0 or if a superscript follows:
            \let\final@kern#3%
            \ifdim\dimen@<\z@ \let\final@kern1\fi
            \if\final@kern1 \kern-\dimen@\fi
        \else
            #1{\rel@kern{-0.6}\kern\dimen@#2}%
        \fi
    }%
    \macc@depth\@ne
    \let\math@bgroup\@empty \let\math@egroup\macc@set@skewchar
    \mathsurround\z@ \frozen@everymath{\mathgroup\macc@group\relax}%
    \macc@set@skewchar\relax
    \let\mathaccentV\macc@nested@a
    %The following initialises \macc@kerna and calls \mathaccent:
    \if#41
        \macc@nested@a\relax111{#2}%
    \else
        %If the argument consists of more than one symbol, and if the first token is
        %a letter, use that letter for the computations:
        \def\gobble@till@marker##1\endmarker{}%
        \futurelet\first@char\gobble@till@marker#2\endmarker
        \ifcat\noexpand\first@char A\else
            \def\first@char{}%
        \fi
        \macc@nested@a\relax111{\first@char}%
    \fi
    \endgroup
}
\makeatother
%%%%%%%%%%%%%%%%%%%%%%%%%%%%%%%%%%%%%%%%%%%%%%%%%%%%%%%%%%%%%%%%%%%

\newcommand\doubleoverline[1]{\overline{\overline{#1}}}

\newcommand\widebar{\wideaccent\overline}
\newcommand\widebarbar{\wideaccent\doubleoverline}
