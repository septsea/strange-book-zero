% Well, I cannot find a better set of fonts that support unicode-math...
% \setsansfont[
%     BoldFont=* SemiBold,
%     BoldItalicFont=* SemiBold Italic
% ]{FiraGO}
% \setmainfont[
%     BoldFont=* SemiBold,
%     BoldItalicFont=* SemiBold Italic
% ]{FiraGO}
% \setmathfont{Fira Math}

% ================================================================================
% Set the fonts. You can use whatever you like, or simply remove this part
% and remove the option "fontset=none".


\def\heiti{\CJKfontspec{Sarasa Mono SC Light}}
\def\songti{\CJKfontspec{Source Han Serif CN}}
\def\kaishu{\CJKfontspec{Sarasa Mono SC Light}}
\def\fangsong{\CJKfontspec{Sarasa Mono SC Light}}
\setmonofont{Sarasa Mono SC Light}
\setsansfont{Source Sans Pro Light}
\setCJKmonofont{Sarasa Mono SC Light}
\setCJKmainfont[ItalicFont=Sarasa Mono SC Light]{Source Han Serif CN}
\setCJKsansfont{Sarasa Mono SC Light}
% I use Sarasa Mono SC instead of Kaishu or Fangsong.
% The operating system I use do not have the 4 fonts:
% Heiti, Songti, Kaishu, Fangsong.

% ================================================================================

% Punctuation style: narrow some punctuation symbols
\punctstyle{kaiming}

% Footnote
% Footnotes per page
\usepackage[perpage]{footmisc}
% Define a new set of footnote symbols
\DefineFNsymbols*{myFootnoteStyle}{
    % If unicode-math is not enabled, the dollar signs can be removed.
    {$\dagger$}{$\ddagger$}{$\S$}{$\P$}{$\parallel$}{$\dagger\dagger$}{$\ddagger\ddagger$}
}
\setfnsymbol{myFootnoteStyle}
\renewcommand{\thefootnote}{\fnsymbol{footnote}}
% My own footnote command
\providecommand{\myFN}{}
\renewcommand{\myFN}[1]{\footnote{\sffamily \ #1}}

% Deprecated. Now I use the theorem layout
% Numbering style for exercises
% \usepackage{enumitem}
% \renewcommand{\labelenumi}{\textbf{\theenumi.}}
% \setlist[enumerate]{
%   leftmargin=0em,
%   listparindent=\parindent,
%   itemindent=*,
%   labelindent=\parindent,
%   align=left,
%   parsep=\parskip
% }

% This is used to occupy some space.
\newenvironment{foobar}{\vfill \itshape \sffamily  \small \begin{center}  \begin{tabular}{l}}{\end{tabular}\end{center} \vspace{2ex}}

% Proof and solution style
\usepackage{amsthm}
% Use \nopunct to remove the dot: https://tex.stackexchange.com/questions/268912/can-i-change-the-dot-to-a-colon-after-proof-in-amsthm
% Use \vspace{-\topsep} to remove extra line: https://tex.stackexchange.com/questions/59755/space-before-proof
% \newenvironment{pf}{\begin{proof}[\indent\bf 证 \hspace{-.1em}\hspace{-\labelsep}\hspace{1em}\nopunct]\vspace{-\topsep}}{\end{proof}}
% \newenvironment{solution}{\begin{proof}[\indent\bf 解 \hspace{-.1em}\hspace{-\labelsep}\hspace{1em}\nopunct]\vspace{-\topsep}}{\end{proof}}
% \newenvironment{pas}{\begin{proof}[\indent\bf 证明与解 \hspace{-.1em}\hspace{-\labelsep}\hspace{1em}\nopunct]\vspace{-\topsep}}{\end{proof}}
\newenvironment{pf}{\begin{proof}[\indent\bf 证 \hspace{-.1em}\hspace{-\labelsep}\hspace{1em}\nopunct]}{\end{proof}}
\newenvironment{solution}{\begin{proof}[\indent\bf 解 \hspace{-.1em}\hspace{-\labelsep}\hspace{1em}\nopunct]}{\end{proof}}
\newenvironment{pas}{\begin{proof}[\indent\bf 证明与解 \hspace{-.1em}\hspace{-\labelsep}\hspace{1em}\nopunct]}{\end{proof}}

\newenvironment{bibi}{\sffamily \itshape}

\newtheoremstyle{exer} % ⟨name⟩
{} % ⟨Space above⟩
{} % ⟨Space below⟩
{} % ⟨Body font⟩
{\parindent} % ⟨Indent amount⟩
{\bfseries} % ⟨Theorem head font⟩
{.} % ⟨Punctuation after theorem head⟩
{.5em} % ⟨Space after theorem head⟩
{} % ⟨Theorem head spec (can be left empty, meaning 'normal')⟩
\theoremstyle{exer}
\newtheorem{exercise}{}[subsection]

\newtheoremstyle{exer*} % ⟨name⟩
{} % ⟨Space above⟩
{} % ⟨Space below⟩
{} % ⟨Body font⟩
{\parindent} % ⟨Indent amount⟩
{\bfseries} % ⟨Theorem head font⟩
{} % ⟨Punctuation after theorem head⟩
{0em} % ⟨Space after theorem head⟩
{} % ⟨Theorem head spec (can be left empty, meaning 'normal')⟩
\theoremstyle{exer*}
\newtheorem*{exercise*}{}

\newtheoremstyle{remarkStyle} % ⟨name⟩
{} % ⟨Space above⟩
{} % ⟨Space below⟩
{} % ⟨Body font⟩
{\parindent} % ⟨Indent amount⟩
{\bfseries} % ⟨Theorem head font⟩
{} % ⟨Punctuation after theorem head⟩
{1em} % ⟨Space after theorem head⟩
{} % ⟨Theorem head spec (can be left empty, meaning 'normal')⟩
\theoremstyle{remarkStyle}
\newtheorem*{supplement}{补充题}
\newtheorem*{proposition}{命题}
\newtheorem*{definition}{定义}
\newtheorem*{example}{例}
\newtheorem*{remark}{评注}

% https://oomake.com/question/175655
\renewcommand{\theexercise}{\arabic{exercise}}
% \newenvironment{pf}{\begin{proof}[\indent\bf 证 \hphantom{.\ }\nopunct]\vspace{-\topsep}}{\vspace{-\topsep}\end{proof}}
% \newenvironment{solution}{\begin{proof}[\indent\bf 解 \hphantom{.\ }\nopunct]\vspace{-\topsep}}{\vspace{-\topsep}\end{proof}}
% \newenvironment{pas}{\begin{proof}[\indent\bf 证明与解 \hphantom{.\ }\nopunct]\vspace{-\topsep}}{\vspace{-\topsep}\end{proof}}
% Use a black square instead
% \renewcommand{\qedsymbol}{$\blacksquare$}

% Use some special characters.
\usepackage{pifont}

\def\period{\text{。}}

% Allow page breaks in display formulae
\allowdisplaybreaks[4]
